% !Mode:: "TeX:UTF-8"%確保文檔utf-8編碼
%新加入的命令如下: reduline showendnotes 
%新加入的环境如下:solution solutionorbox solutionorlines solutionordottedlines

\documentclass[12pt]{exam}%实际双面打印加上twoside
\newlength{\textpt}
\setlength{\textpt}{12pt}

\usepackage{teachingplan}

%写上答案或者不写上答案%1  
\printanswers  


%讲知识点然后测试,测试通过继续下一个知识点。不通过给予小提示,然后继续测试,如果通过那么通过,如果还是不会做,那么继续讲解知识点,并讲解这个题目,然后重新给出一个题目重新测试。

%如果有精力,后面再准备一套中考真题。

%\excludecomment{someextra}
%\includecomment{someextra}


\begin{document}
\begin{coverpages}
\title{比一比认一认}
\author{万泽}
\maketitle
\begin{abstract}
本文是数学的第一课,不需要前置知识。讲课内容类似于小学语文课,而讲课的目的是通过认识一个物体(比如苹果,梨子等)或者几个物体来获得关于数量的直观认知。数量不要超过5个,数字有一个,两个,三个,很多个。同时让同学对数字的大小比较,各种不同形状的物体有一个直观的认识。这个课程虽然简单,但作为具体内容设计还是需要花些心思,我有空写一下。
\end{abstract}
\tableofcontents
\end{coverpages}


\section{比一比认一认}




\section{参考资料}




\end{document}



