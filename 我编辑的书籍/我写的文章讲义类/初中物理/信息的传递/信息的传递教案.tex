% !Mode:: "TeX:UTF-8"%確保文檔utf-8編碼
%新加入的命令如下: reduline showendnotes 
%新加入的环境如下:solution solutionorbox solutionorlines solutionordottedlines

\documentclass[12pt]{exam}
\newlength{\textpt}
\setlength{\textpt}{12pt}

\usepackage{teachingplan}
\usepackage{wallpaper}
\usepackage{comment}
\usetikzlibrary{positioning}

%输出方案 
%1.自己用答案,知识点等都有  ——老师版
%2.给学生用基础差的可以考虑加上知识点,但不写答案——基础差版
%3.练习题加提高题——学生版
%4.练习题加提高题加综合题——学霸版

%写上答案或者不写上答案%1  
\printanswers  

%提高题%1  4
%\includecomment{improveexercises}
\excludecomment{improveexercises}
%综合题%1  4  5
%\includecomment{advanceexercises}
\excludecomment{advanceexercises}


\CenterWallPaper{1}{教案模板-2.pdf}
\newcommand{\shortanswerline}{\rule[-2pt]{50pt}{0.4pt}}
\renewcommand{\solutiontitle}{\noindent\textbf{}}

\newcommand{\keti}{热和能}
\newcommand{\zhongdian}{1.分子热运动 2.内能 3.比热容\\  4.热机 5.能量的转化和守恒}
\renewcommand{\section}[1]{{\large\sffamily  #1} \par}
\renewcommand{\subsection}[1]{{\normalsize\sffamily  #1} \par}
\newcommand{\leftnote}[1]{\marginpar{\ttfamily\fontsize{10}{10}\selectfont #1}}



\newcommand{\answer}[2][50pt]{{\setlength{\answerlinelength}{
#1} \answerline*[#2]}}

\begin{document}
\ThisCenterWallPaper{1}{教案模板-1.pdf}
\vspace*{80pt}
\keti \par
\zhongdian \par
\section{分子热运动}
1876年美国发明家贝尔发明了第一部电话

话筒把声信号变成变化的电流,电流沿着导线把信息传到远方,在另一端,电流使听筒的膜片振动,携带信息的电流又变成了声音。(话筒把声信号转化为电信号;听筒把电信号转化为声信号)

电话交换机:为了提高线路的利用率,人们发明了电话交换机。现在除特殊需要的极少数电话还要专线之外,一般电话都是通过电话交换机来转接的。

早期的电话交换机是依靠话务员手工操作来接线和拆线的,1891年出来了自动电话交换机,它通过电磁继电器进行接线。现在的程控电话交换机利用电子计算机技术,自动进行接线操作。

模拟通信和数字通信
模拟信号:声音转换成信号电流时,信号电流的频率、振幅变化的情况跟声音的频率、振幅变化的情况完全一样,“模仿”着声信号的“一举一动”,这种电流传递的信号叫做模拟信号,使用模拟信号的通信方式叫做模拟通信。

数字信号:用不同符号的不同组合表示的信号叫做数学信号,使用数学信号的通信方式叫做数字通信。

模拟信号容易失真;数字信号抗干扰能力强,便于加工处理,可以加密。
在电话与交换机之间一般传递模拟信号,在交换机之间传递数字信号。



电磁波可以在真空中传播,不需要任何介质。

电磁波在真空中的波速为c,大小和光速一样, c=$3$3×108m/s 

\section{练习题}














\begin{improveexercises}
\section{提高题}

\end{improveexercises}

\begin{advanceexercises}
\section{综合题}

\end{advanceexercises}

\ThisCenterWallPaper{1}{教案模板-3.pdf}

\end{document}



