% !Mode:: "TeX:UTF-8"%確保文檔utf-8編碼
%tikz draw the onepage for cv or titlepage or poster

\documentclass[11pt,oneside]{book}
\newlength{\textpt}
\setlength{\textpt}{11pt}


\usepackage{onepage}
\usetikzlibrary{calc}

\begin{document}
\begin{titlepage}
%start drawing
%a4paper size 297:210 mm
\begin{tikzpicture}
%先画网格线
%为了简便起见,约定图画在坐标系的像素第一象
%上间距默认有5mm
\draw[step=1,color=gray!40] (0,0) grid (20,29);
%这样最多得到20,29。
%现在开始定义整个页面的一些点
%左顶点
\path (0,29) coordinate (top-left);
%右顶点
\path (20,29) coordinate (top-right);
%左底点
\path (0,0) coordinate (bottom-left);
%右底点
\path (20,0) coordinate (bottom-right);
%测试从左顶点到右底点
%\draw[color=red] (top-left) -- (bottom-right);
%接下来多使用偏移地址,基本单位是1或者0.5。

\newcommand{\redball}[1]{\shade[ball color=red] #1 circle (0.25);}
\newcommand{\shorttext}[2]{\node[align=center,right] (name) at #1 {#2};}
\path (top-left) ++(1,-2) coordinate (name-point);
\shorttext{(name-point)}{{\HUGE\sffamily 万泽}}
\path (name-point) ++(4,0) coordinate (email-point);
\shorttext{(email-point)}{{\large\sffamily 邮箱:a358003542@gmail.com}}
\path (email-point) ++(0,-1) coordinate (phone-point);
\shorttext{(phone-point)}{{\large\sffamily 电话:123456789}}


\path (1.5,24) coordinate (born-point);
\shade[ball color=red] (born-point) circle (0.25);
\draw[->] (born-point) ++(0.25,0) -- ++(4,0) node[midway,above] {1986年} node[right] {生于湖南常德。} ;

%24-1986
%1个刻度就是5年
%0-2086
\path (born-point) ++(0,-2) coordinate (ten-point);
\draw[color=red,thick] (born-point) -- (ten-point);
\shade[ball color=red] (ten-point) circle (0.25) ;
\draw[->] (ten-point) ++(0.25,0) -- ++(4,0) node[midway,above] {9岁还是10岁} node[right] {黑铁时代,似乎人生的磨难才刚刚开始。} ;

\path (ten-point) ++(0,-2) coordinate (twenty-point);
\draw[color=red,thick] (ten-point) -- (twenty-point);
\shade[ball color=red] (twenty-point) circle (0.25) ;
\draw[->] (twenty-point) ++(0.25,0) -- ++(4,0) node[midway,above] {20岁} node[right] {说点什么。} ;






 
 
\end{tikzpicture}
\end{titlepage}
\end{document}



