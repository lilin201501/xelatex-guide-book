% !Mode:: "TeX:UTF-8"%確保文檔utf-8編碼
\documentclass[tikz,border=12pt,12pt]{standalone}
\usepackage{graphicx}
\usepackage{pgfplots}
\usepackage{fontspec}
\usepackage{xeCJK}
\setCJKmainfont[BoldFont=Adobe 黑体 Std,ItalicFont=Adobe 楷体 Std]
    {Adobe 宋体 Std}%影響rmfamily字體
\setCJKsansfont{Adobe 黑体 Std}%影響sffamily字體
\setCJKmonofont{Adobe 楷体 Std}%影響ttfamily字體
\usetikzlibrary{mindmap,trees}
\begin{document}
%配色建议 主要是三层  父层统一黑色(白字)  自己层(根据学科内容不同和黑色混)  子层和自己层相同方案 red   blue  orange  green  olive brown  purple teal violet lime   pink cyan magenta gray

\begin{tikzpicture}[scale=1]
\path[mindmap,grow cyclic,align=flush center,text=white]
    node[concept,font=\huge,concept color=black] (shuxue){数学}
    child[concept color=red!50!black]{
      node[concept,font=\Large] (wuli) { 物理}
      child[concept color=orange!50!black] { node[concept] (huaxue) {化学} }
      child[concept color=olive!50!black] { node[concept](shengwu) {生物} }
      }
      child[concept color=blue!50!black] {node[concept,font=\Large] (jisuanji) { 计算机}
      child[] {node[concept] (shujuchuli) { 数据处理} } 
      child[] {node[concept] (paibanzhitu) {排版制图} } 
      } 
     node[concept,concept color=green!50!black,font=\huge] (sixiang) at(-6,2) {思想}
     child[grow=180,concept color=cyan!75!black] {node[concept,font=\Large] (xinlixue) {心理学} }
     child[grow=-90,concept color=brown!50!black] {node[concept,font=\Large] (guojiazhengzhi) {国家政治}}
     node[concept,concept color=magenta,font=\huge] (yishu) at(0,6) {艺术} 
     node[concept,concept color=gray,font=\huge] (gongyejishu) at(0,-6) {工业技术}
     child[grow=-160,concept color=purple!50!black] {node[concept,concept color=purple!50!black] (jingjixue) {经济学}} ;
     
     
     
     
     %物理到数据处理 
   \path (wuli) to[circle connection bar switch color=from (red!50!black) to (blue!50!black)] (shujuchuli);
   %艺术到排版制图
   \path (yishu) to[circle connection bar switch color=from (magenta) to (blue!50!black)] (paibanzhitu);
   %工业技术到国家政治
    \path (gongyejishu) to[circle connection bar switch color=from (gray) to (brown!50!black)] (guojiazhengzhi);
    %工业技术到物理
     \path (gongyejishu) to[circle connection bar switch color=from (gray) to (red!50!black)] (wuli);
      \path (gongyejishu) to[circle connection bar switch color=from (gray) to (orange!50!black)] (huaxue);
      %化学到生物
   \path (huaxue) to[circle connection bar switch color=from (orange!50!black) to (olive!50!black)] (shengwu);
   %物理到计算机
   \path (wuli) to[circle connection bar switch color=from (red!50!black) to (blue!50!black)] (jisuanji);
   %思想到数学到艺术
   \path (sixiang) to[circle connection bar switch color=from (green!50!black) to (black)] (shuxue); 
   \path (sixiang) to[circle connection bar switch color=from (green!50!black) to (magenta)] (yishu);
   \path (sixiang) to[circle connection bar switch color=from (green!50!black) to (brown!50!black)] (guojiazhengzhi);
   %国家政治到经济学
   \path (jingjixue) to[circle connection bar switch color=from (purple!50!black) to (brown!50!black)] (guojiazhengzhi);
   \path (jingjixue) to[circle connection bar switch color=from (purple!50!black) to (gray)] (gongyejishu);
   \path (sixiang) to[circle connection bar switch color=from (green!50!black) to (cyan!75!black)] (xinlixue);
   
\end{tikzpicture}


\end{document}