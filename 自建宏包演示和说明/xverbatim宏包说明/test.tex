% !Mode:: "TeX:UTF-8"%確保文檔utf-8編碼
%新加入的命令如下:addchtoc addsectoc reduline printendnotes hlabel
%新加入的环境如下:common-format  fig scalefig

\documentclass[11pt,oneside]{book}
\newlength{\textpt}
\setlength{\textpt}{11pt}
\newif\ifphone
\phonefalse

\usepackage{myconfig}
\usepackage{mytitle}


\usepackage{pdfpages}


\begin{document}
\frontmatter

\titlea{xverbatim}
\titleb{display source code or execute it}
\titlec{一个显示代码或者执行代码的环境}
\author{万泽}
\authorinfo{作者:湖南常德人氏}
\editor{wanze}
\email{a358003542@gmail.com}
\editorinfo{编者:wanze.}
\version{1.0}
\titleLC

\addchtoc{前言}
\chapter*{前言}
\begin{common-format}
开头说的话

%这里空一行。

\end{common-format}


\addchtoc{目录}
\setcounter{tocdepth}{2}
\tableofcontents

\begin{common-format}
\mainmatter


\chapter{bash语言}
默认的可选项是1,就是只显示代码,所以不用写。
\begin{xverbatim}{bash}
a=1
echo $a
\end{xverbatim}

\chapter{python语言}
\section{基本可执行模式}
这是一段测试文字,这是一段测试文字,这是一段测试文字,这是一段测试文字,这是一段测试文字,这是一段测试文字,这是一段测试文字,

这是一段测试文字,这是一段测试文字,这是一段测试文字,这是一段测试文字,这是一段测试文字,这是一段测试文字,这是一段测试文字,这是一段测试文字,
\begin{xverbatim}[129]{py}
for i in range(1, 5):
    print (i)
else:
    print ("The for loop is over")
\end{xverbatim}
这是一段测试文字,这是一段测试文字,这是一段测试文字,这是一段测试文字,这是一段测试文字,这是一段测试文字,这是一段测试文字,这是一段测试文字,



%\section{有交互的情况}
%有交互的情况使用139模式,但是需要手动控制输出到sys.argv[0].out这个文件中。
%\begin{xverbatim}[1]{py}
%import sys
%x=input('请输入一个实数:')
%string001='你输入的这个实数乘以2等于:'+ str(float(x)*2)
%pyout=open(sys.argv[0]+'.out',"w")
%print('请输入一个实数:',file=pyout)
%print(string001,file=pyout)
%pyout.close()
%\end{xverbatim}

\chapter{c语言}
\begin{xverbatim}[1]{c}
#include <stdio.h>
main()
{
printf ("Hello World!\n");
}
\end{xverbatim}

\chapter{lisp语言}
common-lisp语言可执行,后缀是lisp。
\begin{xverbatim}[129]{lisp}
(format t "hello")
\end{xverbatim}

\chapter{tex语言}
\begin{xverbatim}[1]{tex}
% !Mode:: "TeX:UTF-8"%確保文檔utf-8編碼
%新加入的命令如下:addchtoc addsectoc reduline printendnotes hlabel
%新加入的环境如下:common-format  fig scalefig

\documentclass[11pt,oneside]{book}
\newlength{\textpt}
\setlength{\textpt}{11pt}
\newif\ifphone
\phonefalse

\usepackage{myconfig}
\usepackage{mytitle}

\begin{document}
\frontmatter

\titlea{书籍}
\titleb{使用\LaTeX 排版}
\titlec{一种良好的风格}
\author{作者}
\authorinfo{作者:}
\editor{编者}
\email{a358003542@gmail.com}
\editorinfo{编者:}
\version{1.0}
\titleLC


\end{xverbatim}


\chapter{php语言}
\begin{xverbatim}{php}
<?php
echo "<html>
<head>
<title>Hello World Page</title>
</head>
<body>
Hello World!
</body>
</html>"
?>
\end{xverbatim}

\chapter{html语言}
\begin{xverbatim}{html}
<HTML>
<HEAD>
<TITLE>Hello, World Page!</TITLE>
</HEAD>

<BODY>
Hello, World!
</BODY>
</HTML>
\end{xverbatim}

\chapter{java语言}
\begin{xverbatim}{java}
class HelloWorld {
        public static void main (String args[]) {
        for (;;) {
                 System.out.print("Hello World ");
                 }
        }
}
\end{xverbatim}

\chapter{perl语言}
\begin{xverbatim}{pl}
print "Hello, World!\n" while (1);
\end{xverbatim}




\chapter{后缀名总览txt mode}
\begin{xverbatim}{txt}
Pygments version 1.6, (c) 2006-2013 by Georg Brandl.

Lexers:
~~~~~~~
* Clipper, XBase:
    FoxPro (filenames *.PRG, *.prg)
* Cucumber, cucumber, Gherkin, gherkin:
    Gherkin (filenames *.feature)
* RobotFramework, robotframework:
    RobotFramework (filenames *.txt, *.robot)
* abap:
    ABAP (filenames *.abap)
* ada, ada95ada2005:
    Ada (filenames *.adb, *.ads, *.ada)
* ahk:
    autohotkey (filenames *.ahk, *.ahkl)
* antlr-as, antlr-actionscript:
    ANTLR With ActionScript Target (filenames *.G, *.g)
* antlr-cpp:
    ANTLR With CPP Target (filenames *.G, *.g)
* antlr-csharp, antlr-c#:
    ANTLR With C# Target (filenames *.G, *.g)
* antlr-java:
    ANTLR With Java Target (filenames *.G, *.g)
* antlr-objc:
    ANTLR With ObjectiveC Target (filenames *.G, *.g)
* antlr-perl:
    ANTLR With Perl Target (filenames *.G, *.g)
* antlr-python:
    ANTLR With Python Target (filenames *.G, *.g)
* antlr-ruby, antlr-rb:
    ANTLR With Ruby Target (filenames *.G, *.g)
* antlr:
    ANTLR 
* apacheconf, aconf, apache:
    ApacheConf (filenames .htaccess, 
    apache.conf, apache2.conf)
* applescript:
    AppleScript (filenames *.applescript)
* as, actionscript:
    ActionScript (filenames *.as)
* as3, actionscript3:
    ActionScript 3 (filenames *.as)
* aspectj:
    AspectJ (filenames *.aj)
    
* aspx-cs:
    aspx-cs (filenames *.aspx, *.asax, 
    *.ascx, *.ashx, *.asmx, *.axd)
* aspx-vb:
    aspx-vb (filenames *.aspx, *.asax, 
    *.ascx, *.ashx, *.asmx, *.axd)
* asy, asymptote:
    Asymptote (filenames *.asy)
* autoit, Autoit:
    AutoIt (filenames *.au3)
* awk, gawk, mawk, nawk:

    Awk (filenames *.awk)
* basemake:
    Base Makefile 
* bash, sh, ksh:
    Bash (filenames *.sh, *.ksh, *.bash, *.ebuild,
     *.eclass, .bashrc, bashrc, .bash_*, bash_*)
* bat:
    Batchfile (filenames *.bat, *.cmd)
* bbcode:
    BBCode 
* befunge:
    Befunge (filenames *.befunge)
* blitzmax, bmax:
    BlitzMax (filenames *.bmx)
* boo:
    Boo (filenames *.boo)
* brainfuck, bf:
    Brainfuck (filenames *.bf, *.b)
* bro:
    Bro (filenames *.bro)
* bugs, winbugs, openbugs:
    BUGS (filenames *.bug)
* c-objdump:
    c-objdump (filenames *.c-objdump)
* c:
    C (filenames *.c, *.h, *.idc)
* ca65:
    ca65 (filenames *.s)
* cbmbas:
    CBM BASIC V2 (filenames *.bas)
* ceylon:
    Ceylon (filenames *.ceylon)
* cfengine3, cf3:
    CFEngine3 (filenames *.cf)
* cfm:
    Coldfusion HTML (filenames *.cfm, *.cfml, *.cfc)
* cfs:
    cfstatement 
* cheetah, spitfire:
    Cheetah (filenames *.tmpl, *.spt)
* clojure, clj:
    Clojure (filenames *.clj)
* cmake:
    CMake (filenames *.cmake, CMakeLists.txt)
* cobol:
    COBOL (filenames *.cob, *.COB, *.cpy, *.CPY)
* cobolfree:
    COBOLFree (filenames *.cbl, *.CBL)
* coffee-script, coffeescript:
    CoffeeScript (filenames *.coffee)
* common-lisp, cl:
    Common Lisp (filenames *.cl, *.lisp, *.el)
* console:
    Bash Session (filenames *.sh-session)
* control:
    Debian Control file (filenames control)
* coq:
    Coq (filenames *.v)
* cpp, c++:
    C++ (filenames *.cpp, *.hpp, *.c++, *.h++, *.cc, *.hh, *.cxx, *.hxx, *.C, *.H, *.cp, *.CPP)
* cpp-objdump, c++-objdumb, cxx-objdump:
    cpp-objdump (filenames *.cpp-objdump, *.c++-objdump, *.cxx-objdump)
* croc:
    Croc (filenames *.croc)
* csharp, c#:
    C# (filenames *.cs)
* css+django, css+jinja:
    CSS+Django/Jinja 
* css+erb, css+ruby:
    CSS+Ruby 
* css+genshitext, css+genshi:
    CSS+Genshi Text 
* css+lasso:
    CSS+Lasso 
* css+mako:
    CSS+Mako 
* css+mako:
    CSS+Mako 
* css+myghty:
    CSS+Myghty 
* css+php:
    CSS+PHP 
* css+smarty:
    CSS+Smarty 
* css:
    CSS (filenames *.css)
* cuda, cu:
    CUDA (filenames *.cu, *.cuh)
* cython, pyx:
    Cython (filenames *.pyx, *.pxd, *.pxi)
* d-objdump:
    d-objdump (filenames *.d-objdump)
* d:
    D (filenames *.d, *.di)
* dart:
    Dart (filenames *.dart)
* delphi, pas, pascal, objectpascal:
    Delphi (filenames *.pas)
* dg:
    dg (filenames *.dg)
* diff, udiff:
    Diff (filenames *.diff, *.patch)
* django, jinja:
    Django/Jinja 
* dpatch:
    Darcs Patch (filenames *.dpatch, *.darcspatch)
* dtd:
    DTD (filenames *.dtd)
* duel, Duel Engine, Duel View, JBST, jbst, JsonML+BST:
    Duel (filenames *.duel, *.jbst)
* dylan-console, dylan-repl:
    Dylan session (filenames *.dylan-console)
* dylan-lid, lid:
    DylanLID (filenames *.lid, *.hdp)
* dylan:
    Dylan (filenames *.dylan, *.dyl, *.intr)
* ec:
    eC (filenames *.ec, *.eh)
* ecl:
    ECL (filenames *.ecl)
* elixir, ex, exs:
    Elixir (filenames *.ex, *.exs)
* erb:
    ERB 
* erl:
    Erlang erl session (filenames *.erl-sh)
* erlang:
    Erlang (filenames *.erl, *.hrl, *.es, *.escript)
* evoque:
    Evoque (filenames *.evoque)
* factor:
    Factor (filenames *.factor)
* fan:
    Fantom (filenames *.fan)
* fancy, fy:
    Fancy (filenames *.fy, *.fancypack)
* felix, flx:
    Felix (filenames *.flx, *.flxh)
* fortran:
    Fortran (filenames *.f, *.f90, *.F, *.F90)
* fsharp:
    FSharp (filenames *.fs, *.fsi)
* gas:
    GAS (filenames *.s, *.S)
* genshi, kid, xml+genshi, xml+kid:
    Genshi (filenames *.kid)
* genshitext:
    Genshi Text 
* glsl:
    GLSL (filenames *.vert, *.frag, *.geo)
* gnuplot:
    Gnuplot (filenames *.plot, *.plt)
* go:
    Go (filenames *.go)
* gooddata-cl:
    GoodData-CL (filenames *.gdc)
* gosu:
    Gosu (filenames *.gs, *.gsx, *.gsp, *.vark)
* groff, nroff, man:
    Groff (filenames *.[1234567], *.man)
* groovy:
    Groovy (filenames *.groovy)
* gst:
    Gosu Template (filenames *.gst)
* haml, HAML:
    Haml (filenames *.haml)
* haskell, hs:
    Haskell (filenames *.hs)
* haxeml, hxml:
    Hxml (filenames *.hxml)
* html+cheetah, html+spitfire:
    HTML+Cheetah 
* html+django, html+jinja:
    HTML+Django/Jinja 
* html+evoque:
    HTML+Evoque (filenames *.html)
* html+genshi, html+kid:
    HTML+Genshi 
* html+lasso:
    HTML+Lasso 
* html+mako:
    HTML+Mako 
* html+mako:
    HTML+Mako 
* html+myghty:
    HTML+Myghty 
* html+php:
    HTML+PHP (filenames *.phtml)
* html+smarty:
    HTML+Smarty 
* html+velocity:
    HTML+Velocity 
* html:
    HTML (filenames *.html, *.htm, *.xhtml, *.xslt)
* http:
    HTTP 
* hx, haXe:
    haXe (filenames *.hx)
* hybris, hy:
    Hybris (filenames *.hy, *.hyb)
* idl:
    IDL (filenames *.pro)
* iex:
    Elixir iex session 
* ini, cfg:
    INI (filenames *.ini, *.cfg)
* io:
    Io (filenames *.io)
* ioke, ik:
    Ioke (filenames *.ik)
* irc:
    IRC logs (filenames *.weechatlog)
* jade, JADE:
    Jade (filenames *.jade)
* jags:
    JAGS (filenames *.jag, *.bug)
* java:
    Java (filenames *.java)
* jlcon:
    Julia console 
* js+cheetah, javascript+cheetah, js+spitfire, javascript+spitfire:
    JavaScript+Cheetah 
* js+django, javascript+django, js+jinja, javascript+jinja:
    JavaScript+Django/Jinja 
* js+erb, javascript+erb, js+ruby, javascript+ruby:
    JavaScript+Ruby 
* js+genshitext, js+genshi, javascript+genshitext, javascript+genshi:
    JavaScript+Genshi Text 
* js+lasso, javascript+lasso:
    JavaScript+Lasso 
* js+mako, javascript+mako:
    JavaScript+Mako 
* js+mako, javascript+mako:
    JavaScript+Mako 
* js+myghty, javascript+myghty:
    JavaScript+Myghty 
* js+php, javascript+php:
    JavaScript+PHP 
* js+smarty, javascript+smarty:
    JavaScript+Smarty 
* js, javascript:
    JavaScript (filenames *.js)
* json:
    JSON (filenames *.json)
* jsp:
    Java Server Page (filenames *.jsp)
* julia, jl:
    Julia (filenames *.jl)
* kconfig, menuconfig, linux-config, kernel-config:
    Kconfig (filenames Kconfig, *Config.in*, external.in*, standard-modules.in)
* koka:
    Koka (filenames *.kk, *.kki)
* kotlin:
    Kotlin (filenames *.kt)
* lasso, lassoscript:
    Lasso (filenames *.lasso, *.lasso[89])
* lhs, literate-haskell:
    Literate Haskell (filenames *.lhs)
* lighty, lighttpd:
    Lighttpd configuration file 
* live-script, livescript:
    LiveScript (filenames *.ls)
* llvm:
    LLVM (filenames *.ll)
* logos:
    Logos (filenames *.x, *.xi, *.xm, *.xmi)
* logtalk:
    Logtalk (filenames *.lgt)
* lua:
    Lua (filenames *.lua, *.wlua)
* make, makefile, mf, bsdmake:
    Makefile (filenames *.mak, Makefile, makefile, Makefile.*, GNUmakefile)
* mako:
    Mako (filenames *.mao)
* mako:
    Mako (filenames *.mao)
* maql:
    MAQL (filenames *.maql)
* mason:
    Mason (filenames *.m, *.mhtml, *.mc, *.mi, autohandler, dhandler)
* matlab:
    Matlab (filenames *.m)
* matlabsession:
    Matlab session 
* minid:
    MiniD (filenames *.md)
* modelica:
    Modelica (filenames *.mo)
* modula2, m2:
    Modula-2 (filenames *.def, *.mod)
* monkey:
    Monkey (filenames *.monkey)
* moocode:
    MOOCode (filenames *.moo)
* moon, moonscript:
    MoonScript (filenames *.moon)
* mscgen, msc:
    Mscgen (filenames *.msc)
* mupad:
    MuPAD (filenames *.mu)
* mxml:
    MXML (filenames *.mxml)
* myghty:
    Myghty (filenames *.myt, autodelegate)
* mysql:
    MySQL 
* nasm:
    NASM (filenames *.asm, *.ASM)
* nemerle:
    Nemerle (filenames *.n)
* newlisp:
    NewLisp (filenames *.lsp, *.nl)
* newspeak:
    Newspeak (filenames *.ns2)
* nginx:
    Nginx configuration file 
* nimrod, nim:
    Nimrod (filenames *.nim, *.nimrod)
* nsis, nsi, nsh:
    NSIS (filenames *.nsi, *.nsh)
* numpy:
    NumPy 
* objdump:
    objdump (filenames *.objdump)
* objective-c++, objectivec++, obj-c++, objc++:
    Objective-C++ (filenames *.mm, *.hh)
* objective-c, objectivec, obj-c, objc:
    Objective-C (filenames *.m, *.h)
* objective-j, objectivej, obj-j, objj:
    Objective-J (filenames *.j)
* ocaml:
    OCaml (filenames *.ml, *.mli, *.mll, *.mly)
* octave:
    Octave (filenames *.m)
* ooc:
    Ooc (filenames *.ooc)
* opa:
    Opa (filenames *.opa)
* openedge, abl, progress:
    OpenEdge ABL (filenames *.p, *.cls)
* perl, pl:
    Perl (filenames *.pl, *.pm)
* php, php3, php4, php5:
    PHP (filenames *.php, *.php[345], *.inc)
* plpgsql:
    PL/pgSQL 
* postgresql, postgres:
    PostgreSQL SQL dialect 
* postscript:
    PostScript (filenames *.ps, *.eps)
* pot, po:
    Gettext Catalog (filenames *.pot, *.po)
* pov:
    POVRay (filenames *.pov, *.inc)
* powershell, posh, ps1:
    PowerShell (filenames *.ps1)
* prolog:
    Prolog (filenames *.prolog, *.pro, *.pl)
* properties:
    Properties (filenames *.properties)
* protobuf:
    Protocol Buffer (filenames *.proto)
* psql, postgresql-console, postgres-console:
    PostgreSQL console (psql) 
* puppet:
    Puppet (filenames *.pp)
* py3tb:
    Python 3.0 Traceback (filenames *.py3tb)
* pycon:
    Python console session 
* pypylog, pypy:
    PyPy Log (filenames *.pypylog)
* pytb:
    Python Traceback (filenames *.pytb)
* python, py, sage:
    Python (filenames *.py, *.pyw, *.sc, SConstruct, SConscript, *.tac, *.sage)
* python3, py3:
    Python 3 
* qml, Qt Meta Language, Qt modeling Language:
    QML (filenames *.qml)
* racket, rkt:
    Racket (filenames *.rkt, *.rktl)
* ragel-c:
    Ragel in C Host (filenames *.rl)
* ragel-cpp:
    Ragel in CPP Host (filenames *.rl)
* ragel-d:
    Ragel in D Host (filenames *.rl)
* ragel-em:
    Embedded Ragel (filenames *.rl)
* ragel-java:
    Ragel in Java Host (filenames *.rl)
* ragel-objc:
    Ragel in Objective C Host (filenames *.rl)
* ragel-ruby, ragel-rb:
    Ragel in Ruby Host (filenames *.rl)
* ragel:
    Ragel 
* raw:
    Raw token data 
* rb, ruby, duby:
    Ruby (filenames *.rb, *.rbw, Rakefile, *.rake, *.gemspec, *.rbx, *.duby)
* rbcon, irb:
    Ruby irb session 
* rconsole, rout:
    RConsole (filenames *.Rout)
* rd:
    Rd (filenames *.Rd)
* rebol:
    REBOL (filenames *.r, *.r3)
* redcode:
    Redcode (filenames *.cw)
* registry:
    reg (filenames *.reg)
* rhtml, html+erb, html+ruby:
    RHTML (filenames *.rhtml)
* rst, rest, restructuredtext:
    reStructuredText (filenames *.rst, *.rest)
* rust:
    Rust (filenames *.rs, *.rc)
* sass, SASS:
    Sass (filenames *.sass)
* scala:
    Scala (filenames *.scala)
* scaml, SCAML:
    Scaml (filenames *.scaml)
* scheme, scm:
    Scheme (filenames *.scm, *.ss)
* scilab:
    Scilab (filenames *.sci, *.sce, *.tst)
* scss:
    SCSS (filenames *.scss)
* shell-session:
    Shell Session (filenames *.shell-session)
* smali:
    Smali (filenames *.smali)
* smalltalk, squeak:
    Smalltalk (filenames *.st)
* smarty:
    Smarty (filenames *.tpl)
* sml:
    Standard ML (filenames *.sml, *.sig, *.fun)
* snobol:
    Snobol (filenames *.snobol)
* sourceslist, sources.list:
    Debian Sourcelist (filenames sources.list)
* sp:
    SourcePawn (filenames *.sp)
* spec:
    RPMSpec (filenames *.spec)
* splus, s, r:
    S (filenames *.S, *.R, .Rhistory, .Rprofile)
* sql:
    SQL (filenames *.sql)
* sqlite3:
    sqlite3con (filenames *.sqlite3-console)
* squidconf, squid.conf, squid:
    SquidConf (filenames squid.conf)
* ssp:
    Scalate Server Page (filenames *.ssp)
* stan:
    Stan (filenames *.stan)
* systemverilog, sv:
    systemverilog (filenames *.sv, *.svh)
* tcl:
    Tcl (filenames *.tcl)
* tcsh, csh:
    Tcsh (filenames *.tcsh, *.csh)
* tea:
    Tea (filenames *.tea)
* tex, latex:
    TeX (filenames *.tex, *.aux, *.toc)
* text:
    Text only (filenames *.txt)
* trac-wiki, moin:
    MoinMoin/Trac Wiki markup 
* treetop:
    Treetop (filenames *.treetop, *.tt)
* ts:
    TypeScript (filenames *.ts)
* urbiscript:
    UrbiScript (filenames *.u)
* vala, vapi:
    Vala (filenames *.vala, *.vapi)
* vb.net, vbnet:
    VB.net (filenames *.vb, *.bas)
* velocity:
    Velocity (filenames *.vm, *.fhtml)
* verilog, v:
    verilog (filenames *.v)
* vgl:
    VGL (filenames *.rpf)
* vhdl:
    vhdl (filenames *.vhdl, *.vhd)
* vim:
    VimL (filenames *.vim, .vimrc, .exrc, .gvimrc, _vimrc, _exrc, _gvimrc, vimrc, gvimrc)
* xml+cheetah, xml+spitfire:
    XML+Cheetah 
* xml+django, xml+jinja:
    XML+Django/Jinja 
* xml+erb, xml+ruby:
    XML+Ruby 
* xml+evoque:
    XML+Evoque (filenames *.xml)
* xml+lasso:
    XML+Lasso 
* xml+mako:
    XML+Mako 
* xml+mako:
    XML+Mako 
* xml+myghty:
    XML+Myghty 
* xml+php:
    XML+PHP 
* xml+smarty:
    XML+Smarty 
* xml+velocity:
    XML+Velocity 
* xml:
    XML (filenames *.xml, *.xsl, *.rss, *.xslt, *.xsd, *.wsdl)
* xquery, xqy, xq, xql, xqm:
    XQuery (filenames *.xqy, *.xquery, *.xq, *.xql, *.xqm)
* xslt:
    XSLT (filenames *.xsl, *.xslt, *.xpl)
* xtend:
    Xtend (filenames *.xtend)
* yaml:
    YAML (filenames *.yaml, *.yml)

Formatters:
~~~~~~~~~~~
* bbcode, bb:
    Format tokens with BBcodes. 
    These formatting codes are used by
     many bulletin boards, so you can highlight
      your sourcecode with pygments before posting it there. 
* bmp, bitmap:
    Create a bitmap image from source code. 
    This uses the Python Imaging Library to generate a
     pixmap from the source code. (filenames *.bmp)
* gif:
    Create a GIF image from source code.
     This uses the Python Imaging Library to 
     generate a pixmap from the source code. (filenames *.gif)
* html:
    Format tokens as HTML 4
     ``<span>`` tags within a ``<pre>`` tag, wrapped in a ``<div>`` tag. 
     The ``<div>``'s CSS class can be set by the `cssclass` option. (filenames *.html, *.htm)
* img, IMG, png:
    Create a PNG image from source code. 
    This uses the Python Imaging Library to generate a
     pixmap from the source code. (filenames *.png)
* jpg, jpeg:
    Create a JPEG image from source code. This uses 
    the Python Imaging Library to generate a pixmap from
     the source code. (filenames *.jpg)
* latex, tex:
    Format tokens as LaTeX code. This needs the `fancyvrb`
     and `color` standard packages. (filenames *.tex)
* raw, tokens:
    Format tokens as a raw representation for storing
     token streams. (filenames *.raw)
* rtf:
    Format tokens as RTF markup. 
    This formatter automatically outputs full RTF documents 
    with color information and other useful stuff. Perfect for
    Copy and Paste into Microsoft® Word® documents. (filenames *.rtf)
* svg:
    Format tokens as an SVG graphics file.  This formatter 
    is still experimental. Each line of code is a ``<text>`` 
    element with explicit ``x`` and ``y`` coordinates containing 
    ``<tspan>`` elements with the individual token styles. (filenames *.svg)
* terminal, console:
    Format tokens with ANSI color sequences, for output in 
    a text console. Color sequences are terminated at newlines, 
    so that paging the output works correctly. 
* terminal256, console256, 256:
    Format tokens with ANSI color 
    sequences, for output in a
     256-color terminal or console. 
     Like in `TerminalFormatter` 
     color sequences are terminated
      at newlines, so that paging 
     the output works correctly. 
* text, null:
    Output the text unchanged without
     any formatting. (filenames *.txt)
\end{xverbatim}

%这里空一行

\end{common-format}
\end{document}



