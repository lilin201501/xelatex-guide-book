% !Mode:: "TeX:UTF-8"%確保文檔utf-8編碼
%新加入的命令如下:addchtoc addsectoc reduline showendnotes hlabel
%新加入的环境如下:common-format  fig linefig xverbatim

\documentclass[11pt,oneside]{book}
\newlength{\textpt}
\setlength{\textpt}{11pt}
\newif\ifphone
\phonefalse

\usepackage{myconfig}


\begin{document}

\begin{common-format}

\chapter{万泽}
\vspace{15pt}

\section{基本信息}

\noindent
\begin{tabular}{p{0.45\linewidth} p{0.45\linewidth}}
\textbf{籍贯:}湖南常德人 & \textbf{出生日期:}1986年2月 \\ 
\textbf{邮箱:}\href{mailto: a358003542@gmail.com}{a358003542@gmail.com} & \textbf{电话:}156******** \\ 
\end{tabular} 


\section{基本技能}
\vspace{-5mm}
\begin{enumerate}
\item 熟悉数学物理化学的基础理论知识,尤其精通化学学科,其中对纳米材料、金属氧化物、催化材料等有一定的研究。
\item 熟悉化学实验室各个基本的实验操作,对一般分析手段比如:气相,红外,XRD等都很熟悉。会查阅资料、分析数据、制图计算等。
\item 熟悉windows下的各个软件操作,熟悉ubuntu下的各个软件操作。
\item 精通\LaTeX 排版,熟悉tikz制图,会基本的python编程。
\end{enumerate}


\section{背景资料}
万泽于1986年出生于湖南常德,在临澧读完初中和高中之后。于2004年在湖南文理学院应用化学专业就读并顺利毕业。后在上海理研塑料有限公司担任技术员一职,对PVC塑料相关知识有一定的了解。于2010年在四川理工学院就读应用化学专业研究生,主攻纳米金属氧化物的光催化研究。目前已完成毕业论文答辩,今年6月份毕业。决定从事教育事业。

\section{其他信息}
\vspace{-5mm}
\begin{itemize}
\item 爱好阅读评论排版书籍,并喜欢学习和编程。不断成长,将来可能会用自己的计算机知识来解决工作中实际的问题。
\item 个性比较独立内向,不太擅长人与人之间的客套寒暄。
\end{itemize}


%这里空一行

\end{common-format}
\end{document}



