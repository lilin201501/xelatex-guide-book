% !Mode:: "TeX:UTF-8"%確保文檔utf-8編碼
%新加入的命令如下: reduline showendnotes 
%新加入的环境如下:solution solutionorbox solutionorlines solutionordottedlines

\documentclass[12pt]{exam}%实际双面打印加上twoside
\newlength{\textpt}
\setlength{\textpt}{12pt}

\usepackage{teachingplan}

%写上答案或者不写上答案%1  
\printanswers  


%讲知识点然后测试,测试通过继续下一个知识点。不通过给予小提示,然后继续测试,如果通过那么通过,如果还是不会做,那么继续讲解知识点,并讲解这个题目,然后重新给出一个题目重新测试。

%如果有精力,后面再准备一套中考真题。

%\excludecomment{someextra}
%\includecomment{someextra}


\begin{document}
\begin{coverpages}
\title{引言}
\author{万泽}
\maketitle
\tableofcontents
\end{coverpages}
\begin{flushright}
\begin{notecard}[blue!30]{14em}
\ttfamily
未经反思自省的人生不值得活。

{\hfill\sffamily —苏格拉底}
\end{notecard}
\end{flushright}
\section{人为什么活着}
记得在我读高一的时候,我和我的同学们在上晚自习。大家都在复习,而我也在复习历史。因为马上就要考试了。在我记着各种各样历史年代和名词的时候,我累了。背靠着后面的桌子,头望着那混浊的日光灯。突然脑子里蹦出一个奇怪的问题:“人活着为了什么?”。这个问题在后来的人生时隐时显,但一直都还在那里。

我的灵魂孤寂而又高傲,人生有很多事情我竟大多是事后才知晓,总是稀里糊涂地过着日子,一颗好玩的心从没改变过。包括我在知识上的思想上的那种狂热,亦不过是一个孩子心使然。我本应过早地知道死亡的意义,但我没有。而那时我本应像大家一样专心学习,却一再去追寻这死亡的意义。

在高二的时候,我读书求知几乎达到疯狂的境地,从早自习起我就手捧着一本书,上课的时候我也“悄悄地”在读,到晚上有的时候我甚至手捧个书直到半夜。别人都在外面打篮球,而我却依然体弱多病,为的就是看别人是怎么想的,看别人是怎样活的。那个时候我的自我意识刚刚萌发,愚昧地相信着绝对自由意志论,我坚信我能决定我的一切。所以那个时候我有一句话:“我想怎样活就怎样活。”对!生活,整个人生,多么幸福灿烂的人生摆在我的面前,以至那时我觉得,几辈子也活不完啊!

我并不是一个十分合群的人,不过这也似乎解释不了我独特的阅读兴趣,但无疑那时我力图使自己成为一个高雅的人。时下流行的玄幻小说我一本都没看过,我喜好名人传记和哲学小品。那种内心的坦白和真诚深深地吸引了我。以至到现在我都是这么认为的,人第一要紧的是真诚!

我看书的时候完完全全忘记了自己。自己的一切。以至我经常觉得书中的主人公太像我了。而在阅读别人的思想言论时,我更是设身处地地为别人着想,让别人的思想如野马一般在我的脑海中驰骋。

我的模仿性是如此之强,所以有一天,我在教师里看书。一个漂亮的女孩和一个男孩子在我旁边说说笑笑。

我抬起了一张忧郁的老脸,一本正经地说道:尔等系良家女子,自应言检行慎,不应为区区小事而狂笑不止。自然吾诘难尔等,非显吾心境之不佳,实为尔等着想也。\sidenote{《堂吉诃德》}

谁不是从模仿开始自己的人生旅程的呢?


\section{活出意义来}
但我们不可能一辈子永远去模仿别人的人生(别人的成功,荣耀,庸俗或者堕落。)人们活得越久就越发现自己是一个和地球上任何一个,和自古以来的任何一个人类都不相同,而每一个人都必须选择一种自己的活法,这可比简单的模仿别人的人生要难的多了。

我遇到这样的一个人,他(她)告诉我我没有什么特别的追求,就希望简单生活,吃好点穿好点。我是个多么不讨人喜欢的人呵,我直截了当地指出:希望生活简单点,吃好点穿好点,这些都不是追求,这是人的本性,因为是人都希望少动点脑子,吃好点穿好点,所以这些都算不上追求。

模仿别人重复别人或者将人的本性提升到追求的高度都不助于人精神层面的升华(没有这种升华人的灵性就缺少某种感动,人的灵魂就缺少力量。),你们看啊,四周的植物动物,他们都遵循着自己的本性,只有人不同,人与人之间长相都差不多(一个脑袋两个胳膊),但你根本无法越测他下一秒将会干什么,因为他有着丰富的内心的世界。

把不同的人放在相同的环境中,比如说牢笼中,他们的反应也会完全不同的。是的,我现在谈论的就是人身上具有的一种潜能,那就是超越自身本性超越周遭环境的潜能。人性,不是动物性的加强版,人,因为具有智慧,超越了动物短视的视角,从而获得一种跨越空间和时间的视角,在这种视角之下,人的存在意义,就绝不是为别人给予或者别人指定的,也不是环境命定的,而是自身寻求的。

关于读者您的运命和存在的意义,我不能说的更多了。这里我也不会谈及宗教或者其他哲学,只是这样简单而肯定地告诉你,不要试图模仿和复制别人的人生,想一想你横跨寰宇从古到今都找不到一个和你一模一样的存在,如何你去简单模仿和复制别人的人生,那是多么的浪费呵(而且极易失败)。请用您自己的头脑思考人生,了解自己认识自己,活出您自己人生的意义吧。\sidenote{《活出意义来》}


\section{我的告白}
关于我的个性或者优缺点之类的我就不多说了,长话短说就是我不为任何公司工作,我为上帝工作。上帝告诉我他赐给我的每一天鲜活的生命就是最好的礼物,他告诉我要爱人,而自己也是人,所以当然要爱好自己,因为把自己爱好了服务好了然后推己及人将这些爱和服务分享给他人,就是最大的最真心的爱了。关于要发挥自己的特长等等之类的话我就不多说了,现在让我将我心中的一些想法和计划说出来吧。

\begin{linefig}[0.85]{科学知识体系}
\caption{科学知识体系}
\label{fig:科学知识体系}
\end{linefig}

\subsection{知识大网络}
我一直对科学类思想类书籍都特别感兴趣,到我大概初三的时候脑子里就有了一个雄心壮志,就是想写一本书,书名叫“人类知识体系纲要”,希望这本书将我所学的所有的科学知识都汇总起来。且不论这本书,但这个知识大网络的想法正是我现在要介绍的,请看图\ref{fig:科学知识体系}。

现在想像我们有一张特别巨大的知识大网络的地图——也就是常说的思维导图,那么上面这张图只是这张大地图的最外面的某几个节点的情况,我们可以想像科学发展到现在知识是如此的繁杂,物理课上讲的东西数学里面又讲,化学里面讲的东西生物里面又讲,如果我们有这样一张知识大网络地图,那么下一代的学生学习知识将不带一点重复,而且完全个性化的自己想走那条路就走那条路,对什么知识感兴趣就学什么知识,在专门的老师带领下根本不会迷路,这样将会大大提升他们的学习效率。

在我进入社会之后,提出这个构想,中国的公司没有这样的眼光,当然也没有这样的能力,他们不认可。而且他们教的学生大多是差生,也就是中国的民办教育针对的是差生的教育,也难怪越教越差越办越差了(sorry)。他们认为这样的计划对差生是没有帮助的,但是我认为这样的计划不仅对爱学习优秀的学生有帮助帮助很大,而且对差生也有很大的帮助——因为差生可以根据这张的知识大网络地图发现自己还有那些知识点是没有掌握的(那些自身不愿意学的差生除外,我不认为在这个世界上有那个老师能够教的好自身不愿意学的学生,就算能够辅助监督那也只是暂时有点效果罢了,然后那个学生离开那个老师成绩将会变得更遭而且更加无法改造了,因为他(她)对那个老师产生依赖心理了。)

\subsection{边学边测模式}
这个知识大网络的内部最小的节点是关于某一块知识点比如长度、自然数等收集的资料,主文件是讲义,讲义里面可能还会附带其他试卷,图片,音频,视频等。而这个讲义除了要做到高质量活泼风趣之外,在学习流程上最大的一个特色就是边学边测模式,也就是每讲一个小知识然后提供一个测试题来测验,之后测验通过了才能进入下一个知识点的学习,然后一个稍大一点的知识点提供一个对应的综合性质的测验题,然后整个这一块知识点提供一小套测验题或者某一个综合性质的大题。

这样的边学边测模式算不上什么新东西,但结合在这个知识大网络里面将显得格外的有用,因为知识大网络绘制了要掌握某个技能或者解决某个问题要在这个知识大网络里面最少需要掌握那些知识点,而在具体到这些知识点之后用边学边测模式实现精准的知识点下典型的题目,这样将实现学生做最少的题目达到最佳效果的目的,而这又将极大的解放学生的学习和解决问题时间。时间是这个世界最宝贵的东西,被解放出来的时间意味着我们在这个世界更多的\uwave{成功可能性}。


\subsection{简单的计划?}
我个人的时间和精力是有限的,而这个计划的繁荣无疑需要更多的参与者加入进来。刚开始我并不指望会有人感兴趣,但如果以后这个计划不断繁荣发展下去我会很高兴的。

目前的简单计划就是写讲义,不断写下去。因为我用了一个思维导图管理软件,可以创建自由节点和自动整理图形,所以这方面的管理工作我需要费心了。至于编写内容上,如果一切顺利的话。。小学数学, 初中数学物理化学,高中的数学物理化学,一言蔽之,高中及以下的所有科学知识(可能包含一些思想类的知识),这是我编写的第一个目的地。

等这个目的地达到之后,将会在这颗树上继续扩展,主要就我感兴趣的知识点继续补充进来,分类就不属于讲义类了,在习题和知识点上将更加自由(因为不需要过分考虑中考和高考了。)这些我已知的我已经掌握的知识点和技能都加入进去之后——这就是第二个目的地了\footnote{感觉这个目标这辈子可能很难达到了。}。

第三个目的地就是开创性质的,也就是完全根据自己的兴趣和最近的探索成果加进去就是了,比如说我想研究下塑料的硬度和可塑剂之间的数学关系(需要融合实验,测量,数据处理-软件,预测,检验等过程),那么弄好了之后包括讲义视频等等资料封装为一个文件夹放进去,然后节点加入知识大网络相应位置并建立好连接关系即可。

第三个目的地我完全以普通个人的姿态进入这个知识大网络了,所有的人这个时候都可以参与进来,从一些趣味实验视频到什么高端的玩意儿。这个时候可能会有很多信息管理上的问题,anyway,太远的事情了,就此打住吧。




\end{document}



