% !Mode:: "TeX:UTF-8"%確保文檔utf-8編碼
%新加入的命令如下: reduline showendnotes 
%新加入的环境如下:solution solutionorbox solutionorlines solutionordottedlines

\documentclass[12pt]{exam}
\newlength{\textpt}
\setlength{\textpt}{12pt}

\usepackage{teachingplan}
\usepackage{wallpaper}
\usepackage{comment}
\usetikzlibrary{positioning}

%输出方案 
%1.自己用答案,知识点等都有  ——老师版
%2.给学生用基础差的可以考虑加上知识点,但不写答案——基础差版
%3.练习题加提高题——学生版
%4.练习题加提高题加综合题——学霸版

%写上答案或者不写上答案%1  
\printanswers  

%提高题%1  4
%\includecomment{improveexercises}
\excludecomment{improveexercises}
%综合题%1  4  5
%\includecomment{advanceexercises}
\excludecomment{advanceexercises}


\CenterWallPaper{1}{教案模板-2.pdf}
\newcommand{\shortanswerline}{\rule[-2pt]{50pt}{0.4pt}}
\renewcommand{\solutiontitle}{\noindent\textbf{}}

\newcommand{\keti}{热和能}
\newcommand{\zhongdian}{1.分子热运动 2.内能 3.比热容\\  4.热机 5.能量的转化和守恒}
\renewcommand{\section}[1]{{\large\sffamily  #1} \par}
\renewcommand{\subsection}[1]{{\normalsize\sffamily  #1} \par}
\newcommand{\leftnote}[1]{\marginpar{\ttfamily\fontsize{10}{10}\selectfont #1}}



\newcommand{\answer}[2][50pt]{{\setlength{\answerlinelength}{
#1} \answerline*[#2]}}

\begin{document}
\ThisCenterWallPaper{1}{教案模板-1.pdf}
\vspace*{80pt}
\keti \par
\zhongdian \par
\section{分子热运动}
物质由\answerline*[分子]组成。如何估算单位立方米体积下有多少个分子?

不同的物质在相互接触时,彼此进入对方的现象叫做\answerline*[扩散]现象。气体、液体、固体都能发生这种现象,这种现象说明了一切物质的分子都在\dotuline{不停地}做\answerline*[无规则]的运动,这种运动也叫分子的\answerline*[热运动]。因为温度\answerline*[越高],这种运动就越剧烈;这种现象还说明了分子之间有\answerline*[间隙]。
\leftnote{溶解和蒸发?}

分子间的作用力:分子间有\answerline*[引力]和\answerline*[斥力]。引力使固体、液体保持一定的体积。分子间的斥力使分子已离得很近的固体、液体很难进一步被压缩。
\leftnote{用弹簧模型引伸开去}


\section{内能}
物体内部分子因为\answerline*[热运动]具有动能,分子和分子之间由于吸引和排斥还具有势能,我们叫它\answerline*[分子势能],这两种能的总和叫做物体的\answerline*[内能]。一切物体,不论温度高低,都具有\answerline*[内能]。

影响物体内能大小的因素有温度,质量,材料,存在状态等。其中物体的质量,材料、状态相同时,温度越高物体内能\answerline*[越大];物体的温度、材料、状态相同时,物体的质量越大,物体的内能\answerline*[越大]。

\leftnote{内能和机械能的不同?}

改变物体的内能两种方法:\answerline*[做功]和\answerline*[热传递],这两种方法对改变物体的内能是等效的。

做功:对物体做功,物体的内能增加;物体对外做功,本身的内能会减少。

热传递:温度不同的物体相互接触,低温的物体温度升高,高温的物体温度降低,这个过程叫热传递。发生热传递时,高温物体内能减少,低温物体内能增加。

热量:在热传递过程中,传递的内能的多少叫热量(物体含有多少热量的说法是错误的)。

热量,内能,功都属于能量范畴,单位都是焦耳。

解释温室效应?


\section{比热容}
比热容(c):单位质量的某种物质温度升高(或降低)1℃,吸收(或放出)的热量叫做这种物质的比热。
  比热容是物质的一种属性,它不随物质的体积、质量、形状、位置、温度的改变而改变,只要物质种类和状态相同,比热就相同。
  比热容的单位是:J/(kg•℃),读作:焦耳每千克摄氏度。
  水的比热容是:C=4.2×103J/(kg•℃),它表示的物理意义是:每千克的水当温度升高(或降低)1℃时,吸收(或放出)的热量是4.2×103焦耳。
  热量的计算:
  ①Q吸=cm(t-t0)=cm△t升(Q吸是吸收热量,单位是J;c是物体比热容,单位是:J/(kg•℃);m是质量;t0是初始温度;t是后来的温度。
  ②Q放=cm(t0-t)=cm△t降 

\section{练习题}














\begin{improveexercises}
\section{提高题}

\end{improveexercises}

\begin{advanceexercises}
\section{综合题}

\end{advanceexercises}

\ThisCenterWallPaper{1}{教案模板-3.pdf}

\end{document}



