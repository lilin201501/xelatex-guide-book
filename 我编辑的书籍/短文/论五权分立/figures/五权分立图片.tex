% !Mode:: "TeX:UTF-8"%確保文檔utf-8編碼
%新加入的命令如下:addchtoc addsectoc reduline showendnotes hlabel
%新加入的环境如下:common-format  fig linefig xverbatim

\documentclass[11pt,oneside]{book}
\newlength{\textpt}
\setlength{\textpt}{11pt}
\newif\ifphone
\phonefalse

\usepackage{myconfig}
\usetikzlibrary{intersections,calc,external,positioning}

\tikzset{external/system call={xelatex \tikzexternalcheckshellescape -halt-on-error
-interaction=batchmode -jobname "\image" "\texsource";
dvips -o "\image".ps "\image".dvi;
ps2eps "\image.ps"}}
\tikzexternalize

\tikzexternalize % activate!

\begin{document}


\begin{tikzpicture}
\begin{scope}[rotate=18]
\def\n{5}
\pgfmathsetmacro\i{\n-1}
\foreach \x in {0,...,\i}
{
\def\pointname{\x}
\coordinate (\pointname) at ($(0,0) +(\x*360/\n:4cm)$)  ;
}
\end{scope}

\node[circle,draw] (huo) at (0) {党};
\node[circle,draw]  (mu) at (1) {法};
\node[circle,draw] (shui) at (2) {政};
\node[circle,draw] (jin) at (3) {商};
\node[circle,draw] (tu) at (4) {军};

\draw [-latex] (mu) to [bend left] node[above] {生}  (huo);
\draw [-latex] (huo) to [bend left=15] node[right] {生} (tu);
\draw [-latex] (tu) to [bend left=5] node[below] {生}   (jin);
\draw [-latex] (jin) to [bend left=15] node[left] {生} (shui);
\draw [-latex] (shui) to [bend left] node[above] {生} (mu);

\draw [-latex] (mu) to node[right] {克}  (tu);
\draw [-latex] (huo) to  node[below] {克} (jin);
\draw [-latex] (tu) to  node[below] {克}   (shui);
\draw [-latex] (jin) to  node[left] {克} (mu);
\draw [-latex] (shui) to  node[above] {克} (huo);

\end{tikzpicture}





%这里空一行

\end{document}



