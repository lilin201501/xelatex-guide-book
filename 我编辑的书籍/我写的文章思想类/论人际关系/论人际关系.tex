% !Mode:: "TeX:UTF-8"%確保文檔utf-8編碼
%新加入的命令如下:addchtoc addsectoc reduline showendnotes hlabel
%新加入的环境如下:common-format  fig linefig 

\documentclass[11pt,oneside]{article}
\newlength{\textpt}
\setlength{\textpt}{11pt}
\newif\ifphone
\phonefalse

\usepackage{articleconfig}




\begin{document}
\title{论人际关系}
\author{德山书生}
\date{}
\maketitle

\begin{abstract}
小文章不用目录,用摘要方便读者快速了解内容。
\end{abstract}


\begin{common-format}

\section{引论}
人之本性,趋利避害也。趋利,就是追逐利益,这个好理解;而避害就是恐惧,恐惧将来可能的灾祸 。当人真正被灾难吞没的时候,人只是在靠本能应急反应,近似条件反射似的进行。而真正影响人的行为决策的就是两点:一是利益,二是恐惧。

不管人们承认与否,这个世界控制人有两种手段,一是利益二是恐惧。你看水里的金鱼,一个陌生人来了,尽管那个陌生人无意伤害它,但是它还是慢慢远离那个未知。而那个陌生人抓起一把食物给金鱼,然后金鱼欢拥而至了。

为什么读书越多越“反动”?一,读书消灭无知,而无知是恐惧的源泉,所以读书消灭恐惧。二,读书消灭盲目的利益追逐,而追逐利益的人最好被操纵。无所畏惧,不为利益所动,这样的人是独立自由的,也是最“反动”的。

人之间的关系分为两种:操纵与被操纵的关系;交易或者合作关系。人们通常不愿意承认人与人之间的操纵和被操纵关系,但是这就好像八卦中的阴和阳一样缺一不可:如果人们之间没有操纵被操纵关系那么合作关系也将建立不起来;而人们因为没有利益互补建立不起合作关系那么也就不会建立操纵与被操纵关系。

广告的作用很大程度上就是起到了消除人们对于未知的恐惧。 你可以想像一下你在超市里面前那么多牌子,那么人们为什么会潜意识地被广告影响呢?人们都被广告洗脑了然后相信广告里面说的那个牌子的东西多么多么好吗?不是的,人们选择那个牌子只是因为他听说过,然后广告消除了他们内心说不清道不明的对于未知事物的恐惧,使得他们有勇气选择那个牌子。

\section{男女关系的分析}
现在让我们来分析男女朋友关系是如何建立的,我以为男女关系的发展分为以下几个阶段:
\begin{description}
\item[广告阶段] 在这一阶段,男女之间刚刚认识,(现在让我以男人追女人为例来说明,其实女的也可以如此追男人。)男的首先开始对女的进行广告宣传,这一宣传阶段男的甚至可以进行某些虚假的美化自己的行为(其实人们潜意识的就在这样做,比如化妆,整洁自己,炫耀等)。这是可以的,而男的要记住,你所做的一切宣传都不是为了让女人真的相信,女人不相信也没有关系,你要做的只是消除他们内心对于未知的恐惧。当你把这个重点把握了,你就明白,女人在这一阶段需要有这种感觉,那就是那个男人对自己毫无保留,我把他看得清清楚楚。这怎么可能?人都是有自己的隐私的,人怎么可以完全暴露自己?这里适当的虚假宣传却是有效的。广告阶段的成功判断标准就是女方开始走进男方,说简单点就是走进男方的超市了,男方可以卖出产品来和女方建立合作关系了。

\item[利益阶段] 在这一阶段,男女之间有点认识了,这个时候男的要开始改变策略了,而不能停留在自己的吹嘘上,而是要对目标女方给予实际的利益好处。这一阶段的操作有点类似于动物行为上的激励机制(当然人们出于自尊不大愿意承认这点)。比如你给了女方什么好处,然后就对女方的行为提出什么建议,当女方满足你的要求之后,你就继续给予女方什么好处。一般热恋阶段的男女双方大多处于此阶段,用通俗的话说就是顾客和客户建立的稳定的交易关系。

\item[恐惧阶段] 有很大一部分男人在追求女人和他们建立了亲密关系之后,慢慢就分手了,为什么?因为那些男人没有意识到你已经不能在给予女人额外的什么好处了——因为在利益阶段你已经将底牌亮完了——这个时候你必须转型为用恐惧操纵女人。在恐惧阶段男人要做的和在利益阶段要做的可能恰好是个反的,在利益阶段你可以甜言蜜语夸女人漂亮,但是在恐惧阶段你就不能这样做了,你的恐惧都指向这样的目的:让女人觉得离开你之后她的处境将会很惨。所以你需要时不时的轻描淡写地暗示那个女人她的容貌已经不如以前漂亮了,然后暗示她身子不那么纯洁了等等总之时不时的打击女人的自信心,让那个女人不会过于狂妄而自认为自己离开了你也会活得很好。其次你必须丑化你的其他竞争对手,为了达到这点,你需要做的就是时不时夸大别的男人的不负责任,玩玩女人就走了,夸大别的男人徒有其表花心等等,这些话语的指向只有一点,让女人对未知的那些男人感受到的不是可能的利益而是恐惧。

很多男人最后都会收到一张好人卡,然后男人或女人就分手了或者离婚了。这个时候那些男人会认为女人是愚蠢的(是的,女人大多是愚蠢的。)不过那些男人要宣称自己对之前那个女人是真爱我就十分怀疑了。现在我在这里把话说白了,如果你真的希望爱那个女人,希望和那个女人长久在一起,那么请做好恐惧阶段的工作。

\item[婚姻阶段] 这里说的婚姻阶段不是说男女双方结婚了就进入婚姻阶段了,那倒不是,因为有太多鲁莽结婚的事例了。这里说的婚姻阶段就是真正的男人和女人心都踏实了,然后彼此之间既不需要广告阶段的虚假宣传,也不需要利益阶段过分的讨好,也不需要恐惧阶段过分的恐惧手段。一切都变得平淡了,双方之间关系完全稳固而变成了亲情关系。说的再简单一点就是男的和女的心都踏实了,然后双方都不需要额外的利益和恐惧的手段来操控对方了,他们完全进入了稳定的合作关系——婚姻——阶段。

但是婚姻阶段绝不是理想的那么完美,随着时间的推移,将会发生各种事情,我奉劝男人和女人的是,如果你们真的希望自己的婚姻稳定下去,那么接下来就不再是本来的追逐恋爱游戏了,而是理智的生活,在这样的生活中,没有人有那么多精力去处理和操控别人。也就是如果男女双方之间合作关系利益互补上彻底破裂了,那么任何操控手段都是没用的,这个我在前面谈及人的两种关系的阴阳平衡属性中提及过。
\end{description}


%这里空一行

\end{common-format}
\end{document}



