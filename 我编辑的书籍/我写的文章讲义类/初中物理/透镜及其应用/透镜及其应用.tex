% !Mode:: "TeX:UTF-8"%確保文檔utf-8編碼
%新加入的命令如下: reduline showendnotes 
%新加入的环境如下:solution solutionorbox solutionorlines solutionordottedlines

\documentclass[12pt]{exam}
\newlength{\textpt}
\setlength{\textpt}{12pt}

\usepackage{teachingplan}

%输出方案 
%学生版 学霸版 老师版

%写上答案或者不写上答案%1  
\printanswers  

%题目类型划分:A类题,基本知识题;B类题,中等难度;C类题, 难题;D类题,中考题。
\excludecomment{Aquestions}
%\includecomment{Aquestions}
\excludecomment{Bquestions}
%\includecomment{Bquestions}
\excludecomment{Cquestions}
%\includecomment{Cquestions}
\excludecomment{Dquestions}
%\includecomment{Dquestions}

\CenterWallPaper{1}{教案模板-2.pdf}


\newcommand{\keti}{透镜及其应用}
\newcommand{\zhongdian}{1.凸透镜成像的规律 \\2.生活中的透镜 }
\firstpageheader{}{}{\today For XX}

\begin{document}
\ThisCenterWallPaper{1}{教案模板-1.pdf}
\vspace*{80pt}
\keti \par
\zhongdian \par
\section{凸透镜成像的规律}
透镜分为:凸透镜和凹透镜。凸透镜对光线有\textbf{汇聚}作用,凹透镜对光线有\textbf{发散}作用。

对于那些判断是凹透镜或者凸透镜的题目,关键点就是理解汇聚和发散这两个字。汇聚就是光线向内偏折,\tikz{\draw[->] (0,-0.5) -- (0,-0.2); \draw[->] (0,0.5) -- (0,0.2); } ;发散就是光线向外偏折,\tikz{\draw[->] (0,-0.2) -- (0,-0.5); \draw[->] (0,0.2) -- (0,0.5); } 。 

\subsection{光线经过透镜的三大规律}
在黑板上画图具体说明这三条路线。
\subsubsection{经过光心方向不变}
每个透镜主光轴上都有一个特殊的点,这个点叫\textbf{光心}。任何光线对着光心经过透镜方向不变。(垂直不发生折射)——光线经过透镜规律之一。

\subsubsection{平行光射入汇聚于焦点}
平行于主光轴的光线射入凸透镜,则所有光线汇于一点,这一点叫\textbf{焦点}。焦点到光心的距离叫\textbf{焦距}。(凹透镜是反向延伸光线汇聚于另一边的焦点。)——光线经过透镜规律之二。

\subsubsection{通过焦点的光射入后平行}
焦点发出的光或通过焦点的光线到达凸透镜后平行于主光轴射出。(这实际上是上一条规律和光线在相同条件正向和逆向都走同样的线路的必然结论。这里凹透镜是指向另一边焦点的光线经过凹透镜后平行射出,你能想像的到吗?)


\subsection{具体分析凸透镜}
\begin{fig}{凸透镜原理}
\end{fig}

这幅图是描述的成实像的情况,也就是一个蜡烛的光的一个点到了凸透镜的另外一边又能汇聚为一点,这样可以在屏幕上直接看到光点。而如果蜡烛移动到焦点以内(也就是物距$u$是小于焦距$f$的),这样蜡烛的光点在凸透镜另外一边是汇聚不成一点的,这样就成不了\textbf{实像}。这个时候的情况就是\textbf{放大镜}的情况。这个像不能在屏幕上看到,只能人眼看到,我们称它为\textbf{虚像}。


具体讲解他们的数量关系?

于是我们推的:
\begin{solutionorbox}[15ex]
\begin{equation*}
\frac{h'}{h}=\frac{x'}{f}
\end{equation*}

\begin{equation*}
\frac{h'}{h}=\frac{f}{x}
\end{equation*}

\begin{equation*}
xx'=f^2
\end{equation*}
\end{solutionorbox}

给学生观看动态凸透镜原理演示幻灯片。

根据上面的公式,然后还有已知物距$u=x+f$,像距$v=x'+f$。于是我们有:
\begin{description}
\item[$u=2f$时] 这时$x=f$,所以$\frac{h'}{h}=1$,也就是这个物体原来的高和成像后的高度一样,这个时候我们称放大率为1,成的是等大倒立的实像。为什么是倒立从上面这幅图一眼就看出来了。
\item[$f<u<2f$时] 这时物体向凸透镜移动了一点,这样本来一个两边对称的菱形另一边拉长了,$x$变小了,放大率变大了。也就是成放大的倒立的实像。这就是\textbf{投影仪}的工作原理。而且我们看到因为$xx'$是一个常数,所以必然有$x$变小,那么$x'$增大,也就是物距增大必然像距减小,物距减小必然像距增大。所以我们将蜡烛朝凸透镜移动,屏幕需要怎样移动啊?
\item[$u>2f$时] 类似的我们看到,如果物体远离凸透镜,那么物距增大,像距减小。放大率变小。也就是相对于$u=2f$这个点来说是成缩小倒立的实像。你们可以想像的到,这个物体越来越越来越远离,那么那边成的像越来越越来越靠近焦点,并缩小得几乎只有那么一个点,之所以如此因为物体的光越来越原理凸透镜之后相对于凸透镜它的光变得几乎是平行光射入了,也就是我们之前学过的平行光的那种情况了。因为太阳光离地球很远,情况也和这里的讨论类似。\textbf{照相机}的工作原理和这里的讨论类似,因为照相机照的东西都很远吗。
\end{description}

现在我们总结一下,在$u>f$的情况下,也就是物距大于焦距,成立的都是倒立的实像。其中$u=2f$是一个点,这个点物体离焦点一个焦距,成像点离那边的焦点也是一个焦距,这个时候是一个很对称的菱形。然后接下来的情况就是这边缩短那边延长,这边延长那边缩短了。当然我们也可以假设$u=f$的时候可以成一个假设的很大很大很远很远的倒立的实像,那么接下来的事情就是物体远离,那个很大很远的倒立的实像越来越靠近焦点,变的越来越小。(所以照相机眼睛都是看很远的物体的,他们成的就是很小的很靠近的焦点的一个倒立的实像,想明白了吗?)


例题:\\
在做凸透镜成像实验时,小明将蜡烛放在离凸透镜$24cm$处,光屏上得到一个倒立、缩小的清晰的像,则此凸透镜的焦距可能是(\answerline*[A])

\begin{oneparchoices}
\choice $10cm$
\choice $12cm$
\choice $20cm$
\choice $48cm$
\end{oneparchoices}


\subsection{改变焦距}
这里的讨论部分\uwave{难度偏高},给那些成绩特别好的学霸。但是不表示简单的题目就不会涉及改变焦距的情况。(因为眼睛照相机都是可改变焦距的,所以来自生活中的题目是俯拾皆是的。)

在之前的讨论中我们都假定焦距不变,那么如果焦距可变会是什么情况呢?首先改变焦距整个过程比较复杂,所以不可能考的很深入,但只要稍微涉及就是一个很难的题目。

在做改变焦距的题目时想是想不出来了,因为变化情况很多的。一个基本点就是牢牢把握$\frac{h'}{h}=\frac{f}{x}$和$\frac{h'}{h}=\frac{x'}{f}$这两个式子,这样我们继续推的:$\frac{h'}{h}=\frac{v}{f}-1$和$\frac{h'}{h}=\frac{1}{\frac{u}{f}-1}$。

如是我们有:物距不变,焦距变大则放大率变大也就是成像变大;反之有焦距变小则放大率变小,也就是成像变小。此外我们还有扩展$\frac{1}{u}+\frac{1}{u}=\frac{1}{f}$,则有物距不变,焦距变大,则像距变大。(所以照相机镜头前伸就是提高像距,物距不变,焦距变大,成像放大的情况)

例题:\\
小明站在同一位置用数码相机(焦距可改变)对着无锡著名景点锡山先后拍摄了两张照片甲和乙,如图所示。他产生了这样的疑问:物体通过凸透镜在光屏上所成像的大小与透镜的焦距有何关系呢?小明对此进行了探究,小明先后选用了三个焦距分别为15cm、10cm、5cm的凸透镜,在保持物距始终为35cm的情况下依次进行实验,发现所成的像越来越小.小明在拍摄照片\answerline*[乙](甲/乙)时,照相机镜头的焦距较大,在探究过程中,小明总是让物距保持不变,请你说出他这样做的理:

\begin{solutionorbox}[6ex]
因为像的大小与物距和焦距都有关系,根据控制变量法,探究像的大小与焦距的关系,应保持物距不变,改变焦距,观察像的大小变化。
\end{solutionorbox}



\section{生活中的透镜}
\subsection{放大镜}
放大镜看东西的时候物距是小于焦距的,根据生活中的经验我们知道我们看到的是放大的正立的虚像。那么生活中你使用放大镜的时候你注意到没有,什么情况下放的最大呢?是不是物距很小,也就是放大镜很靠近报纸上的字的时候并没怎么放大,稍微远离一点就会放大一点。这告诉我们在$u<f$的情况下,$u$越靠近$f$放大率越高,是吧?但是又不能靠焦点太近,因为太近了你手一抖,可能就超过焦点了,这样你又看不到字了。那么如果你是放大镜生产厂商,是不是设计一个合适的焦距也很重要啊?

\subsection{投影仪}
投影仪前面我们说过了,在投影仪生活实践中,最关键的一点是像距是确定的,比如一般都是半的教室长。所以为了适应这个像距,我们需要实际的调一下物距,好让图像更清晰。投影仪成的是放大的倒立的实像。

\subsection{照相机}
照相机成的是缩小的倒立的实像,照相机面临的问题和投影仪面临的问题是个反的,也就是物距是确定的,需要调的是像距。(似乎照相机也可以调焦距?)

\subsection{人眼睛}
人眼睛的工作原理和照相机类似,也是成缩小的倒立的实像,不过区别在于人眼睛看的物距不可以调,成像的像距也不可以调,人眼睛可以调凸透镜(\textbf{晶状体})的焦距。


\subsubsection{近视眼}
近视眼的成因:人看近处的东西眼睛的睫状体需要收缩,让晶状体变厚,也就是凸透镜的焦距变小。(我们来看一下为什么?还记得$xx'=f^2$吗?物距变小,像距不变,那么焦距就要变小。)。要是人老是看近处的东西,睫状体老是收缩,最后回复不过来了(刚开始假性近视的时候做保健调节还是能够回复的,所以你们如果因为某段时间用眼过度,不要轻易戴眼睛,注意休息和保养!),于是焦距太小,晶状体汇聚能力太强,像在视网膜的前面成像了。

如果近视眼真的无法回复了,只好待上凸透镜的反面(凹透镜)眼镜了。


\subsubsection{老花眼}
老花眼也叫远视眼,顾名思义,和近视眼是个反的。所以你们应该知道是怎么个情况了,就是睫状体收缩能力下降了,成像成到视网膜后面去了。这个时候可以戴上老花镜来矫正,我想你们也猜到了,老花镜就是一个凸透镜。

很多近视眼都以为,等自己老了有了老花眼自己的近视眼就好了,呵呵。


\subsection{显微镜和望远镜}
显微镜和望远镜的结构都是类似的,一个是望小小的细胞,一个是望遥远的星空。所以大小无类啊。

显微镜和望远镜作用原理都是差不多的,靠近眼睛的凸透镜叫目镜,靠近被观察物体的(细胞或星空)叫物镜。目镜的作用就是类似放大镜,物镜的作用就是让光线汇聚(所以大型天文望远镜用凹面镜也是可以的)



\begin{Aquestions}
\newpage
\section{练习题}

\end{Aquestions}


\begin{Bquestions}
\newpage
\section{一般题}
\end{Bquestions}




\begin{Cquestions}
\newpage
\section{难题}
\end{Cquestions}



\begin{Dquestions}
\newpage
\section{中考题}
\end{Dquestions}




%
\ThisCenterWallPaper{1}{教案模板-3.pdf}

\end{document}



