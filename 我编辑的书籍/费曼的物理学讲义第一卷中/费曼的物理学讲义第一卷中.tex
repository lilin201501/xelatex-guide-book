% !Mode:: "TeX:UTF-8"%確保文檔utf-8編碼
%新加入的命令如下:addchtoc addsectoc reduline showendnotes hlabel
%新加入的环境如下:common-format  fig scalefig xverbatim

\documentclass[11pt,oneside]{book}
\newlength{\textpt}
\setlength{\textpt}{11pt}
\newif\ifphone
\phonefalse


\usepackage{myconfig}
\usepackage{mytitle}



\begin{document}
\frontmatter

\titlea{费曼的}
\titleb{物理学讲义}
\titlec{第一卷}
\author{费曼}
\authorinfo{作者:这里等下填上费曼教授的一些信息}
\editor{万泽}
\email{a358003542@gmail.com}
\editorinfo{编者:wanze,从网络中找到前五章(不太完整)的txt,感谢这个txt的制作人。}
\version{0.03}
\titleLC

\addchtoc{前言}
\chapter*{前言}
\begin{common-format}
开头说的话

%这里空一行。

\end{common-format}


\addchtoc{目录}
\setcounter{tocdepth}{2}
\tableofcontents

\begin{common-format}
\mainmatter

\chapter{光学:最短时间原理}

\section{光}


\chapter{几何光学}


\section{放大率}
到现在为止,我们只讨论了轴上点的聚焦作用。现在我们来讨论不完全在轴上而稍微离开它一点的物体的成象,这样可以使我们了解\uwave{放大率}的性质。当我们装置一个透镜把来自灯丝的光聚焦在屏上一“点”时,我们注意到,在屏上得到同一灯丝的“图象”,只是其大小比实际的灯丝大一些或小一些而已。这必然意味着灯丝上\uwave{每一点}发出的光都会聚到一焦点上。为了更好地理解这一点,我们来分析图27-7中所示的薄透镜系统。

\begin{fig}{薄透镜成象的几何图}
\label{fig:薄透镜成象的几何图}
\end{fig}

我们知道下列事实:
\begin{enumerate}
\renewcommand{\labelenumi}{(\arabic{enumi})}
\item 从一边射来的任一平行于轴的光线都朝另一边称为焦点的特殊点行进,这个点与透镜相距$ f $。
\item 任一从一边的焦点发出而到达透镜的光线,都在另一边平行于轴射出。
\end{enumerate}

这就是我们用几何方法建立公式\emph{label}所需要的全部知识,具体步骤如下:假定离焦点某一距离$ x $处有一物体,其高为$ y $。于是我们知道光线之中有一条光线(如$ PQ $)将经透镜偏折而通过另一边的焦点$ R $。如果现在这个透镜能完全使$ P $点聚焦的话,那么只要找出另外一条光线的走向,就能找出这个焦点在哪里,因为新的焦点应在两条光线再次相交的地方。因此我们只要设法找出另外\uwave{一}条光线实际方向,而我们记得平行的光线通过焦点,\uwave{反之亦然}:即通过焦点的光线将平行地射出!所以我们画出一条光线$ PT $通过$ U $。(诚然参与聚焦的实际光线可能比我们所画的两条光线的张角小得多,但它们画起来较为困难,所以我们假设能作这条光线。)既然它将平行射出,我们就画出$ TS $平行于$ XW $。交点$ S $就是所要求的点。这个点决定了象的正确位置和正确高度。我们把高度称为$ y' $,离焦点的距离称为$ x' $。现在我们可以导出一个透镜公式,应用相似三角形 $  PVU $和$  TXU $,得

\begin{equation}
\label{Eq:I:27:13}
\frac{y'}{f}=\frac{y}{x}.
\end{equation}










%这里空一行

\end{common-format}
\end{document}



