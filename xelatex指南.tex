% !Mode:: "TeX:UTF-8"%確保文檔utf-8編碼
%新加入的命令如下:addchtoc addsectoc reduline printendnotes 
%新加入的环境如下:common-format  fig scalefig 

\documentclass[11pt,oneside]{book}
\newlength{\textpt}
\setlength{\textpt}{11pt} 
\newif\ifphone
\phonefalse

\usepackage{myconfig}
\usepackage{mytitle}

\includeonly{parts/ubuntu知识}

\begin{document}
\frontmatter   

\titlea{\XeLaTeX 指南}
\titleb{在Ubuntu下排版专业的pdf文章}
\author{万泽}
\authorinfo{作者:万泽,湖南常德人氏。}
\editor{万泽}
\email{a358003542@gmail.com}
\editorinfo{编者:万泽,}
\version{1.11}
\titleLTB

\addchtoc{前言}
\chapter*{前言}
\begin{common-format}
我试图收集整理各方面来的xelatex相关的信息,包括自己的实践经验总结,作为后学者的指南手册。

一切在尊重版权的基础上出于爱好兴趣和相信自由分享的价值原则上进行。

全书主要分为五大块:
\begin{description}
\item[1. ubuntu知识] \hfill \\
主要是一些最基本的ubuntu知识或者和本文其他内容相关的知识。当然有时出于全面性考虑会加入一点其他的linux知识进来,不过这部分需要做到不过分扩充内容。
\item[2.xelatex基础] \hfill \\
这部分内容会就xelatex或者latex的基础知识内容做全面的整理归类排版,争取做到言简意赅并面面俱到和实用性很强的手册性质。
\item[3.latex高级篇] \hfill \\
这部分内容会就latex的一些较高级的知识做出说明,有时会就tex的一些命令做出说明。这部分内容主要以各个不同的专题议题的形式展开,不求全,但求就某一个实用的问题解释清楚。
\item[4.其他问题的讨论] \hfill \\
这部分如标题所言,是我根据实际需要接触到的很有用的技巧,以问题的形式展开。这部分内容能够做到分解成为前面三部分内容的将会分解,觉得不好分解的就放到这里,这部分内容不应过分扩充。
\item[5.后面的珍宝] \hfill \\
这部分会将本书涉及到的有用的资源收集起来给出下载链接或者打印出来。
\end{description}

\end{common-format}




\addchtoc{目录}
\setcounter{tocdepth}{2}    
\tableofcontents



\begin{common-format}
\mainmatter 

\part{ubuntu知识}
\chapter{准备工作}
\section{通过U盘安装ubuntu}
\begin{enumerate}
\item  在安装之前请先把硬盘中的资料做一些调整,空出一个大于20G的硬盘做将来要安装 ubuntu根目录的地方。然后还需要一个大约为你内存两倍的硬盘分区等下要作为linux系统的swap交换分区。
\item 到ubuntu的官网上下载系统的光盘映像。
\item 用ultroiso软件(其他具有类似功能的软件也行)(注意ultroiso最好选择最新的版本,ubuntu10.10之后的一定要用9.3版本之上的)将该光盘映像写入到你的U盘中去。
\item 重启计算机,BIOS稍作改动使计算机变成从U盘启动。
\item 进入安装过程,其他过程都比较直观,就是硬盘分区设置上我们选择高级手动,然后将你分出来的那个20G硬盘作为/的加载点,并设置格式化成ext4日志系统(其他文件系统如ext3等也行)。然后设置交换分区,安装完成。
\end{enumerate}


\section{初入ubuntu}
\begin{itemize}
\item 设置ubuntu的root用户,这个以后可能会有用。按住ctrl+alt+t弹出终端,然后输入:\\ sudo passwd \\设置好密码之后,以后输入:\\ su\\就可以进入root账户了。
\item 安装一个有用的工具:\\sudo apt-get install nautilus-open-terminal\\这个工具可以让你在视窗背景下鼠标右键之后打开当前目录下的终端,有时还是很快捷的,尤其是遇到一些怪怪的文件名的时候。
\item 系统安装好之后第一件要做的事是选一个好的源,然后安装更新。我的ubuntu版本是12.04,在右上角的哪里,软件更新,软件源设置,在下载自的哪里就是软件源的服务商,你最好还是自己搜索一个速度最快的源。然后在终端中执行以下命令来升级系统软件包:\\sudo apt-get update  (更新源)\\sudo apt-get upgrade   (更新源下已经安装了的软件)\footnote{如果有很多软件需要升级的时候推荐使用命令:sudo apt-get dist-upgrade 这样不容易出错些。}
\item 一些闭源的有用的东西:\\sudo apt-get install ubuntu-restricted-extras\\这个安装到adoubeflashplugin的时候会有点卡住,最好还是耐心点吧。或者在ubuntu软件中心(Ubuntu 额外的版权受限程序)中安装也是可以的。
\item ubuntu系统清理,运行以下两个命令即可:\\sudo apt-get autoremove (清理软件残余)\\sudo apt-get autoclean  (清理缓存)。
\item 删除通过apt-get安装的软件:\\sudo apt-get remove 要删除的软件名\\删除相应的软件配置文件:\\sudo apt-get purge 软件名
\item 长按super键或者说是win键会弹出一些ubuntu桌面常用的快捷键,比如说ctrl+win+d是显示桌面,按住win+f搜索文件,win+a搜索程序等。
\end{itemize}

\section{linux最基本的命令介绍}
\begin{description}
\item[ \${} 和\~{}] 当我们打开终端的时候,看到一个美元\$ 符号,如果我们输入su命令,然后进入root账户,看到开头有一个\#{} 符号,其中\$ 表示普通用户,\#{} 表示现在是超级用户。然后我们看到一个波浪号$\sim$  ,这个波浪号的意思就是当前用户的个人主文件夹,比如现在我这里$\sim$ 的意思就表示目录/home/wanze 。
\item[ cd] 一般dir或者ls之后列出该目录所包含的文件夹或者文件,然后执行如下命令:\\cd 某个文件夹名 \\ 就进入这个文件夹了。如果cd一个文件就会报错。然后特殊的有:\\ cd $\sim$ 回到个人主文件夹\\cd . 其中点表示当前目录\\cd .. 表示返回上一级目录\\cd 某个目录 ,比如cd /etc 那么就直接跳到系统的/etc目录下了。
\item[ cp] 复制文件命令:\\ cp~~~要复制的文件~~~目标文件目录 
\item[ which] 查看系统某个命令的位置
\item[ touch] 创建某个文件或者对已存在的文件更新时间戳
\item[ rm ]删除某个文件,一些具体选项我就不罗嗦了,如果有不明了的请自己用 -{}- help来查看帮助信息。比如这里的-r选项就可以强制删除一个目录下所有的文件。
\item[ mkdir] 创建文件夹
\item[ rmdir] 删除文件夹
\item[ mv ]移动某个文件或者重命名某个文件
\end{description}

\section{通过ppa安装软件}
ppa也就是软件源,一般推荐ppa安装软件即该软件还处于活跃开发期中,所以推荐尽快更新到官方最新的版本。过程就是添加ppa源,然后用前面讲的apt -get upgrade来更新源下的软件。添加ppa源的格式如下:\\
sudo add-apt-repository ppa:后面就是什么官方源地址了\\
然后更新源下的软件列表:\\
sudo apt-get update\\
接下来就是通过apt-get  install 来安装该软件了。upgrade是更新源下已经安装了的软件,所以第一次装还是要用install命令。
\footnote{\href{http://linuxers.org/howto/how-install-any-software-ubuntu-ppa}{how-install-any-software-ubuntu-ppa}}

\section{ubuntu软件推荐}
以下编号:Ⓐ 表示可以从ubutnu软件中下载到,Ⓑ 表示可以从终端apt-get命令中下载到,Ⓒ 表示推荐从ppa安装。
\begin{description}
\item[新立得软件包管理器] 这个有时安装一套软件组合很有用的。Ⓐ
\item[chromium] 不清楚和google-chrome的区别。它的插件很方便,有的时候网页显示相比较于firefox不会出错些。Ⓐ\\
chromium插件推荐:

\begin{description}
\item[Daum Equation Editor]  本文不会讨论数学公式的输入问题,这里有个简单的解决方案,就是在这个插件里面编辑公式,下面就有对应的\LaTeX 的代码。
\item[代理助手] 好吧,我讲这个完全是因为天朝的防火墙,不学会翻墙严重干扰我们搜索信息。具体设置是http代理填写相关地址,china list这个选项也勾上吧。下面还有详细说明。
\end{description}

\item[翻墙软件] 要搜东西真的要用google,在天朝你懂的,因为要跑防火墙慢死了。轮子的自由门还可以。\footnote{\href{http://115.com/lb/5lbdvqz2o9i}{下载freegate的链接}}翻完墙之后到弹出来的那个网站或者voa网站哪里多下几个类似的翻墙软件吧,有备无患。freegate在ubuntu下需要使用wine模拟,在模拟前先用winetricks装好mfc42.dll,然后就行了。freegate进去之后在通道哪里选择经典模式。google-chrome那边的设置在代理助手哪里,就是使用freegate提供的那个http地址加上端口号即可——在内容那一页。一般代理助手就点系统自动连接吧,毕竟是免费代理,别想质量有多好。
\item[texmaker] 写\LaTeX 非常好的界面环境,和windows下的texstudio类似,但是感觉界面更加的漂亮。Ⓐ
\item[shutter] 一个非常好的截图软件。ubuntu系统自带的截图软件截完图不能拖动改变大小,这个可以。Ⓐ
\item[ubuntuone] 在ubuntu系统下很好的文档同步软件,在手机上安装之后可以方便进行某些文档的传输。Ⓐ
\item[wine] 虽然wine软件一直在更新,但是老实说总感觉不是十分令人满意。这里提到wine是因为前面谈到的翻墙软件需要用wine来模拟。wine的安装不同于其他软件,尽量下载最新版本的吧。到官网上查看最新的稳定版本号,比如wine1.6。然后在终端上通过ppa来安装,ppa地址见脚注。Ⓒ\footnote{ppa:ubuntu-wine/ppa}
\item[workrave] 人们对久坐的危害总是低估了,这个软件帮助提醒人们做一段时间之后最好起来活动活动。Ⓐ 
\item[Gcolor2] 这个小软件取色,RGB转换等很有用。Ⓐ
\item[wps for linux] 目前写毕业论文不得已还是要用doc格式,而在ubuntu下我试过很多解决方案,比如wine,libreoffice,或者网上的在线编辑。结果都不尽如人意。因为我需要的是最好完完全全兼容microoffice2003的格式。金山公司的这个wps for linux 目前还在开发中,但是效果还是可以了。下载地址见脚注。Ⓓ\footnote{\href{http://linux.wps.cn/}{wps for linux}}
\end{description} 
其他还有很多有用的软件看官自己探索吧。


\section{ubuntu下好的输入法}
更多信息请参考rime官方wiki\endnote{\href{https://code.google.com/p/rimeime/w/list}{rime官方wiki}}。
还有一种ppa安装方法,ppa地址见尾注\endnote{ppa地址是:[ppa:lotem/rime] }。
\begin{enumerate}
\item 安装ibus-rime:\\ sudo apt-get install ibus-rime
\item ibus输入框不见了:\\ ibus-daemon  ~~~  -drx
\item 重啓ibus:有时ibus会出现一些奇怪的问题,在终端运行下面命令重啓ibus也许会解决:\\ killall ~~ibus-daemon\\
             ibus-daemon ~~~ -d
\item Rime输入法的其他设置:在输入框调到Rime的状态的时候,按一下F4,就弹出一些设置选项。
\item 用Rime输入常见的特殊符号:比如我要输入*,按下shift和数字8按键,就出来一些选项供你选择。其他很多符号类似,具体请参考官网上的说明。
\item Rime输入法是架构在ibus框架之上的,所以在系统输入法首选项哪里,可以设置Rime候选项是横着显示或者竖着显示,还有下面可以自定义显示的字体和字体大小。推荐用sans-serif字体。然后字体大小要比你输出之后的效果稍微大一点,这样醒目些。
\end{enumerate}

\section{rime输入法的备份和还原}
输入法用久了,你的自造词还有你输入相关词的词频这些信息是非常有用的,能够极大地提高你的工作效率。rime备份的核心词是命令:\emph{rime\_{}dict\_{}manager}。你需要在rime的用户配置文件夹里面打开终端,然后输入这个命令,你就可以看到提示信息了。

首先输入-l选项,你可以看到现在已经有的词典,比如luna\_{}pinyin。然后用-s同步,这样你就会看到一个sync文件夹,里面放着快照文件。-b是选择性是备份某一个词典。-r是还原快照文件。-e,-i是txt格式操作,txt保存数据会有点损失。将用户rime配置文件夹整个复制到同步盘即可。

\section{把QQ拼音上你的词库导入rime}
\begin{enumerate}
\item 在windows下在QQ拼音设置里面找到那个导出词库的命令。就是一个txt文件。
\item 将这个txt文件另存为utf-8编码
\item 进入linux(以下windows用户原理类似)找个在线的简繁转换网站,如果你使用的是简体略过这步。然后转换成为繁体,由于词库比较大,建议你耐心点,推荐笨笨网站简繁转换,不容易卡死。实在不行就只好下载软件了。我的一万多条还是能够转换出来。
\item 打开libreoffice ,将前面的txt文件,复制,粘贴。然后会弹出一个选项,分隔符注意勾选空格。
\item 前三列有用,后面的意义不大,统统删除。
\item 按ctrl+f,下面弹出查找于替换。输入,  后面输入空格   全部替换。输入'     后面输入空格   全部替换。
\item 将第一列第二列位置互换,就是剪切粘贴操作。
\item 全部选择前三列,复制,粘贴到文本。
\item rime\_dict\_manager  -i luna\_pinyin   上面谈及的你创建的文本命令。
\item 注意上面要在rime 用户目录下操作  即.config/ibus/rime。
\item 查看   dict\_manager -b luna\_pinyin  
\end{enumerate}


\section{定制rime输出特殊符号}
\begin{fancycolorbox}
我希望通过rime输入法能够快速地将Unicode中的符号打出来。本文主要关注这个问题,其他定制请参阅官网wiki。
\end{fancycolorbox}


\subsection{rime基本情况说明}
Rime的ubuntu版本叫做中州韵,不管这么多,一般称作ibus-rime,下面简称rime。下面的讨论只使用于ubuntu,我目前的版本是13.04,12.10试过没有问题。

rime的程序在哪里我不关心,在ubuntu下所有程序的配置是用户配置优先级高于程序自带的配置,所以我们在rime的用户文件夹里面DIY就是了。rime的用户资料夹在:\\
$ \sim $/.config/ibus/rime


进入这个文件夹之后,我们可以点击看以下里面的内容,这些文件情况说明如下:
\begin{description}
\item[cangjie5.schema.yaml] 仓颉五代配置文件。我用的是朙(míng)月拼音,这个文件就不管了。
\item[default.yaml] 这个文件存放着输入法的一些全局设定,重新部署rime时会重新生成。
\item[installation.yaml] 这个文件存放着rime的安装信息,不用管它。
\item[luna\_{}pinyin.schema.yaml] 这个是默认的朙月拼音的配置文件,要仔细看看。
\item[luna\_{}pinyin\_{}fluency.schema.yaml] 这个是朙月拼音语句流的配置文件,我不关心。
\item[luna\_{}pinyin\_{}simp.schema.yaml] 这个是明月拼音简化字的配置文件,也就是输入法切换到简体字输入之后起作用的。
\item[luna\_{}pinyin\_{}tw.schema.yaml] 这个是朙月拼音台湾正体的配置文件,没用过。
\item[symbols.yaml] 这个是rime自带的一个输出symbol的配置文件,但是奇怪的是刚开始并没有配置好。
\item[user.yaml] 这个是用户状态信息。
\end{description}

\subsection{开始DIY}
其中官网wiki中讲了一些default.custom.yaml的定制,那些我似乎都没有需要,然后略过。现在主要讨论lunna\_{}pinyin.custom.yaml的配置问题。
具体配置如下:   
\begin{verbatim}  
patch:
    punctuator:
        import_preset: symbols
    recognizer:
        import_preset: default
        patterns:
        reverse_lookup: "`[a-z]*'?$"
        punct: "^/[a-z\\[\\]]*$"  
\end{verbatim}

我想你已经看到了这些配置文件都是.yaml后缀,它们是用一种什么yaml数据描述语言写的,类似于XML的标记语言。然后里面还有一些正则表达式的东西。关于这些内容我是能够略过就略过吧。

刚开始我写这个patch的时候总是不成功,然后看到佛振的那个帖子,链接到另外一个源码上。\href{https://github.com/lotem/brise/blob/master/preset/luna_pinyin_fluency.schema.yaml#L103}{请参看这个网站}。然后直接修改luna\_{}pinyin.schema.yaml文件,就是把default改成symbols,也就是指向那个symbols.yaml文件,然后将patterns的punct:那一行复制过去。就发现可以了。

现在patch可以正常工作了,原因是之前我的缩进不正确,由于我对yaml语言不清楚,就此打住。

其中punct那一行涉及到的正则表达式知识有符号\^{}表示匹配开始,\$表示匹配结束,[]和里面的内容一起表示一个字符,这个字符可能是a-z所有的小写字母,然后我想加上符号[和],因为我想新建一个命令/[]就能输入很多形式的括号。最后发现这个问题还有点小麻烦,\href{http://www.infoq.com/cn/news/2011/01/regular-expressions-1}{请参看这个网站}。他解释了最好前面加上两个\textbackslash ,然后才能更好的工作。最后那个*表示重复零次h或者更多次。
\endnote{你可以到这个网站继续学习正则表达式:\href{http://deerchao.net/tutorials/regex/regex.htm}{正则表达式30分钟入门教程}}

\subsection{symbols.yaml文件的说明}
我在github网站上新建了\href{https://github.com/a358003542/rime-symbols-yaml}{一个项目}。里面有我编辑的symbols.yaml文件。现在将我所做的修改工作简单说明如下:

前面是全角和半角部分,如果你有需要可以自己定制下,这里我略过。然后看到symbols:哪里,开始自己定制命令。前面那个代码我们看到了匹配是以/开始的,所以下面所有的命令都要以符号/开始,如果你想有其他模式修改正则表达式就可以了,比如说可以换成命令以符号\textbackslash 开始。

yaml的缩进表达我不太清楚,实际上我不太喜欢那种缩进的语法。yaml还提供了另外一种语法形式。就是一个系列可以用一个方括号明确标识出来。\footnote{\href{http://www.dev.idv.tw/mediawiki/index.php/YAML}{请参看这个网站}}在这里格式很简单,就是建立了很多符号调出命令,对应的后选词选项用方括号包围起来。

然后还需要一提的就是yaml语言支持锚点,意思就是某一个变量,跟着\& 任意的名字,后面就可以用* 那个名字来表示那个词条了,这个有点类似于取别名的意思。anyway,用这样的方法可以给你建立的命令取几个别名。

\subsection{重新部署rime}
最后将上面两个文件放入用户rime配置文件夹哪里,然后删除default.yaml文件,然后运行\verb+ibus-daemon   -drx+。等着rime重新部署完毕即可。

\subsection{具体输入符号的命令}
现在我在rime下输入/gcs,就会弹出很多符号。类似的还有很多,就不一一介绍了。

\section{配置\LaTeX 编写环境}
\begin{enumerate}
\item sudo apt-get install texlive   (下载安装ubuntu下有名的texlive)
\item sudo apt-get install texlive-full   (下载安装texlive的各个包)
\item 在ubuntu软件中心中下载安装texmaker软件。
\item ibus似乎在这里面有点小问题,有点不稳定。在终端中运行\\sudo apt-get install ibus-qt4\\则问题解决。
\item 在texmaker的选项和配置texmaker里面,设置快速构建,点上用户自定义那一栏,然后输入如下命令:\\
xelatex -interaction=nonstopmode \%.tex| \\
这是使用xelatex来对目标tex文件进行编译,而不是传统的latex或者pdflatex方式,之所以这样是因为多方对比之后,觉得其在字体处理方面是未来的趋势。
\end{enumerate}


\section{ubuntu下\LaTeX 新宏包的安装}
安装texlive-full之后,如果还遇到没有的宏包,可以先到CTAN官网上下载到这个宏包之后,然后将这个宏包解压到系统目录:\\
/usr/share/texmf/tex/latex 里面即可。

当然你也可以在另外一个文件夹里面,这里必须是你的主文件夹下,新建一个文件夹texmf,然后里面新建一个tex,然后再新建一个目录latex,然后在这个latex里面放着你下载下来的宏包。具体比如说有一个宏包名字叫做config,那么latex下面就是config文件夹,然后里面就是config.sty文件。你自己写的宏包扔进去一样有效。
唯一要额外做的操作就是在texmf目录之下运行命令:\\
sudo texhash  \\
让texlive把这个目录也加入搜索范围。


\section{texmaker技巧}
\subsection{自动补全命令}
在菜单里找到用户自定义的customize completion 也就是自动完成,里面加入你想要的命令。比如:\\
\textbackslash textbackslash 然后点击add,这样以后想输入显示命令前面的那个斜线的时候会方便点。如果括号里面加入@符号,那么就会出现类似系统自带命令\textbackslash section\{•\}的那个黑点•。

\subsection{自定义命令}
在菜单哪里用户自定义,你看到可以用户自定义命令,填好之后就是快速构建下面那些备用的1:2:3:哪里将成为有意义的命令。

\subsection{保存你的设置文件}
在选项哪里有保存设置文件的功能,主要是自定义命令,自动补全命令等可以保存下来。

\section{用rime快速输出\LaTeX 命令}
具体效果就是我按下/tex等命令,就会弹出很多\LaTeX 命令,这个也是修改symbols.yaml文件来达到的,也是在我新建的那个\href{https://github.com/a358003542/rime-symbols-yaml}{github项目}里面,我做了很多优化工作,有兴趣的可以研究下。
\endnote{在rime中文模式下,直接按enter键输出的也是英文,所以一些简单地命令直接输入也是很快捷的。}


\section[其他\TeX 编辑器评测]{其他\TeX 编辑器评测\footnote{虽然texmaker软件也有些小缺陷,但综合起来我觉得是最棒的。}}
\begin{description}
\item[emacs] 安装emacs24之后,用package manager安装auctex宏包。然后一些基本的latex编辑器功能都有了。稍微熟悉下就行了。唯一遗憾的是没有那种左侧很方便的目录导航功能。然后emacs的各种快捷键让我压力好大。
\item[winefish latex] 太高端了,刚进去一片空白?
\item[gedit] gedit安装好latex的插件之后也很好。因为latexila编辑器在编译大文件的时候速度好慢,而gedit是直接调用终端模式所以没有损伤的。值得一提的是latexila编辑器的一些功能在gedit里面可以通过外部工具这个插件自己编写终端命令来实现。
\item[texworks] 没怎么评测。
\item[kile] 没怎么评测。
\item[texstudio] 我刚进去什么都没动,它好像中毒一样到处乱操作。。
\item[gummi] 打开tex文档或者做些小的测试还是可以的,但是似乎不支持include文档。
\item[latexila] 这个软件刚进去我就感觉设计理念非常的简洁,和gedit一样。然后我特别在意的文档结构图显示,语法染色都有。\endnote{latexila编辑器使用xelatex引擎生成文档,一样要自己编写一个生成命令,在创建→首选项那里→标签就写xelatex,写其他的也可以。扩展名写.tex,然后命令和texmaker一样写上这个:xelatex -interaction=nonstopmode  \$filename 。}
\end{description}


\section{gedit的一些技巧简要说明}
在编辑→首选项那里请安装好latex插件和外部工具插件,恩,代码注释插件和文件浏览器插件推荐。

我在这里重点介绍一下外部工具插件的用法。看到工具→外部工具。
请选择最下面的那个管理外部工具,看到在此处打开终端,gnome-terminal后面去除掉,然后就能正常工作了。\endnote{本信息来自ubuntu13.04,gedit版本号为3.6.2。}

现在我们新建一个命令,名字叫做xelatex,具体内容如下:
\begin{verbatim}
#!/bin/sh
filename=$GEDIT_CURRENT_DOCUMENT_NAME
shortname=`echo $filename | sed 's/\(.*\)\.tex$/\1/'`
xelatex -interaction batchmode -src $filename
\end{verbatim}
那么这个小工具就实现xelatex编译功能了。

然后我们也可以再新建一个小工具,名字叫清理:
\begin{verbatim}
#!/bin/sh
filename=$GEDIT_CURRENT_DOCUMENT_NAME
shortname=`echo $filename | sed 's/\(.*\)\.tex$/\1/'`
rm -f  $shortname.aux $shortname.ent  $shortname.out 
$shortname.lot $shortname.idx $shortname.lof 
$shortname.ilg $shortname.ind $shortname.log 
$shortname.toc $shortname.bbl $shortname.blg
\end{verbatim}
这样运行它就可以清理临时文件了。


\section{结合github进行项目管理}
以下主要参考\href{http://rogerdudler.github.io/git-guide/index.zh.html}{这个网站}\endnote{在项目网站右下角settings哪里进去有很多项目管理内容,其中最下面有删除项目的功能,请慎重使用。}
到github上注册,创建新的项目,还有安装git软件(ubuntu12.10自带的有)都很简单的,我就不多说了。下面就git命令使用的基本流程说明如下:
\subsection{基本命令}
\paragraph{远程仓库文件到本地}
网上创建项目之后,你需要将网上的存档下载到本地,在你希望下载的地点,打开终端:\\
git ~~clone ~~https://github.com/a358003542/xelatex-guide-book.git\footnote{后面的这个链接地址在你创建的项目的右下角哪里,写着HTTPS clone URL}

git init 命令用于本地创建的文件夹上传到远程仓库,我估计git clone 下来的仓库文件已经索引了,但git init 命令要不要用还不确定。这里认为不需要用。\endnote{参考了\href{http://www.cnblogs.com/findingsea/archive/2012/08/27/2654549.html}{这个网站}}
\paragraph{本地仓库文件进入索引}
下载下来的本地仓库文件进入git的索引,该文件夹内的所有文件都进入索引则在终端中输入如下命令:\\
git ~~add ~~.  \\
因为我们在github创建项目的时候已经创建了一个配置文件,比如我选择的是latex语言,然后它自动会处理将某些文件不上传。

\paragraph{将索引中改动的文件提交到head}
不太清楚这个索引,head具体是什么意思,anyway,过程就是这样的。\\
git ~~commit ~~-m ~~'2013-08-25:19:00' \\
后面的文字串等下在github网站中会看到的,表示这个文件的标示符吧,你也可以取其他的名字。我就喜欢取当时的日期和时间了。

\paragraph{将head中改动的文件更新到远程仓库}
第一次你需要给你的远程服务器取个简单点的名字:\\
git ~~remote ~~add ~~origin ~~https://github.com/a358003542/xelatex-guide-book.git\footnote{这里的地址和前面的一样}

然后以后都可以用这个简单的命令来更新了:\\
git~~ push~~ origin~~ master\footnote{这里的origin的意思前面说了,master是远程仓库默认的一个分支,后面会讲到你可以创建其他的分支。}

\paragraph{远程仓库的改动更新到本地}
需要说的是git对文件的操作是合并式的,也就是只是替换最新改动的文件。\\
git ~~pull~~origin~~master

\subsection{日常改动提交流程}
一般情况下就用前面讲的的三步,add .~~ commit ~~-m~~ 然后push。这是基本的日常维护提交流程。

如果你在网站上对远程仓库做了一些修改,记得先用pull命令将远程仓库的改动更新到本地。


\section{安装字体}
\label{sec:安装字体} 
\subsection{找字体文件}
如果你装了windows系统,那么你可以到windows下copy这些字体文件。比如windows常用的宋体,times new roman等,在C盘的windows的fonts文件夹里面。本文用的就是adobe中文系列:adobe 宋体 std, adobe 黑体 std , adobe 楷体 std 。很奇怪,在pdf上我觉得这几个字体感觉很好,但是在屏幕上就觉得不太好了。

\subsection{放置字体文件}
推荐都放在ubuntu的主目录的.fonts文件夹里面\footnote{这是一个隐藏目录},如果没有请新建一个。这是通常默认用户新加字体放置的目录。当然你也可以放在其他目录里面,比如你的同步盘里面,然后用font-manager安装字体也是可以的。\footnote{这是本文推荐的方式,安装卸载字体都方便些。}

\subsection{命令行安装字体}
运行命令:\\
fc-cache ~~ -f ~~ -v  

字体就安装好了,如果你要看现在你的系统上有那些可用的中文字体,在终端运行命令\footnote{|表示linux命令中的通道,第一个命令的输出信息流会流向sort命令,排序之后重定向到ziti.txt文件里面。然后终端的数据就保存在这个文件里面了。}:\\
fc-list :lang=zh | sort >ziti.txt 

打开ziti.txt,里面就是你的可用中文字体的信息,比如:\\
/home/wanze/.fonts/simsun.ttc: \\宋体,SimSun:style=Regular \\
其中第一个是字体文件所在的目录,第二列信息是可以调用的名字,有宋体和SimSun。

\subsection{通过fontmanager安装字体}
你可以安装其他软件来安装和管理字体,比如fontmanager:\\
sudo  apt-get install font-manager  \\
这个软件查看安装卸载或者禁用某些字体都很方便的,需要提醒的是这个软件占用了默认的用户配置文件.fonts.conf。然后你的字体DIY需要到~/.config/font-manager/local.conf哪里去设置。这个下面会讲到。

\section{ubuntu系统字体的配置}
不是特别难看的情况就没必要改动系统字体,因为我们不要低估了人眼的适应能力。这里的配置主要是指由于系统升级带来的字体的改变,特别是中文字体的改变,手动配置将其固定下来。\endnote{这个代码主要参考了\href{http://www.freedesktop.org/software/fontconfig/fontconfig-user.html}{这个网站}}
\begin{verbatim}
<?xml version="1.0"?>
<!DOCTYPE fontconfig SYSTEM "fonts.dtd">
<fontconfig>

<match target="font">
	<edit name="rgba" mode="assign"><const>rgb</const></edit>
</match>

<match>
	<test name="lang" compare="contains"><string>zh</string></test>
	<test name="family"><string>serif</string></test>
	<edit name="family" mode="prepend"><string>微软雅黑</string></edit>
</match>

<match>
	<test name="lang" compare="contains"><string>zh</string></test>
	<test name="family"><string>sans-serif</string></test>
	<edit name="family" mode="prepend"><string>微软雅黑</string></edit>
</match>

<match>
	<test name="lang" compare="contains"><string>zh</string></test>
	<test name="family"><string>monospace</string></test>
	<edit name="family" mode="prepend"><string>微软雅黑</string></edit>
</match>	

</fontconfig>
\end{verbatim}
有很多内容没有深究,第一个是打开rgba模式,优化液晶显示的。然后下面就是对三大字族设置,如果是zh中文的话那么就prepend也就是插入微软雅黑,也就是在搜索队列中微软雅黑优先级最高。这个可以通过fc-match  -s  serif等来查看。然后其他字体设置不想涉及了,只希望他们能够稳定下来。感觉设置雅黑字体了,系统的主题换为Radiance更好看些。然后用font-manager针对微软雅黑高级设置加上AA和AH。一个是反锯齿一个是自动粗细设置吧。就这样了。
\endnote{chromium字体也都设置成为微软雅黑吧,然后我感觉页面稍微放大点更好看。}


\section[ubuntu的备份和还原]{ubuntu的备份和还原\footnote{以下主要参考了\href{http://www.matthartley.com/how-to-backup-your-ubuntu-software/}{这个网站}}}
首先是利用系统自带的备份软件将home文件夹里面的一些内容备份好,注意配置文件夹排除法则。

备份步骤简介如下:
\begin{enumerate}
\item 备份你的PPA,也就是你通过PPA装的软件。有的可能都没有通过PPA装过软件,那么这一步和下面的备份PPA的key都可以省略。命令如下:\\
sudo~~cp~~-r ~~/etc/apt/sources.list.d~ ~/Nutstore/ubuntu-config
\item 备份你的PPA的key:\\
sudo~~apt-key~~exportall~>~~~/Nutstore/ubuntu-config/myppakey
\item 通过新力得软件包管理器生成你安装的所有软件包的列表:选择文件F→将标记的项目另存为A→然后下面保存全部状态,不仅仅是变更选项勾上→然后选择一个地址取一个有意义的名字。
\end{enumerate}

如果重装系统了,还原步骤相当于前面备份步骤的反向操作吧:
\begin{enumerate}
\item 还原你的PPA:这里就不写命令了,和上面类似也是cp   -r,不同的是现在将你保存的PPA还原到新系统的etc的apt目录下。
\item 还原你的PPA key,假设对照上面的情况,我们有命令:\\
\verb+sudo    apt-key    add    ~/Nutstore/ubuntu-config/myppakey+
\item 通过新力得软件包全部安装你之前保存的软件包列表 :选择文件F→读取标记的项目R→然后类似正常的通过新力得安装软件包的步骤,点击应用即可。
\item 通过上面的步骤,你的新系统的所有软件全新安装好了,至于其他配置文件前面说了一些你在意的配置保存,然后选择相应的位置,一般在.config文件夹里面,复制进去就行了。
\end{enumerate}


\printendnotes


\include{parts/xelatex基础}

\part{\LaTeX 高级篇}
\section{自建环境或修改原环境}
\label{sec:自建环境}

用如下命令格式来新建一个环境\\
\textbackslash begin \{environmentname\}\\
text\\
\textbackslash end \{environmentname\}\\
新建环境有很多用途,这里就最直接的插入图片来说明,比如说上面那个图片我用:\\
\textbackslash begin\{figure\}\\
什么什么\\
\textbackslash end\{figure\}\\
给包围起来,然后加入:\\
\textbackslash caption \{这是用于\textbackslash LaTeX 的一个编辑器界面\}\\
caption命令加入图片标题。然后如果你想在文章中引用这个图片的话,加入如下命令:\\
\textbackslash label \{figure1\}

还有很多高级知识这里能够略过就略过了。

\section{盒子和glue}
\label{sec:盒子和glue}
\begin{fancycolorbox}
这里参考了Knuth的The Texbook,但是我并没有将其写入参考文献,因为只是觉得关于box和glue的概念最好参考原初定义,但是并不推荐读者阅读这本书,现在已经是\LaTeX 时代了,不推荐读者使用原始的\TeX 命令,一是不太实用,二是兼容性可能不太好。除非你是宏包编写者,但就作为一般的使用者真的没有必要接触那些原始命令了。还参考了\cite{boxes}
\end{fancycolorbox}
\endnote{\begin{fancycolorbox}
在这里box翻译为盒子没什么问题,就是这个glue翻译为胶水或者橡皮都让我不太满意。在后面我都使用的术语是距离或者间距或者干脆用英文glue。从某种意义上讲glue的直译确实应该译为胶或者胶水,不过觉得距离这个词汇更能够让人们有感觉些:\TeX 排版就是不同的盒子编写和彼此之间距离的设置问题。
\end{fancycolorbox}}
\subsection{基本知识}
最小的盒子就是基于Unicode的字符,这些字符然后组成更大的盒子──单词,然后单词组成更大的盒子──行等等。行是一个盒子,段落也是一个盒子,图片是一个盒子,表格也是一个盒子。而这些盒子按照Knuth的描述都是用glue胶水粘合起来的,或者我们称之为这些盒子之间都存在着空间胶合层。下面就是一个盒子的详细参数:

\begin{fig}{box参数}
\label{fig:box参数}
\end{fig}

盒子就是这么一个长方形的区域,如上图所示,它有参数:height,width,depth。baseline和reference point在后面讲的hbox和vbox中会用到。在\TeX 看来,从字体而来的Unicode 字符就是一个最简单的盒子。字体的设计者已经决定了这个字符的高度,宽度和深度以及它在这个盒子里面看起来如何。\TeX 就是用这些维度将盒子黏合到一起,并最终决定所有字符的reference point 参考点在页面上的位置。

\TeX 的盒子如果全部涂上颜色,一般是黑色,那么就成了一个黑盒子。这样的黑盒子还有一个名字叫做rule box。也就是线条。这个在后面会谈论到的。

不管是字符盒子还是黑盒子,他们要某是水平排列要某是垂直排列。水平排列要做的就是让这些盒子的参考点在一条水平线上。类似的垂直排列要做的就是让这些盒子的参考点在一条垂直线上。

好吧,介绍两个\TeX 命令:hbox和vbox。hbox命令就是让所有的盒子在一条水平线上,而vbox命令就是把一些hbox命令垂直排列,比如下面的代码:\\
\verb+\vbox{\hbox{恭}\hbox{喜} \hbox{发}\hbox{财}}+

\vbox{\hbox{恭}\hbox{喜} \hbox{发}\hbox{财}}

glue也就是各个盒子之间的间距。下图是glue的具体图示:
\begin{fig}{glue说明}
\label{fig:glue说明}
\end{fig}
前面说到盒子的reference point水平排列,然后他们之间还有叫做glue的间距。间距有三个属性:正常间距量(space),拉伸量(stretch),缩减量(shrink)。比如这个图片中第一个glue的正常间距是9个单位,拉伸量为3个单位,缩减量为1个单位。而总的情况是正常间距是5(box1)+9+6+9+3+12+8=52个单位。现在假设一行宽58个单位,\TeX 就要调整使得这一行盒子的宽度刚好等于58个单位,于是还需要增加6个单位的宽度,而这6个单位的宽度\footnote{为了简单起见这里不考虑缩减量,具体缩减量如何计算我也不大清楚。}需要从这一行所有glue里的拉伸量中找出来。于是总的拉伸量加法是3+6+0=9。也就是6个单位的宽度要分成9等分再分配给他们,即第一个glue的拉伸量是$3*\Large{\frac { 6 }{ 9 }} $。这样第一个间距的总长度就是$9+3*\Large{\frac { 6 }{ 9 }} =11$。

经过计算所有的glue间距都确定下来了,那么整个页面布局就确定了。我在这里就戛然而止了,毕竟这里只是对box和glue的基本概念的阐明。

\subsection{盒子in \LaTeX}
前面稍微介绍了点\TeX 的原初命令hbox和vbox。我们就能够感受到靠着这两个命令已经能够干很多事情了。比如我想写一幅对联。

\begin{verbatim}
{\CJKfontspec[Scale=3,Color=red]{方正魏碑_GBK}
\vbox {
\centerline{呵呵}
    \hbox to \hsize {恭大}
    \hbox to \hsize {喜吉}
    \hbox to \hsize {发大}
    \hbox to \hsize {财利}}}
\end{verbatim}

\vspace{30pt}
{\CJKfontspec[Scale=3,Color=red]{方正魏碑_GBK}
\vbox {
\centerline{呵呵}
    \hbox to \hsize {恭大}
    \hbox to \hsize {喜吉}
    \hbox to \hsize {发大}
    \hbox to \hsize {财利}}}

其中hbox和vbox都支持这样的to 多少尺寸和 spread 增加多少尺寸的形式,这里的hsize相当于linewidth吧。然后我用这个例子主要说明用hbox和vbox能够做很多排版工作了。而我们在\LaTeX 里面则用makebox替代了hbox,raisebox替代了vbox。并且排版思路变成了主要是一行一行排版,然后才考虑raisebox微调。

\section{box 命令介绍}
\href{http://tex.stackexchange.com/questions/83930/what-are-the-different-kinds-of-boxes-in-latex}{参考了这个网站}

\subsection{makebox}
\verb+\makebox[width][allignment]{some text}+
还有一个mbox,不过makebox命令更加全面,推荐使用。makebox就是制造一个水平的盒子,注意这个box里面的文本是不能换行的,也就是一行之内的盒子。第一个可选项width指这个盒子的长度,第二个可选项allignment是里面文本的对齐方式,有〖l c r s〗几个选项,\emph{l}表示左对齐;\emph{c}表示居中;\emph{r}表示right;\emph{s}表示两端对齐。






\section{带颜色或者线框的盒子}
\subsection{colorbox}
colorbox
\verb+\colorbox{yellow}{this is a test line.}+\\
\colorbox{yellow}{this is a test line.}


\section{fancycolorbox}
本文定义了一个fancycolorbox命令,具体代码如下:\\
\begin{fancycolorbox}
为什么是这么复杂的一个代码?这个需要说明一下。首先系统自带的colorbox命令不能换行,必须colorbox里面加上minipage环境才行,9,10行是minipage环境的格式优化,然后整个构成一个新环境我还想传递颜色,就成这个样子了。主要参考的\href{http://tex.stackexchange.com/questions/127612/color-text-and-bg-of-verbatim-without-affecting-fancyvrb-line-numbers}{这个网站}。
\end{fancycolorbox}
\vspace{20pt}
\begin{verbatim}
\definecolor{bgcolor-0}{HTML}{CCFFCC} 

\newsavebox{\tempbox}
\newenvironment{fancycolorbox}[1][bgcolor-0]
 {\noindent%%
  \renewcommand{\tempcolor}{#1}
  \begin{lrbox}{\tempbox}%
  \begin{minipage}{\linewidth-10pt}
  	\setlength{\parskip}{1.6ex plus 0.2ex minus 0.2ex}   %段落間距
	\setlength{\parindent}{\baselineskip * \real{0.06} + \textpt * \real{2.4}}}  	
 {\ignorespacesafterend%
  \end{minipage}%
  \end{lrbox}%
  \colorbox{\tempcolor}{\usebox{\tempbox}}}  
\end{verbatim}
fancycolorbox环境的效果前面你应该已经看过了。这里就不做演示了。


\section{大型文档}


\part{其他问题的讨论}

\section{语录,引用和诗歌环境}
quote为语录环境,quotation用于超过几段的引用环境,verse用于诗歌环境。下面是quote语录环境:
\begin{quote}
“When the winds of change blow, some people build walls and others build windmills.”
\end{quote}


\section{插入摘要}
使用的环境就是abstract,\endnote{\reduline{book类不能使用摘要环境,只有article和report才有。}}命令如下:\\
\begin{verbatim}
\begin{abstract}
这是一段摘要文字。
\end{abstract}
\end{verbatim}

\section{文本中的上标还有下标}
上标使用命令textsuperscript命令,下标感觉自己缩小点就差不多了吧。示例如下:\\
上标\textsuperscript{上标}下标{\scriptsize 下标}

更常见的是化学分子式里面的上标和下标,推荐使用mhchem宏包,至于数学模式里的上标和下标自不必说了。


\section{多张图片并列显示}
\label{sec:多张图片并列显示}

\begin{figure}[h]

\label{fig:四栏图片}
\begin{multicols}{4}
\includegraphics[width=\linewidth ,totalheight=\textheight , keepaspectratio]{temp.jpg}
i want  some test to show there is a text and not to column break.\\
\centerline{test}
\columnbreak

\includegraphics[width=\linewidth ,totalheight=\textheight , keepaspectratio]{temp.jpg}
\centerline{test}
\columnbreak

\includegraphics[width=\linewidth ,totalheight=\textheight , keepaspectratio]{temp.jpg}
\centerline{test}
\columnbreak

\includegraphics[width=\linewidth ,totalheight=\textheight , keepaspectratio]{temp.jpg}
\centerline{test}

\end{multicols}
\caption{denosie fig}

\end{figure}

具体代码如下:
\begin{verbatim}
\begin{figure}[h]

\label{fig:四栏图片}
\begin{multicols}{4}
\includegraphics[width=\linewidth ,totalheight=\textheight , 
        keepaspectratio]{temp.jpg}
i want  some test to show there is a text and not to column break.\\
\centerline{test}
\columnbreak
\includegraphics[width=\linewidth ,totalheight=\textheight , 
         keepaspectratio]{temp.jpg}
\centerline{test}
\columnbreak
\includegraphics[width=\linewidth ,totalheight=\textheight ,
         keepaspectratio]{temp.jpg}
\centerline{test}
\columnbreak
\includegraphics[width=\linewidth ,totalheight=\textheight ,
         keepaspectratio]{temp.jpg}
\centerline{test}
\end{multicols}
\caption{denosie fig}

\end{figure}
\end{verbatim}
这段代码的分栏还有插入图片知识都已经介绍了,值得一提的就是widetext环境,也就是有changepage宏包而来的临时改变页面布局环境和浮动体环境figure以及table不兼容。比如放入浮动体环境内才能起作用。然后caption命令似乎只是默认的linewidth居中。所以如果想要图片和表格在扩大的文本布局中居中对齐,那么需要在浮动体环境内部使用改变页面布局命令,然后使用居中命令。

\section[生成字体所有已有的字形]{生成字体所有已有的字形\endnote{主要参照了\href{http://tex.stackexchange.com/questions/23863/generating-a-table-of-glyphs-with-xetex}{这个网站}}}
\label{sec:字体已有字形}
\begin{verbatim}
\documentclass[landscape]{article}
\usepackage{geometry}
\usepackage{fontspec}
\setmainfont{DejaVu Sans}
\usepackage{multicol}
\setlength{\columnseprule}{0.4pt}
\usepackage{multido}
\setlength{\parindent}{0pt}
\begin{document}
\begin{LARGE}
\begin{multicols}{5}
\multido{\i=0+1}{"196607"}{% from U+0000 to U+2FFFF
%后面的还会继续扩展,目前一般还没使用。
  \iffontchar\font\i
    \makebox[4em][l]{\i}%
    \symbol{\i}\endgraf
  \fi
}
\end{multicols}
\end{LARGE}
\end{document}
\end{verbatim}


\section{方便手机上观看的pdf}
原文档内容都不需要修改,只需要修改geometry的设置就可以满足要求,然后在手机上看的时候选择adobe pdf软件的连续观看功能会有更好的体验。具体配置如下:
\endnote{主要参考了\href{http://tex.stackexchange.com/questions/78920/generating-smartphone-readable-pdf}{这个网站}。}
\begin{verbatim}
\ifphone
%for phone
\RequirePackage[
  paperwidth=105mm, %除去旁註其他沒變,115,再稍微小點
  paperheight=190mm,%太長了縮短點,
  bindingoffset=0mm,%裝訂線
  top=15mm,  %上邊距 包括頁眉
  bottom=15mm,%下邊距 包括頁腳
  left=5mm,  %左邊距or inner
  right=5mm,  %右邊距or  outer
  headheight=10mm,%頁眉
  footskip=10mm,%頁腳
  includemp=true,% 旁註寬度計入width
  marginparsep=0mm, %沒有旁註
  marginparwidth=0mm,  %沒有旁註
  ]{geometry}
\else

\RequirePackage[a4paper, %a4paper size 297:210 mm
  bindingoffset=0mm,%裝訂線
  top=45mm,  %上邊距 包括頁眉
  bottom=40mm,%下邊距 包括頁腳
  left=35mm,  %左邊距or inner
  right=40mm,  %右邊距or  outer
  headheight=25mm,%頁眉
  footskip=25mm,%頁腳
 includemp=true,% 旁註寬度計入width
 marginparsep=0mm, %旁註與正文間距
marginparwidth=0mm,  %旁註寬度
  ]{geometry}
\fi
\end{verbatim}
这里使用了一个条件命令。在加载myconfig.sty宏包的时候已经新建了一个条件变量:\\
\verb+\newif\ifphone+\\
\verb+\phonefalse+
ifphone就是一个条件判断命令,这里涉及到的命令我也不大懂,总之,如果我改成:\\
\verb+\phonetrue+\\
就会自动生成适合在手机上观看的pdf。

\printendnotes






\part{后面的珍宝}
\begin{quotation}
"The opera ain't over until the fat lady sings."

好戏还在后头,等著瞧吧。
\end{quotation}

我在Ubuntu的模板文件夹里面新建了一个tex.tex文件。方便我用鼠标右键快速创建模板。

第一个模板我打算使用一个通用的配置,写一个myconfig宏包,用于加载这个通用配置。还有一个mytitle宏包用于加载封面。然后整个模板就专注于写作,很简洁的样子。

然后我在Ubuntu的home文件目录或者称之为主文件夹下面新建了一个文件夹texmf,然后里面新建了一个文件夹tex,然后里面新建了一个文件夹latex,然后在里面新建了一个文件夹myconfig。myconfig.sty文件就放在里面。然后在终端上运行命令:\verb+sudo~ texhash+即可。mytitle宏包的安装类似。

myconfig宏包的内容是我以前之前的XeLaTeX指南的精华部分,很多冗余删除了。关于myconfig和mytitle宏包的编写,我参考了\href{http://tex.stackexchange.com/questions/70166/create-a-function-that-generates-a-title-page}{这个网站}。

三个文件的代码在本文后面列出。现在就一些必要的信息说明如下:

首先sty文档里面@符号不需要用makeatletter这个命令了,@在里面已经默认就是一个符号。
其次文档的基本结构
\begin{verbatim}
\NeedsTeXFormat{LaTeX2e}
\ProvidesPackage{mytitle}
…
\endinput
\end{verbatim}
这个不用多说。

然后就是sty文档里面所有usepackage都换成RequirePackage。这样sty文档兼容性更好,前面没有documentclass也行。

myconfig宏包的详细讨论我在XeLaTeX指南一书中会详细讨论。
mytitle宏包前面那部分是定义新的输入命令,第二行是用于处理可能出现错误的反应。就照这个模式写。然后gdef命令后面就跟著新定义的命令的形式,gdef和def命令的不同是他定义的是global的,我估计之前我那种写法不行可能就是没有gdef。
后面就是基本的封面设计了。

特别提醒在Ubuntu下右键新建之后,是未命名 然后空格 然后tex,这种文件名XeTeX处理似乎不很好。建议改成没有空格的文件名。

\section{tex.tex代码}
\begin{verbatim}
% !Mode:: "TeX:UTF-8"%確保文檔utf-8編碼
%新加入的命令如下:addchtoc addsectoc reduline printendnotes 
%新加入的环境如下:common-format  fig scalefig 

\documentclass[11pt,oneside]{book}
\newlength{\textpt}
\setlength{\textpt}{11pt} 
\newif\ifphone
\phonefalse

\usepackage{myconfig}
\usepackage{mytitle}

\begin{document}
\frontmatter   

\titlea{书籍}
\titleb{使用\LaTeX排版}
\titlec{一种良好的风格}
\author{作者}
\authorinfo{作者:}
\editor{编者}
\email{a358003542@gmail.com}
\editorinfo{编者:}
\version{1.0}
\mytitle

\addchtoc{开头说的话}
\chapter*{开头说的话}
\begin{common-format}
开头说的话

%这里空一行。

\end{common-format}


\addchtoc{目錄}
\setcounter{tocdepth}{1}    
\tableofcontents

\begin{common-format}
\mainmatter 

\chapter{章节开始}

开始写作。。。




%这里空一行

\end{common-format}  
\end{document}




\end{verbatim}


\section{myconfig.sty}
\begin{verbatim}
% 一种书籍的风格,对中文处理较好。
% 
% by wanze
% LPPL LaTeX Public Project License
%  
%

\NeedsTeXFormat{LaTeX2e}[1994/06/01]
\ProvidesPackage{myconfig}

%\def\hi{Hello, this is my own package}
%\let\myDate\date
%\newcommand\GoodBye[1][\bfseries]{{#1Good Bye}}

\RequirePackage{calc,float,multicol,moresize} 
\RequirePackage[doublespacing]{setspace}
\newcommand{\addchtoc}[1]{ 
	\cleardoublepage   
	\phantomsection    
	\addcontentsline{toc}{chapter}{#1}}
	
\newcommand{\addsectoc}[1]{ 
	\phantomsection    
	\addcontentsline{toc}{section}{#1}}

\newenvironment{common-format}{ %
	\setlength{\parskip}{1.6ex plus 0.2ex minus 0.2ex}   %段落間距
	\setlength{\parindent}{\baselineskip * \real{0.06} + \textpt * \real{2.4}}}  
    {}


\ifphone
%for phone
\RequirePackage[
  paperwidth=105mm, %除去旁註其他沒變,115,再稍微小點
  paperheight=190mm,%太長了縮短點,
  bindingoffset=0mm,%裝訂線
  top=15mm,  %上邊距 包括頁眉
  bottom=15mm,%下邊距 包括頁腳
  left=5mm,  %左邊距or inner
  right=5mm,  %右邊距or  outer
  headheight=10mm,%頁眉
  footskip=10mm,%頁腳
  includemp=true,% 旁註寬度計入width
  marginparsep=0mm, %沒有旁註
  marginparwidth=0mm,  %沒有旁註
  ]{geometry}
\else

\RequirePackage[a4paper, %a4paper size 297:210 mm
  bindingoffset=0mm,%裝訂線
  top=45mm,  %上邊距 包括頁眉
  bottom=40mm,%下邊距 包括頁腳
  left=35mm,  %左邊距or inner
  right=40mm,  %右邊距or  outer
  headheight=25mm,%頁眉
  footskip=25mm,%頁腳
 includemp=true,% 旁註寬度計入width
 marginparsep=0mm, %旁註與正文間距
marginparwidth=0mm,  %旁註寬度
  ]{geometry}
\fi

\RequirePackage[table]{xcolor}  
\definecolor{bgcolor-0}{HTML}{CCFFCC} 

\newsavebox{\tempbox}
\newenvironment{fancycolorbox}[1][bgcolor-0]
 {\noindent%%
  \renewcommand{\tempcolor}{#1}
  \begin{lrbox}{\tempbox}%
  \begin{minipage}{\linewidth-10pt}
  	\setlength{\parskip}{1.6ex plus 0.2ex minus 0.2ex}   %段落間距
	\setlength{\parindent}{\baselineskip * \real{0.06} + \textpt * \real{2.4}}}  	
 {\ignorespacesafterend%
  \end{minipage}%
  \end{lrbox}%
  \colorbox{\tempcolor}{\usebox{\tempbox}}}  


%================字體================%
\RequirePackage{xltxtra,fontspec,xunicode} %必備三件套
\RequirePackage[CJKnumber=true]{xeCJK} %中文環境宏
\xeCJKsetup{PunctStyle=plain}
\defaultCJKfontfeatures{Scale=1.2}   %放大全局CJK字體。中文字應該稍微高於英文字
\setCJKmainfont[BoldFont=Adobe 黑体 Std,ItalicFont=Adobe 楷体 Std]
    {Adobe 宋体 Std}%影響rmfamily字體
\setCJKsansfont{Adobe 黑体 Std}%影響sffamily字體
\setCJKmonofont{Adobe 楷体 Std}%影響ttfamily字體
 %設置英文字體
\setmainfont[Mapping=tex-text]{DejaVu Serif} 
\setsansfont[Mapping=tex-text]{DejaVu Sans}
\setmonofont[Mapping=tex-text]{DejaVu Sans Mono}

%=============新的字符===========%
\newfontfamily{\libertine}[Scale=1.5]{Linux Libertine O}
\newfontfamily{\ubuntu}[Scale=3]{Ubuntu}
\RequirePackage{newunicodechar}
\newunicodechar{Ⓐ}{{\libertine{Ⓐ}}}
\newunicodechar{Ⓑ}{{\libertine{Ⓑ}}}
\newunicodechar{Ⓒ}{{\libertine{Ⓒ}}} 
\newunicodechar{Ⓓ}{{\libertine{Ⓓ}}}
\newunicodechar{①}{{\libertine{①}}}
\newunicodechar{②}{{\libertine{②}}}
\newunicodechar{③}{{\libertine{③}}}
\newunicodechar{④}{{\libertine{④}}}
\newunicodechar{⑤}{{\libertine{⑤}}}
\newunicodechar{}{{\ubuntu{}}}

%%===============中文化=========%
\renewcommand\contentsname{目~录}
\renewcommand\listfigurename{插图目录}
\renewcommand\listtablename{表格目录}
\renewcommand\bibname{参~考~文~献}
\renewcommand\indexname{索~引}
\renewcommand\figurename{图}
\renewcommand\tablename{表}
\renewcommand\partname{部分}
\renewcommand\appendixname{附录}
\renewcommand\today{\number\year年\number\month月\number\day日}

\RequirePackage{fancyhdr}   %頁眉頁腳
\pagestyle{fancy}
\fancypagestyle{plain}{
    \fancyhf{}
    \renewcommand{\headrulewidth}{0pt}
    \renewcommand{\footrulewidth}{0pt}
    \renewcommand{\chaptermark}[1]{\markboth{第\CJKnumber{\arabic{chapter}}章~~#1}{}} 
     \renewcommand{\sectionmark}[1]{\markright{第\CJKnumber{\arabic{section}}節~~#1}{}} 
%    \fancyhf[HL]{\ttfamily \footnotesize \leftmark }
    \fancyhf[HR]{\ttfamily \footnotesize \rightmark }
    \fancyhf[FR]{\thepage}
    \fancyhfoffset[R]{\marginparwidth+\marginparsep}
    }
\pagestyle{plain} 

%=========章節標題設計=========%
\RequirePackage{titlesec}
%修改part
\titleformat{\part}{\huge\sffamily}{}{0em}{} 
%修改chapter
\titleformat{\chapter}{\LARGE\sffamily}{}{0em}{} 
%修改section
\titleformat{\section}{\Large\sffamily}{}{0em}{}
%修改subsection
\titleformat{\subsection}{\large\sffamily}{}{0em}{}
%修改subsubsection
\titleformat{\subsubsection}{\normalsize\sffamily}{}{0em}{}


%================目录===============%
\RequirePackage{titletoc}

%==============超鏈接===============%
\RequirePackage[colorlinks=true,linkcolor=blue,citecolor=blue]{hyperref} %設置書簽和目錄鏈接等

%=================文字強調=========%
\RequirePackage{ulem} %下劃線,加點
\normalem

\newcommand\reduline{\bgroup\markoverwith
{\textcolor{red}{\rule[-0.8ex]{1em}{0.4pt}}}\ULon}
\renewcommand\emshape{\color{red}}

%==================插入圖片=======%
\RequirePackage{graphicx}
\graphicspath{{figures/}}
\newenvironment{fig}[1]
	{\begin{figure}[h]
	\includegraphics[width=\linewidth ,totalheight=\textheight , keepaspectratio]{#1}
	\caption{#1}
	\end{figure}}
	{}	
\newenvironment{scalefig}[2][0.4]
	{\begin{figure}[h]
	\includegraphics[scale=#1]{#2}
	\caption{#2}
	\end{figure}}
	{}	

%==============插入表格========%
\RequirePackage{booktabs}
\renewcommand{\arraystretch}{1.3}

%============插入代码============%
\RequirePackage{fancyvrb} 
\DefineVerbatimEnvironment%
	{verbatim}{Verbatim}
	{numbers=left,frame=lines,tabsize=4 ,baselinestretch=2,
	xleftmargin=6pt, fontsize=\footnotesize , numbersep=2pt}  	
	
\newlength{\fancyvrbtopsep}
\newlength{\fancyvrbpartopsep}
\makeatletter
	\FV@AddToHook{\FV@ListParameterHook}
	{\topsep=\fancyvrbtopsep\partopsep=\fancyvrbpartopsep}
\makeatother
\setlength{\fancyvrbtopsep}{-15pt}   %代碼環境之上空白高度
\setlength{\fancyvrbpartopsep}{0pt}  %段落之上空白高度

%=============插入注释=========%
\RequirePackage{endnotes}
\RequirePackage{hyperendnotes}
\renewcommand\makeenmark{(\theenmark)}
\renewcommand\notesname {注释和参考资料}
\newcommand{\printendnotes}
    {\theendnotes}






\endinput

\end{verbatim}


\section{mytitle.sty}
\begin{verbatim}
\NeedsTeXFormat{LaTeX2e}
\ProvidesPackage{mytitle}


\def\titlea#1{\gdef\@titlea{#1}}
\def\@titlea{\@latex@warning@no@line{No \noexpand\titlea given}}
\def\titleb#1{\gdef\@titleb{#1}}
\def\@titleb{\@latex@warning@no@line{No \noexpand\titleb given}}
\def\titlec#1{\gdef\@titlec{#1}}
\def\@titlec{\@latex@warning@no@line{No \noexpand\titlec given}}

\def\authorinfo#1{\gdef\@authorinfo{#1}}
\def\@authorinfo{\@latex@warning@no@line{No \noexpand\authorinfo given}}
\def\email#1{\gdef\@email{#1}}
\def\@email{\@latex@warning@no@line{No \noexpand\email given}}
\def\editorinfo#1{\gdef\@editorinfo{#1邮箱:\href{mailto: \@email}{\@email}。}}
\def\@editorinfo{\@latex@warning@no@line{No \noexpand\editorinfo given}}

\def\editor#1{\gdef\@editor{#1}}
\def\@editor{\@latex@warning@no@line{No \noexpand\editor given}}
\def\version#1{\gdef\@version{#1}}
\def\@version{\@latex@warning@no@line{No \noexpand\version given}}


%默认的title样式  继续添加的有mytitlea,mytitleb,mytitlec。。。
\newcommand{\mytitle}{
	\begin{titlepage}
	\begin{flushleft}	
	 \vspace*{\stretch{2}}
{\HUGE\bfseries \@titlea}\\[\stretch{1}]
{\Huge\bfseries \@titleb}\\[\stretch{1}]
{\LARGE\itshape \@titlec}\\[\stretch{2}]
{\Large \@author\footnote{\@authorinfo}}\quad\rule{0.8pt}{3ex}\quad
{\large \@editor\footnote{\@editorinfo}}\\[\stretch{2}]
\vfill
{\ttfamily 版本: \@version}
\end{flushleft}
\end{titlepage}
}


\endinput

\end{verbatim}

\chapter{源码和参考资料打包下载地址}




\addchtoc{插图目录}
\listoffigures 

\addchtoc{表格目录}
\listoftables  

\backmatter
\addchtoc{参考文献}
原谅我,我对参考文献的格式美化比较困惑。
\begin{thebibliography}{9}
\bibitem{lshort} 有名的《一份不太简短的\LaTeX2e 介绍》,原版作者:Tobias Oetiker,Hubert Partl,Irene Hyna等,版本:2001-08-09,中文翻译:CTeX用户小组,版本:2002-05,pdf下载链接:\href{http://www.ctan.org/tex-archive/info/lshort/chinese}{http://www.ctan.org/tex-archive/info/lshort/chinese}。
\bibitem{latex123}  《大家来学\LaTeX 》,原版作者:李果正,版本:2004-03-08,网站链接:\href{http://edt1023.sayya.org/tex/latex123/}{http://edt1023.sayya.org/tex/latex123/}。
\bibitem{wikibook-latex} 维基图书latex篇英文版,版本:2013-08,网站链接:\href{http://en.wikibooks.org/wiki/LaTeX}{http://en.wikibooks.org/wiki/LaTeX}。\emph{这本书甚至都没有cite它一下,因为不知道插在哪里,几乎到处都有对它的引用,谢谢所有wikibook的编写者。}
\bibitem{boxes} using boxes and glue in \TeX{} and \LaTeX ,原版作者:Nelson H. F. Beebe,版本:2009-03-28,pdf下载链接:链接地址不好复制,请google搜索书名之。
\bibitem{latex-companion}  The Latex Companion 
\end{thebibliography}


\end{common-format}  
\end{document}



