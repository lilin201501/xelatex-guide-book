% !Mode:: "TeX:UTF-8"%確保文檔utf-8編碼
%新加入的命令如下:addchtoc addsectoc reduline showendnotes hlabel
%新加入的环境如下:common-format  fig linefig 

\documentclass[11pt,oneside]{article}
\newlength{\textpt}
\setlength{\textpt}{11pt}
\newif\ifphone
\phonefalse

\usepackage{articleconfig}




\begin{document}
\title{定王台弘道文化观察报告}
\author{德山书生}
\date{}
\maketitle

\begin{abstract}
主要是对定王台最上面的弘道文化的观察,也对下面一些书店做了草草的观察,然后对目前整个中国书市进行了一些思考,其中的核心问题是:现在中国什么书卖得最好。
\end{abstract}


\begin{common-format}

\section{对人的观察}
\begin{itemize}
\item[孩子爷爷类] 这一类是爷爷带着孩子出来玩的,所以最后肯定会为孩子买一些书。
\item[单独专业男] 文学,计算机,考研,通俗文学等等都会涉猎。
\item[单独专业女] 一般是外语类。
\item[男女朋友类] 男的来的作用就是陪女朋友买书,一般女方会买专业类书,大多是外语类。
\item[学生类] 学生结伴出来会看教辅,学生单独出来一般小学的喜欢看科普、漫画等,高中喜欢玄幻。
\item[父母孩子类] 一般都是儿童吧,买的书主要是儿童绘画类,儿童文学类,字帖等。
\item[男男朋友类] 他们会多逛逛思想类书籍。\footnote{所以男人用思想交流,女人用感情交流。} 
\end{itemize}

\section{对书籍数量的观察}
书籍主要分类如下:外语类,考研类,计算机类,其他专业类,思想类(名出版社),儿童文学类,美术类,通俗文学思想类,少儿文学类,中国古典类,玄幻类,中国现当代文学类,漫画类,音乐电影类,作业类,试卷类,字帖类,考试教辅类,其他类等等。

其中我观察到一个非常显眼的现像就是外语类竟然占了75个柜子,如果加上考研英语之类那么将有79~80个柜子。这可能占了整个弘道文化书展示柜的五分之一。在外语类中,包括出国,托福之类,也包括其他小语种如日语德语等,也包括常规的英语语法书之类的,但引起我注意的是:

\textbf{那里有大量的国外名著英汉对照双语版。}

我现在假定书市里面放的最多的类型是卖的最好的类型,那么这就引起了我对这个现像的思考。


\section{对当代中国人阅读的思考}

\subsection{实用居上}
不管是专业书籍,还是各种考试还是美术等,实用居上这是中国人决定买书的主要动机。客观来讲,名著都很晦涩艰深,需要细细品读,并不适合读双语版。而中国人愿意选择名著的双语版甚至纯英文版的目的只是一条:学英语。那么学英语的目的为了什么?即使高考英语比重下降,中国人学英语的热情我看还将继续高涨下去,因为他们都有一个梦,一个向往国外文化甚至是移民的梦。你可以看作这是国外的文化入侵,但我看不如看作中国国人对国外美好生活哪怕是有点庸俗的向往吧。

\subsection{名气是选择的重要标准}
这个书市大部分书都可以贴上以下某个标签,所以才在这里躺著:名作家,名著,有名气的书,流行(当前有名的电视电影等的衍生物),名出版社等。

\subsection{中国当代文学的缺失}
我问了一个喜欢文学的人,我问:“名作家是你的选择标准吗?”他答道:“主要还是看那个作品是否打动了他的内心。”

我问了另外一个伙计:“你觉得今日中国还活着的作家谁最有名?”他说:“都不大喜欢。”我说:“哦,你喜欢已经死了的作家,你喜欢名著是吧。”他说:“是的。”

我总结到:“那边放着七十几柜英文相关的书籍,很多都是英汉对照的名著。这一方面反映了人们对翻译质量的不满和对西方文化的向往;另一方面也放映了中国当代文学的缺失。你看这边的中国当代文学,和我预期的相反,就那么三四柜。”

中国当代文学和思想的缺失,很大程度上是由于当今社会上甚嚣尘上的物质主义,而政府的言论不自由高压管制也有部分原因,当代中国教育精神的贫穷也是部分原因,但我觉得最大的原因是:人们生活在一个精神贫穷的时代,同时也生活在一个物质贫穷的年代,任何风花雪月任何高谈阔论都和这个时代格格不入,而人们又渴求着另外一个别样的天地,另外一个别样的生活。于是只好沉浸在国外的名著或者中国古代一两柜传统文化名著里获得片刻的陶醉,但一旦有人告诉他们,今天,就是今天,我们也可以有另外一种别样的美,那将是他们不敢想像的,不过幸运的是现在也很少有这样的作家。而多的是那些才资平凡者,舞弄文字者和泛点小资情调的古怪玩意儿,然后剩下的剩下的就只剩下搞怪戏谑哗众取宠了,剩下的就只有电视电影的一些私生仔了。

书,依附于电视的书,当然是比粗俗流行的电视更加的低下了,这样的书得是多么低下啊。而书,高贵的书,自当独自撑起一片天地,引人思考,引人流泪。怎能...

不说了。


\section{我的其他一些感想}
我的网上开淘宝卖电子书的计划继续进行,不过我不打算排版试卷,考试类了,也不打算屈从流行了。所以我就需要好好找份工作。 那么淘宝上卖什么了,我自己写的东西,我十分喜爱的书。 

然后我学到的一点是,英文版的排版工作也同步进行和同步发行,这样人们会更加喜爱一些。而且可能(如果版权没什么问题),还可以用于国外卖电子书赚国外的钱,国外这个市场很活跃和成熟,这样也许真能开出一条路子出来。

%这里空一行

\end{common-format}
\end{document}



