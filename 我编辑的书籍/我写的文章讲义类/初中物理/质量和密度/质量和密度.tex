% !Mode:: "TeX:UTF-8"%確保文檔utf-8編碼
%新加入的命令如下: reduline showendnotes 
%新加入的环境如下:solution solutionorbox solutionorlines solutionordottedlines

\documentclass[12pt,twoside]{exam}
\newlength{\textpt}
\setlength{\textpt}{12pt}

\usepackage{teachingplan}


%写上答案或者不写上答案%
%\printanswers  


%讲知识点然后测试,测试通过继续下一个知识点。不通过给予小提示,然后继续测试,如果通过那么通过,如果还是不会做,那么继续讲解知识点,并讲解这个题目,然后重新给出一个题目重新测试。

%如果有精力,后面再准备一套中考真题。

%\excludecomment{knowledge}
\includecomment{knowledge}
%\excludecomment{Aquestions}
\includecomment{Aquestions}

\begin{document}
\begin{coverpages}
\title{压强}
\author{德山书生}
\maketitle
\tableofcontents
\end{coverpages}
\begin{knowledge}
\begin{flushright}
\begin{notecard}{16em}
\ttfamily
歪着脖子的智慧才是你自己的智慧。
\end{notecard}
\end{flushright}
\section{质量}
物体所含\answer[80pt]{物质的多少}叫质量。质量通常用字母$m$表示。质量和物体的形状,位置,状态等都无关,就好比一个实实在在的东西放在那里,除非这个物体里面有东西(带有质量)跑出去或者跑进来,否则这个物体质量是不变的。(物体的内部质量在某些情况下会发生变化,这些情况目前你们初中还不需要考虑,我知道你们都看过这个公式了:$E=mc^2$。)


\textbf{(2012广东,2,3分)}一瓶矿泉水放入冰箱结冰后,下列物理量\dotuline{不发生}改变的是(\answerline*[A])

\begin{oneparchoices}
\choice 质量
\choice 温度
\choice 内能
\choice 密度
\end{oneparchoices}


\textbf{(2013广西南宁,16,2分)}利用橡皮擦将纸上的字擦掉之后,橡皮擦的质量\answerline*[变小],密度\answerline*[不变](以上两空选填“变小”、“变大”、“不变”)。


\subsection{质量的单位}
质量的国际制基本单位是\answerline*[千克](kg),其他常用单位:吨(t)、克(g)、毫克(mg)。

其中:1t=1000kg,1kg=1000g,1g=1000mg。


\textbf{(2011北京,1,2分)}在国际单位制中,质量的单位是(\answerline*[A])

\begin{oneparchoices}
\choice 千克
\choice 牛顿
\choice 帕斯卡
\choice 焦耳
\end{oneparchoices}

\subsection{质量的生活经验}
在我们的日常生活中,人们常常用斤来表示物体的质量。比如我们说一个成年人约为一百多斤重,这个斤和国际单位的换算关系是:1斤=0.5公斤=0.5kg。也就是一个成年人约五十多千克的样子。

需要说明的是“斤”这个概念主要在东亚文化圈中使用,而且各个地方一斤和千克的换算公式都不是一样的,这里说的1斤等于0.5公斤只是就中国大陆而言(其他地方香港台湾日本等一斤都大约为600g的样子。也就是你如果跑到香港买一斤苹果是要比在内地买一斤苹果要重些的。)

此外我们生活中还会接触“两”这个概念,一斤等于十六两(记忆成语“半斤八两”,也就是1两=31.25g)。一个苹果大约三四两的样子,一斤大约能买四个苹果,如果人家写着苹果一斤8块钱,那么平均2块钱一个苹果吧。


\textbf{(2013江苏苏州,4,2分)}根据你对生活中物理量的认识,下列数据中最接近生活实际的是(\answerline*[A])
\begin{choices}
\choice 人体的密度约为\SI{1.0d3}{kg/m^3}
\choice 中学生的课桌高约为1.5m
\choice 一个中学生的质量约为5kg
\choice 人的正常体温为38\si{\degreeCelsius}
\end{choices}


\section{天平}
在实验室里常用\answerline*[天平]测物体的质量。生活中的台秤、杆秤等都和天平的原理类似。(不过现在不管是生活中还是实验室里电子秤用的越来越多了。)

\begin{linefig}[0.9]{天平}
\end{linefig}

\subsection{天平的原理}
如上图所示,天平的原理有点类似于跷跷板,两边的情况都是对称的,当两边两个盘上放的物体质量都是一样大的时候,这个天平就平衡了。

\subsection{游码的理解}
天平的结构两边是对称的,唯一的不同就是游码。我们看到游码拨到最左边的时候实际上是给左边加重量,而游码拨到中间实际上并没有给天平施加力。人们为了理解简单,游码给设计成了砝码的补充。也就是拨到最左边的时候天平才是平衡的。

 所以天平不放东西游码归零之后,人们期望的是天平是平衡的,也就是那个指针指在最中间。在测量物体之前一定要检查这点,如果发现天平没有平衡,那么需要调整横梁两边的螺母。

横梁两边两个螺母,就相当于跷跷板上额外的两个东西,如果指针偏右,(这说明右边重了)如上图所示,那么右边的螺母就要向左移动,或者左边的螺母向左移动。\leftnote{可以讲下杠杆的作用原理}

也正是因为游码左边归零,所以左边的盘子必须是放待测物体的,而右边的盘子是放砝码的。经过上面的空载调节天平平衡之后,然后我们才有:\\
\textbf{被测物体的质量=砝码的质量+游码所对的刻度值}

\subsubsection{游码未归零就平衡天平了}
这种情况不是不可能出现,如果你游码没有归零,就平衡天平了。比如说你放好待测物体,加好砝码,等你去调游码的时候才发现游码没有归零,这个时候是有补救措施的。关键的一点是要理解现在游码的位置就是游码的0刻度位置,然后向右比如说2g,那么$m_\textrm{游码}$读作+2g,如果向左移动了1g,那么$m_\textrm{游码}$读作-1g。

\textbf{问题:}某同学在使用天平时,游码固定在了4g处,将天平平衡。在测量物体质量时,右盘加上18g砝码,再把游码移到2g的位置,天平横梁再次平衡。则所称物体质量为(\answerline*[B])

\begin{oneparchoices}
\choice 12g
\choice 16g
\choice 20g
\choice 22g
\end{oneparchoices}

\pagebreak
如果待测物放右边,砝码放左边,然后其他一切正常,怎么补救?这种情况我觉得你还是重新做实验吧,就这种问题我觉得也不大可能出题,但这里还是简单说一下:游码为负值,显示2g那么$m_\textrm{游码}$就是-2g。

当然可能有学生问,如果游码又没归零,天平又没平衡,还有补救措施吗?我想是没有了吧。。


\subsection{天平的使用}
除了前面谈到的天平的正常使用步骤,这里再提及一些天平使用过程中需要注意的事项:
\begin{enumerate}
\item 天平要放在水平工作台上
\item 先大砝码再小砝码,然后是游码调节(这个是一个测量技巧,方便你快速秤量)
\item 被测物体的质量不能超过天平所能称的最大质量(超过了你把所有的砝码都加上天平也不会平衡的)
\item 用镊子拿砝码,不要用手接触砝码。不能把砝码弄湿弄脏(这样砝码就不准了)
\item 潮湿的物体和化学药品不能直接放在天平的盘中。(这样天平就损坏不准了)
\end{enumerate}

最后天平使用完了记得砝码回盒,游码归零,取下物体,台面收拾干净。

天平的使用经常以实验探究的大题形式出现,需要学生真正理解弄懂:

\textbf{(2012山东潍坊,20,6分)}小强同学用托盘天平测量一块橡皮的质量,调节天平平衡时,将游码调到0刻度后,发现指针停在分度盘的右侧,如下图所示。要使天平平衡,应使右端的平衡螺母向\answerline*[左]移动。天平平衡后,在左盘放橡皮,右盘添加砝码,向右移动游码后,指针停在分度盘中央,所加砝码数值和游码的位置如下图所示,则橡皮的质量是\answerline*[33.2]g。如果调节天平平衡时,忘记将游码调到0刻度,则他测量的橡皮质量比真实值\answerline*[大]。(填“大”或“小”)
\begin{linefig}[0.9]{2012山东潍坊20}
\end{linefig}

\textbf{(2013广西南宁,25,6分)}在使用托盘天平测量物体质量的实验中\\
(1)将托盘天平\answerline*[水平]台面上,将游码移至零刻度处,发现指针位置下图所示,此时应向\answerline*[右](选填“左”或“右”)旋转平衡螺母,直到指针静止时指在分度盘的中线处。\\
(2)在测量物体质量时,应将物体放在\answerline*[左](选填“左”或“右”)盘,往另一个盘增减砝码时要使用\answerline*[镊子]。\\
(3)多次增减砝码,当加入5g的砝码后,指针静止时,指在分度盘中线附近,此时应移动\answerline*[游码]使横梁恢复平衡。若盘中砝码和游码位置下图所示,则被测物体的质量为\answerline*[78.4]g。
\begin{linefig}[0.9]{2013广西南宁25}
\end{linefig}


\section{质量的测量}
质量的测量一般是用天平或电子秤直接秤量。

\subsection{累积法}
对于某些小的物体可以用累积法——比如测量一滴水的质量,测量一张邮票的质量等。

累积法类似的还可以反过来用,比如假设你有很多五毛钱的硬币,你可以抓起一把五毛钱的硬币,称有多重,然后看看有多少个,然后再将那一大堆硬币的重量称一下,这样那一大堆硬币总共有多少钱你就可以很快的计算出来了。

如果情况很复杂里面有五毛钱和一块钱的硬币混在一起,你怎么办?同样采用上面的方法,真正的技巧在于如何让五毛钱的硬币和一块钱的硬币混合趋向于均匀,然后你再抓一把,数数多少钱,再抓一把数数多少钱,多做几次实验,大致统计一个平均数出来,然后总的质量称一称即可。当然混合不是均匀的也没关系,但这样对如何取样,以及取样的次数还有统计的方法都有一定的要求了,这一块内容就比较高深了,我也不大清楚了。

\subsection{间接秤量}
在实际测量过程中,可能会遇到某些不方便直接秤量的情况,这个时候可以采用间接秤量的方法,比如先称烧杯的质量,再称烧杯和待测物体的质量。

\subsection{根据密度}
某一些形状规则的固体体积是可计算的,然后我们根据密度定义有:$m=\rho V$,这样可以算出该物体的质量。当然还有其他一些得到待测物的体积的方法,在这里道理都是一样的。

\subsection{根据力}
关于力的定量计算高中才涉及,这里简单提一下,力作用于质量不同的物体上产生的效果是不同的,可以根据各个力的公式,比如万有引力公式来计算星体的质量。



\pagebreak
\section{量筒}
\vspace*{20pt}
\noindent
\begin{minipage}{\textwidth}
\begin{minipage}[c][6cm][c]{0.8\textwidth}
一般液体的体积可以直接用量筒(或量杯等)测出。关于量筒(量杯)需要说的有:
\begin{enumerate}
\item  读数的时候视线要和量筒的凹液面底部相平,如果是凸液面比如说(汞)则对准其液面的顶部。
\item 有些液体比如比较粘稠的并不适宜用量筒,有些腐蚀性强的挥发性大的也不适宜用量筒,通常是已知密度的情况下秤量质量来控制物质的量的。
\item 量筒有很多种,200ml的,100ml的,50ml的等等,根据实际需要选择从待测量值最接近最大量程的那个。
\item 注意量筒的最小分度值一格表示的数值具体是多少,别读错了。
\end{enumerate}
 
\end{minipage}\hfill
\begin{minipage}[c][6cm][c]{0.2\textwidth}
\begin{linefig}[0.8]{量筒}
\end{linefig}
\end{minipage} 
\end{minipage} 

\vspace*{20pt}


\section{体积}
\subsection{体积的生活经验}
体积生活中常用的单位有升和毫升,其中:1L=\num{d3}mL,1mL=1\si{cm^3}。

水的密度约1\si{g/cm^3},也就是1毫升水重1克,1升水重一千克。一般一瓶矿泉水约二三百毫升,家用水桶有大的有小的一般十几升的样子,饮用桶装水标准是5加仑——18.9升,大约19公斤,38斤的样子。

在化学上还提到一个概念,那就是平均一滴水大约0.05mL。这个概念在实验时遇到那些用量小又不方便测量体积的液体时可以使用。当然你可以设计一个类比实验,让你的估计更加精确,就是用累计法滴很多一滴一滴的待测液体,然后一起秤量看看平均一滴待测液体大约为多重,如果你能控制好滴液体的手势(比如用机器手)那么这种方法也是可以接受的。



\section{密度}
一种物质在一定的条件下质量与体积的比值是一定的,物质不同,其比值一般不同,这反映了该物质的一种特性。物理学中用密度表示这种特性。单位体积的某种物质的质量叫做这种物质的密度。

\subsection{密度的计算公式和单位}
密度的公式:\answerline*[$\rho =m/V$],密度的国际单位是\answerline*[\si{kg/m^3}]。此外密度的常用单位还有\si{g/cm^3},\si{g/cm^3}单位大,比如说水的密度是1\si{g/cm^3},就等于\answer[80pt]{\num{1.0d3}}\linebreak \si{kg/m^3}。

\textbf{(2013湖南长沙,30,4分)}浏阳腐乳以其口感细腻、味道纯正而远近闻名,深受广大消费者喜爱。现测得一块腐乳质量约为10g,体积约8\si{cm^3},则其密度为\answerline*[1.25]\si{g/cm^3}。若吃掉一半,剩余部分的密度将\answerline*[不变]。


\subsection{物质混合问题}
两个已知密度的物质混合成一个新的物质,混合后的密度也知道了,题目一般假定混合前后总体积不变,然后要求原两个物质的混合比例的问题。这种类型题目如果是定性讨论就是计算{\large $\frac{\rho_\textrm{A}+\rho_\textrm{B}}{2}$}——两个物质密度的中间值,如果混合后的密度和这个中间值相比偏大那么密度大的物质加的多些,如果混合后的密度和这个中间值相比偏小那么密度小的物质加的多些。

如果需要具体加入的质量比值,那么就利用:
\begin{align*}
V_\textrm{A}+V_\textrm{B}=V_\textrm{总} \\
\frac{m_\textrm{A}}{\rho_\textrm{A}} + \frac{m_\textrm{B}}{\rho_\textrm{B}}=\frac{(m_\textrm{A}+m_\textrm{B})}{\rho_\textrm{总}}
\end{align*}
这样计算出$m_\textrm{A}$和$m_\textrm{B}$的比值。

如果需要计算他们的体积比,就利用:
\begin{align*}
m_\textrm{A}+m_\textrm{B}=m_\textrm{总} \\
\rho_\textrm{A} \times V_\textrm{A} +\rho_\textrm{B} \times V_\textrm{B}=\rho_\textrm{总} \times (V_\textrm{A}+V_\textrm{B})
\end{align*}
这样计算出$V_\textrm{A}$和$V_\textrm{B}$的比值。

\textbf{(2012新疆乌鲁木齐,7,3分)}用一块金和一块银做成一个合金首饰,测得首饰的密度是15.0\si{g/cm^3}。已知金的密度是19.3\si{g/cm^3},银的密度是10.5\si{g/cm^3}。下列说法正确的是(\answerline*[A])
\begin{choices}
\choice 金块的体积比银块的体积稍大一点
\choice 金块的体积比银块的体积大很多
\choice 银块的体积比金块的体积稍大一点
\choice 银块的体积比金块的体积大很多
\end{choices}

\textbf{(2011内蒙古包头,9,2分)[难]}甲物质的密度为5\si{g/cm^3},乙物质的密度为2\si{g/cm^3},各取一定质量混合后密度为3\si{g/cm^3}。假设混合前后总体积保持不变,则所取甲、乙两种物质的质量之比是(\answerline*[C])

\begin{oneparchoices}
\choice 5:2
\choice 2:5
\choice 5:4
\choice 4:5
\end{oneparchoices}



\section{密度的测量}
密度的测量的原理一般是显式的测出待测物的质量(天平)和体积(计算或排水法等),然后按照公式{\large $\rho=\frac{m}{V}$}计算得到。



\subsection{等体积法}
不过其中的体积可以不显式求出来,在实验中可以通过另外一种物质来将待测体积传递出去,和排水法的不同在于,排水法最后还是求出了那个体积,而这里讨论的等体积法是利用液体密度已知(一般指水)然后质量间接称出的方法来推测原来待测物质的体积。我们以下面这个例题来具体说明之。

\textbf{(2012江苏连云港21,6分)}石英粉是重要的化工原料,小明爸爸在石英粉厂工作,他想知道石英粉的密度,可是身边只有天平。他求助于正在九年级就读的儿子。聪明的小明利用天平(含砝码)、一个玻璃杯、足量的水,就完成了测量该石英粉密度的实验。($\rho_\textrm{水}$已知)\\
下面是小明同学设计的实验步骤,请你帮他补充完整。\\
(1)用天平测出空玻璃杯的质量$m_0$;\\
(2)给玻璃杯中装满石英粉,测出\answer[180pt]{玻璃杯和石英粉的总质量$m_1$};\\
(3)\begin{solutionorlines}[5em]
将石英粉倒出,给玻璃杯中装满水(可以多洗几次将石英粉全洗出,外表干燥即可),测出玻璃杯和水的总质量$m_2$
\end{solutionorlines}
(4)用已知量和测量量对应的字母写出石英粉密度的表达式$\rho_\textrm{粉}$=\answer[80pt]{$\frac{m_1-m_0}{m_2-m_0}\times \rho_\textrm{水}$}

\leftnote{用下面这道题检验}

\textbf{(2013福建福州,28,8分)}在“测量酸奶密度”的实验中,\\
(1)小明的实验方案:用天平和量筒测密度。\\
① 他用已调节好的天平测得空烧杯的质量$m_0$为37.4g;接着把酸奶倒入烧杯中,测得烧杯和酸奶的总质量$m_1$,下图左所示,则$m_1$=\answerline*[81.4]g;然后把烧杯中的酸奶倒入量筒中,如下图右所示,则$V_\textrm{奶}$=\answerline*[40]\si{cm^3};则酸奶的密度$\rho$=\answerline*[1.1]\si{g/cm^3}。\\[20pt]
\noindent
\begin{minipage}{\textwidth}
\begin{minipage}[c][6cm][c]{0.82\textwidth}
\begin{linefig}{2013福建福州28-1}
\end{linefig}
\end{minipage}\hfill
\begin{minipage}[c][6cm][c]{0.18\textwidth}
\begin{linefig}{2013福建福州28-2}
\end{linefig}
\end{minipage} 
\end{minipage} 

\vspace{20pt}
② 在交流讨论中,小雨同学认为小明测得的酸奶密度值偏大,其原因是\\
\answer[\textwidth]{酸奶倒入量筒中会有部分残留在烧杯中,使测量的体积偏小。}

(2)小雨的实验方案:巧妙地利用天平、小玻璃瓶(有盖)和水测酸奶密度。\\
请你简要写出小雨的实验过程和酸奶密度的计算表达式(用测量的物理量符号表示)。
\begin{solutionorbox}[6em]
1.用天平测出小玻璃瓶(含盖)的质量$m_0$\\
2.在瓶内装满酸奶,盖上盖子,测出瓶和酸奶的总质量$m_1$\\
3.倒酸奶,用水洗酸奶,装满水,擦干净瓶表面,盖上盖子,测得瓶和水的总质量$m_2$\\
酸奶密度的计算公式为:{\large $\rho_\textrm{奶}=\frac{m_1-m_0}{m_2-m_0} \times \rho_\textrm{水}$}
\end{solutionorbox}

\section{密度的应用}
\subsection{热气球}
气体受热体积碰撞,密度变小,所以热气球上升,直到周围的空气和内部的热空气密度差不多为止。

\textbf{(2013山东烟台,5,2分)}鸡尾酒是由几种不同的酒调配而成的。经过调配后,不同颜色的酒界面分明,这是由于不同颜色的酒具有不同的(\answerline*[D])

\begin{oneparchoices}
\choice 重力
\choice 质量
\choice 体积
\choice 密度
\end{oneparchoices}

一般物体都是热胀冷缩,我们知道水在4℃的时候密度最大,冬天湖最低下的一层是温度为4\si{\degreeCelsius}。然后冬天湖面漂浮着冰块,你能告诉我冰的密度和水的密度谁大?

\subsection{物质鉴别}
这个在物质混合问题那一小节已讨论过,这里略过。

\section{总结}
\subsection{排水法测量物体体积}
\leftnote{这些情况具体体积怎么算简单板书一下即可。}
对于那些可以沉到水下的不溶于水的小固体可以放入量筒内直接排水测体积,如果固体形状比较大,可以用溢水法——也就是水装满大烧杯,固体放入看看排除了多少水。

对于那些不能沉到水下的固体可以采用助沉法,如和一个可以沉入水中的物体绑在一起沉下去,或者用一个细小的针帮助固体浸入水中。还有另外的助沉法,请看下面一个题目。

排水法测量物体体积这个物理情景常常和压强浮力密度等知识综合起来作为一个大题出现,这里给出一个大题作为本章的结束检验。


\subsection{有效数字保留问题}
顺便提一下作答中有效数字的保留问题:之前谈过测量值的估读问题,那么根据测量值,得到的数字有一个有效位数\leftnote{有效数字的位数?},这个有效位数的构成就是之前测量值包括估读值也就是不准确位数的那位。现在他们进入计算之后,在计算过程中,原则上是越精确越好(为了简化计算有一些修约手段这里不做讨论)。现在的问题是计算完之后如何保留的问题。如果题目有要求,那么按照题目的意思来。没有说明的是按照进入运算的测量中中最小有效数字位数来保留。

比如你测量一个长度是0.23m,有效数字的位数是2位,(那么这个长度的估读值是?),还有一个时间值4.56s,然后把它们代入计算公式经过一番运算之后,假设得到一个值12.356...,这个时候你要四舍五入怎么保留?答:保留之前的最小的有效数字位,也就是两位,也就是12。

\textbf{(2013重庆B,16,8分)[难]}在一次物理兴趣小组的活动中,某小组同学准备用弹簧测力计、烧杯、水、吸盘、滑轮、细线来测量木块(不吸水)的密度。

(1)在一定的范围内拉伸弹簧时,弹簧受到的拉力越大,弹簧的伸长量就越\answerline*[大]。使用弹簧测力计前,应先检查指针是否在\answerline*[零刻度线]的位置,若不在此位置,应进行调整后再使用。

(2)如下图甲所示,用弹簧测力计测出木块在空气中的重力为\answerline*[0.6]N。

(3)将滑轮的轴固定在吸盘的挂钩上,挤出吸盘内部的空气,吸盘在\answerline*[大气压]的作用下被紧紧压在烧杯底部,如下图乙所示。在烧杯中倒入适量的水,将木块放入水中后,用弹簧测力计将木块全部拉入水中,如下图丙所示,此时弹簧测力计示数为0.4N。

(4)如果不计摩擦和绳重,图丙所示的木块受到的浮力为\answerline*[1]N,木块的密度为\answerline*[\num{0.6d3}] \si{kg/m^3}。

(5)如果将烧杯中的水换成另一种液体,用弹簧测力计将该木块全部拉入该液体中时,弹簧测力计示数为0.2N,该液体的密度为\answerline*[\num{0.8d3}] \si{kg/m^3}。

(6)如果实验中先用弹簧测力计将木块全部拉入水中,然后取出木块直接测量木块的重力,从理论上分析,按这样的实验顺序测得的木块密度值\answerline*[偏大](选填“偏大”“偏小”或“不变”)。

\begin{linefig}[0.8]{2013重庆B16}
\end{linefig}


\end{knowledge}





\begin{Aquestions}
\newpage
\section{题库A}


\end{Aquestions}




\end{document}





