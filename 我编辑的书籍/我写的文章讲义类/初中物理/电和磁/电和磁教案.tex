% !Mode:: "TeX:UTF-8"%確保文檔utf-8編碼
%新加入的命令如下: reduline showendnotes 
%新加入的环境如下:solution solutionorbox solutionorlines solutionordottedlines

\documentclass[12pt]{exam}
\newlength{\textpt}
\setlength{\textpt}{12pt}

\usepackage{teachingplan}
\usepackage{wallpaper}
\usepackage{comment}
\usetikzlibrary{positioning}

%输出方案 
%1.自己用答案,知识点等都有  ——老师版
%2.给学生用基础差的可以考虑加上知识点,但不写答案——基础差版
%3.练习题加提高题——学生版
%4.练习题加提高题加综合题——学霸版

%写上答案或者不写上答案%1  
\printanswers  

%提高题%1  4
%\includecomment{improveexercises}
\excludecomment{improveexercises}
%综合题%1  4  5
%\includecomment{advanceexercises}
\excludecomment{advanceexercises}


\CenterWallPaper{1}{教案模板-2.pdf}
\newcommand{\shortanswerline}{\rule[-2pt]{50pt}{0.4pt}}
\renewcommand{\solutiontitle}{\noindent\textbf{}}

\newcommand{\keti}{热和能}
\newcommand{\zhongdian}{1.分子热运动 2.内能 3.比热容\\  4.热机 5.能量的转化和守恒}
\renewcommand{\section}[1]{{\large\sffamily  #1} \par}
\renewcommand{\subsection}[1]{{\normalsize\sffamily  #1} \par}
\newcommand{\leftnote}[1]{\marginpar{\ttfamily\fontsize{10}{10}\selectfont #1}}



\newcommand{\answer}[2][50pt]{{\setlength{\answerlinelength}{
#1} \answerline*[#2]}}

\begin{document}
\ThisCenterWallPaper{1}{教案模板-1.pdf}
\vspace*{80pt}
\keti \par
\zhongdian \par
\section{分子热运动}
古代中国人发明了一个装置,在航海的时候用来指示方向,这个装置叫罗盘,罗盘里面的核心部件是一个名叫司南的东西,司南就是现在说的指南针。

指南针是用磁铁制作而成的。比如上面说的司南就是古代中国人找到的天然磁石打磨而成的。

磁铁能够吸引钢铁一类的物质。其中磁铁的两端吸引力最强,叫做\answerline*[磁极]。指南针都有两个磁极,一个叫\answerline*[南极](S),一个叫\answerline*[北极](N)。那么北极指的是北面吗?

磁极间的相互作用:同名磁极互相\answerline*[排斥],异名磁极互相\answerline*[吸引]。

若两个物体互相吸引,则有两种可能:①一个物体有磁性,另一个物体无磁性,但含有钢铁、钴、镍一类物质;②两个物体都有磁性,且异名磁极相对。

一些物体在磁体或电流的作用下会获得磁性,这种现象叫做\answerline*[磁化]。钢和软铁都能被磁化:软铁被磁化后,磁性很容易消失;钢被磁化后,磁性能长期保持。所以钢针被磁化后可以做指南针。


\section{磁场}
在磁铁的周围,有一种物质,正是这种物质让它旁边的磁针发生了偏转,我们把这种物质叫做磁场。

我们在一块磁铁的周围放上很细小的小磁针,这些小磁针在磁场的作用下会排列成一条一条的线,我们把这些曲线叫做\answerline*[磁感线]。这些曲线还有一个方向,就好像磁感线从什么地方流出来一样,我们定义磁感线的方向是小磁针北极所指的方向。这样磁感线就是从磁铁的北极流了出来,然后流回到磁铁的南极。

典型的磁感线  图

磁感线的一些认识:
①磁感线是在磁场中的一些假想曲线,本身并不存在,作图时用虚线表示;
②在磁体外部,磁感线都是从磁体的N极出发,回到S极。在磁体内部正好相反。
③磁感线的疏密可以反应磁场的强弱,磁性越强的地方,磁感线越密,磁性越弱的地方,磁感线越稀;
④磁感线在空间内不可能相交。

为什么指南针N极总是北方?因为地球本身就是一个大磁体,我们把地球周围空间存在的磁场叫做地磁场。那么地磁场的磁极怎么放置的?

地理的两极和地磁的两极并不重合,磁针所指的南北方向与地理的南北极方向稍有偏离(地磁偏角),世界上最早记述这一现象的人是我国宋代的学者沈括。(《梦溪笔谈》)


\section{电生磁}
\subsection{奥斯特实验}
丹麦物理学家奥斯特发现电流通过导线,它旁边的磁针发生了偏转。这个实验称之为奥斯特实验,如下所示:

奥斯特实验 图

对比甲图、乙图,可以说明:通电导线的周围有磁场;对比甲图、丙图,可以说明:磁场的方向跟电流的方向有关。

通电螺线管外部的磁场方向和条形磁体的磁场一样。通电螺线管的两端相当于条形磁体的两个极,通电螺线管两端的极性跟螺线管中电流的方向有关。

安培定则:用右手握螺线管,让四指指向螺线管中电流的方向,则大拇指所指的那端就是螺线管的N极。


\section{电磁铁}
电磁铁:
定义:插有铁芯的通电螺线管。
特点:①电磁铁的磁性有无可由通断电控制,通电有磁性,断电无磁性;
②电磁铁磁极极性可由电流方向控制;
③影响电磁铁磁性强弱的因素:电流大小、线圈匝数、:电磁铁的电流越大,它的磁性越强;电流一定时,外形相同的电磁铁,线圈匝数越多,它的磁性越强。


\section{电磁继电器}
电磁继电器是利用低电压、弱电流电路的通断,来间接地控制高电压、强电流电路的装置。
电磁继电器是利用电磁铁来控制工作电路的一种开关。
电磁继电器的结构:电磁继电器由电磁铁、衔铁、弹簧、动触点和静触点组成,其工作电路由低压控制电路和高压工作电路组成。


\section{扬声器}
扬声器是将电信号转化成声信号的装置,它由固定的永久磁体、线圈和锥形纸盆构成。
扬声器的工作原理:声音信息以时刻变化的电流的形式进入线圈,使得在一个瞬间和下一个瞬间产生不同方向的磁场,线圈就不断地来回振动,纸盘也就振动起来,便发出了声音。


\section{电动机}
通电导体在磁场里受力的方向,跟电流方向和磁感线方向有关。(当电流方向或磁感线方向两者中的一个发生改变时,力的方向也随之改变;当电流方向和磁感线方向两者同时都发生改变时,力的方向不变。)

电动机由两部分组成:能够转动的部分叫转子;固定不动的部分叫定子。

电动机是根据通电线圈在磁场中因受力而发生转动的原理制成的,是将电能转化为机械能的装置。


当直流电动机的线圈转动到平衡位置时,线圈就不再转动,只有改变线圈中的电流方向,线圈才能继续转动下去。这一功能是由换向器实现的。换向器是由一对半圆形铁片构成的,它通过与电刷的接触,在平衡位置时改变电流的方向。换向器的作用是每当线圈刚转过平衡位置时,能自动改变线圈中电流的方向,使线圈连续转动。

提高电动机转速的方法:增加线圈匝数、增加磁体磁性、增大电流。


\section{磁生电}
英国物理学家法拉第首先发现了利用磁场产生电流的条件和规律。当闭合电路的一部分在磁场中做切割磁感线运动时,电路中就会产生电流。这个现象叫电磁感应现象,产生的电流叫感应电流。

\subsection{发电机}
发电机是根据电磁感应现象制成的,是将机械能转化为电能的装置。


没有使用换向器的发电机,产生的电流,它的方向会周期性改变方向,这种电流叫交变电流,简称交流电。(AC)
在交变电流中,电流在每秒内周期性变化的次数叫做频率,频率的单位是赫兹,简称赫,符号为Hz。
我国供生产和生活用的交流电,电压是220V,频率是50Hz,周期是0.02s,即1s内有50个周期,交流电的方向每周期改变2次,所以50Hz的交流电电流方向1s内改变100次。

使用了换向器的发电机,产生的电流,它的方向不变,这种电流叫直流电。(实质上和直流电动机的构造完全一样,只是直流发电机是磁生电,而直流电动机是电生磁)。从电池得到的电流的方向不变,通常叫做直流电。(DC)

实际生活中的大型发电机由于电压很高,电流很强,一般都采用线圈不动,磁极旋转的方式来发电,而且磁场是用电磁铁代替的。发电机发电的过程,实际上就是其它形式的能量转化为电能的过程。



\section{练习题}














\begin{improveexercises}
\section{提高题}

\end{improveexercises}

\begin{advanceexercises}
\section{综合题}

\end{advanceexercises}

\ThisCenterWallPaper{1}{教案模板-3.pdf}

\end{document}



