% !Mode:: "TeX:UTF-8"%確保文檔utf-8編碼
\documentclass[border=2pt]{standalone}
\usepackage{tikz}
\usepackage{pgfplots}
\usetikzlibrary{intersections,calc,positioning}
\usetikzlibrary{shapes.geometric}%菱形
\usetikzlibrary{patterns}
\pgfplotsset{compat=newest}
\usepackage{fontspec}
\usepackage{xeCJK}
\setCJKmainfont[BoldFont=Adobe 黑体 Std,ItalicFont=Adobe 楷体 Std]
    {Adobe 宋体 Std}%影響rmfamily字體
\setCJKsansfont{Adobe 黑体 Std}%影響sffamily字體
\setCJKmonofont{Adobe 楷体 Std}%影響ttfamily字體


\begin{document}
\begin{tikzpicture}
%力的数值
\def\forcenum{0.6}
\def\forcepoint{-10*\forcenum/4}
\pgfmathsetmacro{\coilsegmentlen}{1+\forcenum/5}
%刻度
\begin{scope}[scale=0.5]
%主线轴
\def\starty{0}
\def\endy{-25}
\pgfmathsetmacro{\stepy}{\starty-1}
\pgfmathsetmacro{\stepyfive}{\starty-5}
\pgfmathsetmacro{\stepyten}{\starty-10}

\coordinate (startpoint) at (1.5,\starty) ;
\coordinate (endpoint) at (1.5,\endy) ;
\draw [](startpoint) -- (endpoint);


%细小刻度
\foreach \x in {\starty,\stepy,...,\endy}
  \draw [](1.5,\x) -- (2.5,\x) ;   

%5分之一刻度
%\foreach \y in {\starty,\stepyfive,...,\endy}
%  \draw [] (1.6,\y) -- (0,\y);


%10分之一刻度
\foreach \i in {\starty,\stepyfive,...,\endy}
  \draw [] (3.1,\i) -- (1.5,\i)%2.2
  node[right=1cm] {\Huge \pgfmathprint{int(-\i/5)}};
\end{scope}




%翻转
\begin{scope}[scale=0.5,xscale=-1]

%主线轴
\def\starty{0}
\def\endy{-25}
\pgfmathsetmacro{\stepy}{\starty-1}
\pgfmathsetmacro{\stepyfive}{\starty-5}
\pgfmathsetmacro{\stepyten}{\starty-10}

\coordinate (startpoint) at (1.5,\starty) ;
\coordinate (endpoint) at (1.5,\endy) ;
\draw [](startpoint) -- (endpoint);


%细小刻度
\foreach \x in {\starty,\stepy,...,\endy}
  \draw [](1.5,\x) -- (2.5,\x) ;   

%5分之一刻度
%\foreach \y in {\starty,\stepyfive,...,\endy}
%  \draw [] (1.6,\y) -- (0,\y);


%10分之一刻度
\foreach \i in {\starty,\stepyfive,...,\endy}
  \draw [] (3.1,\i) -- (1.5,\i)%2.2
  node[left=1cm] {\Huge \pgfmathprint{int(-\i/5)}};
\end{scope}



%额外的修改
%外边框
\draw (-3,-14) rectangle (3,4);
\fill (0,4) circle (0.2);
\fill[gray!50] (-3,4) rectangle (3,4.2);
%unit
\node [above right=1cm] {\Huge N};
%上面的弹簧
\draw[decoration={aspect=0.3, segment length=\coilsegmentlen mm, amplitude=2mm,coil},decorate] (0,4) -- (0,\forcepoint); 

%菱形指针和下面的钩子
\draw (0,\forcepoint) -- +(0,-15)  arc (90:-180:0.6);
\node [draw,scale=2,diamond,fill=orange!50,aspect=2] at (0,\forcepoint) {}; 
%立方体
\begin{scope}[yshift=-22cm,xshift=-1.2cm]
\fill[gray!30] (0,0) rectangle (2,2.4);
\fill[gray!50] (2,0) -- (2.8,0.4) -- (2.8,2.8) -- (2,2.4);
\fill[gray!10] (0,2.4)-- (1,2.8) -- (2.8,2.8) -- (2,2.4);
\draw[ultra thick] (1.2,2.4) -- (1.2,4.3);
\end{scope}
\node at (0,-25) {\Huge 甲};
%烧杯和定滑轮
\begin{scope}[xshift=10cm,yshift=-20cm]
\draw  (0,3)  circle (1) ;
\draw[ultra thick] (-1,0.4) -- (1,0.4);
\draw (-0.2,3) rectangle (0.2,0.4);
\draw (-5,0.4) -- (5,0.4) --(5,12) -- (-5.4,12) -- (-5,11.5) --cycle ; 
\node at (0,-5) {\Huge 乙};
\end{scope}

%烧杯,定滑轮和水
\begin{scope}[xshift=25cm]
%力的数值
\def\forcenum{0.4}
\def\forcepoint{-10*\forcenum/4}
\pgfmathsetmacro{\coilsegmentlen}{1+\forcenum/5}
%刻度
\begin{scope}[scale=0.5]
%主线轴
\def\starty{0}
\def\endy{-25}
\pgfmathsetmacro{\stepy}{\starty-1}
\pgfmathsetmacro{\stepyfive}{\starty-5}
\pgfmathsetmacro{\stepyten}{\starty-10}

\coordinate (startpoint) at (1.5,\starty) ;
\coordinate (endpoint) at (1.5,\endy) ;
\draw [](startpoint) -- (endpoint);


%细小刻度
\foreach \x in {\starty,\stepy,...,\endy}
  \draw [](1.5,\x) -- (2.5,\x) ;   

%5分之一刻度
%\foreach \y in {\starty,\stepyfive,...,\endy}
%  \draw [] (1.6,\y) -- (0,\y);


%10分之一刻度
\foreach \i in {\starty,\stepyfive,...,\endy}
  \draw [] (3.1,\i) -- (1.5,\i)%2.2
  node[right=1cm] {\Huge \pgfmathprint{int(-\i/5)}};
\end{scope}




%翻转
\begin{scope}[scale=0.5,xscale=-1]

%主线轴
\def\starty{0}
\def\endy{-25}
\pgfmathsetmacro{\stepy}{\starty-1}
\pgfmathsetmacro{\stepyfive}{\starty-5}
\pgfmathsetmacro{\stepyten}{\starty-10}

\coordinate (startpoint) at (1.5,\starty) ;
\coordinate (endpoint) at (1.5,\endy) ;
\draw [](startpoint) -- (endpoint);


%细小刻度
\foreach \x in {\starty,\stepy,...,\endy}
  \draw [](1.5,\x) -- (2.5,\x) ;   

%5分之一刻度
%\foreach \y in {\starty,\stepyfive,...,\endy}
%  \draw [] (1.6,\y) -- (0,\y);


%10分之一刻度
\foreach \i in {\starty,\stepyfive,...,\endy}
  \draw [] (3.1,\i) -- (1.5,\i)%2.2
  node[left=1cm] {\Huge \pgfmathprint{int(-\i/5)}};
\end{scope}



%额外的修改
%外边框
\draw (-3,-14) rectangle (3,4);
\fill (0,4) circle (0.2);
\fill[gray!50] (-3,4) rectangle (3,4.2);
%unit
\node [above right=1cm] {\Huge N};
%上面的弹簧
\draw[decoration={aspect=0.3, segment length=\coilsegmentlen mm, amplitude=2mm,coil},decorate] (0,4) -- (0,\forcepoint); 

%菱形指针和下面的钩子
\draw (0,\forcepoint) -- +(0,-15)  arc (90:-180:0.6);
\node [draw,scale=2,diamond,fill=orange!50,aspect=2] at (0,\forcepoint) {}; 
%立方体
\begin{scope}[yshift=-20cm,xshift=-3.25cm]
\fill[gray!30] (0,0) rectangle (2,2.4);
\fill[gray!50] (2,0) -- (2.8,0.4) -- (2.8,2.8) -- (2,2.4);
\fill[gray!10] (0,2.4)-- (1,2.8) -- (2.8,2.8) -- (2,2.4);
\draw[ultra thick] (1.2,0) -- ++(0,-1) arc (180:360:1)   -- ++(0,3.8) ;
\end{scope}
\node at (0,-25) {\Huge 丙};
%烧杯和定滑轮
\def\ml{8}
\pgfmathsetmacro{\mlnum}{\ml+0.13}
\begin{scope}[xshift=-1cm,yshift=-24cm]
\path[draw,dashed] (-5,0.4) -- (5,0.4) --(5,\mlnum)  .. controls ($(0.8,\mlnum) + (0,-0.2)$) and ($(-0.8,\mlnum) + (0,-0.2)$) ..  (-5,\mlnum) --cycle ; 
\draw  (0,3)  circle (1) ;
\draw[ultra thick] (-1,0.4) -- (1,0.4);
\draw (-0.2,3) rectangle (0.2,0.4);
\draw (-5,0.4) -- (5,0.4) --(5,12) -- (-5.4,12) -- (-5,11.5) --cycle ; 
\end{scope}
\end{scope}


\end{tikzpicture}
\end{document}