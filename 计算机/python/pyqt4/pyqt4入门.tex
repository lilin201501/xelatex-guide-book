% !Mode:: "TeX:UTF-8"%確保文檔utf-8編碼
%新加入的命令如下:addchtoc addsectoc reduline showendnotes hlabel
%新加入的环境如下:common-format  fig scalefig xverbatim

\documentclass[11pt,oneside]{book}
\newlength{\textpt}
\setlength{\textpt}{11pt}
\newif\ifphone
\phonefalse


\usepackage{myconfig}
\usepackage{mytitle}
\usepackage{python}



\begin{document}
\frontmatter

\titlea{pyqt4入门}
\author{万泽}
\authorinfo{作者:}
\editor{编者}
\email{a358003542@gmail.com}
\editorinfo{编者:}
\version{1.0}
\titleLA

\addchtoc{前言}
\chapter*{前言}
\begin{common-format}
本文的编译使用了python宏包,这个宏包新建了一个python环境,在这个环境中写入的代码将会下如pdf文档中。如果你的py文件写完了,那么你可以使用dopython命令来生成整个py文件并执行这个py文件。命令格式如下:\\
\verb+\dopython[0或者1]{py文件名}+\\
可选项如果不填的话是默认1也就是生成py文件并执行。py文件名的格式是比如你填入test,那么将生成test.py文件。如果你填入可选项0,那么只是进行常规完结py文件操作(dopython命令在你新开一个py文件之前一定要有。),而不执行该py文件。

py文件的执行是python3,也就是所谓的py文件的编写按照python3格式来。

%这里空一行。

\end{common-format}


\addchtoc{目录}
\setcounter{tocdepth}{2}
\tableofcontents

\begin{common-format}
\mainmatter

\chapter{第一个例子}
\section{刚开始}
文件:expample001.py
\begin{python}
#!/usr/bin/env python3
# -*- coding: utf-8 -*-
####序言部分
import sys
from PyQt4  import QtGui
##############
app001 = QtGui.QApplication(sys.argv)

widget001 = QtGui.QWidget()
widget001.resize(800, 600)
widget001.setWindowTitle('第一个程序')
widget001.show()

sys.exit(app001.exec_())
\end{python}
前面的注释部分就不用说了,然后导入sys,是为了后面接受sys.argv参数。导入QtGui是为了后面创建QWidget类的实例。

任何程序都需要创建一个QApplication类的实例,这里是app001,后面跟著数字001就是为了强调这是一个实例。

然后接下来创建QWidget类的实例widget001,首先是引用类的\verb+__init__+方法,然后QWidget类里面有resize方法,这个方法调整等下生成的程序窗口的大小。而setWindowTitle方法设置等下程序窗口上面的标题。show方法就是显示这个窗口。

后面我们看到系统要退出是调用的app001实例的exec\_ 方法,这一句还不太清楚。


\dopython[0]{example001}
\section{加上图标}
\begin{python}
#!/usr/bin/env python3
#-*- coding: utf-8 -*-
\end{python}
现在我在前面第一个程序的基础上稍作修改,来给这个程序加上图标。为了模拟Texmaker,程序的名字就叫做Texmaker。
\begin{python}
import sys
from PyQt4  import QtGui
#######################
class MyQWidget(QtGui.QWidget):
    def __init__(self,parent=None):
        QtGui.QWidget.__init__(self,parent)

        self.setGeometry(0, 0, 800, 600)
        #坐标0 0 大小1360 768
        self.setWindowTitle('Texmaker')
        self.setWindowIcon(QtGui.QIcon\
        ('icons/texmaker.ico'))

app001 = QtGui.QApplication(sys.argv)
widget001 = MyQWidget()
widget001.show()
sys.exit(app001.exec_())
\end{python}
因为自己的DIY开始变多了,所以这里新建了一个类,名字就简单叫做MyQWidget,然后重新定义了这个类的初始函数。首先是继承自QtGui.QWidget类,然后延续了该类的初始函数,而parent被默认为None。

然后用QWidget的setGeometry方法来调整窗口的左上顶点的坐标和窗口的X,Y的大小。这里0,0表示从屏幕的最左上点开始显示,同样800,600类似前面的resize函数的配置。

setWindowTitle方法前面谈论过了,这里加入图标是通过setWindowIcon方法来做到的。这个方法调用了QtGui.QIcon方法,不管这么多,后面跟的就是图标的存放路径,使用相对路径。在运行这个例子的时候,请随便弄个图标文件过来。

后面的和前面类似就不多说了。
\dopython[0]{example002}

\section{窗口弹出提示信息}
\begin{python}
#!/usr/bin/env python3
#-*- coding: utf-8 -*-
import sys
from PyQt4  import QtGui
#######################
\end{python}
接下来要做的DIY就是让这个窗口可以弹出提示信息,就是鼠标停置一会儿会弹出一段小文字。
\begin{python}
class MyQWidget(QtGui.QWidget):
    def __init__(self,parent=None):
        QtGui.QWidget.__init__(self,parent)

        self.setGeometry(0, 0, 800, 600)
        #坐标0 0 大小1360 768
        self.setWindowTitle('Texmaker')
        self.setWindowIcon(QtGui.QIcon\
        ('icons/texmaker.ico'))
        self.setToolTip('<b>看什么看、、</b>')
        #<b></b> 加粗
        QtGui.QToolTip.setFont(QtGui.QFont\
        ('微软雅黑', 10))

app001 = QtGui.QApplication(sys.argv)
widget001 = MyQWidget()
widget001.show()
sys.exit(app001.exec_())
\end{python}
上面这段代码和前面的代码的不同就在于MyQWidget类的初始函数新加入了两条命令。其中setToolTip方法设置具体显示的文本内容,而<b></b>之间的文字会加粗。然后后面那条命令是设置字体和字号的,我不太清楚这里随便设置系统的字体微软雅黑是不是有效。
\dopython[0]{example003}

\section{退出的时候询问}
目前程序点击那个叉叉图标关闭程序的时候将会直接退出,这里新加入一个询问机制。
\begin{python}
#!/usr/bin/env python3
#-*- coding: utf-8 -*-
import sys
from PyQt4  import QtGui
#######################
\end{python}
接下来要做的DIY就是让这个窗口可以弹出提示信息,就是鼠标停置一会儿会弹出一段小文字。
\begin{python}
class MyQWidget(QtGui.QWidget):
    def __init__(self,parent=None):
        QtGui.QWidget.__init__(self,parent)

        self.setGeometry(0, 0, 800, 600)
        #坐标0 0 大小1360 768
        self.setWindowTitle('Texmaker')
        self.setWindowIcon(QtGui.QIcon\
        ('icons/texmaker.ico'))
        self.setToolTip('<b>看什么看、、</b>')
        #<b></b> 加粗
        QtGui.QToolTip.setFont(QtGui.QFont\
        ('微软雅黑', 10))
        
    def closeEvent(self, event):
        #重新定义colseEvent
        reply = QtGui.QMessageBox.question\
        (self, '信息',
            "你确定要退出吗?",
             QtGui.QMessageBox.Yes,
             QtGui.QMessageBox.No)

        if reply == QtGui.QMessageBox.Yes:
            event.accept()
        else:
            event.ignore()

app001 = QtGui.QApplication(sys.argv)
widget001 = MyQWidget()
widget001.show()
sys.exit(app001.exec_())
\end{python}
这段代码重新了原来的colseEvent方法,这里调用的那个方法内部“信息”两个字是弹出的信息框的标题,后面是信息框里面显示的文字。这里具体代码我还不是很懂。
\dopython[0]{example004}

\section{居中显示窗体}
\begin{python}
#!/usr/bin/env python3
#-*- coding: utf-8 -*-
import sys
from PyQt4  import QtGui
#######################
\end{python}
接下来要做的DIY是让窗体弹出的时候居中显示,前面是设置了窗体的起点坐标的,这里新建了一个center方法来确认窗体居中显示。
\begin{python}
class MyQWidget(QtGui.QWidget):
    def __init__(self,parent=None):
        QtGui.QWidget.__init__(self,parent)

        #self.setGeometry(0, 0, 800, 600)
        #坐标0 0 大小1360 768
        self.resize(800,600)
        self.center()
        self.setWindowTitle('Texmaker')
        self.setWindowIcon(QtGui.QIcon\
        ('icons/texmaker.ico'))
        self.setToolTip('<b>看什么看、、</b>')
        #<b></b> 加粗
        QtGui.QToolTip.setFont(QtGui.QFont\
        ('微软雅黑', 10))
        
    def center(self):
        screen = QtGui.QDesktopWidget().screenGeometry()
        #接受屏幕几何
        size =  self.geometry()
        self.move((screen.width()-size.width())/2,\
         (screen.height()-size.height())/2)       

    def closeEvent(self, event):
        #重新定义colseEvent
        reply = QtGui.QMessageBox.question\
        (self, '信息',
            "你确定要退出吗?",
             QtGui.QMessageBox.Yes,
             QtGui.QMessageBox.No)

        if reply == QtGui.QMessageBox.Yes:
            event.accept()
        else:
            event.ignore()

app001 = QtGui.QApplication(sys.argv)
widget001 = MyQWidget()
widget001.show()
sys.exit(app001.exec_())
\end{python}
这里做的改动就是新建了一个center方法,接受实例。然后对这个实例也就是窗口的具体位置做一些调整。前面使用了resize和center两个方法来调整窗口的大小和窗口的位置。

从center方法中我们可以看到move方法的X,Y是从屏幕的坐标原点(0,0)开始计算的。第一个参数X表示向右移动了多少宽度,Y表示向下移动了多少高度。
\dopython[0]{example005}

%这里空一行

\end{common-format}
\end{document}