% !Mode:: "TeX:UTF-8"%確保文檔utf-8編碼
%新加入的命令如下:addchtoc addsectoc reduline showendnotes hlabel
%新加入的环境如下:common-format  fig scalefig xverbatim

\documentclass[11pt,oneside]{book}
\newlength{\textpt}
\setlength{\textpt}{11pt}
\newif\ifphone
\phonefalse


\usepackage{myconfig}
\usepackage{mytitle}
%\DeclareMathSizes{13}{13}{13}{13}

\begin{document}
\frontmatter

\titlea{费曼的}
\titleb{物理学讲义}
\titlec{第一卷第二部分}
\author{费曼}
\authorinfo{理查德•费曼(Richard Phillips Feynman,1918年5月11日-1988年2月15日),美国物理学家。1965年诺贝尔物理奖得主。提出了费曼图、费曼规则和重整化的计算方法,这些是研究量子电动力学和粒子物理学的重要工具。}
\editor{德山书生}
\email{a358003542@gmail.com}
\editorinfo{编者:湖南常德人,纯手工敲入,数学公式和图片借鉴了英文官网。}
\version{0.01}
\titleLC

\addchtoc{前言}
\chapter*{前言}
\begin{common-format}
源码在github上。

%这里空一行。

\end{common-format}


\addchtoc{目录}
\setcounter{tocdepth}{2}
\tableofcontents

\begin{common-format}
\mainmatter

\chapter{万有引力理论}


\chapter{运动}



%这里空一行

\end{common-format}
\end{document}



