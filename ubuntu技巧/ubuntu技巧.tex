% !Mode:: "TeX:UTF-8"%確保文檔utf-8編碼
%新加入的命令如下:addchtoc addsectoc reduline printendnotes hlabel
%新加入的环境如下:common-format  fig scalefig

\documentclass[11pt,oneside]{book}
\newlength{\textpt}
\setlength{\textpt}{11pt}
\newif\ifphone
\phonefalse

\usepackage{myconfig}
\usepackage{mytitle}

\begin{document}
\frontmatter

\titlea{\color[HTML]{DE4815}{}}
\titleb{ubuntu技巧}
\titlec{各种linux下的实用问题解答集}
\author{万泽}
\authorinfo{作者:湖南常德人氏}
\editor{万泽}
\email{a358003542@gmail.com}
\editorinfo{编者:wanze.}
\version{1.0}
\titleLTC

\addchtoc{开头说的话}
\chapter*{开头说的话}
\begin{common-format}
开头说的话

%这里空一行。

\end{common-format}


\addchtoc{目录}
\setcounter{tocdepth}{2}
\tableofcontents

\begin{common-format}
\mainmatter

\chapter{批量文件管理}

\section{shell编程基本知识}
\subsection{变量赋值}
i=2
就给i赋值2了。注意变量和值之间的等号是相连的,不能用空格隔开。

\subsection{echo命令}
echo查看某个变量的值或者直接输出一行字符串。\\
echo 'this is a test line.'

比如显示上面变量i的值:
echo \${}i

\${}跟在变量符号前面表示引用某个变量。

\subsection{if条件}
if条件语句格式是:
\begin{verbatim}
if  [ test expression ]
then    do what
fi  
\end{verbatim}
在then的后面还可以跟上else或者继续下一个条件语句elif。

分号; 的作用是如果bash shell 命令是一行的时候用分号表示换行来表达格式。所以上面的命令写成一行是:\\
\verb+if [ test expression ]; then do what ; fi+

还值得提醒一下的是:条件判断语句要和那个方括号[]有一个空格表示分开。

\subsection{for循环}
\href{http://www.cyberciti.biz/faq/bash-for-loop/}{bash for loop}
for循环语句格式如下:
\begin{verbatim}
for  var in 1 2 3
do 
    do what
    do what
done
\end{verbatim}

\subsection{批量创建文件}
在文件夹里面输入如下命令:\\
\verb|for (( i=1; i<=10; i++ )); do  touch file$i.txt; done|
这样会在该文件夹里面批量创建10个文件,文件名依次为file1.txt,file2.txt......file10.txt。
这个for语句类似于C语言的for语句,记住这个格式方便运用。


\section{所有文件合并到一起}
\href{http://unix.stackexchange.com/questions/3770/how-to-merge-all-text-files-in-a-directory-into-one}{merge all file}

cat *
文件联接

echo *
反映顺序


> 信息流导向,前面语句输出的信息流会流向这个文件。比如:\\
\verb+cat * > test.txt+
就会把这个文件夹里面所有的文件都合并成为一个文件。

\section{所有的文件依次重命名}


\chapter{批量文本替换}
一般的编辑器都有一般的文本查找和替换功能,这里谈及的是更加复杂的一些情况。

\section{将"一串字符"都替换为中文标点的“同样一串字符”}
sed 全局替换

\verb+sed -e 's/good/goodmorning/g'  test >test1+

将test文本中所有出现的good 字符替换为goodmoring 然后输出为test1文本

\verb+sed -e 's/foo/bar/' myfile.txt+
上面的命令将 myfile.txt 中每行第一次出现的 'foo'(如果有的话)用字符串 'bar' 替换,然后将该文件内容输出到标准输出。请注意,我说的是每行第一次出现。

在linux下怎么大量将"一串字符"都替换为中文标点的“同样一串字符”。 gedit下有替换工具,不清楚怎么表示任意一串字符,而且就算知道这样一串字符,还必须同样将同样这串字符传递给等下要替换的同样那串文字中。 具体效果就是这样的: "测试文字" → “测试文字”
这个文本很大,必须用批处理的方式,请问有好的方法吗?

\verb+sed -e 's/\("\)\([^"]*\)\("\)/“\2”/g'  test>test1+



\chapter{ubuntu备份技巧}
\section{让你的ubuntu有个分身}
移动home和让ubuntu拥有分身的技术。\href{http://wangmm2008.blog.163.com/blog/static/1812740122011111112842470/}{主要参考了这个网站}

\begin{enumerate}   
\item 新开一个分区,格式化为ext4格式。
\item 将你的home目录复制过去。建议用root账户操作:\\
\verb+su+\\
\verb+sudo cp -afrv /home/* /media/wanze/data+\\
这里建议使用gparted分区工具将分区的卷标加号,比如上面我加上了data卷标,然后挂载就成了/media/wanze/data的地址,当然你也可以用ubuntu的文件浏览器设置那里选择输入位置,那个位置你复制了就是的。
上面选项加上v就是为了防止你出现系统卡死的错觉。。
\item 复制的时候你可以开始修改/etc/fstab文件。我之前没有/home设置,所以需要重新加上这样一句:
\begin{verbatim} 
# home was on /dev/sda6 during installation
UUID=e56f656e-8ebb-4ab8-8f76-c1e26aba22a4 /home ext4 defaults 0 0
\end{verbatim}
上面的UUID也就是你新分区的那个UUID,通过命令:\\
\verb+ls -l /dev/disk/by-uuid+
查看。日期时间后面那个就是,然后设置/home挂载点,其他就是defaults 0 0 0了。
\item 稳妥起见,我发现复制完了新的wanze文件夹权限不一样了,用sudo nautilus 修改下权限,和你之前的一模一样就行了。
\item 重启,发现简直一模一样,包括网络硬盘同步程序等等都没出错。
\item 将fstab那一句注释掉,重启,发现又进入原来的wanze主目录了。然后将不重要的音乐,下载的文件图片等,因为重复了,所以删除掉节省点根目录的空间。
\item 将fstab那句不注释,发现又进入新的主目录文件夹了。这样就感觉有了两个系统,毕竟个人电脑用户出错就出错在home文件夹里面的设置上,这样算是有了双保险了把。
\end{enumerate}





%这里空一行

\end{common-format}
\end{document}



