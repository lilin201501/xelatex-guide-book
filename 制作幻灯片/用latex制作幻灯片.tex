% !Mode:: "TeX:UTF-8"%确保文档utf-8编码
%commands:
%environments:  
\documentclass[xetex,aspectratio=169]{mybeamer}

\usetheme{Berkeley}
%theme list:
%Antibes    Bergen    Berkeley    Berlin    Copenhagen    Darmstadt    Dresden    Frankfurt    Goettingen
%Hannover    Ilmenau    JuanLesPins    Luebeck    Madrid    Malmoe    Marburg    Montpellier
%PaloAlto    Pittsburgh    Rochester    Singapore    Szeged    Warsaw    boxes    default 
\title{\LaTeX 用制作幻灯片}
\subtitle{主要讨论beamer class}
\author{万泽}

\begin{document}



\maketitle



%\begin{frame}
%  \tableofcontents
%\end{frame}



%\section{基本情况说明}
%\subsection{基本命令}
%\section{beamer class一般讨论}
%\section{全屏}
%\section{目录和封面}
%\section{标题}
%\section{建议废弃的功能}
%\section{表格}
%
%\subsection{页眉页脚}
%\subsection{旁注}
%\section{脚注}
%\section{文字强调}
%
%
%\section{delay}
%
%\section{插入图片}
%\section{theme}
%\section{color theme}
%\section{页面布局}
%\section{参考文献}


\begin{frame}
\frametitle{基本情况说明}
本文讨论基於\XeLaTeX 指南一书的讨论,并由此而来。首先是documentclass使用了beamer class。然後前面的可选项加上xetex,其他可选项後面再慢慢讨论。 

本幻灯片的中文字显示所用代码完全参考\XeLaTeX 指南一书中的字体处理部分,还额外将中文化的一些命令复制进来了。
\end{frame}

%\section{插入代码}
%\begin{frame}[fragile]
%\frametitle{基本命令}
%\begin{xverbatim}{txt}
%test\\\
%\end{xverbatim}
%\end{frame}

\section{列表}
\begin{frame}
\frametitle{Overlays with {\tt pause}}
\setbeamercovered{invisible}
Practical \TeX\ 2005\\  \pause
Practical \TeX\ 2005\\ \pause
Practical \TeX\ 2005
这里放着第一张幻灯片的内容。 beamer automatically loads xcolor
\end{frame}

  \begin{frame}
    \frametitle{This is the first slide}
    %Content goes here
  \end{frame}
  \begin{frame}
    \frametitle{This is the second slide}
    \framesubtitle{A bit more information about this}
    %More content goes here
  \end{frame}
  
  \begin{frame}
  The usual environments (itemize, enumerate, equation, etc.) may be used.
  
  Inside frames, you can use environments like block, theorem, proof, ... Also, maketitle is possible to create the frontpage, if title and author are set.
  Trick: Instead of using begin{frame}...end{frame}, you can also use frame{...}.
  
  \end{frame}



\begin{frame}[allowframebreaks]
  \frametitle<presentation>{Weiterf¸hrende Literatur}    
  \begin{thebibliography}{10}    
  \beamertemplatebookbibitems
  \bibitem{Autor1990}
    A.~Autor.
    \newblock {\em Einf¸hrung in das Pr‰sentationswesen}.
    \newblock Klein-Verlag, 1990.
  \beamertemplatearticlebibitems
  \bibitem{Jemand2000}
    S.~Jemand.
    \newblock On this and that.
    \newblock {\em Journal of This and That}, 2(1):50--100, 2000.
  \end{thebibliography}
\end{frame}


\end{document}
