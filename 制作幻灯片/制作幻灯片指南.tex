% !Mode:: "TeX:UTF-8"%確保文檔utf-8編碼
%新加入的命令如下:addchtoc addsectoc reduline showendnotes hlabel
%新加入的环境如下:common-format  fig linefig xverbatim

\documentclass[11pt,oneside]{book}
\newlength{\textpt}
\setlength{\textpt}{11pt}
\newif\ifphone
\phonefalse


\usepackage{myconfig}
\usepackage{mytitle}




\begin{document}
\frontmatter

\titlea{制作幻灯片指南}
\author{万泽}
\authorinfo{作者:}
\editor{德山书生}
\email{a358003542@gmail.com}
\editorinfo{编者:德山书生,湖南常德人氏。}
\version{0.1}
\titleLA

\addchtoc{前言}
\chapter*{前言}
\begin{common-format}
这个小册子整理关于制作幻灯片的知识,

参考资料:
1.Chales Batts Beamer tutorial

%这里空一行。

\end{common-format}


\addchtoc{目录}
\setcounter{tocdepth}{2}
\tableofcontents

\begin{common-format}
\mainmatter

\chapter{章节开始}

\section{使用模板}
最好还是建立自己喜欢的模板

\section{插入封面}
封面内容的插入命令有:\textbf{title}、\textbf{subtitle}、\textbf{date}、\textbf{author}、\textbf{institute}、\textbf{logo}等。其中logo引入的图标会贯穿整个幻灯片。

然后在正文文档里面写上下面代码即引入封面。
\begin{Verbatim}
\begin{frame}
\titlepage
\end{frame}
\end{Verbatim}

\section{加一张幻灯片}
\begin{Verbatim}
\begin{frame}[c]
\frametitle{}
\end{frame}
\end{Verbatim}
加一张幻灯片代码如上,一张幻灯片就是一个frame环境,然后frametitle命令写上该幻灯片的标题。

后面跟着选项:\textbf{c},\textbf{b},\textbf{t}。表示该幻灯片内容居中对齐(默认),居底和居顶。

\section{插入目录}
幻灯片不必书籍,目录不一定需要,而且目录内容过多还会有越界问题。
\begin{Verbatim}
\begin{frame}
\frametitle{目录}
\setcounter{tocdepth}{1}
\tableofcontents[pausesections]
\end{frame}
\end{Verbatim}
这里的pausesections选项会让各级目录产生逐渐显示的效果。


\section{插入代码}
frame后面要加上选项\textbf{fragile},这样就可以使用verb等其他命令插入代码。

\section{semiverbatim环境}
beamer类新定义了一个semiverbatim环境,和verbatim类似,除了\textbackslash 和\{和\}继续保持原义。

\section{改变字体}
使用命令\verb+\usefonttheme{serif}+就可以将整个幻灯片默认的粗体换成正常book类默认的serif字体。

此外还有:\textbf{structurebold},\textbf{structuresmallcapsserif},\\ \textbf{structureitalicserif}。

\section{字体大小}
整个文档字体大小设置和book类类似,在加载类的前面可选项里面填上:10pt,11pt(默认)等。

\section{显示顺序}
\subsection{pause命令}
插入一个pause命令,这里就会暂停一下。

\subsection{其他控制技巧}
比如对于列表环境,item后面跟一个<1>,那么这个item第一张\footnote{这里所谓的第一个的意思是产生了overlay,然后同一个显示环境下不同的内容分成了几张pdf页面依次显示。}显示。类似的有<2> 第二张显示 ;<1-2>1,2都显示;<1->从1一直显示。

上面谈论的情况同样适用于alert命令,这个命令让文本强调显示。此外这些命令也可以:textbf,textit,textsl,textrm,textsf,color,structure。

此外列表环境后面跟上可选项[<+->]表示这个列表环境的item逐渐显示。

其他环境也可以overlay,比如theorem,proof环境的等,整个环境只放在一个幻灯片内。

\section{block环境}
block是一个类似colorbox的盒子结构,可以用来强调某些特别的文本和图形。其他的block环境还有:
一般  block\\
数学中的定理  theorem\\
引理\footnote{引理是数学中为了取得某个更好的结论而作为步骤被证明的命题,其意义并不在于自身被证明,而在于为达成最终目的作出贡献。}   lemma\\
证明  proof\\
推论 corollary\\
例子 Example\\
高亮标题 alertblock


\section{column环境}
分栏环境大体配置如下。
\begin{Verbatim}
\begin{columns}
\column{0.5\textwidth}
blabla
\column{0.5\textwidth}
blabla
\end{columns}
\end{Verbatim}

\section{分栏并排的block}
\begin{Verbatim}
\begin{columns}[t]
\column{.5\textwidth}
\begin{block}{Column 1 Header}
Column 1 Body Text
\end{block}
\column{.5\textwidth}
\begin{block}{Column 2 Header}
Column 2 Body Text
\end{block}
\end{columns}
\end{Verbatim}

\section{整个幻灯片的主题}
usetheme命令,默认可用的选项有:Antibes Boadilla Frankfurt Juanlespins Montpellier Singapore
Bergen Copenhagen Goettingen Madrid Paloalto Warsaw Berkeley Darmstadt Hannover Malmoe Pittsburgh Berlin Dresden Ilmenau Marburg Rochester。

\section{colortheme}
\begin{Verbatim}
\usecolortheme{rose}
%albatross fly crane seagull beetle wolverine dove beaver
% innercolortheme lily orchid rose       
        the colors of blocks. 
%outer color theme whale seahorse dolphin     
     headline, footline, and sidebar
\usefonttheme{serif}
\end{Verbatim}


\section{紧凑导航栏}
在documentclass选项里面插入\textbf{compress}选项可使导航栏更加紧凑,具体参见\href{http://tex.stackexchange.com/questions/42917/how-to-hide-the-navigation-bar-in-latex-beamer}{这个网站}


%这里空一行

\end{common-format}
\end{document}



