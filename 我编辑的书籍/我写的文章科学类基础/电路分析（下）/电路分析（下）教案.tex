% !Mode:: "TeX:UTF-8"%確保文檔utf-8編碼
%新加入的命令如下: reduline showendnotes 
%新加入的环境如下:solution solutionorbox solutionorlines solutionordottedlines

\documentclass[12pt]{exam}
\newlength{\textpt}
\setlength{\textpt}{12pt}

\usepackage{teachingplan}
\usepackage{wallpaper}
\usepackage{comment}
\usetikzlibrary{positioning}

%输出方案 
%1.自己用答案,知识点等都有  ——老师版
%2.给学生用基础差的可以考虑加上知识点,但不写答案——基础差版
%3.练习题加提高题——学生版
%4.练习题加提高题加综合题——学霸版

%写上答案或者不写上答案%1  
\printanswers  

%提高题%1  4
%\includecomment{improveexercises}
\excludecomment{improveexercises}
%综合题%1  4  5
%\includecomment{advanceexercises}
\excludecomment{advanceexercises}


\CenterWallPaper{1}{教案模板-2.pdf}
\newcommand{\shortanswerline}{\rule[-2pt]{50pt}{0.4pt}}
\renewcommand{\solutiontitle}{\noindent\textbf{}}

\newcommand{\keti}{热和能}
\newcommand{\zhongdian}{1.分子热运动 2.内能 3.比热容\\  4.热机 5.能量的转化和守恒}
\renewcommand{\section}[1]{{\large\sffamily  #1} \par}
\renewcommand{\subsection}[1]{{\normalsize\sffamily  #1} \par}
\newcommand{\leftnote}[1]{\marginpar{\ttfamily\fontsize{10}{10}\selectfont #1}}



\newcommand{\answer}[2][50pt]{{\setlength{\answerlinelength}{
#1} \answerline*[#2]}}

\begin{document}
\ThisCenterWallPaper{1}{教案模板-1.pdf}
\vspace*{80pt}
\keti \par
\zhongdian \par
\section{分子热运动}
第二章  声现象
  第1节  声音的产生与传播                 
  第2节  声音的特性                       
  第3节  声的利用                         
  第4节  噪声的危害和控制                
(本章节内容相对独立,比较容易掌握,可多结合生活常识进行认知。本章节出题量不大,分值在中考中三分到五分之间,也可能不会出现本章题型。主要以填空、选择为主,偶尔会结合机械运动声速等知识出现简单的计算题,分值会增大。)


  【教学目标】
  1、知识和技能目标  
  知道声波是振动在介质中以疏密相间的形式向外传播。
  知道常温下声音在空气中传播速度,了解不同介质中声速不同。
  了解不同介质的传声本领不同。
  知道声波的反射规律及其应用。
  2、过程和方法目标 
  体会根据“类比法”学习物理知识和探究性的实验方法。
  使学生具备运用物理语言归纳和表达观点的能力。
  3、情感、态度和价值观目标  
  通过对物理现象的分析,激发学生学习物理知识的热情,正确理解生活中的物理知识。
  学会协作交流,形成积极向上、主动学习的习惯,体现自我价值。




\section{练习题}














\begin{improveexercises}
\section{提高题}

\end{improveexercises}

\begin{advanceexercises}
\section{综合题}

\end{advanceexercises}

\ThisCenterWallPaper{1}{教案模板-3.pdf}

\end{document}



