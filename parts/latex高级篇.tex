\part{\LaTeX 高级篇}
\section{自建环境或修改原环境}
\label{sec:自建环境}

用如下命令格式来新建一个环境\\
\textbackslash begin \{environmentname\}\\
text\\
\textbackslash end \{environmentname\}\\
新建环境有很多用途,这里就最直接的插入图片来说明,比如说上面那个图片我用:\\
\textbackslash begin\{figure\}\\
什么什么\\
\textbackslash end\{figure\}\\
给包围起来,然后加入:\\
\textbackslash caption \{这是用于\textbackslash LaTeX 的一个编辑器界面\}\\
caption命令加入图片标题。然后如果你想在文章中引用这个图片的话,加入如下命令:\\
\textbackslash label \{figure1\}

还有很多高级知识这里能够略过就略过了。

\section{盒子和glue}
\label{sec:盒子和glue}
\begin{fancycolorbox}
这里参考了Knuth的The Texbook,但是我并没有将其写入参考文献,因为只是觉得关于box和glue的概念最好参考原初定义,但是并不推荐读者阅读这本书,现在已经是\LaTeX 时代了,不推荐读者使用原始的\TeX 命令,一是不太实用,二是兼容性可能不太好。除非你是宏包编写者,但就作为一般的使用者真的没有必要接触那些原始命令了。还参考了\cite{boxes}
\end{fancycolorbox}
\endnote{\begin{fancycolorbox}
在这里box翻译为盒子没什么问题,就是这个glue翻译为胶水或者橡皮都让我不太满意。在后面我都使用的术语是距离或者间距或者干脆用英文glue。从某种意义上讲glue的直译确实应该译为胶或者胶水,不过觉得距离这个词汇更能够让人们有感觉些:\TeX 排版就是不同的盒子编写和彼此之间距离的设置问题。
\end{fancycolorbox}}
\subsection{基本知识}
最小的盒子就是基于Unicode的字符,这些字符然后组成更大的盒子──单词,然后单词组成更大的盒子──行等等。行是一个盒子,段落也是一个盒子,图片是一个盒子,表格也是一个盒子。而这些盒子按照Knuth的描述都是用glue胶水粘合起来的,或者我们称之为这些盒子之间都存在着空间胶合层。下面就是一个盒子的详细参数:

\begin{fig}{box参数}
\label{fig:box参数}
\end{fig}

盒子就是这么一个长方形的区域,如上图所示,它有参数:height,width,depth。baseline和reference point在后面讲的hbox和vbox中会用到。在\TeX 看来,从字体而来的Unicode 字符就是一个最简单的盒子。字体的设计者已经决定了这个字符的高度,宽度和深度以及它在这个盒子里面看起来如何。\TeX 就是用这些维度将盒子黏合到一起,并最终决定所有字符的reference point 参考点在页面上的位置。

\TeX 的盒子如果全部涂上颜色,一般是黑色,那么就成了一个黑盒子。这样的黑盒子还有一个名字叫做rule box。也就是线条。这个在后面会谈论到的。

不管是字符盒子还是黑盒子,他们要某是水平排列要某是垂直排列。水平排列要做的就是让这些盒子的参考点在一条水平线上。类似的垂直排列要做的就是让这些盒子的参考点在一条垂直线上。

好吧,介绍两个\TeX 命令:hbox和vbox。hbox命令就是让所有的盒子在一条水平线上,而vbox命令就是把一些hbox命令垂直排列,比如下面的代码:\\
\verb+\vbox{\hbox{恭}\hbox{喜} \hbox{发}\hbox{财}}+

\vbox{\hbox{恭}\hbox{喜} \hbox{发}\hbox{财}}

glue也就是各个盒子之间的间距。下图是glue的具体图示:
\begin{fig}{glue说明}
\label{fig:glue说明}
\end{fig}
前面说到盒子的reference point水平排列,然后他们之间还有叫做glue的间距。间距有三个属性:正常间距量(space),拉伸量(stretch),缩减量(shrink)。比如这个图片中第一个glue的正常间距是9个单位,拉伸量为3个单位,缩减量为1个单位。而总的情况是正常间距是5(box1)+9+6+9+3+12+8=52个单位。现在假设一行宽58个单位,\TeX 就要调整使得这一行盒子的宽度刚好等于58个单位,于是还需要增加6个单位的宽度,而这6个单位的宽度\footnote{为了简单起见这里不考虑缩减量,具体缩减量如何计算我也不大清楚。}需要从这一行所有glue里的拉伸量中找出来。于是总的拉伸量加法是3+6+0=9。也就是6个单位的宽度要分成9等分再分配给他们,即第一个glue的拉伸量是$3*\Large{\frac { 6 }{ 9 }} $。这样第一个间距的总长度就是$9+3*\Large{\frac { 6 }{ 9 }} =11$。

经过计算所有的glue间距都确定下来了,那么整个页面布局就确定了。我在这里就戛然而止了,毕竟这里只是对box和glue的基本概念的阐明。

\subsection{盒子in \LaTeX}
前面稍微介绍了点\TeX 的原初命令hbox和vbox。我们就能够感受到靠着这两个命令已经能够干很多事情了。比如我想写一幅对联。

\begin{verbatim}
{\CJKfontspec[Scale=3,Color=red]{方正魏碑_GBK}
\vbox {
\centerline{呵呵}
    \hbox to \hsize {恭大}
    \hbox to \hsize {喜吉}
    \hbox to \hsize {发大}
    \hbox to \hsize {财利}}}
\end{verbatim}

\vspace{30pt}
{\CJKfontspec[Scale=3,Color=red]{方正魏碑_GBK}
\vbox {
\centerline{呵呵}
    \hbox to \hsize {恭大}
    \hbox to \hsize {喜吉}
    \hbox to \hsize {发大}
    \hbox to \hsize {财利}}}

其中hbox和vbox都支持这样的to 多少尺寸和 spread 增加多少尺寸的形式,这里的hsize相当于linewidth吧。然后我用这个例子主要说明用hbox和vbox能够做很多排版工作了。而我们在\LaTeX 里面则用makebox替代了hbox,raisebox替代了vbox。并且排版思路变成了主要是一行一行排版,然后才考虑raisebox微调。

\section{box 命令介绍}
\href{http://tex.stackexchange.com/questions/83930/what-are-the-different-kinds-of-boxes-in-latex}{参考了这个网站}

\subsection{makebox}
\verb+\makebox[width][allignment]{some text}+
还有一个mbox,不过makebox命令更加全面,推荐使用。makebox就是制造一个水平的盒子,注意这个box里面的文本是不能换行的,也就是一行之内的盒子。第一个可选项width指这个盒子的长度,第二个可选项allignment是里面文本的对齐方式,有〖l c r s〗几个选项,\emph{l}表示左对齐;\emph{c}表示居中;\emph{r}表示right;\emph{s}表示两端对齐。






\section{带颜色或者线框的盒子}
\subsection{colorbox}
colorbox
\verb+\colorbox{yellow}{this is a test line.}+\\
\colorbox{yellow}{this is a test line.}


\section{fancycolorbox}
本文定义了一个fancycolorbox命令,具体代码如下:\\
\begin{fancycolorbox}
为什么是这么复杂的一个代码?这个需要说明一下。首先系统自带的colorbox命令不能换行,必须colorbox里面加上minipage环境才行,9,10行是minipage环境的格式优化,然后整个构成一个新环境我还想传递颜色,就成这个样子了。主要参考的\href{http://tex.stackexchange.com/questions/127612/color-text-and-bg-of-verbatim-without-affecting-fancyvrb-line-numbers}{这个网站}。
\end{fancycolorbox}
\vspace{20pt}
\begin{verbatim}
\definecolor{bgcolor-0}{HTML}{CCFFCC} 

\newsavebox{\tempbox}
\newenvironment{fancycolorbox}[1][bgcolor-0]
 {\noindent%%
  \renewcommand{\tempcolor}{#1}
  \begin{lrbox}{\tempbox}%
  \begin{minipage}{\linewidth-10pt}
  	\setlength{\parskip}{1.6ex plus 0.2ex minus 0.2ex}   %段落間距
	\setlength{\parindent}{\baselineskip * \real{0.06} + \textpt * \real{2.4}}}  	
 {\ignorespacesafterend%
  \end{minipage}%
  \end{lrbox}%
  \colorbox{\tempcolor}{\usebox{\tempbox}}}  
\end{verbatim}
fancycolorbox环境的效果前面你应该已经看过了。这里就不做演示了。


\section{大型文档}
