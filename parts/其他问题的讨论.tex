\part{其他问题的讨论}

\section{语录,引用和诗歌环境}
quote为语录环境,quotation用于超过几段的引用环境,verse用于诗歌环境。下面是quote语录环境:
\begin{quote}
“When the winds of change blow, some people build walls and others build windmills.”
\end{quote}


\section{插入摘要}
使用的环境就是abstract,\endnote{\reduline{book类不能使用摘要环境,只有article和report才有。}}命令如下:\\
\begin{verbatim}
\begin{abstract}
这是一段摘要文字。
\end{abstract}
\end{verbatim}

\section{文本中的上标还有下标}
上标使用命令textsuperscript命令,下标感觉自己缩小点就差不多了吧。示例如下:\\
上标\textsuperscript{上标}下标{\scriptsize 下标}

更常见的是化学分子式里面的上标和下标,推荐使用mhchem宏包,至于数学模式里的上标和下标自不必说了。


\section{多张图片并列显示}
\label{sec:多张图片并列显示}

\begin{figure}[h]

\label{fig:四栏图片}
\begin{multicols}{4}
\includegraphics[width=\linewidth ,totalheight=\textheight , keepaspectratio]{temp.jpg}
i want  some test to show there is a text and not to column break.\\
\centerline{test}
\columnbreak

\includegraphics[width=\linewidth ,totalheight=\textheight , keepaspectratio]{temp.jpg}
\centerline{test}
\columnbreak

\includegraphics[width=\linewidth ,totalheight=\textheight , keepaspectratio]{temp.jpg}
\centerline{test}
\columnbreak

\includegraphics[width=\linewidth ,totalheight=\textheight , keepaspectratio]{temp.jpg}
\centerline{test}

\end{multicols}
\caption{denosie fig}

\end{figure}

具体代码如下:
\begin{verbatim}
\begin{figure}[h]

\label{fig:四栏图片}
\begin{multicols}{4}
\includegraphics[width=\linewidth ,totalheight=\textheight , 
        keepaspectratio]{temp.jpg}
i want  some test to show there is a text and not to column break.\\
\centerline{test}
\columnbreak
\includegraphics[width=\linewidth ,totalheight=\textheight , 
         keepaspectratio]{temp.jpg}
\centerline{test}
\columnbreak
\includegraphics[width=\linewidth ,totalheight=\textheight ,
         keepaspectratio]{temp.jpg}
\centerline{test}
\columnbreak
\includegraphics[width=\linewidth ,totalheight=\textheight ,
         keepaspectratio]{temp.jpg}
\centerline{test}
\end{multicols}
\caption{denosie fig}

\end{figure}
\end{verbatim}
这段代码的分栏还有插入图片知识都已经介绍了,值得一提的就是widetext环境,也就是有changepage宏包而来的临时改变页面布局环境和浮动体环境figure以及table不兼容。比如放入浮动体环境内才能起作用。然后caption命令似乎只是默认的linewidth居中。所以如果想要图片和表格在扩大的文本布局中居中对齐,那么需要在浮动体环境内部使用改变页面布局命令,然后使用居中命令。

\section[生成字体所有已有的字形]{生成字体所有已有的字形\endnote{主要参照了\href{http://tex.stackexchange.com/questions/23863/generating-a-table-of-glyphs-with-xetex}{这个网站}}}
\label{sec:字体已有字形}
\begin{verbatim}
\documentclass[landscape]{article}
\usepackage{geometry}
\usepackage{fontspec}
\setmainfont{DejaVu Sans}
\usepackage{multicol}
\setlength{\columnseprule}{0.4pt}
\usepackage{multido}
\setlength{\parindent}{0pt}
\begin{document}
\begin{LARGE}
\begin{multicols}{5}
\multido{\i=0+1}{"196607"}{% from U+0000 to U+2FFFF
%后面的还会继续扩展,目前一般还没使用。
  \iffontchar\font\i
    \makebox[4em][l]{\i}%
    \symbol{\i}\endgraf
  \fi
}
\end{multicols}
\end{LARGE}
\end{document}
\end{verbatim}


\section{方便手机上观看的pdf}
原文档内容都不需要修改,只需要修改geometry的设置就可以满足要求,然后在手机上看的时候选择adobe pdf软件的连续观看功能会有更好的体验。具体配置如下:
\endnote{主要参考了\href{http://tex.stackexchange.com/questions/78920/generating-smartphone-readable-pdf}{这个网站}。}
\begin{verbatim}
\ifphone
%for phone
\RequirePackage[
  paperwidth=105mm, %除去旁註其他沒變,115,再稍微小點
  paperheight=190mm,%太長了縮短點,
  bindingoffset=0mm,%裝訂線
  top=15mm,  %上邊距 包括頁眉
  bottom=15mm,%下邊距 包括頁腳
  left=5mm,  %左邊距or inner
  right=5mm,  %右邊距or  outer
  headheight=10mm,%頁眉
  footskip=10mm,%頁腳
  includemp=true,% 旁註寬度計入width
  marginparsep=0mm, %沒有旁註
  marginparwidth=0mm,  %沒有旁註
  ]{geometry}
\else

\RequirePackage[a4paper, %a4paper size 297:210 mm
  bindingoffset=0mm,%裝訂線
  top=45mm,  %上邊距 包括頁眉
  bottom=40mm,%下邊距 包括頁腳
  left=35mm,  %左邊距or inner
  right=40mm,  %右邊距or  outer
  headheight=25mm,%頁眉
  footskip=25mm,%頁腳
 includemp=true,% 旁註寬度計入width
 marginparsep=0mm, %旁註與正文間距
marginparwidth=0mm,  %旁註寬度
  ]{geometry}
\fi
\end{verbatim}
这里使用了一个条件命令。在加载myconfig.sty宏包的时候已经新建了一个条件变量:\\
\verb+\newif\ifphone+\\
\verb+\phonefalse+
ifphone就是一个条件判断命令,这里涉及到的命令我也不大懂,总之,如果我改成:\\
\verb+\phonetrue+\\
就会自动生成适合在手机上观看的pdf。

\printendnotes