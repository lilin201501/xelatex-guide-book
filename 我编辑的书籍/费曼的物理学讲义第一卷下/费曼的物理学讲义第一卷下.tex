% !Mode:: "TeX:UTF-8"%確保文檔utf-8編碼
%新加入的命令如下:addchtoc addsectoc reduline showendnotes hlabel
%新加入的环境如下:common-format  fig scalefig xverbatim

\documentclass[11pt,oneside]{book}
\newlength{\textpt}
\setlength{\textpt}{11pt}
\newif\ifphone
\phonefalse


\usepackage{myconfig}
\usepackage{mytitle}



\begin{document}
\frontmatter

\titlea{费曼的}
\titleb{物理学讲义}
\titlec{第一卷}
\author{费曼}
\authorinfo{作者:这里等下填上费曼教授的一些信息}
\editor{万泽}
\email{a358003542@gmail.com}
\editorinfo{编者:wanze,从网络中找到前五章(不太完整)的txt,感谢这个txt的制作人。}
\version{0.03}
\titleLC

\addchtoc{前言}
\chapter*{前言}
\begin{common-format}
开头说的话

%这里空一行。

\end{common-format}


\addchtoc{目录}
\setcounter{tocdepth}{2}
\tableofcontents

\begin{common-format}
\mainmatter

\chapter{色视觉}

\section{人眼}
颜色的现象一部分是由物理世界决定的。我们讨论肥皂膜等等的颜色时,认为它们是由干涉所产生。但是当然也取决于眼睛,或是在眼睛后面和大脑中所发生的过程。物理学描述光在进入眼睛时的特性,但在进入以后,我们的感觉是光化学神经过程和心理反应的结果。

有许多有趣的现象往往和视觉联系在一起,这些现象是物理现象与生理过程的一种混合,而要完全理解各种自然现象,象我们所看到的那样,则必定超出了通常意义下的物理学范围。我们并不因为离开正题去讨论其他领域的内容而感到内疚,因为各个领域的划分,正如我们已经强调的,仅仅是由于人们的方便,而且也是一种很不自然的事。自然界对我们的这种划分并不感兴趣,而许多有趣的现象就在各个领域之间的缝隙上架起了一座座桥梁。

在第三章中我们已经一般地讨论过物理学和其他科学的关系,但现在我们将稍微详细地研究一下一个特殊的领域,在这个领域里物理学和其他科学是非常紧密地互相联系着的。那就是\uwave{视觉}这个领域。我们特别要讨论\uwave{色的视觉}。在这一章中将讨论视觉的生理方面,无论是人的还是其他动 物的。

一切都是从眼睛开始,所以, 为 了 要理解我们所看到
的是一些什么现象, 就需要具备某些有关眼睛的知识.在
下一章中, 我们将 比较详细地讨论眼睛的各干部分是怎
样工作的, 以 及它们 和神经系统怎样相互连结.
但在目
前1 我们只想简单地描述-下眼睛是怎样起 作 用 的 (困
85-1) .
光通过角膜进入眼睛s 我们 已经讨论过它如何被弯
曲而成象在眼睛背后叫做视网膜的部膜上, 以使视网膜
的不同 部分接收到从外界视场不同部分射来的光. 视网
膜不是绝对均匀的. 在我们的视场中心有一个地方, 郎 一
个 斑 点 , 当 我们试 图非 常仔细地观察物 体时 就 使 用 这 个
国 回斗
眼镜
斑点p 而且在这里我们有最大的视觉敏锐性, 叫做 史去旦垦重累 , 眼睛的旁边部分, 正象我
们 注视物体时所获得的经验立刻促使我们意识到的那样, 对于看清楚物体上的细节不象眼
睛 中央部分来得有效.
视网膜上还有另一个斑点, 输送各种信息 的神经就是从这里伸展出
去的, 那是盲点了 这里没有视网膜的敏感部分, 并且可以这样来证明: 如果我们闭上比如说
左暇, 用右眼对直观察某-物体, 然后把一个手指或者一个小的物体慢慢地从视场中移开P
那么在某个地方宫会突然消失不见.
关于这个事实, 我们所知道的唯一的实际用处就是某
个生理学家因 为 向 法国国王指出 了这-点而成 了宫廷中的宠 臣; 在国王和他的大臣们举行
的令人厌倦的例行现会上, 国王可 以用"砍掉他们的头" 即町着一个头而看着另一个头消朱

来 自 我消遣.
图 35-2 以 比较简单的形式显示视 网服内 部的放大图象- 视网版的不r.;J部分具 有 不 同
的销掏. 靠近视网膜外回所出现的那些比校商
店的物体叫 做投达到庭 , 而靠近中央四 处 除 了
这 些仔状 细胞外 ,
我 们 还 看 到 例 锥 细胞 .
这些细胞的结灼, 我 们 将放到以后去讲,
近中央四处圆锥细胞的数 日 越多
关于
越接
阳在中央凹
处, 事实上别无他物, 只有困锥细胞ß 't:们革拢
得非常紧密, 以至世这些凶锥细胞比任何其他地
方都细得多或狭得多 所 以我们必须 意 识 剖 ,
混们是用 处在视场正 中 的 以l锥主liJ出来 脱 去 的 ,
但是在外国地方, 则有另 一种细胞, 即轩状细
固 3!>-2
副院似的性的(先从下面也U
跑.
现在有趣的 是, 视网隙中每一个对于光敏
感的细胞不是用 一根纤维直接与视神经相连结, "1M是与许多 自 相连结的细胞联系在一起的.
此外, 还有好多种细胞: 有 向 视神经输送信息的细胞, 也有主要是"[司类"相互连绪在一起的
其他细胞. 实质上有 四 种 细胞, 但是我们现在不预备深入讨论这些细 节. 我们妥强调 的 L
这 就是 说 来 自 各 种 细 胞 的 信 息不 是 一 点 - 点地直接
通往大脑, 而是在视网脱 巾, 已经把来自 几个视觉接收器的信í.@.组合起来, 而使一定数量的
要-点就是光倩号 巴被"考虑"过了
信息得到整理. 这里重要的是应理解到,有些大脑功能现象是在眼睛本身中发生的.
� 35-2
U鱼依赖于光的强度
视觉最惊人的现象之一是眼睛对黑暗的适应性. 假如我们从明亮的 房 间 走 ill 黑 暗 中
去, 那么有一段时间就看不清楚, 但是渐渐地物体变得越来越清楚, 而最后在我 们 以前看不
到东西的地方能够看到一些东西 如果光的强度非常弱, 我们看到的东西是没有颜色的. 现
在我们知道,这种适应黑暗 的视觉几乎应完全归功于杆状细胞, 而适应亮光的规觉则应归功
于圆锥细胞.
作为这 方 面 的 一 个 结 果 , 有好些 现 象 我 们 可以很容易把它们理艇 为 由于功能
的这种转换, 即 圆 锥细胞和杆状细胞的共同作用 转换为只有杆状细胞作用所引起,
在许多情形中, 如果光的强度比较强, 我们就能看到颜色, 而且还会发现这些东西极其
英丽. 一个例子是, 通过望远镜观察微 弱 的 墨云时, 我们几乎总是看到它的"黑 町'象, 但是
Ob阳rva.wrj田) 的密勒先生 (W. C. MjJ.
lor) 却有耐心曾给某些星云拍摄了彩色图象. 从来没有人曾经真正 肘 肉 眼 看 到 过 这 些 颜
色, 然而这些颜色并不是人为的颜色, 只是光的强度还不足以便找们 暇时J 中 的 国锥细胞能
够看到它们 . 这些星 云 中 比较壮丽的有环状星云和 巨蟹座星云. 前者呈现荣丽的 蓝 色 内
威尔逊和帕 罗 马 天文台 (M也 WilSOll alld Palomar
核, 并带有亮红色的外晕,后者呈现蓝色的云雾,并带有明亮桔红色的细丝掺入其中.
在亮光中,忏状细胞的灵敏度显然非常低,但在黑暗中, 随着时间的流逝3它们逐渐获得
了能够看到光的本领. 人们所能适应的光的强度变化超过了一百万比一的范固. 大 自 然 并
不是只用一种 细胞来完成所有 这一切,而是把她的职能从看 到亮光的细胞, 即看到颜色 的 细
胞, 也就是因锥细胞转移到看到低强度的细胞, JllI 适 应,咱睹的细胞, 也就是籽状细胞.在这­
转移所沪 生的布趣的结果之中, 首先是没有颜色, 其/X;是颜色不同的物体其相对亮度也个

在黑暗中 曲事F状细胞来承担任务以及在中央剖必没有抨状细胞这一 事 实 的 另 一 效 应
是 , 当 我们在黑暗 中直接观察某一物体时, 就们的视觉不2>t向一边看时来得敏锐. 对弱的星
或星云, 当我们精确偏向一边观察时, 有时会比直接对着它观察更穷清楚, 因为在申央凹处
的 中心没布灵敏的籽状细胞.
越往视杨旁边回锥细胞的数 目 越是减少这个事实的另 一有趣的效应是, 当物体往一边
移去时, 即使在明亮的亮光中颜色也会消失. 试碰它的方法是朝着某一特定不变筒方向着
去,请一位朋友拿着一些有颜色的卡片从-边寇进来, 在这些卡片拿到你商前主前先试试看
判定它们是什么颜色. 人们发现, 在他能够看见这些卡片在哪里之后很久, 才能确定它们的
颜色. 在做这项试黯时, 最好从与盲点相反的一边走进来, 因为不然的话,就会被捕糊除,
回儿几乎看见了颜色, 一会儿(当经过盲点时〕什么也看不见, 然后又重新看到了颜色.
另一个有趣的现象是视网膜的外国对手运输非常敏感 凰然从我们的照角去看时不可
能看得很清楚, 但是如果有一个中且在爬功J 丽且貌们来将想到那墨有某一种东西在移动,
我们就会立即对官很敏感. 我们都会".骤起来"去寻ft亘在爬到视场边立的那个东西.
Iì 3s←S
每属握自闺测量
现在我们转到国锥细胞视觉, 即亮光 申倒视觉立来, 所涉及的问题是因锥细胞视觉最主
要的特征是什么.. 那就是颜色.
我们知道, 白先可以,JIj撞镜卦"成具有各种被*随整个光
谱, 而这些波长对我们显示出具有不同的颜笆, 这就是颜色的意!义 , 当然这是就外E回来说.
任何光源都可用光栅或榄键加以分析, 并且可 以确 定宫的光谱分布, 也就是每』放长的"份
量" 某-神光可以包含有大量的颜包,相当数量的组笆,以 放一点点费色, 等等. 这在物理
的意义上是非常精确的. 但问题是它看起来 是立金望急? 很明显, 备种不同的鲸鱼在一定

的程度 k依赖 f光的光谱分布, 但是问题在于要去找 出产生各种不同感觉的是光锵分布告。
哪峰怦征
例如, 我们必须怎样去做才能获得绿色? 大家知道, 我们可以简单地从光谱中取
绿色的那部分. 但这是否是得到银色,橙色或任何其他一种颜色的唯-方法呢?
能够产生同样表现视觉效应 的光谱分布是杏不止一种 呢 ' 答案是完全肯定的. 视觉效
应的数 目 非常有限, 而 且。事实上正如我们不久就要看 到的 那样, 它们正好是一个三维施形.
但是对于不同光源发出的光线, 我们所能 画 出 的不 同 曲 线 数 目 是无 限 的 .
的 问 题 是 2 在 什 么 情 况 下 光的不 同分 布 对于 眼 附 会 显示 出 完全 相 同 的 颜色 ?
在判断颜色方面
现在我们要讨论
个最有力的心理 物理技术是 把 眼 睛 用 作 衡捎仪器. 这就是说,
我们并不试图去定义究竟是什么造成绿 色的感觉, 或者去测量在什么情况下我们得到绿色
的感觉, 因为很清楚要这样去做是非常复杂的.
至里区全巴
我们代之去研究在fl 么 条件下两个剌激是
这样, 我 们 就 毋需判 定 在不 同 情况 下 两 个人 是不是会 得 到 同 样 的 感觉, 而 只 去
判定如果两种感觉对于一个人是相同的话, 对于另- 个人是否也相同
我 们 并不需要去判
定, 当 -个人看到某个绿色的物体时 , 在他内 心深处引起的感觉和另外某个人在他看到某个
绿色的物体时是否相同: 关于这一点我们什么也不知道.
为 了 说明这种可能性,我们可以用 一组四只带有滤色片的投影灯, 它们的亮度可以 在 一
个较宽 的 范 围 内 连续调节g 一只灯带有红色滤色片 , 在屏幕上映出 ~个红色光斑. 另一只灯
带有绿色滤 色 片 , 在屏幕上 映 出 一个绿色光斑
第二只灯带有蓝色滤色片 . 第四只灯在屏
幕上映出 - 个 白 色回圃, 它的 中央有一个黑斑. 现在如果残们开亮虹光, 并且靠近它加上→
些绿光, 我们看到, 在两种光重迭的区域里所产生的并不 是我们所说的那种绿色带红的感
觉, 而是一种新的颜色J 在我们这个特例中是黄色. 改变红光和绿光的 比 例, 我们可以得出
各种深浅不 同的橙色, 等等. 如果我们已把它配成某一种黄色,那么我们不通过这两种颜色
的混合而是把另外的一些颜色混合起来也能得到同样的黄色, 也许用黄色滤色 片 和1 白 光 - 或
者 诸如此类的东西混合起来 , 可 以 得到同样的感觉. 换句话说J 可 以 用 不止一种方法把来 自
各种滤色片的光混合起来以 形成各种颜色,
我们刚才左现的这种情况可以用解析方法表述如下,
例如
种特定的黄色可以用某
一符号 Y 表 示 , 它是某一数量的红色滤色光 (B) 和绿色滤色光(G)的"和"
在用两个数字比
如 伊 和 g 描写 (B) 和 (G)有多亮的情况下, 我们可以写下这革中黄色的一个公式.
y,-俨B+gG.
,
(35,1)
现在的问题在于, 是否通过把两种或三种 固定的不同颜色栩加在 - 起 , 就 能 做 成 所 有 各
种不同的颜色? 我们来看一看,在这方面可以得到什么结论
只 把红色和绿色混合起来, 肯
定是不能得到所有各种不同 的颜色的, 因 为比如说在这样的混合物中决不会剧现蓝色目
而在加进一点点蓝色后2 可以使所有三个斑点重迭的中央 区域看来象是
然
种十分美 妙 的 自
色. 把这三种不同的颜色混合起来, 并且现集这个中央区辙, 我们会发现, 通过改变颜色的
比例, 可以在这个区域中得到范围相当宽广的不同颜色,所以 应至pg 颜色可以 用这三种色光
的混合来做成并非是不可能 的.
我们要讨论一下这在多女型度上是真实的, 事实上这一点
基本上是正确的 : 不久我们将看到怎样把这个命题定义得更加完善一些.
为了说明 我们的观点, 在屏幕上移动各个光班, 使它们彼比都落在其他光斑的上面, 然
后试着去配制那种在第四只灯映出的圃环 中所呈现的特定 的颜色. 以前我们曾经一度认为
是"自 鱼"的元, 现在在第四只灯的光线 F都呈现 出 淡黄色.
我 们 可以借助于尝试法尽可能

适当地调节红色、 绿色 和 蓝 色 以 配制那种颜色, 并且发现, 我们能够相当 接近于这种特 殊
浓浓的"奶油"色. 所以不难相信, 我们能够配制成所有的颜色.
我们不久就要试 制 黄 色 ,
倒是在这以前3 必须指出. 有一种颜色可能很难制成. 教颜色这门课程的 人部只 制 成 所
有"鲜明 的"颜色, 但从来投有制成过棕色, 而且人们很难 回忆起曾绕着见挝在:臼的光. 事实
上, 为 了 i牛l}iTI任何舞台效果, 这种光从来没有 被使用过, 人们也从来没有看到使 用 棕色光的
聚光灯3 所 以 我 们 想 3 或许不可能制成棕色光. 为了 弄iIl始是否可 以创成棕色光 , 我\')指
出 , 棕色光仅仅是这样一种光, 如果没有背景的衬托, 我们就不习惯于看它. 事实上, 我们
能够把一些红光和黄光混合起来而制成棕色光. 为了证明我们看到 的是棕色光, 只要培加
阴环背景的亮度, 相对于这个背景, 我们看 到的正是这种光, 它就是我们所说的棕色『 棕色
在靠近比较明 亮的 背景时 , 总是
→
种深暗的 颜色
.
棕色的 特 征很容劫改变
比方悦,如果我
们从中取出 一些绿色J 就得到略带红的棕色, 这显然是f种巧克力似的红棕色. 如果 拥 进
更多的绿色, 那么我们就相应得到那种令人讨厌的所有的军队制服都由它集成的颜色
但
是来 自 这种颜色的光本身并不那样令人讨厌, 它 是略带黄的绿色,但是在明亮的背景眩衬托
下就显得非常可i拌了.
现在我们在第四只 灯的前面放置一抉黄色滤色片,并试图配制出这种颜色
(光的强rr.E
当然必须限于各种不 同 的灯的范围 之 内 , 我们不可能去配制大明亮的东西, 因为我们的灯没
有足够的功率, ) 然而税们能够配制出黄色, 为此只要把绿色和红色提合起来, 甚 年 lJ11 L
点点蓝鱼, 使它更加完美
-
或许我们 已经相信, 在 良好的条件下, 能够完 美地配制出 任 何结
寇的颜色.
现在找们 来讨论颜色混合的定律. 第一,我们曾发现不同光谱分帝的光能够产生同样的
颜色3 其次,我们 曾看到"任何"颜色可以通过把三种特殊的颜色t 红‘蓝和绿加在一起商配刨
出 来. 混合的颜色最有趣的特点是, 设有某一种觉, 我们把它叫做 X, 又设丛眼睛看章宫和
Y 设有什么 区 别 t它可以是一种 与 Y 不同的光谱分布, 但它 建星空 与 Y 是不可区别的) , 那
么我们称这些颜色是"相等"的, 这是从这于意义上来说, �p眼睛看到官们是相等的, 并且 坷
以写成
X 严Y. ,
(35 ,a)
颜色的主要定律之一是: 如果商个光谱分布是不可区别的, 我的给每一个加上某→种光, 比
如 说 Z (如果我们 写成 X +Z, 就意味着把这两种光照射在同一个斑点上) , 然后再取 Y 并
加上同样数量的另-神光 Z. 那么这些盖跑盖全鲤虫是主豆豆组lJ(]i
X十Z国Y+Z;
ι
(85.8)
哉们刚才已经配制出黄色: 如果现在担轮红色翩光 照射到全部物体1上, 它们仍然'阻够匹配.
所以对 已 经 匹 配的光, 加上任何其他的光, 留节的仍然是相匹配的先 . 换句话说, 我们可以
把所有这些颜色现象总结起来, 两种色先在相同情况下彼此靠近观察时, 如果一经峰髓, 那
么这种匹配 将继续保持下去, 而且在任何其他的颜色混合情形 中
种先 可 以 用 另 一 种
光来代替. 事实上, 这证 明 了 -个非常重要和有趣闹情况, 即包党的这种 匹 配 不 依 载 于
眼睛在观察那个时刻的特 使 我仍知迢, 如果我们�时间地注视一+现亮 的 红 色 表 面 或
者明亮的红光, 然后去看一张 白 纸, 那么它看上去略带绿色, 而且其他颜色 也 会 因 我 'U',
时 间 地注视 着 明亮的 红 色而 走 样
.
如果我们 现在报 雨 神 颜 色 ,
铺 如黄 色相 匹 配, 我 们 注
视它们, 然后长时间地去注视一个明亮的 红色表面, 然后再回过来看黄毡, 这时 官 砖 上 去

不是黄色 的 了 ,
我不知道它看上去是什么颜色,
1目者早在不会是黄色. 虽然 如 此 ,
这曲黄
色看上去仍然是PI;配的, 因 比 , 由 于眼睛能适应光的不 同 强 度 , 颜色的{I!;配仍然 发 生 it 用 , 除
非一个明显�Ýí如j列, 那就是当 我们进入一个领域, 在邱坦光的强度如此之弱, 以致我们 必 须
从圆锥细胞转移到杆状细胞 的时候, 这时原来相配阻 的颜色不是相 匹 配 的 了 , 因为我们运用
了不同的系统
颜色混合的第三个原理是
是红、 绿 和蓝三种色光.
经组二致座19.盔豆J;! !Il二致盟鱼组盛 , 在我们的情况中, 就
适 当地把这三种颜 色 混合在一起, 找们就能够配制出任何一种颜
色,:æ象我们在前面两个例于中所表明的那样. 此 外 , 这将定律在数学上也非常有趣. 对于
那些对这方商的数学感兴趣 的人来说, 情况是这样z 假设我们取红、 绿和蓝三种颜色, 用 A、
B 和 10 来标记, 并且把它们叫作原色. 于是任何一种黄色都可以向这主种颜色的一定数量
制成: 比 如 由 颜色 A 的数量 a, 颜色 B 的数量 b 和颜色 U 的数挝 c 制成 X,
X � .A+bB + oO.
现 在假设 另 一种颜色 Y 由 同样这 三 种颜 色 制成
Y - a'A+b'B+ o'O.
(85.4)
(85.5�
于是我们发现这两种光的混合物 (这是我们在前 面 已经提到过的那些定律的结论之-.) 可 以
通过取 X 和 Y 的分罩之和来求得:
Z � X +Y回 (α十.').11+ (b十b')B+ (。十.')0.
(3<5.6)
这正好象数 学 中 的矢量加法, 其 中 怡, b, 0)是→个矢量 的分量, 而 (a'� þ', c') 是另一欠量的
分量, 这时新的J'(; Z 就是这些矢量的"和"
这个问题--直 在 引 起物理学家和数学家们的注
意. 事实上, 薛)Îf湾 曾经 写过一篇有关色觉的精彩论立 , 他 在这篇论文中发展了这个可用于
颜色棍合的矢量分析理论.
现在的问题是, 哪些是所要用 的正确的原色? 就光的混合来说,是没有象正确的原色这
类东西的. 对于实用 的 目 前, 可能有三种颜色在得到比较多 的混合色方面比其他颜色更为有
用, 但是我们现在不þt论 这个问题. 无论哪三种不同 的醺色气 总能用正确 的 比例混合起章以
产生无论哪种颜色. 我们是不是能移证明 这一奇妙的事实呢?若我们在投:ID灯 中改用 江色、
蓝色和黄色来代替红色、绿色和蓝色. 我们是否能用 红色、蓝色和黄色配制成比如说绿色呢?
以各种比例把这三种颜色混合起来, 我们得到范围相当大的一系列不同颜色, 它们几乎
遍及整个光谱. 但是事实上, 经过大量的尝试和失败, 我们发现没有什么东西曾经看上去有
点象绿色. 问题在于我们是否能配制出绿色? 回答是肯定的. 那么如何配制呢?
鱼主意组到应差里鲤鱼堡垒」二 ,
虽二皇IJ:
我们就能用黄色和蓝色的某一混合色来与之相匹配! 就这
样,我们确实把它们匹配了, 只是除去一点, 那就是我们不得不欺骗自 己一下, 把红色放到另
-边去.
但是既然我们掌握了某种数学技巧, 那就能理解到我们实 际上所证明的并不是说
X 总能从比如红色、蓝色和黄色配制,而是在把缸色放在另 一边之后,我们发现红色加上 X
可 以 从 蓝 色和黄色中配制 出 来 . 把它 放 在 等式 的 另 一 边 , 这 可 以 僻释 为 它 是一个 兔览室室 ,
所以如果我们允许象(85.4) 那样的等式中 的系数既可以是正怕也可以是负的, 以及担负的
数量解 释 为把 它组型是二边 , 那 么 任 何 颜色 都 可 以 用任 何 三 种 颜色 来 配 制 , 因 而并没有 象
"
这种
基 本 的原色这 样 的东 西 .
"
我们可以间, 是不是有三种颜色, 官们对于所有混合只有正的数 量. 回 答 是 否 寇 的.

每-组三原色都对某些颜色要求负的数量, 因 而也就没有用以 定义-种原色的唯 一 方 法.
在初等教材中, 它们按说成是纽色、绿色和蓝色, 但那只是因为用这些原1'1对有些组合露宿
用 负号 即可得到较宽的颜色范 围 而 已
9 35-4 色 晶 圆
我们现在从数学的水平作为一个几何学的命题来讨论颜色的组合. 假如任何一种颜色
能用等式 (3õ.4) 来表示, 那么我们 可 以 把它当你一个空间矢量来作图,沿着三根坐标轴画出
矶 b 和 c 的数值, 于是-利l颜色就是二个点-
如果另-种颜色是 民 b'、 c', 那么这种颜色
就处在图中别 的什么地方. 我们知道, 这两者之和就是把它们作为矢量相加而得到的颜色.
我们可以 把这个 图解简化一下, 并且通过如市的观察把所有东西表示在一个平窗上 如果我
们有某种颜色的光, 而且仅仅把 叭 b 和 c 都加倍, 也就是说, 使它们都以同样的比例增强,
那么它还是 同一种颜色, 只是更亮了一些. 所以如果我们约定把所有东西都化为 鲤笠堕差
强, 那么我们就能把所有东西都投影到一个乎商上,
这在困 SIí-4 中就已这样做r. 由化可知,由给定的
两种颜色以某一比例棍合而琪的任何颜色, 将处在
联结这两 点的直线上某一地方.
例如; :li-f'比直十
的混合色将处在它们之阔 的 中 成, 一种色的 1/4 和
另户种色的 3/4 将出现在从一 点 到 71 -' 点倒 1/4
处, 依此类推. 如果我们以蓝色、摄色相红色作为原
色, 那么我们看到所有能用正的系数配制而成的颜
色都处在虚线三刽形之 内 , 这几乎包 含了所有我们
能 够 看 到 的 颜色 , 因 为 这些颜色都 包 围 在 以 幽钱为
边界的 钟形丽职之 中 . 这个面积是从哪里 来 的 呢?
有人曾经 把所有 我们能够看到的癫色 与主种特殊颜
色非常仔细地比较过. 但是我们不必核对所能看见
的 盟主庭鱼 , 而 只要核对纯光谱色, 即 光懵线. 任


定所有颜色的位置 . 这 是 找 出 两 条 弯 曲 边界线的方法. 宫是纯光谱色的轨迹. 任何其他颜
色现在当然都可通过光谱线的栩如得到 , 因而我们发现, 把 曲 线的-个部分和另一部升联结
起来所能产生的任何东西都是一种 自 然 界 中可 以 得到的颜色. 团 中 的直线把光谱 中 紫 色 的
最外一端和红色的最外一端联 系 起来. 这是紫红色的轨迹
在边界之内是那些可 以 用 各 种
先配制的颜色, 而在它之外是不能用光配制的颜色, 这些颜色从来没有人看到过1除非在余
象中町能看 到! ) .
色视觉的相l制
& 35-5
现在, 事情的下-个方町娃这样的问题
为什么颜色的行为竟是如此, 由扬 和主姆霍
兹 提 出 的最简单的现论, 假设眼睛中有三种不同 的能接收光的包素, 它们有不同的吸收光
谱 , 因此一种色素比如说在红色区吸收很强 , 揭 一种色素在蓝色区吸收很强 , 再 一种色素在
绿色区吸收很强.
于是当我们把光照射到它们上面时, 就会在三个区域 内 得到 不 同 敛 璋 的
吸收, 而这三部分伯息在大脑 中、 眼 睛 中 , 或某个地方 以某种方式调节, 以 确 定 这是什么颜
色. 很容 易 证 明 , 所有颜色的混合法则都是这一假说的绪果
关于这个问题曾经有过相当
多 的争论, 因为接下来的 问 越 当 然就是要找出这种色素各 自 的吸收特性曲线. 遗憾的是,我
们友剖, 由 于我 们 能 以 任 何息意的方式变换颜色坐标 , 所 以用混合颜色的实验只能找到吸收
曲线的各种线性组合, 而不是个别色素的吸收曲线.
人们曾用 各种方法试图在得一条特殊
的 曲 线 , 用'ι确实能够捎述眼硝的某种特殊的物理性质. 这种曲线之一是图 35-3 所示的室
里跑线 在这张图上有两条曲线
条是对于处在黑暗 中 的眼 睛 , 另一条是对于处在亮光中
的眼睛: 后者是因锥细胞的亮度曲线. 它是这样测得的, 即-种色光, 其最小数量应是多少
才能使眼睛价好看到它. 这条曲 线 表 明 眼硝在不同光谱区内的灵敏度有多高. 另外, 还有
一个非常有 趣的方法 可 以 测 扯这条曲线.
假如我们取两种颜色, 并使它们显示在同
区域
内 , 再 把 臼们-.个 隔一个来回摇 晃, 那 么如果频率过低, 我们就能看到一种 晃动 . 然而, 随着
频率的地灿,这种晃动终于会在某-频率消失, 这个频率依赖于光的亮度,例如说 每秒来四
16 1X . 现在在 16 周频率时, 如躲我们相对于一种颜色调 节另一种颜色的亮度或强肢 , 那么
到达来-强度, 晃 动就会消失.
频率, 以 便能够看到颜色的晃动.
时的颜色晃动
要用这样调节好的亮度来得到晃动, 就必须回到低得多的
所 以我们 得到频率较高时的所谓亮度晃动以及频本较低
利用这种晃动技术可以使两种颜色在"苑度相等"上相 匹配. 所得结果几乎
与测挝眼附时使用 圆锥细胞观察微弱光线 的灵敏度阙值所得前一样, 但不是完全相 同.
大
多数研究工作者在这方面都用 晃动系统作为亮度 曲 线的定义.
现在, 如躲眼睛 中有三种对颜色灵敏的色肃 , 那 么 问题就是要确定每)种色素 的吸收先
谐的轮廓. 怎样做呢? 我们知道, 有些人一-男性 人 口 中 的百分之八,女性人 口 中 的百分之
朵也盲. 大多数色盲或色视觉不 正常的人对原色的变化 与其他本相比具有不同
程度的灵敏度, 但他们仍需要用三种颜 色来避行匹配 然而, 有 一些人被称为三垒芷鱼直亨
'#点五
(diohroma.创 , 对 于这种人任 何颜色只要用理直原色就可以 匹配.
于是一个明显的 设想E
他们缺少三种色素中 的一种. 如果我们能够找到三 种具有不同颜色浪合 法则的二包住色商
豆 是缺少 重垒 的着色作用 因 而
者, 那么一种应是缺少 丝鱼 , 封 一种应是缺少 壁垒 , 再一种b
通过对所有这些色盲 类型 的 调 址 , 我 们 就能确定三条曲 主,! 结果发现果然有兰科 l类型的一色
性色西 ; Wj种是一般的类 型, 第三种是假稀少 的类型, 从这三种类型就 可 以 推 断 出 色 幸 的 吸


臼 能-6 表 示一种特殊类型的称为患绿色盲者的损色浪告. 对他来说,相同颜色的毓遭
不是一个]个点, 而是一条一条直线, 沿着每一条直绩, -色是相同的. 如果象这种理论所
说 的 , 他缺少三部分倩息中之一是正确的话, 那在所有这些直线应该相交于 -点. 如果我们
在这张图上仔细地进行测量, 那么 它们理I�完全相奕. 因此, 很明显, 这是数学家设想出来
的, 并不表示真实的数据! 事实上,如果我们看」下具有真实数据的最新文献,就会发现, 因
35-6 中所有直线 的焦点劳不准确地位于恰当的位置 上. 利用上田中的直线不可能 找 出 合
理 的 光谱号 在不同区域内, 我伺需要用 负的和E 伪吸收
但是如果用余斯托伐(Yus阳va) 的
新的敛掘, 那么就会发现每 一 条 吸 收 曲 线 到 处 辄 是飞 正 的.
图 且在7 表示另一种色盲, 即 息 红色盲的人的情
在这种
况, 它在革近边 界 曲 线 的 红端有一个焦点.
情况下, 余斯托伐近似地得到 了 同一个位置.
,
利用
三种不同的色育, 三种色素的响应曲线最后接确寇
了下来, 如 阳 85--8 所示. 这是最终的结果吗' 或许
'
是, 但对下列各点还是有一些问题,那就是三种色素
的想法是否正确, 色盲是否 由 于一种色素而引起的
结果, 甚至关于色盲的强色混合数据是否正确. 不
同的研究工作者得出 不 同 的 结果. 这个领域仍在不

弱光 , 何不必 用 许 多 色 亲 来观 察颜色 . �.许东的想法是
芷鱼室里至旦旦旦 , 并用 不ii;
何种方法来测量立 . 他是这样做的. 有→种仪棒叫栓眼镜2 它 把.�通过眼球的晶状体送进
/lR附J 然J,iI!l12射回来的光*然在一起.
使用这种仪器人们可以测量出有 多少先被反射回
来 回 这样, 我们对通过色素两次的光f被眼球的背层所反肘, 并且再饮通过圆锥细胞的色素
出 来 的 ) 测 量 了 它 的反射系数.
自 然界并不总是设计得这样美妙的.
但剧锥 细胞有岖地被
设计成这样, 使得进入团锥细胞的先被来回反射, 最后向 下钻进顶端处的微小的灵敏点 中 .
鬼一直往下进入灵敏点, 在其底部被反射J 而在穿过相当数量的包视觉色素后重新反射回
来; 而且,通过观察 中央凹, 那里就有杆状细胞, 这样入们就不会被视细所搞混. 但是视网肢
的颜色很早以前就 已被人们现绑到 宫是F种带橙色的粉红色3 然后又看到了所有的血管和
背后物质的颜色, 等等. 我们怎么知道看到的就悬这种色素呢? 回答是z 首先, 我们找一个
患有包宙的人, 他的包素较少, 因 此很容易剖他进行分析. 其次, 各种色索象视铺一样 , 当被
光漂白 后强度就有所改变1 当我们祖先照射到.t们身上时, 它们就改变浓度. 所 以 , 在观察

颜色不是光本身的物理学问题. 颜色是感觉, 不同颜色的感觉在不同情况中是卒帽翩.
举例来说, 假如我们有-种由 白 光和红光交叉迭加而成的粉红色光 (用 白 色和 红 色 所 能 配
制的 跟然总是粉红色) , 我们就可 以证明白光可以显示为蓝色
如果我们把一个物体放在光
束 中 , 它投射两个影子一一一个单独为白 光所照亮, 而另一个为红光所照亮. 对大多数人来
说 ! 物体的"白 色"影子看I 去是蓝色, 但是如果我们不断扩大这个影于, 直到它遮盖住整个
屏幕, 那么我们将看 到它突然显示为白色, 而不是蓝色' 将红光、 黄光和白光棍古时,我们能
够得到性质 与此相同的其他放r,y: . 红光、 黄光和自 )�只能产生橙黄鱼, 等等. 所以如果我们
把这些光大藏等量地混合在- 起, 我们只能得到橙色光. 然丽, 当在这束光中tII:射 出不同科'
类的影于时3 那 么 由 于颜色的各种选加, 人们得到一连串 美丽的麟色, 这些颜色并不存在于
先本身 之 中 〔官只是橙色) , 而只存在于我们的感觉之中
光束中的"物理"鲸鱼不 同 .
我们清楚地看到许多颜色完全与
重要的是�意识到视网膜 已经在"考虑"光s 官正在把一个区域
中所能看到的 东西同另 一个区域中所看到的东西道行比枝, 且然是不自觉的. 至于宫是怎
样送行 的 》 我们 在这万副所知道的)切将在 于 一 意中进行讨论





%这里空一行

\end{common-format}
\end{document}



