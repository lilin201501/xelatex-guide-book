% !Mode:: "TeX:UTF-8"%確保文檔utf-8編碼
\documentclass[a4paper,addpoints,11pt]{exam}
\newlength{\textpt}
\setlength{\textpt}{11pt}

\usepackage{mathtools} 
\usepackage{examconfig}
\pagestyle{headandfoot}
\firstpageheader{Math}{First Exam}{\today}
\firstpagefootrule
\runningheader{}{}{}
\firstpagefooter{}{}{\thepage}
\runningfooter{}{}{\thepage}
\runningfootrule

\newcommand{\shortanswerline}{\rule[-2pt]{50pt}{0.4pt}}


\begin{document}
\begin{common-format}


%\begin{center}
%\fbox{\fbox{\parbox{5.5in}{\centering
%请认真作答,不要交头接耳,如果写不下了就写在背面。}}}
%\end{center}
\noindent\makebox[0.49\textwidth]{机构名字:\ \hrulefill}
\makebox[0.49\textwidth]{你的名字:\ \hrulefill}


\section{选择题}
\begin{questions}
\question
设集合$ A $和$B$都是坐标平面上的点集$\{(x,y)|x\in R, y\in R\}$,映射$f:A→B$把集合$A$中的元素$(x,y)$映射成集合$B$中的元素$(x+y,x-y)$,则在映射$f$下,像$(2,1)$的原像是(\hspace{2em})。\\
\begin{oneparchoices}
\choice $ (3,1) $
\choice $ (\frac{3}{2},\frac{1}{2})$
\choice $ (\frac{3}{2},-\frac{1}{2}) $
\choice $ (1,3) $
\end{oneparchoices}

\question
若全集为$U$,集合$A$,$B$是$U$的子集,定义$A$与$B$的运算:$A\ast B=\{x|x\in A, \text{或} x \in B, \text{且} x\notin A\cap B\}$,则$(A\ast B)\ast A=$(\hspace{2em})。\\
\begin{oneparchoices}
\choice $ A $
\choice $ B$
\choice $ (\complement _U A)\cap B $
\choice $ A\cap (\complement _U B) $
\end{oneparchoices}



\question
已知集合$ M=\{(x,y)|y=\sqrt{9-x^2}\} $,$ N=\{(x,y)|y=x+b $,且$ M\cap N=\emptyset $,则$ b $应满足的条件是(\hspace{2em})。\\
\begin{oneparchoices}
\choice $ \left| b \right| \geq 3\sqrt{2} $
\choice $ 0< b <\sqrt{2}$
\choice $ -3\leq b \leq 3\sqrt{2} $
\choice $ b>3\sqrt{2} \text{或} b<-3 $
\end{oneparchoices}


\question
集合$ M =\{x|tan^2x=1\}$,$ N=\{x|cos2x=0\} $,则$ M $,$ N $的关系是(\hspace{2em})。
\begin{oneparchoices}
\choice $ M\supsetneqq N $
\choice $ M\subsetneqq N $
\choice $ M=N $
\choice $ M\cap N=\emptyset $
\end{oneparchoices}
\end{questions}


\section{填空题}
\begin{questions}
\question
若全集$ I=R $,$ f(x) $,$ g(x) $均为$ x $的二次函数,且$ P=\{x|f(x)<0\} $,$ Q=\{x|g(x)\geqslant 0\}  $,则不等式组$ \begin{cases} f(x)<0 \\ g(x)<0  \end{cases} $的解集可用$ P $,$ Q $表示为\rule[-2pt]{50pt}{0.4pt} 。

\question
从自然数$ 1\sim20 $这20个数中,任取2个数相加,得到的和作为集合$ M $的元素,则$ M $的非空真子集共有\shortanswerline 个。

\question
集合$ \left\{ a, \frac{b}{a}, 1 \right\} $,也可以表示为$ \left\{ a^2, a+b, 0 \right\} $,则$ a^{2005}+b^{2006}= $\shortanswerline 。
\end{questions}




\newpage
\section{解答题}
\begin{questions}
\question
设$ A=\{x|x^3-7x^2+14x-8=0\}, B=\{x|x^3+2x^2-c^2x-2c^2=0, c>0\} $。
\begin{parts}
\part
求$ A $,$ B $的各个元素;
\part
以集合$ A\cup B $的任意元素$ a $,$ b $作为二次方程$ x^2+px+q=0 $的两个根,试在$ f(x)=x^2+px+q $的最小值中,求出最大的或最小的。
\end{parts}
%\makeemptybox{16em}
\vspace{\stretch{1}}
\question
某班25名学生参加数学、物理、化学三种竞赛,已知:(1)每名学生至少参加一科;(2)在没有参加数学竞赛的学生中,参加物理竞赛的人数是参加化学竞赛的人数的2倍;(3)只参加数学竞赛的学生比余下的学生中参加数学竞赛的人数多1名;(4)在只参加一科的学生中,有一半没有参加数学竞赛,问:共有多少名学生只参加物理竞赛?
\vspace{\stretch{1}}
\end{questions}





\end{common-format}
\end{document}



