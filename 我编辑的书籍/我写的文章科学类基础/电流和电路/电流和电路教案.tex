% !Mode:: "TeX:UTF-8"%確保文檔utf-8編碼
%新加入的命令如下: reduline showendnotes 
%新加入的环境如下:solution solutionorbox solutionorlines solutionordottedlines

\documentclass[12pt]{exam}
\newlength{\textpt}
\setlength{\textpt}{12pt}

\usepackage{teachingplan}
\usepackage{wallpaper}
\usepackage{comment}
\usetikzlibrary{positioning}

%输出方案 
%1.自己用答案,知识点等都有  ——老师版
%2.给学生用基础差的可以考虑加上知识点,但不写答案——基础差版
%3.练习题加提高题——学生版
%4.练习题加提高题加综合题——学霸版

%写上答案或者不写上答案%1  
\printanswers  

%提高题%1  4
%\includecomment{improveexercises}
\excludecomment{improveexercises}
%综合题%1  4  5
%\includecomment{advanceexercises}
\excludecomment{advanceexercises}


\CenterWallPaper{1}{教案模板-2.pdf}
\newcommand{\shortanswerline}{\rule[-2pt]{50pt}{0.4pt}}
\renewcommand{\solutiontitle}{\noindent\textbf{}}

\newcommand{\keti}{热和能}
\newcommand{\zhongdian}{1.分子热运动 2.内能 3.比热容\\  4.热机 5.能量的转化和守恒}
\renewcommand{\section}[1]{{\large\sffamily  #1} \par}
\renewcommand{\subsection}[1]{{\normalsize\sffamily  #1} \par}
\newcommand{\leftnote}[1]{\marginpar{\ttfamily\fontsize{10}{10}\selectfont #1}}



\newcommand{\answer}[2][50pt]{{\setlength{\answerlinelength}{
#1} \answerline*[#2]}}

\begin{document}
\ThisCenterWallPaper{1}{教案模板-1.pdf}
\vspace*{80pt}
\keti \par
\zhongdian \par
\section{电荷}
电荷:
  带电体:物体有了吸引轻小物体的性质,我们就说是物体带了电(荷)。这样的物体叫做带电体。
  自然界只有两种电荷——被丝绸摩擦过的玻璃棒带的电荷是正电荷(+);被毛皮摩擦过的橡胶棒上带的电荷叫做负电荷(-)。
电荷间的相互作用:同种电荷互相排斥,异种电荷互相吸引。
带电体既能吸引不带电的轻小物体,又能吸引带异种电荷的带电体。
电荷:电荷的多少叫做电荷量,简称电荷,符号是Q。电荷的单位是库仑(C)。
2、检验物体带电的方法:
  ①使用验电器。
   验电器的构造:金属球、金属杆、金属箔。
   验电器的原理:同种电荷相互排斥。
   从验电器张角的大小,可以判断所带电荷的多少。但验电器不能检验带电体带的是正电荷还是负电荷。
  ②利用电荷间的相互作用。
  ③利用带电体能吸引轻小物体的性质。
3、使物体带电的方法:
  (1)摩擦起电:
   定义:用摩擦的方法使物体带电。
   背景:
     宇宙是由物质组成的,物质是由分子组成的,分子是由原子组成的,原子是由位于中心的原子核和核外的电子组成的,原子核的质量比电子的大得多,几乎集中了原子的全部质量,原子核带正电,电子带负电,电子在原子核的吸引下,绕核高速运动。原子核又是由质子和中子组成的,其中质子带正电,中子不带电。
     在各种带电微粒中,电子电荷量的大小是最小的,人们把最小电荷叫做元电荷,通常用符号e表示。任何带电体所带电荷都是e的整数倍。6.25×1018个电子所带电荷等于1C。
     在通常情况下,原子核所带的正电荷与核外所有电子总共带的负电荷在数量上相等,整个原子呈中性,也就是原子对外不显带电的性质。
    原因:由于不同物质原子核束缚电子的本领不同。两个物体相互摩擦时,原子核束缚电子的本领弱的物体,要失去电子,因缺少电子而带正电,原子核束缚电子的本领强的物体,要得到电子,因为有了多余电子而带等量的负电。
   注意:①在摩擦起电的过程中只能转移带负电荷的电子;
      ②摩擦起电的两个物体将带上等量异种电荷;
     ③由同种物质组成的两物体摩擦不会起电;
      ④摩擦起电并不是创造电荷,只是电荷从一个物体转移到另一个物体,使正负电荷分开,但电荷总量守恒。
  能量转化:机械能-→电能
 (2)接触带电:物体和带电体接触带了电。(接触带电后的两个物体将带上同种电荷)
 (3)感应带电:由于带电体的作用,使带电体附近的物体带电。
4、中和:放在一起的等量异种电荷完全抵消的现象。
  如果物体所带正、负电量不等,也会发生中和现象。这时,带电量多的物体先用部分电荷和带电量少的物体中和,剩余的电荷可使两物体带同种电荷。
  中和不是意味着等量正负电荷被消灭,实际上电荷总量保持不变,只是等量的正负电荷使物体整体显不出电性。
5、导体和绝缘体:
容易导电的物体叫做导体;不容易导电的物体叫做绝缘体。
常见的导体:金属、石墨、人体、大地、湿润的物体、含杂质的水、酸碱盐的水溶液等。
常见的绝缘体:橡胶、玻璃、塑料、油、陶瓷、纯水、空气等。
  导体容易导电的原因:导体中有大量的自由电荷(既可能是正电荷也可能是负电荷),它们可以脱离原子核的束缚,而在导体内部自由移动。
  绝缘体不容易导电的原因:在绝缘体中电荷几乎都被束缚在原子范围内,不能自由移动。(绝缘体中有电荷,只是电荷不能自由移动)
  金属导体容易导电靠的是自由电子;酸碱盐的水溶液容易导电靠的是正负离子。
  导体和绝缘体之间并没有绝对的界限,在一定条件下可相互转化。一定条件下,绝缘体也可变为导体。
  绝缘体不能导电但能带电。



1、自然界的两种电荷:正电荷和负电荷
2、摩擦起电的原因:电荷发生了转移
3、电流的形成:电  荷  的定向移动形成电流。(任何电荷的定向移动都会形成电流)。
2、电流的方向:把 正 电 荷 定向移动的方向规定为电流方向。(而负电荷定向移动的方向和正电荷移动的方向相反,即与电流方向 相 反 )。
3、电荷[量](Q):电 荷 的多少叫电量。(单位: 库 仑 (c))。
4、电流的大小用电流强度(简称电流)表示。电流强度等于1秒钟内通过导体横截面的电量。
5、定义式:,(),式中I是电流、单位是:安(A);Q是电量、单位:库仑(C);t是通电时间、单位是:秒(S)。
6、电流I的单位是:国际单位是:安培(A);常用单位是:毫安(mA)、微安(µA)。1安培=103毫安=106微安(µA)。
7、电路组成:由电源、导线、开关和用电器组成。
8、电路有三种状态:(1) 通路:接通的电路叫通路;
          (2) 开路:断开的电路叫开路;
          (3) 短路:直接把导线接在电源两极上的电路叫短路。
9、电路图:用符号表示电路连接的图叫电路图。
10、串联:把元件逐个顺序连接起来,叫串联。(电路中任意一处断开,电路中都没有电流通过)
11、并联:把元件并列地连接起来,叫并联。(并联电路中各个支路是互不影响的)
12、导体:容易导电的物体叫导体。如:金属,人体,大地,酸、碱、盐的水溶液等。
13、绝缘体:不容易导电的物体叫绝缘体。如:橡胶,玻璃,陶瓷,塑料,空气,油,纯水等。
14、导体和绝缘体是没有绝对的界限,在一定条件下可以互相转化
15、测量电流的仪表是:电流表,
它的使用规则是:①电流表要  串 联  在电路中;
        ②接线柱的接法要正确,使电流从“+”接线柱入,从“-”接线柱出;
        ③被测电流不要超过电流表的 量 程 ;
        ④绝对不允许不经过用电器而把电流表连到 电源的两极 上。
16、实验室中常用的电流表有两个量程:①0~0.6安,每小格表示的电流值是0.02安;
                  ②0~3安,每小格表示的电流值是0.1安。
17、电压(U):电压是使电路中形成电流的原因,电源是提供电压的装置。电源是把其他形式的能转化为 电 能。如干电池是把 化 学 能转化为电能。发电机则由 机 械 能转化为电能。
18、有持续电流的条件:必须有 电 源 和 电 路 闭 合。
19、电压U的单位是:国际单位是:伏 特(V);常用单位是:千伏(KV)、毫伏(mV)、微伏(µV)。1千伏= 10 3 伏= 106 毫伏= 109 微伏。
20、测量电压的仪表是: 电 压 表,
它的使用规则是:① 电压表要 并 联 在电路中;
        ② 接线柱的接法要正确,使电流从“+”接线柱入,从“-”接线柱出;
        ③ 被测 电 压 不要超过电压表的量程;
实验室中常用的电压表有两个量程:① 0~3伏,每小格表示的电压值是 0. 1 伏;  
                    ② 0~15伏,每小格表示的电压值是 0. 5 伏。
22、熟记的电压值:
①1节干电池的电压 1. 5 伏;
②1节铅蓄电池电压是 2  伏;
③家庭照明电压为220伏;
④安全电压是:不高于36伏;
⑤工业电压380伏。
23、电阻(R):表示导体对  电 流 的 阻 碍 作用。(导体如果对电流的阻碍作用越大,那么电阻就 越 大 ,而通过导体的电流就  越 小 )。
24、电阻(R)的单位:国际单位:欧姆(Ω);常用的单位有:兆欧(MΩ)、千欧(KΩ)。		    1兆欧=103千欧;       1千欧=103欧。
25、决定电阻大小的因素:导体的 电 阻 是导体本身的一种 性 质 ,它的大小决定于导体的:材料、长度、横截面积和温度。(电阻与加在导体两端的 电 压 和通过的 电 流 无关)
26、变阻器:(滑动变阻器和变阻箱)
   (1)滑动变阻器:
原理:改变电阻线在电路中的 长  度 来改变电阻的。
作用:通过改变接入电路中的 电 阻 来改变电路中的 电 流 和 电 压。
铭牌:如一个滑动变阻器标有“50Ω2A”表示的意义是:最大阻值 是50Ω,允许通过的最大电流是2A。
正确使用:A  应 串 联 在电路中使用;
       B  接线要“一上一下”;
       C 通电前应把阻值调至最大的地方。
   (2)变阻箱:是能够表示出 电 阻 值 的变阻器。



\section{练习题}














\begin{improveexercises}
\section{提高题}

\end{improveexercises}

\begin{advanceexercises}
\section{综合题}

\end{advanceexercises}

\ThisCenterWallPaper{1}{教案模板-3.pdf}

\end{document}



