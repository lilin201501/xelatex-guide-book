% !Mode:: "TeX:UTF-8"%確保文檔utf-8編碼
%新加入的命令如下:addchtoc addsectoc reduline printendnotes
%新加入的环境如下:common-format  fig scalefig 

\documentclass[11pt,oneside]{book}
\newlength{\textpt}
\setlength{\textpt}{11pt} 
\newif\ifphone
\phonefalse



\usepackage{myconfig}
\usepackage{mytitle}

\makeatletter
\@addtoreset{@EndnoteCounter}{chapter}
\makeatother

\begin{document}
\frontmatter   

\titlea{意义}
\titleb{人类的永恒追求}
\titlec{一种存在疗法的介绍}
\author{V•E•弗兰克尔}
\authorinfo{翻译:赵可式,沈锦惠。}
\editor{wanze}
\email{a358003542@gmail.com}
\editorinfo{根据网上txt核对pdf版本精确校对排版而成,并加上注解。这个翻译版本《活出意义来》比另外一个版本《追寻生命的意义》明显要好些。需要tex源码的请email我。}
\version{1.2}
\titleLC

\addchtoc{目录}
\setcounter{tocdepth}{1}    
\tableofcontents

\begin{common-format}
\mainmatter 

\addchtoc{代序}
\chapter*{代序}
{\large 戈登•欧伯}

作者弗兰克博士,是一位精神医学家。他经常问遭逢巨痛的病人:“你为什么不自杀?”病人的答案,通常可以为他提供治疗的线索。譬如,有的是为了子女,有的是因为某项才能尚待发挥,有的则可能只是为了保存一个珍贵难忘的回忆。利用这些纤弱的细丝,为一个伤心人编织出意义和责任——这便是“意义治疗法”(logotherapy)的目标和挑战,也正是弗兰克博士在现代存在分析上的创见。

在本书中,弗兰克博士现身说法,详述他如何由亲身经验,发明“意义治疗法”。他曾是集中营里的囚犯,漫长的牢狱生涯,使得他除了一息尚存之外别无余物。他的双亲、哥哥、妻子,不是死在牢营里,就是被送入煤气间。一家人全都死了,仅剩下他和妹妹。像这样一个丧尽一切,饱受饥寒凌虐,随时都有死亡之虞的人,怎么会觉得人生还值得活下去呢?一位曾亲身经历过这种惨绝人寰遭遇的精神医学家,他的话必然值得我们洗耳恭听。他这种人,必然能够以容智和悲悯的眼光来盱衡人类的处境。

本书处处流露着坦率和真诚,因为都是刻骨铭心的实录,容不下丝毫欺瞒。以弗兰克博士目前在维也纳大学医学院中的地位,加上意义治疗诊所在世界各地声誉日隆,且都仿效他的维也纳综合医院精神科的情形来看,他所说的一切,自然深具威信。

我们不能不把弗兰克的主张与治疗方法,拿来跟他的前辈大师弗洛依德互作比较。两位大师最关切的便是精神官能症的性质及治疗。弗洛依德由起因于矛盾与潜意识动机的焦虑中找出失调的症结,弗兰克则把精神官能症区分为数类,并把其中数类(譬如心灵性神经官能症〔Noogenicneurosis〕)归因于病人无法由自己的存在中找出意义与责任感。弗洛依德强调性挫折的重要,弗兰克则强调寻求意义之意志受挫的重要。

今天,欧洲人士纷纷舍弗洛依德而就存在分析,而意义治疗学派就是存在分析的一种。弗兰克的见解具有包容的气度。他并不排斥弗洛依德,反而以后者的学术贡献作为其学说架构的基础。他也不和别的存在治疗学派闹纠纷,反而把他们当成同门兄弟。

本书虽小,结构却十分巧妙,读来扣人心弦。本人两度拜读,皆一口气读完,直如中了魔咒。在故事部分中,弗兰克医生曾介绍他个人对“意义治疗”的体会。由于他是在故事行进当中以温和含蓄的笔法引介的,所以读者只有在全书读毕之后,才会领悟到那一段原来别有深意,而不只是集中营里另一个残酷插曲而已。

这个自传式的段落,十分发人深省。读者从中可以窥知:一个人在恍悟到自己“除了这寒伦可笑的一身之外别无余物可供丧失”之时,会有怎样的表现。在弗兰克笔下,这种既感叹又超然的心理,最是扣人心弦。当事人先是对自己的命运怀着淡漠而超然的好奇心;而后,虽然生还的机会微乎其微,仍然想尽办法保住残生。至于饥饿、屈辱、恐惧以及对惨无人道的愤慨,也都因为心中珍藏着爱侣亲人的倩影,或怀着不绝如缕的幽默感,或因为宗教信仰,甚或是对花草树木、晨曦夕照的一瞥,而变得差可忍受。

然而,这些慰藉除非能帮助当事人由状似毫无意义的痛苦中看出一些道理来,否则仍不足以鼓舞生存的意志。而这,正是存在主义的中心思想所在:活着便是受苦,要活下去,便要由痛苦中找出意义。如果人生真有一点目的,痛苦和死亡必定有其目的。可是,没有人能告诉别人这个目的究竟是什么。每个人都得自行寻求,也都得接受其答案所规定的责任。如果他找到了,则他即使受尽屈辱,仍会继续成长。弗兰克特别喜欢引用尼采的一句话:\reduline{“懂得‘为何’而活的人,差不多‘任何’痛苦都忍受得住。”}

集中营里的每件事,都是为叫囚犯丧失自主权而刻意设计的。生活上一切熟悉的目标全遭掠夺,剩下的只是“人的最后一件自由”:在既定的境遇中采取个人态度的能力。这项最终极的自由,古代斯多噶学派和当代存在主义学者都曾提及;而在弗兰克的故事中,尤其带有鲜活的意蕴。集中营的囚犯都只是平凡的人,却至少还有几个人能够决定使自己“苦得有价值”,因而证实了人超越其外在命运的能力。

作者身为精神治疗专家,当然希望知道人可以借着怎样的帮助,来获得这惟独人才有的能力。我们当如何唤醒一个病人,让他感觉到自己无论处境多么悲惨,都有责任为生命找出一个意义来呢?弗兰克就曾和他的难友一同举行过一次集体治疗会议。这次会议,在本书中有动人的描述。

弗兰克医生应出版商之邀,在自传之外增列了阐述“意义治疗法”基本概念的第二部分。以往,这维也纳精神治疗法第三学派(前两派为弗洛依德学派和阿德勒学派)的出版品大部分都在德国发行。因此,弗兰克博士这番增补工夫,相信会受到读者的欢迎。

弗兰克与欧洲许多存在主义学者不同;他既不悲观,也不反对宗教。相反地,他体验过痛苦的如影随形、无所不在,也面对过邪恶的力量,但他却能认定人类有足够的潜力来超越困境,发现一个能提携其成长的真理。

我由衷地向读者推荐这本小书,因为书中戏剧般的故事,其实就是在探讨人类最深切的问题。本书富有文学与哲学的双重价值。捧读本书,在不知不觉之中,对当前最重要的心理学发展必亦略有所窥。


\chapter{集中营的经历}
\section{一场硬仗}
本书并不以集中营实录自诩。书中所载,只是数百万集中营俘虏反复身受的痛苦经验。这是一个集中营的内在故事,由一位生还者所述。书中没有那屡经描绘而其实不太有人相信的大恐怖,有的只是多如牛毛、层出不穷的小折磨。换言之,本书只想为这个问题寻找答案:“一个普通的俘虏每天生活在集中营里,会有怎样的感触?”

本书所描述的事件,大多不是发生在著名的大型集中营里,而是发生在屡见残杀的小集中营里。书中故事,不是英雄烈士的苦难事迹,也不是“酷霸”\footnote{即Capos,意思是监狱里罪犯的头头。}或知名俘虏的生活点滴。它所关切的,不是有权势,有地位的人所受的苦,而是诸多默默无闻、名不见经传的俘虏所遭遇的苦刑、苛虐及死亡。“酷霸”真正瞧不起的,正是这些平凡无奇、袖子上一无标记的俘虏。他们几乎无以果腹,而“酷霸”却从不知饥饿为何物。事实上,许多“酷霸”在营期间的膳食,比这辈子的其他时候还要享受。但他们对俘虏的态度,比警卫还要苛薄;打起人来,也比纳粹挺进队员还要狠。当然,“酷霸”是由众多囚犯中精挑细选而来的。他们的个性,恰恰适合担任这种酷虐的角色;如果“工作”不力,有负所托,立刻就会被刷下来。因此,他们一个个都卖力表现,俨若纳粹挺进队员和营中警卫。象这种例子,也可以用同样的心理学观点来衡量。

局外人对集中营生活,很容易抱着一种带有怜悯与感伤的错误观念,至于对营中俘虏为图生存而奋力挣扎的艰辛,则不甚了了。这种挣扎,正是为了日常口粮,为了生命本身,为了自己或好友而不得不全力以赴的一场硬仗。

\section{挣扎生存时的道德问题}
且以换营为例。换营消息,是由官方发布的,表面上说是要把一批俘虏转运到另一个营区。然而你如果料想这所谓的“另一个营区”其实就是指煤气间,你的推测可以说八九不离十。病弱而无力工作的俘虏,都会遭到淘汰,并且遣送到设有煤气间和火葬场的大型集中营里。淘汰的方法,是叫全体俘虏来一场群殴,或者分队格斗。当其时,每个俘虏心中最记挂的便是:努力把自己和好友的名字,排除于黑名单之外——尽管大家知道拯救某人,有可能会被发现。

每次换营,总有一定数量的俘虏非走不可。然而,由于每个俘虏不过是个号码,所以究竟走了哪些人并没多大关系。俘虏在入营之时,随身证件和其他物品就已经遭到没收了(至少奥斯维辛集中营是这样做的),因此,每个人都有机会虚报姓名职业。许多人为了各种理由,就都这么做。当局所关注的,只是俘虏的号码。这个号码,就刺在各人的皮肤上,也绣在衣裤的某个地方。任何警卫若想“整”一个俘虏,只要对该俘虏的号码“瞟”一眼就行了(这一“瞟”,即可教我们心惊肉跳),根本不必查问姓名。
\endnote{本书描写的是纳粹党卫军建立起来的集中营环境,实际上这样的环境和现在的监狱环境也有几分类似,我们可以想到《肖申克的救赎》电影里面的情景,而与之类似的环境还有劫匪持枪建立起来的环境等。这些环境有一个共同点,那就是某一少部分人拿着武器将一大群人用暴力的方式监禁起来。在这种暴力威权之下,人类建立起来的一切理性道德等等都受到巨大的冲击,一边是无法无天的撒旦,一边是时时处于生存边缘的小民。其实一个国家的政治环境也可能慢慢演变成为这样一种类监狱环境。一句话,在这个环境里面,实际上是拳头说了算。}

言归正传,换营队伍行将离去时,营中俘虏是既不愿也没有时间去顾虑道德或伦理问题的,每个人心中只有一念,那就是:为等候他回去的家人而活下去,并且设法营救朋友。所以,他会毫不犹豫地想尽办法弄到另一个人,另一个“号码”,来代替他加入换营行列。

我曾提过,挑选“酷霸”的方法十分消极.只有最残暴的俘虏才会被挑出来担任这个差事(虽然也有些侥幸的例外)。不过,除了由挺进队负责挑选之外,还有一种毛遂自荐的办法是在全体俘虏之间全天候进行的。一般说来,只有经过多年辗转迁徙,为挣扎生存已毫无顾忌,并且能够不择手段,或偷或抢,甚至出卖朋友以自保的俘虏,才有可能活下来。我们这些仗着许多机运或奇迹——随你怎么称呼——而活过来的人,都知道我们当中真正的精英都没有回来。

\section{鼓足勇气,现身说法}
有关集中营的报道和实录,多已有案可稽。可是,事实真相只有附属于一个人的经验时才有其深意。本节所要描述的,正是这些经验的特质。笔者愿意以当今人类所拥有的知识,为曾陷身集中营的人阐释当时的经验,并帮助未曾身历其境的人理解、体会这极少数浩劫余生、如今却万难适应正常生活的人所曾身受的历炼。这些历劫归来的生还者常说:“我们不喜欢谈过去的经验。身历其境的人,不必别人多费唇舌来替他解说;没有经验过的人,不会了解我们当时和现在的感受。”

要有条不紊地阐述这个主题,实在相当困难。毕竟,心理学家总该维持其学术上的超然。可是,一个坐囚期间从事其研究观察的人,是否拥有这必要的超然呢?局外人必定有这种超然,可是往往因为相距太远,事不关已,而无法作出真正有价值的论述。这种事,只有局内人最清楚。他的判断容或不够客观、不够公允,但这原是无可避免的。如果他想要避免任何个人的偏见,就必须付出更多的心血和努力。这也便是撰写这样的一本书的困难所在。有时,作者必须鼓足勇气,写出极其隐私的经验,我在撰写当时,就曾经打算隐匿真实姓名,只附上我坐囚期间的俘虏编号。可是脱稿之时,我又发觉如果匿名出版,本书的价值势必减半,更何况我必须有勇气公开陈述我的信念。因此,我尽管十分不愿暴白自己,却没有删去任何章节。

把本书内容浓缩成理论的工作,我将留待他人去做。这些理论,对第一次世界大战以来深受瞩目,而其“铁丝网恐惧症”为众所知的监狱生活心理学,可能有所贡献。晚近,人类在“大众精神病理学”(容我引用LeBon的一本著作中的著名词句及书名)上的进展。可以说拜第二次世界大战之赐,因为这场大战制造了神经战和集中营。

\section{苦役的代价}
本书所述,乃是我在集中营中身为一名普通俘虏的经验。因而,我特别要声明的是,被俘期间,我除了最后几个星期之外,并未受雇担任营中的精抻病医生甚或是一般的医生。我提到这一点,难免有些自豪。我有几个同行相当幸运,能够在简陋且仅供应绷带(由破布和废纸作成)的急救站工作。而我,带着一一九一○四的俘虏编号,大部分时间都在铁路沿线上挖土和铺铁轨。有一次,我独力挖掘一条地下水管的通道。这项功绩后来得到了报酬。就在一九四四年圣诞节,我收到一份所谓“奖金联券”的礼物,是由承包该项工程的建设公司发给的。我们这些俘虏,实际上是被集中营当局卖给这家公司当奴役,该公司每天按俘虏人数付给当局一笔固定的工资。每份联券约值五十个芬尼,可以兑换六根香烟。兑换时间,通常在几星期后,不过有时候也会失效。于是乎,我成了个骄傲的“财主”,拥有一份值十二根香烟的礼券。这十二根香烟本身或许无甚意义,却可以兑换十二份肉汤,而十二份肉汤在当时看来,委实是一道消饥救急的大餐。

抽烟的特权,只保留给每星期都有固定奖券配额的“酷霸”,和在仓库、工作场所担任守卫、或领取几支烟以为担当危险职务酬劳的人。除此之外,就只有已丧失生存意志,想“享受”生平最后几天的俘虏,还可以拥有这个特权。因此,我们一旦看到一个同伴在抽烟,就知道他已经失去了活下去的力量和信心,而生存意志一旦丧失,便很难以恢复过来。
\endnote{享受一口烟,然后是死亡。如果事实清晰可见而人们还那么去做,那真是愚蠢透顶了。}

\section{入营第一关}
现有大批资料,为众多俘虏的经验与观察的结晶。当我们仔细审视这些资料,将会发觉众俘虏对集中营生活的心理反应,可分为三个阶段:刚入营之后的阶段、习于集中营例行生活的阶段、释放且重获自由之后的阶段。

第一阶段最显著的征状便是震惊。在某些情况下,俘虏也可能在正式入营之前即已有此征状。

且以我个人入营时的情况为例。当时,共有一千五百人在火车上度过了几天几夜,每节车厢有八十个人,每个人都得躺在自己的行李(即个人仅余的身外物)上。车厢内因为拥挤不堪,鸽灰色的晨曦只能由车窗顶端透进来。每个人都以为火车会驶向某个军需工厂,然后大家会在那儿充当强制劳工。没有人知道我们是否仍在西里西亚,或者已经抵达波兰。火车的汽笛声一如求救的呼喊,听来十分凄厉,像是要为一步步接近地狱的可怜乘客叫冤抱屈似地。不久,火车转辙了,显然已接近一个大站。突然间,一厢厢忧心忡忡的乘客纷纷惊叫:“那儿有个牌子,奥新维辛!”煞时,每个人的血液都降到冰点。“奥斯维辛”是恐怖的代名词,代表着煤气间、火葬场、大屠杀。火车慢慢地、近乎迟疑地行驶着,仿佛希望为乘客拖延真相大白的一刻:奥斯维辛!

晨曦渐露,一座庞大的集中营逐渐现出轮廓。几排长长的带钩铁丝网篱笆,几座守望塔、探照灯,以及一列列憔悴褴褛的人形沿着荒凉的石路蹒跚走着,在灰白的晨曦中,不知要迈向何处。有几声零落的吆喝和指挥的哨声,却不知有何含义。想像中,我仿佛还看到有几座绞刑台,上面吊着晃来晃去的死人。我不觉毛骨悚然,然而这还不算什么,因为随后一个遥无止期的大恐怖,正等着我们去适应哩!

火车终于到站了。一声声吆喝,打破了起初的静默。此后,我们在所有的集中营里,就一再听到这粗鲁而尖锐的噪音。它酷似罹难者临死的哀号,所不同的是,它带着刺耳的沙哑声,仿佛发自一个不得不常如此叫嚷,或一再遭受谋害的人的喉间。车厢门立刻被推开了,一小队着条纹制服、剃光头,看来营养不错的俘虏冲将进来。他们操着各种欧洲语言,而且全都带有一些幽默;只是此情此景,这种幽默听来未免怪异,就像垂死挣扎一样,我骨子里的乐观(这种乐观使我每逢最险恶的境地也常常能克制自己)紧紧攫住这个念头:这些俘虏气色不错,精神似乎很好,甚至还笑得出来。说不定,日后我也可以挣到他们今天这种地位呢!

在精神病学里,有一种状态叫做“缓刑错觉”。死刑犯在处决以前,幻想自己会在最后一分钟获得缓刑。同样地,我们也抱着一线希望,直挨到最后一刻都还相信结果不会这么糟糕。先看到那些俘虏的圆脸和红润的双颊,就已经是一大鼓励了。当时,我们并不知道这批俘虏是经过特选的中坚分子,多年来一直负责接收每天涌入车站的乘客。而所谓“接收”,包括点数新到的俘虏、搜查随身携带的行囊,其中凡是稀有物品或走私来的珠宝,一律没收。在大战的最后几年,奥斯维辛在欧洲想必是一个奇特的地方。珍贵的金银财宝,必定不只锁在硕大的储仓内,还掌握在挺进队员手中。

一千五百名俘虏都被关进一间顶多只能容纳两百人的库房里。我们饥寒交迫,库房内连蹲的地方都不够,更别说躺下来了。四天之中,我们仅靠一片五盎斯重的面包果腹。然而,我却听到几个看管库房的资深俘虏用一枚白金钻石领夹和一名负责接收的俘虏谈交易。大多数的利润,最后都用来买醉——这儿可以买到杜松子酒。足够一晚酣梦的杜松子酒,究竟需要花几千马克才能买到,我已不复记忆;可是,我却知道那些长期受到监禁的俘虏需要杜松子酒。在这种情况下,谁能责怪他们花钱买醉,麻痹自己呢?还有一批俘虏也有酒可喝,并且由纳粹挺进队无限制供应。这些俘虏都在煤气间和火葬场工作,他们深知终有一天,自己会被另一批人取代,也深知自己终究会由目前这不得不干的刽子手角色沦而为罹难人。

我们这一梯次的每个人,差不多都有个痴想:料想自己可以逢凶化吉、消灾解厄。火车到站时,我们还不确定下一步的命运,有人叫我们把行囊留在车上,然后分男女排成两行,以便逐次由一名挺进队的资深长官面前通过。教人吃惊的是,当时我竟胆敢把我的背袋藏在外套里边。我这一队继续前进,一个个从这位长官面前经过。我很清楚,这官员一旦发现我暗藏背袋,必定叫我吃足苦头!根据过去的经验,我知道他至少会狠狠踢我一脚。我本能地挺直腰杆走向这位长官,免得他瞧出我身上的重物。不久,我与他正面相对。他身材高挑,合身的制服纤尘不染;反观我们,漫长的旅途之后,已经是蓬头垢面,一身邋遢,跟他呈强烈的对比。他摆出一副满不在乎、悠然自得的姿态,左手托着右肘,右手直立、并用右手食指悠闲地指向左,或指向右。我们丝毫不知道这家伙的手指头一忽儿指向左,一忽儿指向右,究竟有何不祥的含义。只是,他指向左边的次数占大多数。

轮到我了。早先,有人低声对我说,指向右边表示要工作,指向左边表示无力工作和有病在身,会被送到一个特别的集中营去。于是,我静待发落;身上的背袋沉甸甸的,使我稍微歪向左边,但我奋力站直。挺进队的这位长官打量了我好一会,似乎在犹豫。而后,他伸出双手,搁在我肩上,我努力显出精明的模样。最后,他非常缓慢地把我扳向右边,我便向右边跨去。

当晚,这种“指头把戏”才告揭晓。原来这是第一次的淘汰与判决——判决我们究竟是生存或丧命。我们那一梯次,约有百分之九十的俘虏被判死刑,而且是在几个钟头之内立刻处决。所有被叫到左边的人,当时立刻由火车站直接遣送到火葬场。一个在火葬场工作的人就告诉过我,火葬场那栋建筑的门上,用欧洲各种语文写着“洗澡间”字样。进门时,每名俘虏都会收到一块肥皂,然后——唉!接下来发生的一切,我不提也罢!反正这种恐怖的事情,许多书刊都已经报道过了。

我们这些幸存的少数,当晚就获悉真相。我向几名曾在那边工作过的俘虏打听消息,因为我的一位同行兼好友潘先生被送到那儿了。

“他是被叫到左边的吗?”

“对!”我答。

“那么你可以看到他在那里。”他们说。

“哪里?”我问着,有人伸手指向几百码外的一支烟囱。一股火焰,正由烟囱口喷向灰蒙蒙的波兰天空,消失在一片不祥的烟雾里。

“你的朋友就是在那里,他飘到天堂去了。”我听了,仍然丈二金刚摸不到头脑;对方只好用普通的语句另外解释一次,我这才恍然大悟。

不过,此处所讲述的,并没按照事情发生的先后次序。由心理学的立场来看,从火车站破晓的那一刻起,我们就面临了一段极其漫长的历程,一直要等到我们在营中歇息下来,度过第一夜才止。

在挺进队的警卫持枪戒备之下,我们奉命由火车站穿过通电的带钩铁丝网和营区,奔向清洗站。我们这批通过了第一关的邋遢人,在这儿可以说真正享受到洗澡的舒畅。“缓刑错觉”也因此再度有了个明确保证,连挺进队员似乎都和蔼可亲。可惜不多时,我们看出了和蔼可亲的原因。这些队员只要看到我们手腕上带有手表,对我们便亲切有加,并且鼓起如簧之舌,以万般善意的声调劝我们把手表交出去。既然我们什么东西都得充公,为什么不干脆交给一个看起来比较和气的人呢?说不定,有朝一日他还可以帮个大忙哩!\endnote{斯德哥尔摩综合征所描述的心理反应开始出现。}

我们在一个小房间里等着,那小房间似乎是消毒间的休息室。挺进队员出现了,并摊开几张毯子,要我们把身上一切物品,包括手表、珠宝全扔进去。有几个俘虏还天真地问说:可否留下一枚婚戒、纪念章或幸运符什么的,使得在那儿充当助手的几个资深俘虏为之发笑不已。到那个时候,每个人差不多都已经知道:一切物品会被搜个精光。

我曾试着向一位资深俘虏吐露我的秘密。我偷偷溜到他身边,指着我外套暗袋里的一卷纸说道;“你看,这是一本学术著作的手稿,我知道你会怎么说。你会说我能够保住老命已经该谢天谢地,不敢再有非份的奢想了。可是我实在克制不住。我必须不计一切代价保留这份手稿。这是我这辈子的心血结晶。你知道吗?”

嗯!他是知道了。他脸上慢慢绽出一个笑容,起先带着悲哀,继而变成逗趣,而后现出嘲弄和侮辱的表情,最后他以营中俘虏惯用的一个词汇,答复我的问题:“狗屎! ”就在那一刻,我认清了眼前的现实,并且抵达了我第一阶段的心理反应的最高潮:我挥手斩断过去的一切。

突然间,大伙儿骚动起来,一个个脸色苍白,战战兢兢地站着,并且议论纷纷。此时,刺耳的吆喝声再度响起,我们在哨子的催促下赶忙跑进堂前的休息室,然后在一个挺进队员四周集合起来。此人一直等着所有的俘虏统统到齐,才开口说道:“我给你们两分钟,并且用我的手表计时。在这两分钟内,你们要脱个精光,并且把所有的衣物放在脚板前面。除了鞋子、皮带或吊带,或者疝气带,其余全部不准留在身上。我就要计时了——开始!”

大伙儿不假思索,立刻急匆匆地宽衣解带。时限愈短,每个人就愈形紧张,笨手笨脚地扯着内衣裤和鞋带腰带。不久,一阵鞭打声响起,原来是皮鞭打在赤条条的人体上所发出的响声。

后来,我们被赶到另一个房间剃毛,不惟头发、胡须都要剃掉,连身上任何部位的毛也得剃个精光。接下来便是到淋浴间.大伙儿再度排队。此时,每个人几已面貌全非,彼此间差不多都认不出来了。差可告慰的是,有些人发觉莲蓬头上的确有水滴下来。

等候淋浴时,全身的赤裸,使得我们认清了一个事实:此际,我们除了这光秃秃的一身,的的确确是一无所有了;就连身上的毛发,也已经被剃除净尽,仅余这赤裸光溜的身体。我们还有什么物质上的东西可以同过去的生活产生关连呢?我个人,还有一副眼镜和一条皮带,可是隔没多久,我就不得不用皮带去换取一片面包了。拥有疝气带的,倒是多了一样值得庆幸的东西。当晚,管理我们那间茅舍的资深俘虏在致词欢迎我们的时候,就严正地警告说,如果有谁胆敢把钱钞或珠宝缝进疝气带内,他一定会亲手把那个家伙吊到屋梁上。说着,他指了指上头那根横梁,并且骄傲地说他资格老,按营规他有权这么做。

说到鞋子,事情可没这么简单。我们虽然有权保留鞋子,但拥有适脚鞋的人,最后都不能不予以放弃,换来一双不适脚的。更苦恼的是,有些俘虏听从了资深俘虏在休息室内的善意忠告(表面上似乎是善意的),便把过膝长统靴的上半截切掉,并用肥皂涂去切痕,借以掩饰。可是,挺进队长似乎早就料到了这一招,因此每个有嫌疑的俘虏都被叫到隔壁一间小屋里。不久,皮鞭的呼啸声和挨打者的号叫声隔墙传来,而且持续了好一阵子。

某些人心中尚存的几个幻想,就这样逐一归于破灭。意外的是,大多数人心头渐渐滋生出一股顽强的幽默感。我们知道,除了这可笑的赤裸之身,我们已别无他物可供丧失。当莲蓬头开始喷水,我们全都努力地寻开心,努力开自己和彼此间的玩笑。毕竟,莲蓬头总算还喷得出水来哩!\footnote{喷的是水不是毒气。。}

除了那股奇特的幽默感,我们的心头另外还蟠踞着一种感觉:好奇心。这种好奇心我以前也体验过,那是我碰到某种奇特境遇时的一个基本反应。每当我遭逢意外,处境危险,在紧要关头之中,我所感到的只是好奇。我想知道自己究竟能全身生还,或者负伤而归。

即使在奥斯维辛,冷静的好奇心仍然凌驾一切,使得理智能超越周遭的环境,进而以客观的眼光看待周遭。在当时,培养这种心境,是为了保护自已。我们急于知道下一刻会发生什么事,而且后果又会怎样。譬如,当我们淋浴完毕,身体赤裸而还湿漉漉的,却要站在户外忍受着晚秋刺骨的寒意;当其时,每个人对下一个“节目”就十分好奇。往后几天,这种好奇渐渐转变成惊讶:惊讶于自己居然没有感冒。

大凡新到的俘虏,总有一箩筐类似的惊奇等着他去发掘。如果他是医科出身的,那他一定最先发现教科书全是在扯谎!譬如,我就记得教科书上说过:人如果每天没有睡满一定的钟点数,就活不下去。这真是大谬不然。过去,我一直深信有些事我就是办不到或无法适应:比如,我没有某样东西就睡不着,我没法跟某种人或某种现象共处于同一个屋檐下。可是在奥斯维辛的第一晚,却大大出乎我意料之外。我们睡的是一层层搭架起来的硬木板床。每张床宽约六尺半到八尺,却挤了九条大汉,而且九个人分盖两条毯子。当然,我们只能侧卧且彼此紧挨着身子。这样倒有个好处,因为天气实在太冷了。

按规定,鞋子是不准带上床的,不过,有些人还是偷偷把沾满泥垢的臭鞋垫在头下当枕头,免得使都快脱臼了的手臂还要为“曲肱而枕之”而受罪。怪的是,睡神依旧光临,让大家在黑甜的梦乡里得到几个小时的解脱。

还有些我们居然都能忍受的境遇,也值得一提。我们无法刷牙,维生素又严重缺乏,奇的是,每个人的牙龈反而远比以前健康。同一件衬衫,我们得穿上半年,直到毫无衬衫样为止。由于水管冻结,我们常常一连好几天不能洗澡(即连局部冲洗也不行),然而手上擦伤发炎之处,却不因为工作得满手污垢而化脓(当然,冻疮则又另当别论)。还有像浅眠易醒者,以前只要隔壁稍有轻响,立刻会惊醒过来,如今身边紧挨着一个鼾声如雷的家伙,却睡得香甜万分,丝毫不受干扰。


\hlabel{适应}
陀斯妥耶夫斯基曾断言:人无论任何境遇,都适应得了。现在,如果有人问我这句话究竟对不对,我会说,“对!人什么都适应得了,不过别问我怎么适应的。”只可惜,心理学研究目前还没进展到那个地步;我们俘虏在当时,也还没达到那个境界。当时,我们仍处在心理反应的第一阶段。

每个人差不多都有过自杀的念头(即使为时十分短暂)。这是由于境遇的无望,无时无之无日无之的死亡威胁,以及目睹他人惨死的惊惧使然。我基于个人的信念(这容我稍后再述),在营中的第一晚就私下作了个坚决的许诺:我决不去“碰铁丝网”。“碰铁丝网”是集中营里流行的一句话,意指最常见最普遍的自杀办法——去碰充有电流的带钩铁丝网篱笆。我下这个决心,并不算太困难。自杀可以说毫无意义,因为,一般的俘虏只要客观地估计.且算好一切可能的良机,都会发觉活命的指望极其渺茫。他无法自信能通过连番的淘汰,因为通得过的人实在是少之又少。奥斯维辛的俘虏在满怀惊骇的第一阶段当中并不怕死,经历过最初几天之后,连煤气间的恐怖也不足畏了。

我后来遇到的几位朋友,都告诉我说,入营时那种惊骇,我的还不算特别严重。因为,在奥斯维辛度过第一夜后的翌晨,发生了一个插曲;当时,我只是笑笑,而且是由衷的一笑。事情是这样的:我有个同业,比我早到了几个星期。当局虽严禁擅离属区,这位仁兄还是偷偷溜到我们营舍,想安慰我们,并告诉我们一些事。他变得实在太憔悴,我们好不容易才认出他来。他摆出高度的幽默和漫不在乎的姿态,匆匆关照我们:“别怕!也别担心被淘汰!马医生(挺进队的医科主任)对医生特别照顾。”(这话其实有错。一位六十多岁的医生俘虏就告诉我,他曾经哀求马医生放过他那个被送往煤气间的儿子,马医生无情地拒绝了。)

“不过,请你们牢记一点,”他继续说道。“如果可能,最好每天修脸,即使用玻璃片来修……或即使用你们仅余的一片面包来换取修脸机会,都大大值得。修了脸,看起来比较年轻,脸色也比较红润。\footnote{即使在集中营里也要打起精神来!}如果你们想活命,唯一的办法便是:摆出还能胜任工作的样子。如果你只是跛脚——譬如说,你脚跟起泡,不幸被挺进队员发觉,他会把你叫到一边,然后第二天送你到煤气间。你们知道我们所谓的‘末世脸’是什么意思吗?一个人如果脸色黯淡,形容憔悴,一副病恹恹的样子,而且无法再胜任吃力的苦工,……这人就是个‘末世脸’。迟早——通常是快得很——他就会进入煤气间。所以千万记住:时常修脸,走路或站立都要挺直腰杆。这样就不必怕煤气间。你们这几个虽然只在这儿待了一天,却都不必怕煤气间,除了你——”他指着我,说道:“请恕我直言。”然后又对其他人强调。“你们中,只有他才该害怕下次的淘汰,所以,不必担心!”

当下我笑了。此刻,我相信任何人当时如果碰到我这种情况,反应也会和我一样。

\section{由惊骇到视若无睹}
“丧失理智,一定事出有因,不然就是没有理智。”这句话,大概是诗人莱辛所说的。遇到反常情况而有反常的反应,这是正常的行为。一个人在遭逢巨变——譬如被送进精神病院时,即使是精神医生,也会预料他反常的程度将与他正常的程度成正比。一个人对他被抓进集中营这件事的反应,容或显示他心智异常,然而客观说来,却是正常且典型的反应(这一点容后详述)。如前所言,这些反应在几天后开始有了变化。当事人由第一阶段转入第二阶段——也就是冷漠、无动于衷的阶段。当其时,他达到了一种情绪死亡的境界。

除开已描述过的反应之外,新到的俘虏还尝到其他难以堪之的情绪折磨,也企图予以缓和。其中最难挨的,莫过于对家乡和家人的思念了。思念之情常因为澎湃难抑,令人心如刀割。再来就是嫌恶之感。周遭的一切丑陋现象,即使只是外表的样子,就足以叫人作呕。

大多数俘虏,都可以分发到一套破烂的制服,这套制服穿在稻草人身上倒是能增益其丰采。在营中的幢幢房舍之间,堆着成堆的秽物;愈是努力去清除,愈是不得不要去接触。管理当局特别喜欢把一名新的俘虏分派到扫厕所和挑大粪的工作队里。在挑粪时,如果粪水溅到脸上,只要他一显露出嫌恶的表情或企图揩去污物(通常会这样),“酷霸”立刻会给他一顿毒打,这样一来,他无论如何也会克制他的正常反应了。

新到的俘虏,起初若看到别个工作队受到“游行”惩罚的情景,总会掉头不看。他不忍心看到难友在泥地里忽上忽下地行进,还得随时承受残暴的棍击。几天或几星期后,情形改观了。早晨天色尚暗,他正和队友站在大门口,准备出发前往工地。他听到一声惨叫,然后看见一个难友被打倒后站了起来,旋又再度挨揍而颠仆于地。究竟是为什么呢?原来这人患了热病,申请调入病房,不料时机不对,便被当局视为企图逃避劳役而遭受处罚。

但是,己进入心理反应第二阶段的俘虏,目睹惨状,已不再把眼光掉开。他的感觉已经迟钝,因此即使目睹也无动于衷。且再举一例:他在病房内等着,因为受伤、水肿或发烧,很希望获准在营内做两天轻松的工作。就在这时,有人扶着一名十二岁男童进来。这男孩光着脚(营中没有他能穿的鞋子)在雪地里劳动了几个钟头,脚趾头都冻坏了,值班医生用镊子把已经坏死且冻成黑色的趾头一个个摘掉。这幕光景看在他眼里,丝毫激不起恶心、恐怖或怜悯的情绪。他像个木头人一样站在那儿;因为,几星期来的集中营生活,已使他看惯了痛苦死亡和垂死挣扎,再也也引不起任何感觉了。


\section{冷漠是自卫的绝招}
\hlabel{冷漠}
我曾在专供斑疹伤寒患者居住的茅舍里工作过一段时间。那些病人体温都非常高经常神志昏迷,而且大多都奄奄一息。每当有人死去,我总是冷眼旁观着随之而来且已经司空见惯的一幕:众俘虏一个个挨近犹温的尸体,有的抢到一盘吃剩的马铃薯泥,有的发现死者的木鞋比自己的稍好而来个调换。另一个抢到了死者的外衣,还有一个更因为也抓到了一点东西——一根真正的绳子——而高兴万分。

我以事不关己的冷淡看完这一幕,才叫“看护”来移开尸体。他讪讪然来了,抓住死尸的脚使劲一拖,尸体就掉在两排木板(也就是五十名患者所睡的床)之间的窄道上。他再拖着尸体走过凹凸不平的泥地,来到门口那两级通往户外的台阶前。两级台阶各有六英寸高,对长期挨饿,体力不济的我们,向来是一大考验。在集中营待了几个月之后,我们已无力拾级而上,只得伸手抓住门框,使劲把自己拉上去。

那人走近台阶,虚弱地把自己先拉上去,再拖着尸体:先是脚、再而躯体,最后,紧跟着一阵恐怖的碰撞声之后,尸体的头部总算也拖上了台阶。

当时,我正在该茅舍的另一边,紧靠着唯一的小窗口(窗子离地面很近),以冰冷的双手捧着一碗热汤,贪婪地啜着。无意间,我往窗外一望,恰好看到才移到那儿的死尸,正以呆滞的眼神死盯着我。两个钟头前,我还跟死者说过话哩!然而此刻,我继续啜我的热汤。

我若不是因为职业关系,对自己当时的冷漠大感惊异,很可能早就淡忘了此事。毕竟,这其中简直不含半点感觉啊!


\section{精神创伤}
冷漠寡情,感觉钝化,自觉什么也无法在乎——这正是第二阶段心理反应所特有的征状。这些征状,终能使一个人忍受无时无之的鞭笞而浑无所觉。每个俘虏就靠这种迟钝和麻木,很快把自己裹进一层极为需要的保护膜里头。 
 
我们常因为细故(甚或是无缘无故)而挨打。譬如,面包是在工地分配的,必须排队领取,有一次,我后面那个人站歪了一点点,队伍因此不够整齐,结果惹恼了挺进队的警卫。当时,我压根儿不知道背后发生了什么事,也不明白警卫到底怎么想,可是突然间,我头上吃了两记闷棍。直到那一刻,我才发觉身旁那个警卫出手打人。那种时候,最难受的不是肉体上的痛苦(不论大人或儿童皆然),而是不公正、不合理的待遇所带来的精神创伤。

奇怪的是,在某些情形下无形的打击反而比有形的殴打还难以忍受。有一次,正值大风雪,我那个工作队照常赶工。我站在铁轨上,努力铲石头填补轨道——因为这是取暖的唯一办法。有一会,我停下来靠着铲柄喘气,不巧警卫正好转过头来,以为我在偷懒。令我感到痛苦的,既不是侮辱,也不是殴打。他大概认为对我这种衣衫褴褛、不成人样的怪物,没有开腔的必要,连骂一声都嫌费事。于是,他戏弄似地拣起一颗石子,向我抛来。这个举动,仿佛是要引起一只畜牲的注意,好叫它回到工作岗位上似地。显然,他把我看作一个与他毫无共同处的动物,所以连惩罚都嫌多余了。

挨打时,最痛苦的便是其中所暗含的侮辱,有回,我们扛着长而笨重的梁木,走过冰冷的铁道。一旦有人跌跤,不仅他本人危险,扛着同一条梁木的其他人也都会遭殃。我有位好友患有先天性臀骨脱臼症,由于身体残疾的人一经淘汰,差不多都会被送进煤气间,所以他尽管疼痛难挨,还是庆幸自己能够劳动。他扛着一条特别笨重的梁木,一颠一跛地跨过铁道,眼看着就要跌跤,且连同其他伙伴一块绊倒了。当时,我恰好没扛着梁木,因此我不假思索,便冲上去帮助他。不料,警卫一棍打在我背上,还对我谩骂一阵,命我滚回原处。而几分钟以前,这名警卫还不以为然地说我们这些“猪”太缺乏友爱精神了呢!

又有一次,气温为华氏二度,我们在森林里挖掘已冻得硬邦邦的表土,以便埋设水管。当时,我身体已经变得很虚弱。一名监工走来了。他的两颊丰腴红润,令我明确地联想到一个猪头。我注意到他在这酷寒的天气中,戴着一双温暖宜人的手套。他沉默地盯了我好一会,我感到祸事临头,因为我眼前那堆土,正好显示我究竟挖了多少。

他开口了:“你这懒猪,我从开头就注意到你了。你等着瞧,我会教你怎么工作的。我要你用牙齿来挖,要你像畜牲一样死掉!看着好了,两天之内我会把你干掉!你这辈子从来就没劳动过吗?猪!你以前是干什么的?生意人吗?”

他这番恶声恶气的话,我倒不放在心上。只是,我必须顾虑到他要杀我的威胁。因此,我挺起腰杆,正对着他说:“我以前是医生——专科医生。”\\
\indent“什么?医生?我敢说你一定从病人身上揩了不少油啰!”\\
\indent“正好相反,我在贫民医院工作,常常分文不收。”至此,我显然说得太多了,当下他纵身一扑,把我打倒,还像疯子一样大叫。至于叫些什么,我已记不得了。

我写出这段微不足遭的经历,是为了表示:有些时候,再冷漠的俘虏,也会被激得满腔怒火——不是为残酷或痛苦而发怒,而是为了切身相关的侮辱。那次,我简直热血沸腾,因为我不得不要恭听一个对我毫无所知的人批评我的过去,而这个人(下列这段评语,是我在事后对一个难友所说的。我得承认这番话给了我稚气般的发泄),“样子那么粗俗,那么野蛮;我医院门口的护士,光看他一眼就不会让他进来”。

所幸,我队上的“酷霸”对我深为感激。他对我很有好感,因为我曾在前往工地的漫长步行当中听他吐露他的爱情故事和婚姻问题。我为他作了性格上的诊断,还提出精神治疗方面的建议,令他印象极深。此后,他一直深为感激。这对我大有帮助。以前,他好几次在工作队(约由二百八十名俘虏组成)的前五排中,为我保留了一个与他隔邻的位置。这种恩惠非常重要。天色尚暗,我们一大早就得排队。每个人都怕迟到,也怕排在后面几排中。每遇有讨厌的工作需要人手,一位资深“酷霸”就会出现,并由后面数排中挑选他们所需要的人数。不幸中选的俘虏,就得在陌生警卫的指挥下,动身前往另一个特别令人生畏的工地。偶尔,那位资深“酷霸”也会从前五排中挑选人手,只为了逮住自作聪明的俘虏。人选一旦挑出,任何哀求,抗议都会在几记准确的踢打之下归于沉默,而中选的可怜虫便在吆喝殴打声中被赶往集合地点。

不过,只要我那位“酷霸”感到有倾诉衷曲的必要,这种事就临不到我头上。在他身边,我必定拥有个荣誉席位,而且还有另一个好处。我就像绝大多数的俘虏一样,两脚浮肿,脚上皮肤紧绷得连膝盖都难以弯曲。为了让鞋子容得下一双肿脚,我只得不系鞋带;即使有袜子,也只能弃而不穿。结果,我光溜溜的脚丫老是湿漉漉的,鞋内也老是灌满雪泥。这当然会引起冻疮,因而我每跨一步,都痛彻骨髓。每当行经白雪覆盖的田野,我们的鞋上常结出一块块的冰层。许多人一再滑倒,每一滑倒,后边的人就跟着绊跤,整个队伍因之停顿下来。然而不会耽搁太久的。警卫当中,总有一名立刻出面,以步枪枪柄,使劲往跌跤的俘虏身上一敲,他们很快便纷纷起身。这时候,你排得愈前面,就愈不必停顿下来,更不必为了弥补耽搁掉的时间而以一双痛脚跑步。所以,能够成为“酷霸”阁下的私人医生,并在队伍前排中以平稳的步伐前进,实在很令我开心。

此外,在工地午餐时,只要是分配汤,一轮到我,这位“酷霸”便会把汤杓直接探到桶底,再捞出一些豌豆来给我,算是对我为他服务的一个额外报酬。过去当过军官的他,竟还鼓起勇气,偷偷向曾跟我吵过架的那名监工说:他晓得我是个特别优秀的工人。这虽然无济于事,但他仍然设法营救我(这只是许多次中的一次)。就在我与那名监工发生了那件事之后的第二天,他偷偷把我调到另一个工作队去了。


\section{非人的境遇}
也有些监工同情我们的遭遇,尽量减轻我们的负担——至少在工地是如此。不过,即使是这样的监工,也经常提醒我们说,普通工人有时候干的活跟我们一样多,所花的时间却更短。然而,如果他们知道正常工人每天的饮食不像我们这样,只有十点半盎斯的面包(这是规定上的,实际上更少)和一小碗的稀汤,而且还不必承受精神压力,不必时时面对死亡威胁,一定会知道个中的原因。何况,正常工人不像我们这样,全无家人音讯,更不必担心亲人是不是被关进另一个集中营,或已经被送入煤气间。有一次,我就曾鼓足勇气对一个和善的监工说:“如果你能够以我现在向你学习修路的速度,来跟我学习脑部开刀的技术,我便佩服你啦!”当时,他咧嘴一笑。


\section{比噩梦还恐怖}
第二阶段的主要征状——冷漠——是自我防卫所必需,人一旦冷漠,现实就模糊了;而一切的心力和情感便贯注在一件事上:保住自己和好友的生命。每天傍晚,当俘虏由工地返回营区,常常会松一口气叹道:“呼!幸好又过了一天。”

读者一定不难理解,这种随时随地提心吊胆、力图自保的日子,很容易使俘虏的内在生活倒退成原始状态。营里有几位受过精神分析训练的同业就常说,营中俘虏都有一种“退化现象”——精神生活变得更原始、更接近本能的现象。他的愿望及欲念都在梦中显现出来。

俘虏最常梦到的是什么?是面包、蛋糕、香烟,以及舒服的热水澡。由于这些单纯的欲念未获满足,他便在梦中寻求“愿望实现”(wish-fulfillment)。至于这种梦对俘虏是否有些好处,那是另一回事。反正,作梦人终究必须醒过来,面对集中营的现实,也面对该现实和梦中幻境之问的可怕对比。

我永远忘不了的是:有一夜,我被一个难友的呻吟声吵醒。那家伙虽然睡着,却四处翻滚冲撞,显然正在作恶梦。由于我对作恶梦和发癫的人向来特别同情,当下便想伸手,把那个可怜虫摇醒。才刚伸出去,我突然又缩了回来;想摇醒他的念头,把我吓住了。那一刻间,我深切地意识到一个事实:任何梦任何事就是再恐怖,也不可能比得上集中营的惨酷现实。而我,居然想把这可怜虫唤回到惨酷的现实中。


\section{画饼充饥}
\hlabel{爱好食物}
由于营养严重缺乏,渴望食物乃成为俘虏最主要的原始本能,并为其精神生活的重心。大多数的俘虏在工作时,只要彼此距离够近,且只要未受到严密监视,立刻就会打开话匣子,谈起食物来。其中一个会问另一个同在壕沟中劳动的难友:他最喜欢吃什么菜?当下,两人就会交换食谱,并计划劫后还乡喜相逢那天的菜单。两人就这样津津有味地畅谈不休,把那些佳肴美馔(zhuàn)描绘得淋漓尽致,直到别的俘虏暗中示意:“警卫来了”,才猛然住口。

我一向认为讨论食物十分危险。试想,当你的身体仅能靠一丁点低热量食物勉强支撑,你偏又以这种刻绘入微、叫人馋涎的珍馐(xiū)图给予刺激,岂不增添它的负荷?这种画饼充饥式的幻想,容或能使人暂忘饥火中烧之苦,但就心理学观点来看,却不见得没有危险。

在囚禁的后半期,我们每日的口粮,只有一天一次的稀汤和少量的面包。除此之外,还有所谓的“额外点心”,计为四分之三盎斯的人造奶油,或一片劣等腊肠,或一小块乳酪,或一些人造蜂蜜,或一匙稀汤似的果酱——每天都不相同。这样的食物,热量绝对不够,更何况我们操作的是粗重的苦工,而且经常衣衫单薄于酷寒之中。至于那些受到“特殊照顾”的病患——换句话说,就是获准在茅舍内躺着,不必出外工作的俘虏——他们的情况就更差了。

当最后一层的皮下脂肪消失净尽,我们便活像是披上皮肤和破衣的骷髅,眼看着自己的身体一天天萎缩下去。身体消耗着体内的蛋白质,肌肉渐形消失,而后身体便毫无抵抗力。茅舍内的难友,一个个相继死去。每个人都能够精确地算出下一次会轮到谁,自己又将在什么时候撒手西归。多次的观察,我们已可以洞烛机先、铁口直断。“他差不多了”,或“下次轮到他”——我们常这样子交头接耳。晚上捉虱子时,我们看着自己赤裸的身躯,心里同样都想着:“我这个身子其实已经是一具死尸了。我变成了什么?我不过是挤在铁丝网后寥寥几间破屋里的一大堆人体当中的一小部分罢了。这一大堆人体每天总会有一部分开始腐烂,因为它已经死气沉沉了。”

前面曾提到,俘虏只要偷得到空闲,不知不觉就会想起食物和爱吃的菜肴。在这种情况下,读者想必不难理解,即使是我们中最坚强的一位,也非常渴望能重获大快朵颐的自由。这不是为了品尝美味的食物,而是为了确知这种使我们除了食物之外无法再思索其他事物的非人生活总算是结束了。

未曾身历其境的人,很难以想像一个饥火中烧的人内心的挣扎和意志力削弱的情形,更难以体会一个站在壕沟里挖土的俘虏,苦苦等着哨音宣布上午九点半或十点整(这是半个小时的午餐时间,这期间,只要有面包,通常都会分发下来)的滋味。面包一旦发下,俘虏总把它放在外衣的口袋里。此后,只要监工不是个苛刻的家伙,就会一再问他:“几点钟了?”然后珍惜地摸摸口袋中那片面包;先是用冻僵了的手指头拍一拍,再撕下一小块放进嘴里,但又使出所有的意志力,把那一小块再放回衣袋;因为,他已经暗暗发誓过:不到下午决不再碰面包一下。

光是那每天只发一次(在集中营生活的后半期)的一小片面包,就足够让我们为如何处理它而争论不休了。有的人认为最好立刻把它吃光了,一来可以防止失窃,再则一天至少有一次可以解除饥肠辘辘的痛苦——尽管为时十分短暂。另一批人则以不同的论点,证实分次食用的好处。我几经踌躇,最后也加入了这批人的行列。

一天二十四小时当中,最难挨的时刻莫过于起床时刻了。当其时,天色尚暗,三声尖锐的哨音却无情地把我们从筋疲力竭的睡眠和黑甜的梦乡中吵醒。而后,我们便开始与湿漉漉的鞋子周旋。我们的脚又肿又痛,几乎塞不进鞋内。哀叹和呻吟声此起彼落,因为处处有人碰到了麻烦(譬如,替代鞋带子的那根电线折断了)。有天早上,我就听到一个向来很勇敢很持重的难友哭得像个小娃娃。原来他的鞋子缩水了,他穿不下,必须光着脚在雪地上行走。在这痛苦的时候,我却找到了一点点安慰:我从衣袋中掏出一小块面包,以专注的喜悦大声咀嚼着。


\section{‘性’趣缺缺}
\hlabel{性压抑}
营养不良除了使众俘虏神往于食物之外,很可能也是性冲动普遍阙如的原因所在。在清一色男性的集中营里,心理学家必然会注意到一个现象:这里压根儿没有性倒错(Sexual Perversion)。这和其他纯男性的团体(譬如军队)恰恰相反。究其原因,除开初期的惊骇之外,营养不良似乎是唯一的解释。即使在梦里,俘虏对于“性”仿佛也是兴趣缺缺——尽管他的挫折感,以及较纤细、较微妙的感觉都能在梦中明确地表达出来。\footnote{这可以称之为生存压力之下的性压抑现象。}

近乎原始的生活,以及仅仅为了自保就必须使出浑身解数的生存环境,使得绝大多数的俘虏完全漠视了于自保无益的其他事物。这也便是我们普遍缺乏感情的原因所在。关于这一点,我在由奥斯维辛被调往达荷城的附近一处集中营时,感受特别深刻。当时,我们(约有两千名俘虏)所搭乘的火车经过维也纳。子夜时分,火车路过维也纳的一个小站,而且就要经过我出生的那条街,以及我住了好多年——老实说,一直住到我被捕为止——的房子。

我那节囚车有个窗户,却因钉上了木条,只留下两个小窥孔。车上挤了五十个人,只够其中半数蹲着,其他人只好挤在窥孔旁,枯站数个钟头。我踮起脚尖,从别人的头顶望过去,隔着窗上的木条,我怯怯地瞥了故乡一眼。由于我们都以为会被运往莫豪森的集中营,并且只剩一、两个星期的时间可活,大家都有此去凶多吉少之感。当时,我就清楚地感觉到自己像是从另一个世界回来的幽灵;儿时的街道、广场及住屋,在我眼中看来,恰似一座鬼城。

火车在小站耽搁了几个钟头,终于姗姗离开。那条街——我的街啊!——终于接近了。几个在集中营呆过许多年的年轻小伙子把这趟旅程当作是天大的事。他们紧挨着窥孔,目不转睛地盯着。我只好哀求他们让我在前面站一会。我努力向他们解释在那一刻窗前一瞥对我是多么意义重大,但他们不仅一口拒绝,还半粗鲁半尖酸地冲着我说:“你在这儿住了那么多年啦?那你早就看饱了嘛!”


\section{宗教热}
集中营里,也普遍有一种。“文化冬眠”(Cutural hibernation)的现象,然而政治和宗教却是两个例外。营中处处有人谈论政治,而且几乎是毫不间断地谈。谈论的根据,主要是靠屡遭喝止但又传递极速的谣言。与军事状况有关的谣言经常互相矛盾。一个接一个快速传来的结果,除了增添俘虏的神经紧张之外,别无其他好处。有许多次,被乐观的谣言煽热了的希望——希望战争快快结束——一一归于破灭。有的俘虏因而丧失了一切希望,不过,最惹人发怒的却是那些无可救药的乐天派。

俘虏对宗教的兴趣,打从萌芽开始,就虔诚得令人难以想像。那种信仰的深度和活力,常使新到的俘虏既惊讶又感动。印象最深刻的,要算是即兴的析祷或弥撒了。不论是在茅舍内的某个角落,或搭着载运牲口的卡车由遥远的工地返回营区,尽管又饿又累又冻,周遭一片漆黑,大家仍不忘举行这种宗教仪式。

一九四五年冬春之交,斑疹伤寒的病毒蔓延营中,几乎所有的俘虏都受到感染。身体虚弱的,只要还能够劳动,都必须继续苦干,死亡率因此非常高。病人的营舍小得可怜,根本不够容纳;药品也付诸阙如,看护人员更是形同虚设。这种病有某些症状十分讨厌,譬如,患者对食物感到难以克制的恶心(这不啻是增加生命危脸),发高烧以致神智昏迷等等。我有位朋友就因为神智昏乱极其严重,备受折磨。他自以为就要死了,便想要析祷;然而由于心神狂乱,搜尽枯肠仍找不出祈祷的字句。为防止这种情况发生,我和其他许多人一样,晚上大部分的时间都尽力保持清醒。这几个钟头,我试着构思演说的辞句,后来,我又开始把我在奥斯维辛消毒间内被没收的那份书稿重新撰构起来,并且用速记把重要的词汇写在一张张的小纸片上。

偶尔,营里也会发生一些颇值得科学讨论的事情。有一次,我就亲眼目睹了一件怪事。那种事虽然很合于我的职业兴趣,但我这辈子(即使是在正常生活中)却从未经验过。那是一个招魂会,我是应营医的邀请前往参加的。这位医生也是个俘虏,他知道我是个精神科大夫,招魂会就在病患营舍内一间他的私人小房间里举行。当时,一群人围坐成一个小圈子,其中还包括偷偷溜来参加的一名卫生队准尉军官。

有个人开始念咒招唤鬼魂。那名准尉军官面前搁着一张白纸,无意识地书写着。接下来的十分钟里(十分钟后,灵媒失灵,鬼魂未曾招出,招魂会旋告结束),他的笔在纸上慢慢划出几道线条,拼凑起来,恰恰是清晰可读的“VAE V.”。据说,他从未学过拉丁文,以前也从未听过“Vae Victis”——悲哉败者——这句话。依我看,他以前想必曾听过,只是不曾刻意记住而已。正因为这样,“鬼魂”(其实就是他的潜意识)在那时候才找到这句话。当时,离战争结束和俘虏获释的日子,只有几个月而已。


\section[但教心似金钿坚,天上人间会相见]{但教心似金钿坚,天上人间会相见\endnote{出处:唐朝诗人白居易的《长恨歌》。\begin{verse}
但教心似金钿坚,天上人间会相见。\\
临别殷勤重寄词,词中有誓两心知。\\
七月七日长生殿,夜半无人私语时。\\
在天愿作比翼鸟,在地愿为连理枝。\\
天长地久有时尽,此恨绵绵无绝期。
\end{verse}}}
生活在集中营里,身心方面虽然不得不退化成原始状态,精神生活还是有可能往深处发展。生性敏锐的人过惯了丰富的知性生活,在营中容或会吃足苦头(这种人体格多半柔弱),但他们内在的自我所受到的伤害却少得多。他们能够无视于周遭的恐怖,潜入丰富且无挂无碍的内在生活当中。惟有从这个角度,我们才可以解释这个教人困惑的现象:看来弱不禁风的俘虏,反而比健硕粗壮的汉子还耐得住集中营的煎熬。为了使读者容易了解我的意思,我不得不再用我个人的亲身经验来作说明。容我再谈谈我们每天清晨动身前往工地时的情景吧!

有人喝道:"工作分队,前进!左二三四!左二三四!左二三四!头一名向后转!向左转!向左转!向左转!脱帽!"这些命令,迄今仍在我身边回响着。"脱帽!"令一下,我们遂经过营区大门,探照灯直射在我们身上,凡是精神不够抖擞的,立刻会挨一顿踢打;至于未经许可,即因耐不住寒冻而重行戴上帽子的人,则更加倒霉。

在昏暗的晨曦中,我们沿着处处坑洼石块的道路蹒跚而行。随行的警卫不时吆喝着,并以步枪枪托驱赶我们。两脚肿痛难挨的,就得仰赖隔邻难友的搀扶。一路上,大家默不作声,刺骨的寒风使人不敢开口。我旁边的一个难友,突然用竖起的衣领掩着嘴巴对我说道,“我们的太太这时候要是看到我们,不知会怎样?我倒希望她们全都呆在营里,看不到我们这副狼狈相。”

这使得我想到自己的妻子。此后,在颠簸的数里路当中,我们滑跤、绊倒,不时互相搀扶,且彼此拖拉着往前行进;当其时,我们默无一语,但两个人内心却都知道对方正在思念他的妻子。偶尔我仰视天空,见繁星渐渐隐去,淡红色的晨光由灰黑的云层中逐渐透出,整个心房不觉充满妻的音容。我听到她的答唤,看到她的笑靥和令人鼓舞的明朗神采。不论是梦是真,她的容颜在当时,比初升的旭日还要清朗。

突然间,一个思潮使我呆住了。我生平首遭领悟到偌多诗人所歌颂过,偌多思想家所宣扬过的一个大真理:爱,是人类一切渴望的终极。我又体悟到人间一切诗歌、思想、信念所揭露的一大奥秘:“人类的救赎,是经由爱而成于爱。”我更领会到:一个孑然一身,别无余物的人只要沉醉在想念心上人的思维里,仍可享受到无上的喜悦——即使只是倏忽的一瞬间。人在陷身绝境、无计可施时,唯一能做的,也许就只是以正当的方式(即光荣的方式)忍受痛苦了。当其时,他可以借着凝视爱侣留在他心版上的影像,来度过凄苦的难关。生平首遭,我总算了解到下列这句话的真义:“天使睇视那无限的荣耀,竟至于浑然忘我。”(The angels are lost in perpetual contemplation of an infinite glory)

在我前面,有个人跌倒了,后边几个人跟着一一绊跤。警卫冲过去,挥鞭猛打,我的思路因之中断了几分钟。所幸,我很快就卸下俘虏的身份,飞回另一个世界,继续与妻交谈。我向她发问,她答复了;轮到她提出问题,我也回答了她。

“停!”我们已抵达工地,而且纷纷冲进漆黑的茅舍,巴望抢得到一件像样的工具。不久,每个人手上都有一把锤子或鹤嘴锄。

“快一点不行吗?猪!”大家连忙各就各位,回复到前一天在壕沟里工作的位置。冻得死硬的土壤,随着鹤嘴锄的敲击而迸裂,而溅出火花。众人默无一语,脑部冻得发麻。

妻的影像,仍萦绕在我心头。一个念头掠过我脑际。我连她是生或死都不知道。我只晓得一件事(此事我而今已深为熟稔):爱,远超乎我所爱的人的肉身以外。爱最深刻的含义,就蕴藏在她的精神层次、她的“内在我”当中。不论她是否近在眼前,不论她是否尚在人间,其实都已经无关紧要。

我不知道妻是否尚在人间,也无从查询(被俘期间,不准通邮),可是这在当时并不重要。我已经不需要知道了。任何事物,都动摇不了我的爱情、我的思念,以及我所爱的人的影像。当时,即使我获悉妻已仙逝,我想我还是会平静地瞑想她的音容笑貌,我与她之间的精神晤谈还是会一样生动、一样宽慰我心。毕竟,“但教心似金钿坚,天上人间会相见”啊!


\section{死囚的美感经验}
这样子强化内心生活,就可以在空洞、贫血、孤绝的俘虏生涯中,以遁入过往的方式,找到了一个避难的港口。只要你不自羁绊,就可一任想像力驰骋于过往,咀嚼一些无关宏旨、微不足道的前尘往事。你会以怀旧的心情,把这些前尘往事一一加以美化,使其显得遥不可及,也使得你满心渴望再度身临其中。我自己就常在想像中搭上公共汽车,打开家门,接听电话且捻亮电灯。这些琐事和记忆每每令我低徊不已,乃至潸然泪下。

内在生活一旦活络起来,俘虏对艺术和自然的美也会有前所未有的体验。在美感的影响下,有时连自身的可怕遭遇都会忘得一干二净。从奥斯维辛转往巴伐利亚一集中营的途中,我们就曾透过车窗上的窥孔,凝视萨尔兹堡附近山峦沐浴在落日余晖中的美景。当时,如果有人看到我们的脸容,一定不会相信我们是一批已放弃了一切生命和获释希望的俘虏。尽管(也许正因为)放弃了一切希望,我们仍(才)神往于睽隔已久的大自然美景,并为之心醉情痴。

一个人即使身在集中营里,也可能叫身旁正在劳动的难友抬头观赏落日余晖中的巴伐利亚森林(一如画家丢勒——Dürer——在其一幅名水彩画中所示)。在该处森林中,我们兴建了一座巨大而隐蔽的军需工厂。有天傍晚,我们已经捧着汤碗,疲累万分地坐在茅舍内的地板上休息;一个难友冲进屋里,叫大家跑到集合场上看夕阳。大伙儿于是都站到屋外,看到西天一片酡(tuó)红,朵朵云彩不断变幻其形状与颜色,整个天空真是绚烂之极、生动万分。相形之下,灰黑的破茅舍显出强烈的对比;泥泞的集合场上,大大小小的坑洼则映出灿烂夺目的晚天。大伙儿屏息良久,一个俘虏才慨然一叹:“这世界怎会这么美啊!”

又有一次,我们在壕沟里劳动。周遭是灰蒙蒙的晨曦,头上是灰蒙蒙的天空,眼前下的是灰朴朴的雪,连大伙儿身上的破衣,以及每个人的脸孔,都是清一色的灰黯。当时,我再度默默地与妻交谈——或者该说是我正努力为自己身受的痛苦和凌迟寻找一个原因。就在我与死亡阴影笼罩下的无望感作最后也最激烈的抗辩之时,我意识到我的灵魂挣脱了把我团团困住的阴郁,且超越了这无望、无意义的尘世。突然间,我听到一声胜利的肯定,从某处遥遥传来,仿佛是在答复我针对生存的终极目的而提出的疑问。就在那时,遥远的地平线上,有幢农舍在巴伐利亚灰暗的晨曦中亮起了一盏灯——那盏灯,就这样照亮了昏暗的周遭。一连好几个钟头,我站着挖掘冰冻的雪地,警卫从我身旁走过,辱骂了我几句,我于是再度和妻交谈。我愈来愈感觉她就近在眼前,同我在一起;我甚至觉得自己碰得到她,还可以伸手握住她的手。这个感觉非常强烈。恰在那时,一只鸟悄然无声地飞下来,而且就栖息在我前面——在我刚刚挖出来的土堆上——还目不转睛地望着我。


\section{营中艺术活动}
先前,我曾提到艺术。集中营里,也会有艺术这种东西么?这倒要看你所谓的艺术究竟是指什么而定。营中不时举行一些业余节目。每逢其时,有幢茅舍便会暂时腾出来,排上几条木条凳,还有人负责草拟一张节目单。当晚,营中稍有地位者(也就是像酷霸和一些不必到工地去做工的人)全都到场,大概是专程来笑一阵或哭几声——总之是为了消愁破闷。节目中有歌唱、诵诗、讲笑话等等.有的还暗暗讽刺营中的人、事、物。这一切,全是刻意要帮助我们忘忧的——也的确有所帮助。有些普通俘虏就因为这种节目很有消愁破闷之效,才不惜拖着疲惫的身子或冒着分不到当日口粮的危险而争先往观。

在工地的半个钟头午餐时间里,我们可以在分汤(汤由承包商负责供应,所费不多)时聚集到一间未完工的机房内。进门时,每个人都得到一勺稀汤。大伙儿正啜得起劲,有个俘虏爬到一个桶子上,唱起意大利抒情曲来,我们欣赏了他的歌,他则获得双份“直接由桶底捞上来”的汤——这表示汤里有豌豆!

在集中营里,不只献艺有赏,喝采也有报酬。即如我,就曾因为喝采,而能够从一位素以“杀人魔”著称的酷霸那儿获得保护(幸好我从不需要他的保护)。事情是这样子的:有天晚上,我有幸再度应邀前往曾举行过招魂会的那间房间。里头,仍是营医的那一票密友;而卫生队那位准尉军官也再度偷偷跑来参加。“杀人魔”酷霸凑巧走了进来,当下有人便请他朗诵他在营中相当出名(该说是出了臭名)的一首诗。他毫不迟疑,立刻掏出一本日记似的小册子,并且朗声诵读他的杰作样版。其中有一首情诗,差点没叫我爆笑出来;幸好我竭力咬住嘴唇,且咬到发痛的地步,才勉强忍住不笑。我这条老命,极可能就是靠这种“忍功”拣回来的。此外,我因为不吝于喝采,所以我即使被分发到他的工作队上(以前我曾被调去呆了一天——光是一天,就够我受了),也不必耽心有生命之忧。无论如何,让这位“杀人魔”酷霸对你产生好感,只有百利而无一害。所以当时,我竭尽所能报以热烈的掌声。

当然,营中的一切艺术活动,一般说来都显得有些怪异。我愿意说,一切与艺术有关的活动所给人的真实印象,恰恰都源于活动本身与荒凉的营中生活之间不协调的对比。我永远也忘不了我在奥斯维辛过第二夜,由疲惫已极的熟睡中被一阵音乐吵醒的情景。原来茅舍中那个资深舍监正在他房中举行一种庆典。他的房间就在茅舍的入口处。他酒醉了的嗓子,嚎叫出陈腐的曲调。突然间,一切归于寂静。就在万籁俱寂的夜里,一支小提琴幽幽地唱出一首凄怨欲绝的探戈——一首百听不厌、久奏不腻的仙曲。弦弦掩抑声声思,我也跟着小提琴掩泣起来;因为就在当天,有个人正值二十四岁的生日。那人身在奥斯维辛的另一区,离我可能只有几百码,甚或几千码之遥,然而却与我咫尺天涯,不得相见。那人是谁?是我的妻啊!


\section{集中营幽默}
集中营里,居然也有艺术之类的玩意儿,这个事实局外人想必会大吃一惊。不过,要是他听说营中还有幽默感这东西,很可能更要啧啧称奇了。当然,所谓的幽默感只是淡淡的痕迹,而且为时不过短短数秒钟或数分钟。为求自保,幽默感是另一项精神武器。众所周知,幽默是人类性情当中最能使人超越任何情境的一种。即使超越的时间只是短短数秒也是弥足珍贵的能力。我就曾实地训练一位在建筑工地中与我并肩做工的友人培养幽默感。我建议他,以后我们每天至少要想出一则笑谭趣事——一则与获释之后可能遭遇到的情况有关的趣闻。他是一名外科医生,曾在某大医院充当助理。有次,我就因为对他描述他回复原职之后,将如何改不掉营中习惯,而逗得他捧腹不已。在建筑工地,监工为了叫我们勤快些,常吆喝道;“干呀!干呀!”尤其在督察巡视的时刻,更是吆喝不停。我于是告诉这位友人:“终有一天,你会回到手术房,执行一项腹部大手术。突然间,一个看护人员冲将进来,吆喝道'干呀!干呀!',借以宣布主任大爷的光临。”

有时候,别的难友也会假想一些与未来有关的趣事。譬如,有人就预测在未来某天的一次晚宴上,盛汤时,自己很可能一时忘情,而央求女主人“由桶底直接捞上来”。


\section{苦中作乐}
试着培养幽默感,试着以幽默的眼光观察事物——这是研究生活艺术时必学的一招。人世间尽管处处有痛苦,却仍有可能让生活的艺术付诸实现,即便在集中营里亦然。容我打个比方:痛苦就像是煤气。一个空房间里,如果注入某一定量的煤气,则不论房间多大,煤气都会完全均匀地弥漫。同样地,痛苦不论大小,都会完全充满人的心灵和意识。因此,人类痛苦的“尺度”,绝对是相对的。\footnote{心灵的空间越大,所承受的痛苦的强度就越小。}

也因此,一件极其琐碎的小事,也可以引发莫大的喜悦。我且举个例子:从奥斯维辛转往达荷城附近一集中营的途中,我们一直耽心火车要开往莫豪森营。接近多瑙河上的某座桥时,我们益发紧张起来。因为,据有经验的旅伴说,如果火车要开往莫豪森,一定会经过那座桥。后来,当大伙儿获悉火车“只不过”是开往达荷,并未经过那座桥,整个车厢立刻爆出欢笑和歌舞的喧闹声。那种场面,非身历其境的人简直不能想像!

至于在两天三夜的旅途之后抵达荷城时,又有怎样的遭遇呢?在火车上,由于空间太窄,大多数人只好全程枯站,幸运的少数则轮流蹲在满是尿骚臭的稻草堆上。抵达时,从老俘虏那儿打听到的第一条大消息便是:这个小型集中营(人口仅二千五百名)没有“炉子”、没有火葬场、也没有煤气!这表示所有变成“末世脸”的人,不会直接被送到煤气间,而要等到所谓的“病患护送队”组成以后才被遣回奥斯维辛。这个令人惊喜的大好消息,使得大伙儿心情特佳。奥斯维辛那位资深舍监的愿望终于重视了:我们这么快,就已经来到一个没有“烟囱”的集中营里。当下,我们欢笑作乐,管他紧接着又要忍受什么样的煎熬?
\endnote{我在想这个时候坚持下来的俘虏一定都怀着一个信仰,那就是战争会结束的。尽管这在当时看起来是那么的遥不可及,甚至是痴心妄想。但也许真是这骨子信仰才让人类超越于动物之上。这种信仰的背后当然需要一些道理,否则只是盲目的相信,而更重要的是坚持这种信仰能够让人获得实际的好处,我不是说最终的奖赏,我是说这种信仰的坚持本身就是一种好处。那些不轻言信仰而要一心证伪的人实际上是自愿放弃了人类最强大的一项心灵技能,这样将使得他们的灵魂软弱无力,使得他们的精神生活贫穷困乏。因为在这个世界上有太多的东西是不能直接证伪的, 而那些夸夸其谈宇宙真理的人不过是历史的笑料罢了。}

清点新到者的人数时,当局发现有名俘虏失踪了,要我们在风雨交加的户外等着,直等到寻获失踪者为止。后来,终于在一幢茅舍内找到了那家伙——他因为疲劳过度,在那儿呼呼大睡。点名完毕,我们立刻受到“游行”处分;当晚,还通宵在户外枯站,忍受长途旅行后的疲劳及风雪刺骨的滋味。尽管如此,大伙儿还是非常开心!这儿好歹没有烟囱,奥斯维辛则已经遥遥其远了。

有一次,我们看到一群罪犯路过工地。当时,一切苦难的差距,在我们看来何其明显!我们嫉妒那些罪犯,因为他们似乎活得较有保障、较有条理,且较为快乐。他们当然有定时洗澡的机会啰——我们悲哀地想着。很可能还有牙刷衣刷、草席(而且是一人一张),每个月还有邮件告知亲人的下落或生死;而这一切,我们老早以前就已经无权享受了。

我们之中,也有人特别幸运,能够进工厂,在户内做工,而成为众人争羡的对象。这种救命似的好运道,每个人都梦寐以求。然而所谓的幸运,毕竟是相对的;幸运的尺度,因而可一再延伸。同样是令人生畏的户外工作队(我就是属于这种工作队),其中就有些队是公认比较倒霉的。一旦置身这种工作队中,你自然会羡慕别人不必每天十二小时都得在陡坡上踩着满腿烂泥清理战地铁道的木桶。大多数的意外事件,都发生在这种工作上;而一旦出了意外,往往有丧命之虞。

有些工作队的监工,特别喜欢整人,因而,我们总要比较谁运气好,不必受其指挥,或只是暂时归其管辖。有一次,我不幸奉派到这种工作队上。要不是两个钟头后发生了空袭警报,以致在警报解除后必须重整队伍,我想我可能早就因受不了监工的虐待而躺上专门承载劳累致死或濒死者的雪橇,被运回营去了。在那种情况下,警报所带来的解脱,没有人能够想像——即使是在拳赛中听到一回合终了的铃响,因而避免了致命一击的拳击手,也无法想像。

就连最微不足道的运气,我们也庆幸不已。只要在就寝前有时间捉虱子,我们就高兴得很。倒不是说这有什么乐趣;光着身子站在寒气逼人、天花板上结满冰柱的茅舍内,可不是闹着玩儿的。然而在“捉虱大典”中,只要没熄灯或空袭警报,就值得我们千恩万谢了。因为,这件事没办好,我们一整夜休想睡个好觉。

\hlabel{甚少快乐}
在集中营生活里,这种贫弱的欢娱,为大伙儿提供了消极的快乐——也就是叔本华所说的“苦中作乐”(freedom from suffering)——然而就连这种快乐,也是相对性的。真正的快乐(即使是细微的).可以说几乎没有。记得我有一次曾经草拟一张《快乐明细表》,结果发现,在过去好几个星期中,我总共只有两次快乐的经验。其中一次是这样的:我从工地回来后,苦等良久,终能进入厨房,并且被分发到由冯姓伙夫(也是俘虏)主勺的队伍里。冯伙夫站在一个大锅后,接过每个俘虏递上去的碗,一一盛上汤,众俘虏则一一迅速离开。这人是唯一不看情面、一视同仁、分汤公正的伙夫。他对自己的好友或乡亲,并不会特加关照,为他们捞出锅底的马锋薯,而叫其他人喝薄稀稀的汤。

不过,我无意责怪那些特别关照自己人的俘虏。在那种生死攸关的情况下,谁能苛责别人袒护自己的朋友呢!一个人除非在相同情况下也能够作到绝对的公正无私,否则无权去判断别人。


\section{救命仙丹}
我恢复正常生活(即重获自由)很久以后,一位友人拿了张画刊给我看,上面登了几帧照片,全是集中营俘虏挤躺在木板床上.眼光呆滞地盯着一名访客的镜头。“很可怕,不是吗?那种呆滞的表情底下,隐含了多少恐怖啊!”

“怎么说呢?”我问着,因为我的确不懂得他的意思,也因为在那时候,我仿佛重又身临其中:早上五点正,天色仍一片漆黑,我躺在一间土屋里的硬板床上,同其他约七十名与我一样“受到照顾”的难友挤在一起。我们病了,不必离营做工,不必出操受罚,却可以整天躺在屋里打盹,等着每天照例要分发的面包(当然,病人的份量较少)和汤(病人的汤不仅较稀,量也大减)。虽然事事不算如意,我们却心满意足,衷心快慰。试想,当我们彼此缩在一起,以防暖气外泄;当我们懒得连手指头都不愿一动,屋外的集合场上,却传来尖锐的哨声与吆喝声。值夜班的俘虏刚从工地回来,正等着点名。我们的房门被推开了,风雪长驱直入,一名筋疲力竭的难友满身雪泥,一拐一拐地闯进来,正打算坐下来休息几分钟,可惜却被资深的舍监给撵了出去。在病人营舍,病人尚在接受检验的期间,陌生人是严禁入内的。当时,我多么替那家伙难过,又多么庆幸自己生了病,可以躲在屋里打盹啊!能够在病人营区呆个两天,甚至还可能再多呆几天——这不啻是救命仙丹哩!

我一看到画刊上那些照片,这一切记忆全又浮上脑海。经我解释过后,友人才了解我何以不觉得那帧照片有何恐怖之处。毕竟,照片中的人可能根本就不觉得难受呢!

在病人营舍的第四天,我才刚被分派去值夜班,主任医官就冲进来,请我以自愿方式,前往斑疹伤寒病人区,负责医疗工作。我不顾好友的苦劝,不顾没有一位同业愿效此劳的事实,而决定前往。我知道我在工作队里,必然不久于人世;然而我如果非死不可,总得让自己死得有点意义。我想,我与其茫无目的地苟活,或与其在生产不力的劳动中拖延至死,还不如以医生的身份帮助难友而死去。这种死,我觉得有价值多了。

我这只是权衡轻重而已,并不算什么奉献牺牲。不过,卫生队那位准尉军官却偷偷叫人特别照顾两名自愿到斑疹伤寒营服务的医生。我们一副虚弱模样.使得他生怕自己手上又多了两具尸体,而不是两名医生。


\section{独处的渴望}
\hlabel{不要与众不同}
前曾提到,在集中营里,任何事只要与生存活命没有关系,就没有价值。为了活命,营中人不惜作一切牺牲。但这势必威胁到他向所秉持的理念与价值,因而使他陷入精神的惶乱中,尝到价值失落的痛苦。生活在集中营这草菅人命、夺人心志、蔑视人性尊严、视人如待戮牲口(不过却打算榨尽他最后一滴劳力)的世界里,如果不尽力抗拒这种价值失落的痛苦,努力为自己保留一点自尊,终将丧失生而为人,具有独特心智、独特内在自由及个人价值的意识。当其时,你会认为自己不过是一大群人当中的一个;你的存在将退化到与禽兽无异的地步。事实上,集中营大多数的俘虏就是这样:一大群人,像羊群一样任人随意驱赶,毫无自己的思想和意志;而一小撮无赖,则由四面八方密切监视,并以各种酷虐手段任加折磨。他们不断地驱赶羊群,并以吆喝、踢打、棍击来指示方向;至于我们这群蠢羊,则只是一心一意地想着两件事:如何躲避恶狗与如何挣取一点食物。

羊总是胆怯地挤入羊群中央,我们也一样。每个人都努力往队伍的中心挤,一则比较能避免挨揍(警卫总是在队伍的前后及两侧走着),再则也可以避风。因此,拚命挤进队伍里头,其实就是为了自卫。在队伍里如此,在其他时候亦然。我们总是努力服膺自卫的第一要规,不要显得与众不同!每个人随时随地,都尽力避免引起挺进队员的注意。

当然,如果可能,甚至如果有需要,也该离开群众。大家都知道,在团体生活当中,如果一举一动都要受到监视,人很可能极端渴望离开团体——即使只是离开一下。营中俘虏很渴望独处,也渴望一个人静下来想想。他企盼孤独、企盼隐私,然而不见得能偿宿愿。我在转到所谓的“休养营”(rest-camp)以后,就碰上了难得的运气,有了每次约五分钟之久的独处时间。我工作的那间土屋(里头住了五十名高烧昏迷的病人)后面,靠近双层铁丝网的地方,有个安静的角落,在那里有人用几根木条和树枝,临时搭了个帐篷,权充太平间(营里每天平均有六个人死亡)。那儿还有个坑口,和自来水管相通。我只要没事,就坐在木质的坑口盖上,呆望着缀满鲜花的山坡和铁丝网交错下的蓝蓝远山。我幽幽地梦想着,思绪飘向了北方和东北方,搜寻着记忆中的家园。然而,我举目眺望,但见浮云而已。

身边的死尸爬满跳蚤,我却不以为意。能使我由梦中惊醒的只有过路警卫的脚步声。有时,这脚步声是为了召我回病房或回去点收新到药品(只有五片到十片的阿斯匹灵,却要应付五十名病人几天之内的需要)。我每次点收完毕,就去巡视病人,量一量他们的脉搏,并且分半片药给几个病重的。至于病入膏肓的人,我一律不发给药品;一方面是因为服药己无济于事,再则是因为药品奇缺,须尽量留给有痊愈希望的人;病情轻微的我除了鼓励几句以外,别无药品可给。我就这样在病房内蹒跚穿梭,逐一问诊,而我自己却因为大病初愈,仍然非常虚弱。巡视完毕,我又回到坑口盖上,静享独处的喜悦。

这个坑口,有次偶然拯救了三名难友。就在我们获释前不久,当局计划把大批俘虏运往达荷。这三名难友非常精明,企图逃避外调。他们爬入坑口,躲避警卫的搜索。我则若无其事地坐在坑口盖上,佯作不知情地玩着小孩子的把戏,把一颗颗石子丢向铁丝网。警卫看到我,迟疑了一会,但还是走开了。我总算有机会告诉下面那几个仁兄:要命的阎王已经走啦!


\section{人命如蝼蚁}
集中营里的人命,究竟多么不值,局外人通常很难以理解。营中人心肠虽硬,但每当一个“病人护进队”组成之时,大家就更意识到人命全然不受重视的事实。病人衰弱的身体,往往被丢上二轮马车,由别的俘虏冒着大风雪,拉了好几里路到下一个集中营去。在马车离开以前,如果有哪个病人死了,照样要丢上去——因为名册上非得正确无误不可。唯一重要的——只有名册。一个人的价值,就在于他有个俘虏号码。他名符其实地成了个号码。是死是活倒无关紧要,反正同样是个号码;而一个号码的生命是完全微不足道的。至于这个号码及这个生命背后所含的一切,包括命运、身世、姓名等等,不用说更是无足挂齿了。运送病人时,我因为是医生,必须陪病人从巴伐利亚的一个营转到另一个营。有次,有个年轻俘虏因为他哥哥未被列入名册,必须留下来,便一直哀求不停。管理员被缠得没办法,只好来个对调:把他哥哥和一名在当时较喜留下的俘虏对换过来。可是名册上却必须正确无误!这倒简单,两个人只要对换一下号码,就行了。

我曾经提过,我们一无证件,每个人侥幸仍拥有一个总算还在呼吸的身体。至于身体以外的一切——也就是挂在我们瘦骨架上的那身破衣——只有在我们被调往“病人护送队”时。才会招人觊觎。行将离去的“末世脸”,常遭到厚颜好奇的检视:许多人都想看看他们的衣服鞋子是否比自己的还要好。毕竟,“末世脸”气数已尽;但留在营中、还能卖命的人,则必须想尽一切办法,来改善眼前的生活啊!这些人不会感情用事。他们知道自已的命运,完全取决于警卫的心情。正因为这样,他们才罔视人性,而且变本加厉。


\section{德黑兰的死神}
我在奥斯维辛时,就曾暗自订下一个规则。这规则屡经考验,效果良好,后来大多数的难友都争相效尤。一切问话,我大都照实回答;但若问得不明确,我便缄口不答。问到年龄,我据实以告;问到职业,我答:“医生”,但却并不详细答复。在奥斯维辛的第一个上午,一个挺进队员来到操场,大伙儿必须按四十岁以上、四十岁以下、金属工、机工(以此类推)……分成不同的队伍。后来接受受体检,有疝气的又另组一个新队。我那队被赶到另一间土屋重新整队,经过再一次的分组和问话(关于年龄职业的),我被分到另一个小组,然后又被赶到另一间小屋,再重新组队。就这样一连循环了几次,把我搞得烦死了,尤其我后来发现自己竟处在一群言语不通的陌生人当中,心里真是闷闷不乐。不久,最后一次的分组总算结束;万没想到,我竟又回到最初所属的那一队!主事者根本没注意到我这段时间里换了几个房间,不过,我却明白在这几分钟之内,命运之神用了许多种不同的方式,放了我一马。

病人转运往“休养营”的消息一经发布,我的名字(也就是说,我的号码)赫然在目——因为也需要几名医生。不过,没有人相信目的地的确是休养营。几个星期前,当局就曾筹备过同样的换营计划;当时,每个人也都以为那是要转运到煤气间。结果,当局一宣布愿值夜班(夜班人人避之犹恐不及)者可以除名,立刻有八十二名俘虏自动请缨。一刻钟后,换营计划取消了,那八十二名可怜虫,却仍然列名于夜班名册上。这表示他们中大多数人,在两星期之内都会撒手西归。

如今,转往休养营的计划再度拟定,然而这究竟只是想榨出病人体内最后一滴劳力(即使只是短短的两星期)的阴谋,或其实是要送入煤气间,或竟真的是前往休养营,没有人知道。当晚十点差一刻,对我已颇有好感的主任医官偷偷告诉我说:“我已经向营本部报备过了,十点钟以前,你还可以划掉名字。”

我告诉他说,这不是我处世的方式,我已经习惯于顺其自然了。“这样.我或许可以和我的朋友在一起。”我又说道。他的眼神流露着怜悯,仿佛他知道个中蹊跷似的。当下,他默默地握着我的双手,似乎是祝我平安——不是平安地活着,而是平安地蒙主恩召。我慢慢踱回我的住处,发觉有个好友正等着我。

“你真的要跟他们一起去吗?”他伤感地问着。

“对,我就要走了。”

他的眼眶涌出了泪水,我只好温言相慰。后来,我想到我该做一件事——立遗嘱。

“欧图,你听着,万一我没有回家和我太太见面,而且万一你见得到她,就告诉她说,我每天无时无刻不惦念着她,和她谈话。记住了吗?第二,我爱她远超过任何人。第三,我和她婚后厮守的日子,虽然太短,但在我心目中,却比任何事——包括我们在这儿所受的一切折磨——还要有份量。”

欧图,如今你在哪里?你还活着吗?从那次最后一晤以来,你又碰上怎样的遭遇?你找到你太太了吗?你是不是还记得我不顾你伤心落泪,硬要你一一牢记的每句话?

翌晨,我随队起程了。这一次倒不是阴谋,我们并非走向煤气间,而的的确确是走向休养营。原先怜悯我的那些人,则留在那个不久大闹饥荒的旧营里,而其饥荒现象,远比我们的新营还要严重。那些人力图自救,无奈回天乏术。几个月后,我重获自由,遇到一个从旧营出来的朋友。他告诉我说,当时他因为是个营警,曾经调查死尸堆里遗失的一块人肉。结果发现那张肉正在锅里煮着,便把它没收了。同类相食的事件竟然发生,我那时离开正是时候啊!

这使我不由得想起一则德黑兰死神的故事:一个有财有势的波斯人有天和他的仆人在花园中散步,仆人大叫大嚷,说他刚刚碰上死神威胁要取他的命。他请求主人给他一匹健马,他好立刻起程,逃到德黑兰去,当晚就可以抵达。主人答允了,仆人于是纵身上马,放蹄急驰而去。主人才回到屋里,就碰上死神,便质问他:“你干嘛恐吓我的仆人?”死神答道:“我没有恐吓他呀!我只是奇怪他怎么还在这里而已。今天晚上,我打算在德黑兰跟他碰面哩!”


\section{自由的曙光}
营中人很怕做决定,也怕主动做任何事情。这是因为大家都强烈地感觉到命运是人的主宰,人不能企图改变它,只能任由它自然发展所致。这种感觉,每每因惯常的冷漠而益形加深。有时候,生死攸关的决定,必须在闪电般的瞬间做出。然而每个人都宁愿由命运替他做主。这种逃避行动的现象,在面对是否逃亡的问题时最为明显。当其时(只是短短几分钟),他备尝犹豫不定的煎熬。他尝试逃亡好吗?他该不该冒险?

这种煎熬的滋味,我也尝过。当战火逐渐逼近,我有过逃亡的机会。一位同行由于必须到营外的土屋去作例行巡诊,想趁机带我一块逃命。他打算以某病人需要一位专科医生会诊为由,把我偷偷带出去。营外,有名外国反抗运动分子将供应我们制服和证件。就在最后一刻,碰到一些技术性的问题,必须再度回营。我们就利用这个机会,张罗了一些补给品(几枚烂马铃薯),再寻找一个帆布背包。

我们闯进女营区的一间空屋里,由于女俘已调往他处,营区内空无一人。那间空屋凌乱不堪,显然许多女俘都张罗好补给品逃掉了。屋内散置着破衣服、发霉的食物,和破旧的陶器。有几个碗还算完好,对我们非常有用,但我们还是决定放弃。我们知道,在情势逐渐恶化的最近,这些碗不仅曾用来装食物,还用来盥洗和充当夜壶。(当局严禁在屋内持用任何器皿,不过也有些人——尤其是身体太虚弱、连有人搀扶都无法走到屋外的斑疹伤寒病人——不得不违反禁令。)我在垃圾堆里搜索着,并且找到了帆布背包和一根牙刷。突然间,我在一大堆杂物当中发现了一具女尸。

我又跑回我居住的土屋,收拾我所有的财产:一个饭碗、一双由病死的难友那儿“继承下来”的手套、几张写满速记符号的废纸头(前曾提到.我有一部书稿在奥斯维辛那儿被没收了,后来我就用这些废纸头重新撰写)。然后,我又到各土屋,为正挤卧在屋内两侧朽木板上的病人迅速作最后一次的巡视。我来到我唯一的乡亲面前。我曾经不顾他的病情,竭力营救过他,然而此际他差不多已经奄奄一息。我不得不隐瞒我的逃亡企图,但他似乎嗅出了异样(也许是我表现得有些紧张)。他以疲惫的声音问我:“你也要出去?”我立刻否认,然而我却回避不了他那伤感的眼神。巡视完毕后,我又回到他那儿,再度瞥到他无望的神情;不知何故,我竟觉得那是一项控诉。打从我答应友人愿相偕逃亡以来即蟠踞心头的不快感,此时更加强烈,突然间,我决定在这一次自行操纵命运。我奔出土屋,告诉友人我不能去了。我一说出我已决定留下来陪伴病人,不快之感立刻云散烟消。我不知道以后的几天会有什么遭遇。但我内心,却获得前所未有的平静。我回到土屋中,坐在我乡亲脚旁的木板上,试着安慰他;然后又同别人聊天,试着抚平他们迷乱的神智。

集中营生活的最后一天终于到了。由于战火线逐渐接近,绝大多数的俘虏都已运往他营;管理当局、酷霸和伙夫更是走个精光。这一天,当局发布一道命令,要营中人员在日落前完全撤出,即使是仅余的几个俘虏(病人、医生、和“看护”)也必须离开。当晚,整个营就要放火销毁了。然而,载运病俘的卡车下午并未出现;而营门却突然关闭了,铁丝网一带也加紧戒备以防逃亡。看样子,营中仅余的俘虏注定都要葬身火窟了。我和友人遂决定再度逃亡

我们奉命埋葬铁丝网篱之外的三具尸体。整个营只剩下我们两人还有足够的力气干这件事,其他人差不多全呆在还有用的几间土屋里,被高烧和神智迷乱弄得精疲力竭。我们拟好了计划:运出第一具尸体时,把友人的背包放在充作棺材的旧洗衣桶里,偷偷运出去;运送第二具尸体时,则顺便偷运我的背包。运第三趟时,我们俩就双双溜之大吉。前两趟全照计划进行,并无差错。回营后,友人去张罗逃亡时所需的面包.免得躲在林中的几天会挨饿。我则呆呆地等着。时间一分一分地流逝,他一直没出现,令我愈等愈不耐烦。经过了三年的牢狱生活,我已经满心雀跃地期待着自由,想像着奔赴火线的仙滋妙味了。可是,我们并没进展到那个地步。

友人回来的那一刹那,营门被推开了。一辆漂亮的银色汽车缓缓驶入集合场,车身漆着大大的红十字。一位日内瓦国际红十字会的代表翩然莅临,整个营及营中俘虏都受到他的保护。他就在附近的一幢农舍中驻扎下来,以便在紧急情况时能随时策应。这种时候,谁还去操心逃亡的事呢?一箱箱的药品从车上卸下来,香烟四处分发;我们受到拍照,内心的快慰简直难以言宣。现在,我们不必再冒险奔赴战火线了。

兴奋之余,我们差点把第三具尸体给忘了,于是便把它抬到营外,放到已挖好的墓坑里。随行的警卫(是个比较不讨厌的家伙)突然变得非常温和。他看出情势已经改观,便试图赢取我们的好感。掩土之前,我们为三名死者作了短祷,他也参加了。经过几天来生死交搏的紧张以及几个小时以来的兴奋,我们祈求和平的祷词,其热切的程度比得过人类所曾吐露过的任何言语。

营中生涯的最后一日,就这样在期待自由中过去了。然而我们高兴得过早了。红十字会那位代表曾向我们保证已签署了一项协定,而且该营也不准撤销。可是当晚,纳粹挺进队却率同一批卡车抵达营区,并且带来一道清除营舍的命令,说是营中剩下来的俘虏要搬到一座中央营去,两天之内再从那儿遣送到瑞典,以便和另一批战俘交换。那些挺进队员,我们差点认不出来。他们变得和气万分,还劝我们不必怕登上卡车,说我们该为自己的运气而谢天谢地。力气还够的人,纷纷挤上卡车,病重的和虚弱的则由别人吃力地抬上去。此时,友人和我已不掩饰身上的背包。我们站在最后一队里,等着当局挑选十三人搭上最后第二辆卡车。主任医官挑出了需要的数目,却把我们两人给遗漏了。那十三个人登上车,我们却必须留下来。惊讶、懊丧、失望之余,我们责怪主任医官,他却推说他太累了,分了心,何况他以为我们还想逃走。我们只好背着背包坐下来,不耐烦地和剩下来的几个俘虏一起等着最后一辆卡车。由于必须等很久,我们便在警卫室(己空无一人)里的草席上躺下来。几个钟头以来的紧张与兴奋,希望与绝望,已经把我们搞得精疲力竭。当下,大家和衣而眠,随时准备出发。

步枪和大炮的声音遥遥传来,曳光弹和枪弹的闪光照进屋内。主任医官冲进来,命令我们趴在地上掩护。一名俘虏由床上跳下,穿着鞋的脚丫踩到我的肚子,这下我可完全醒过来啦!不多时,我们总算明白了究竟。战火线已经抵达营区了!枪炮声渐渐消竭,晨光终于破晓,屋外,营门旁的那根柱子上,一面白旗正随风飘扬。

好几个星期以后,我们才发觉命运之神即使在最后的几个小时,还是玩弄了我们这些剩下来的俘虏。我们发觉人的抉择是多么不可靠,尤其在攸关生死的大事上。有人拿了几张在离我们营区不远的一个小营里所摄的照片给我看。原来,那些自以为正要奔向自由的俘虏,当晚都被卡车载到这个小营里,并被锁在土屋内活活烧死。他们的尸体虽然烧焦了一部分,在照片上却依然清晰可辨。我不觉又想起了德黑兰死神的故事。


\section{吃瘪与吃香}
\hlabel{躁急易怒}
俘虏的冷漠,有其自卫的功能,但冷漠本身,也是由其他因素所促成的。除冷漠以外,俘虏的精神状态另有一个特征,那就是躁急易怒。这两种精神状态,都肇因于饥饿和睡眠不足(在正常生活中,也有此可能)。睡眠不足,部分是因为跳蚤太多,不胜其扰。挤得水泄不通的房舍,如果再不讲究卫生,就容易滋生蚊蚋。另方面,缺少尼古丁和咖啡因的刺激,也会使人容易冷漠和躁怒。

\hlabel{自卑情结}
除开这两个生理因素之外,还有几个以情结(Complexes)形式出现的精神因素。大多数的俘虏都有一种自卑情结,并且深以为苦。过去,我们都曾一度自以为“有头有脸”。如今,却受到猪狗不如的待遇。(一个人内在价值的意识,原应建基于较高尚、较属精神层次的事物上,因此不可能为集中营生活所动摇。然而不要说俘虏,即便是享有自由之身的芸芸众生之中,有多少人真正拥有这样一份意识?)一般俘虏不必特别去想,就都感到自己的价值已全然贬低。这种感觉,在看到营中简单的社会结构所显示出来的强烈对比时,尤其明显。较“优秀”的俘虏,诸如酷霸、伙夫、仓库管理员,营警等等,可以说完全不像大多数俘虏那样自感吃瘪,反而自以为升格了!有的人甚至还自认为威风八面哩!至于内心酸溜溜的大多数对这一小撮吃香分子的观感,则有几种不同的表达方式,而开玩笑则是其中一种。譬如,我就曾听过一名俘虏对另一名俘虏谈起某酷霸:“喝!早在他还只是某大银行总经理时,我就认识他了。如今他在这里升得这么快,岂不是时来运转了吗?”

吃瘪的大多数和吃香的少数一旦发生冲突(这种机会多的是,多半起因于食物的分配),后果多半十分吓人。因此,躁急易怒的情绪(其生理因素前已述及,若再遇到这种紧张局面,不啻是火上加油)如果最后演变成一场全武打,那可是一点也不值得惊讶。俘虏由于经常目睹殴打的场面,暴力冲动自然会跟着增强。我在又饿又累时一旦怒火攻心,就常发觉自己双拳紧握。照顾斑疹伤寒患者的期间,我因为必须彻夜生火(当局特准病人使用的),常常累得要命。不过,每当夜阑人静,每当其他人全都入眠或神智昏迷,我往往可以享受到最诗意的几个小时。我可以四仰八叉躺在火炉前,用偷来的炭,烤几个偷来的马针薯。只是翌日,我总是觉得更疲倦、更迟钝,也更躁怒。


\section{临时舍监}
我在斑疹伤寒病患区充任医生时,因为舍监病倒了,只好暂代他的职位,负责保持房舍的清洁(但愿“清洁”两字,还能用来形容那种情况下的环境),以便对当局有所交代。当局所谓的清洁检查,与其说是为了卫生,不如说是为了借机找碴。食物和药品只要多分配些,就大有帮助;然而检查员所关切的,只是走道中央有没有一根稻草?病人那块肮脏破烂、处处跳蚤的毛毯是否折叠得整齐?至于病人的命运如何,他们压根儿不管。我只要把俘虏帽从剪过发的头上猛抽下来,两个脚跟再重重一扣,然后口齿伶俐地报告:“六区九号病房,病俘五十二名,看护二名,医生一名!”他们就会满意,并且就会离开。可是在此之前,我却得把每张毯子一一弄平,把由床板上掉下的每根稻草一一捡起,再大声吆喝那些在床上打滚,扬言要捣乱我辛苦整理好的一切成果的可怜虫;而后才恭候大驾。(问题是,这些检查大员常常姗姗来迟好几个钟头,有时候干脆不来,令我白忙一阵。)吆喝是有必要的,因为发高烧的病人,已经冷漠到除非挨骂否则仍无动于衷的地步。有时候,连叫骂也不管用;这时,我就得使出浑身解数忍住一腔的怒火,才不致于出手打人。毕竟,在面临别人的无动于衷以及因而造成的险恶情势(即渐渐逼近的清洁检查)之时,任何人都特别容易变得暴躁起来。


\section{抉择与自由}
我以这种心理学的精神病理学的角度,试着解析集中营俘虏的典型特征,很可能使读者错以为人乃是完全且无可避免地受制于环境。(以集中营俘虏为例,所谓环境,即是指集中营生活的独特结构,该结构迫使俘虏迁就某一固定模式。)然而人的自由呢?在面对任何既定环境时,人的行为反应当中,难道毫无精神自由可言?有个理论说,人不过是许多生物学、心理学、或社会学条件与环境的因素支配下的产物)这种说法,难道是真的吗?人真只是这些因素凑合下偶然的产儿吗?更重要的是,以俘虏在集中营那种社会里的反应和表现,能够证实人逃不开环境的影响吗?人在面临这种处境时,难道别无选择的余地?

这些问题,不仅可以根据原则,也可以从经验方面来作答复。集中营中的生活经验,显示出人的确有选择的余地。有太多太多的实例(多具有英雄式的特质)足以证实;冷漠的态度是可以克服的,躁怒的情绪也可以控制。人“有能力”保留他的精神自由及心智的独立,即便是身心皆处于恐怖如斯的压力下,亦无不同。

在集中营呆过的我们,都还记得那些在各房舍之间安慰别人,并把自己仅余的一片面包让给别人的人。这种人即使寥如晨星,却已足以证明:\reduline{人所拥有的任何东西,都可以被剥夺,惟独人性最后的自由——也就是在任何境遇中选择一己态度和生活方式的自由——不能被剥夺。}

有待抉择的事情,随时随地都会有的,每个日子,无时无刻不提供你抉择的机会。而你的抉择,恰恰决定了你究竟会不会屈从于强权,任其剥夺你的真我及内在的自由,也恰恰决定了你是否将因自愿放弃自由与尊严,而沦为境遇的玩物及槁木死灰般的典型俘虏。

从这个角度看来,营中人的心理反应,显然比起某种生理及社会环境下的单纯反应要来得意味深长。即使像睡眠不足、缺乏食物、和繁重的精神压力等这些情境可能使人联想到营中人非以某种方式来反应不可,但若分析到最后,我们却可以发现一个俘虏之所以变成怎样的人,实在是他内心抉择的结果,而非纯系环境因素使然。因此,任何人就是处在这种情境下,根本上都可以凭他个人的意志和精神,来决定他要成为什么样子。即使是置身于集中营,他仍可以保有他的人性尊严。陀斯妥耶夫斯基曾说过:“我只害怕一件事;我怕我配不上自己所受的痛苦。”这句话,在我结识营中那些烈士以后,时常萦绕在我心头。他们的痛苦和死亡,在在\footnote{即处处,各方面的意思。}都证明了一个事实:人最后的内在自由,绝不可以失丧。可以说,他们配得上他们所受的苦,他们承受痛苦的方式,是一项实实在在的内在成就。正是这种不可剥夺的精神自由,使得生命充满意义并有其目的。

忙碌而积极的生活,其目的在于使人有机会了解创造性工作的价值;悠闲而退隐的生活,则使人有机会体验美、艺术,或大自然,并引为一种成就。至于既乏创意、又不悠闲的生活,也有其目的:它使人有机会提升其人格情操,并在备受外力拘限的情境下选择其生活态度。集中营俘虏虽与悠闲的生活和创意的生活无缘,但人世间有意义的,并不只是创意和悠闲而已。如果人生真有意义,痛苦自应有其意义。痛苦正如命运和死亡一样,是生命中无可抹煞的一部分。没有痛苦和死亡,人的生命就无法完整。

一个人若能接受命运及其所附加的一切痛苦,并且肩负起自己的十字架,则即使处在最恶劣的环境中,照样有充分的机会去加深他生命的意义,使生命保有坚忍、尊贵、与无私的特质。否则,在力图自保的残酷斗争中,他很可能因为忘却自己的人性尊严,以致变得与禽兽无异;险恶的处境,提供他获致精神价值的机会;这机会,他可以掌握,也可以放弃,但他的取舍,却能够决定他究竟配得上或配不上他所受的痛苦。

读者千万不要以为这些思虑都太超凡绝俗,太与现实生活脱节。的确,有能力达到这样崇高精神境界的人,实在寥寥无几。集中营众多俘虏当中,也只有少数几个人,能够守住完全的内在自由,且获得痛苦所惠予的那些价值。然而,即使只有一个实例,就足以证明人的内在力量,可使人超越于外在的命运。这种人,并非只有集中营里才有。在世界各地,人处处都面对着命运的挑战,面对着经由痛苦而获大成就的机会。

且以病人——尤其是罹患绝症者——的命运为例。有次,我读到一封由某个半身不遂的年轻人写给他朋友的信。信上说,他才刚获悉自己将不久于人世,即使动手术也终归徒劳。他又说,他看过一部影片,里头有个人以勇敢和尊贵的方式等候死亡。当时,他觉得能那样迎接死亡,实在是一大成就。如今——他写道——命运也给了他一个类似的机会。

这部影片,名叫《复活》,是由托尔斯泰名著改编的。几年前观赏过的人,想必也有过同样的念头。影片中所见,都是伟大的命运和伟大的人物。至于我们这些观众,在当时并没有什么了不起的命运,也没有机会去成就这种伟大。电影散场之后,我们走入附近一家咖啡屋里;一杯咖啡一份三明治落肚后,那些曾一度掠过脑际的形而上学思维很快就被我们忘到九霄云外。然而,当我们亲身遭遇一个伟大的命运.当我们必须以同样伟大的精神下决心和它周旋到底,无奈,我们早已经忘怀多年前的青春决断,只好颓然退下,树旗投降。

也许有一天,我们再度看到同一部或类似的影片。然而在此之前,可能早有其它影像掠过我们的心眼,为我们展现出多少生命斗士其远超乎区区一部电影所能展现的丰富成就。我们可能想起某个独特的人,想起他伟大的内在所曾散发出来的点点滴滴,正如我时常想起集中营一名女郎的事迹一样(我亲眼看到她死去)。她的事迹十分简单,简单得不足一道;读者听了,也许会以为是我杜撰的,然而我却觉得这仿如一首诗。

这位女郎知道自己不久于人世,然而当我同她说话,她却显得开朗而健谈。她说。“我很庆幸命运给了我这么重的打击。过去,我养尊处优惯了,从来不把精神上的成就当一回事。”她指向土屋的窗外,又说:“那棵树,是我孤独时唯一的朋友。从窗口望出去,她只看得到那棵栗树的一根枝丫,枝丫上绽着两朵花。”我经常对这棵树说话。“我一听,吓了一跳,不太确定她话中的含义。她神智不清了吗?她偶然会有幻觉吗?我急忙问她那棵树有没有答腔。——”有的。“——答些什么呢?——”它对我说,'我在这儿——我在这儿——我就是生命,永恒的生命。"

我已经说过,营中俘虏的精神状态,与其说是一大堆条分缕析的精神物理学因素所促成,无宁说是\emph{自由抉择}的结果。从心理学立场来研究俘虏,我们已知:惟有容许自己丧失精神防线的人,才会沦为集中营恶势力下的牺牲品。问题是,这所谓的“精神防线”,会是或应该是什么?

曾在营中呆过的人,每谈及当时的经验,都一致宣称最令人颓丧的困扰,就在于无从知道那种非人生活将何时了结。获释的日子,遥不可期(在我的营里,连口头上谈论此事,都毫无意义),每个人的监禁期,不仅不能确定,更是毫无期限,有位专门作研究的知名心理学家就曾指出,集中营的生活,可称为一种“暂时的生存”;我们倒可以进一步把它称作“一种无明确期限的暂时生存”。

初到一个新营的俘虏,对该营的情况通常一无所知。由别的营辗转回来的人,却又不得不守口如瓶。有些营则只见人进去,不见人出来。可是,只要一进营门,所有的俘虏心理上都会来个剧变:不明确之感告终,终局之不明确继之而来。任何人休想预测营中岁月将何时了结,或究竟有无了结之望,因为根本不可能预知。

拉丁字finis,有双重含义:一是终结或结局,一是有待企及的目标。一个人如果看不出他的“暂时存在”将于何时终结,自亦无法朝人生的最终目标迈进。他不再计划未来、安排未来,而这恰恰和生活于正常状况下的人相反。也因此,他整个内在生活的结构将随之改观,衰败的迹象亦将渐渐呈现,并由其他的生命领域(如身体)中暴露出来。举例来说,失业的工人就有类似的处境。他处于暂时性的存在中,就某方面看来,他实在无法替未来作打算,或朝一个既定目标迈进。以失业矿工为研究对象的论著,就显示出这类工人每为时间之“变形”所苦。这种内在时间的“变形”,肇因于失业。集中营俘虏也有这种奇特的时间体验,并且也深以为苦。在集中营,一小段的时间——譬如一天当中由于每一分、每一刻都充满了痛苦和疲惫,感觉上仿佛遥无止期。较长的时间——譬如一个星期——则似乎过得很快。当我说营中一日,长于营中一周,许多难友都表示有同感。这种时间体验,多么怪异啊!我不由得想起汤玛斯曼的名著“奇峰”(The Magic Mountain,书中包含极犀利的心理学观点)。在“奇峰”一书里,一群疗养院的肺结核患者也有类似的心理状况。他们同样不确知何时可以解脱,同样活得没有未来,活得茫无目标。汤玛斯曼即针对此点,研究其精神的演变过程。

有名俘虏曾告诉我,他抵达车站后,随着长长的队伍步行到集中营,当时只觉得好象是走在自己的出殡行列里似地。他的生命仿佛早已死去,有如过眼云烟,毫无未来可言。这种死气沉沉的感觉,更因为其它两个原因而更形强烈。在时间上,营中岁月漫无期限,最令人刻骨铭心;在空间上,居住范围过于狭窄,又令人透不过气。铁丝网外的一切,不仅遥不可及,更显得疑幻疑真。营外的人事物及一切的正常生活在俘虏眼中简直恍如隔世。

人一旦因为看不到未来而自甘沉沦,便容易有满腹的怀旧愁思。在本书前面,我们曾提到营中人喜欢回味过去,借以忘却眼前的痛苦,现状因而变得较不真实。可是,除去现状中的真实特点,很可能伏下一个危机。当事人势必容易忽略现实中的确存在着、而且可堪运用的机会。把目前的“暂时存在”(provisional existence)当成虚幻不实的存在——这种态度本身正是使俘虏丧失其生命力的一大重要原因。人一旦有了这种态度,任何事物看在他眼里都显得毫无意义。他忘了艰困的外在环境通常能给人一个机会,让人超越自己,从而得到精神上的成长。他不把集中营的困境看成是考验内力的试金石,他不看重自己的生命,反而轻蔑它,当它是无足轻重的玩意儿。他宁可阖上眼皮,耽溺于过去。这样的人自然会觉得人生没有意义了。

当然,有能力在精神上达到崇高境界的人只有少数几个。但这少数几个,都有机会表现其人性的伟大(即是借着世俗眼中的死亡或一败涂地来表现这种伟大)。这样的人格,若是换上普通的环境,必然造就不出。至于我们这些泛泛之辈,或许该听听俾斯麦这段话:“生命好比让牙医治牙痛,你老是以为最糟糕的情况还在后头,实际上早已过啦!” 照这句话改变一下,我们也可以说集中营内大多数的俘虏,都相信生命的真正机运早已消逝。其实,现实中永远有着机会和挑战。人可以战胜这些经验,把生命扭转成一个内在的胜利;也可以忽视现有的挑战,茫无目的地过一天算一天——正如大多数俘虏所表现的一样。


\section{超越当前的困境}
任何人若想以心理治疗或心理卫生方法来抗拒集中营对某俘虏身心上的不良影响,就必须为他指出一个可堪期待的未来目标,借以增长他内在的力量。有些俘虏出于本能,也曾设法自行寻找这样的目标。人就这么奇特,他必须瞻望永恒(sub specie eternitatis),才能够活下去。这也正是人在处境极其困厄时的一线生机,即使有时候必须勉强自己,也一样。

我还记得自己的一个亲身经验。有一次,我随着漫长的队伍由营区步向工地。由于穿了双破鞋子,两脚满是冻疮和擦伤;几公里的路程下来,我痛得几乎掉泪。天气十分寒冷,凛烈的风飕飕吹着。我脑海里不断想着这种悲惨生涯中层出不穷的小问题。今晚有什么吃的? 如果额外分配了一截香肠,我该不该拿去换一片面包? 两星期前获得的“奖金”,到现在只剩下一根香烟,该不该拿去换碗汤? 充作鞋带的一根电线断了,我如何才能够再弄一根来? 我是否来得及赶到工地,加入我熟悉的老工作队,或者我必须到另外一个可能有凶恶监工的队里去? 我该如何博取酷霸的好感,好让他分派营内的工作给我,免得我老要长途跋涉到工地作苦工?

这种叫人满脑子只想着这些芝麻小事的处境,我实在是厌倦透了。我强迫自己把思潮转向另一个主题。突然间,我看到自己置身于一间明亮、温暖、高雅的讲堂,并且站在讲坛上,面对着全场凝神静声的来宾发表演说。演说的题目则是关于集中营的心理学!那一刻间我所身受的一切苦难,从遥远的科学立场看来全都变得客观起来。我就用这种办法让自己超越困厄的处境。我把所有的痛苦与煎熬当成前尘往事,并加以观察。这样一来,我自已以及我所受的苦难全都变成我手上一项有趣的心理学研究题目了。斯宾诺萨在他的名著《伦理学》上就曾说过:“我们只要把痛苦的情绪,塑成一幅明确清晰的图像,就不会再痛苦了。”
\endnote{有两种人生态度,一种是置身于生活之中,一种是超然于生活之外。置身于生活之中,痛苦忧愁接踵而来;超然于生活之外,以倾诉者,研究者,叙述者,观察者的姿态来看待自己的生活,那么一切苦难都不过如此。}

\section{精神防线}
对未来——自己的来来——失去信心的俘虏,必然难逃劫数。随着信念的丧失,精神防线亦告丧失;此后,自然甘心沉沦,一任身心日趋衰朽。这种情形通常借着危机的形式而突然发生;而其征兆,营中经验老到的俘虏都十分熟悉。我们每个人都很害怕这种情形发生——倒不是怕发生在自己身上,而是怕发生在好友身上(自己要是已沦落到这个地步,自然就无所谓害怕了)。一个俘虏只要有这种心理危机,刚开始,通常是早上醒来以后不肯穿衣盥洗,不然就是不肯到集合场去集合。你再怎么求他、揍他、恐吓他,都没有用。他只是躺在那里,动也不动。如果这种危机起因于生病,他就拒绝住进病人区或拒绝接受任何冶疗。总之他就是放弃。他呆呆地躺在自己的排泄物当中,天塌下来也不在乎。

信心丧失与全然放弃之间,有着密切的关连。有一次我就遇到了一个非常奇特的例子.我那位资深舍监傅先生,是一位小有名气的作曲家兼作词家。有天他对我吐露心事道,“医生,我想告诉你一件事。我做了个怪梦。梦中有个声音告诉我,我可以许个愿,只要我说出想知道什么,我的一切问题就可以得到圆满的解答。你猜我问了什么?我说我想知道什么时候我可以看到战争结束。你懂得我的意思吗?——'我'可以看到!我想知道什么时候我们可以获释,我们的痛苦可以告终。”

“你什么时候作了这个梦的?”我问。

“一九四五年二月。”他答。当时,已是三月初。

“你梦中那个声音怎么回答?”

他凑到我耳边,悄悄耳语道:“三月三十日。”

傅先生告诉我这个梦的时候,还是满怀希望,深信梦中那个声音一定是铁口直断。然而,预许的日子渐渐接近,传抵营区的战讯却全不像是我们即将在预许当日获释的样子。到了三月二十九日,傅先生突然病倒了,全身发高烧。三月三十日,也就是预言中他会看到战事结束、痛苦告终的日子,他昏迷不醒,失去知觉。三月三十一日\footnote{恰好这个时候,可见人超越短暂生存只有藉著对于未来的某种期许或者某种永恒意义的寄托,如果丧失了这两样东西,大抵也就是所谓的心死了吧。},他死了。从一切外在迹象看来,他死于斑疹伤寒。


\section[参透“为何”,迎接“为何”]{参透“为何”,迎接“为何”\footnote{这一段甚是精彩。}}
心境(包括有无勇气与希望)的良窳(yǔ 良窳即优劣之意),与身体的免疫能力息息相关。懂得这个道理的人,自然会了解人如果突然失去希望和勇气,很可能因而致死。我的朋友傅先生之死,就是因为预期中的获释未曾实现,致令他陷入绝望使然。突如其来的绝望,减低了他身体上抵抗传染病的能力。由于对未来的信心及活下去的意志皆告瘫痪,身体对病毒便毫无招架之力。结果,他只好一死了之。他梦中那个声音毕竟没错。

这个案例的研究及心得结论,与营区主任医官提醒我的一件事正相符合。一九四四年圣诞节到一九四五年元旦,一星期当中,营里的死亡率大为增加,并且是前所未有的现象。照主任医官的看法,这种现象并非肇因于工作环境较恶劣、伙食配给递减或气候变化甚或新的传染病;而是因为大多数的俘虏都抱着一个天真的希望,以为他们会在圣诞节以前重归故里。当佳节渐渐逼近,佳音依旧杳然,许多俘虏逐渐都失去了勇气,因而万念俱灰,大大削弱了身体的抵抗力,结果便一个个相继死去。

前曾说过,若想重振营中俘虏的内在力量,首先就得为他指出一个未来的目标。尼采说过:“懂得为何而活的人,几乎'任何'痛苦都可以忍受。”这句话,所有与囚犯或俘虏接触的心理专家,都应奉为圭皋(guī gāo 即信奉为准则之意。)。只要有机会,就该给他们一个活下去的目的,才能够增强他们忍受“任何”煎熬的耐力。看不出个人生命有何意义、有何目标,因而觉得活下去没什么意思的人,最是悲惨了。他很快就会迷失。而这种人一听到鼓励和敦促的话,典型的反应便是,“我这辈子再也没什么指望了。”碰到这种反应,你还能说什么?

我们真正需要的是从根本上改革我们对人生的态度。我们应自行学习——并且要教导濒于绝望的人——认清一个事实。\emph{真正重要的不是我们对人生有何指望,而是人生对我们有何指望。我们不该继续追问生命有何意义,而该认清自已无时无刻不在接受生命的追问。面对这个追问,我们不能以说话和沉思来答复,而该以正确的行动和作为来答复。到头来,我们终将发现生命的终极意义,就在于探索人生问题的正确答案,完成生命不断安排给每个人的使命。}

这些使命因人因时而异,生命的意义亦然。因此,我们不可能以概括的方式来解释生命的意义;而这类的问题也绝无法用泛论来解答。“生命”并不是模棱两可的玩意儿,而是非常真切具体的东西,正如人生的使命也非常真切具体一样。这些使命构成了人的命运;每个人的命运都独一无二且各有不同,无法同别人互作比较。同样的境遇不会重复出现,每个境遇需要当事人给予不同的反应。置身在某种情境当中,人有时候必须以行动来塑造自己的命运;有时候则最好趁机深思熟虑,借以领悟人生的道理!又有时候,光是接受命运,承担个人的十字架即足矣尽矣。总之,每个情境因其特点、性质而迥然有别,其所提出的难题,也永远只有一个确切的解决方法。\footnote{人人皆有自己的命运。}

人一旦发觉受苦即是他的命运,就不能不把受苦当作是他的使命——他独特而孤单的使命。他必须认清:即使身在痛苦中,他也是宇宙间孤单而独特的一个人。没有人能替他受苦或解除他的重荷。他唯一的机运就在于他赖以承受痛苦的态度。

曾经在集中营内呆过的我们,都不认为这只是与现实脱节的空论。这是唯一对我们有帮助的见解,即使在毫无逃生之望的时候,我们也能够借着这种看法而免于绝望。很久以来,我们即已不再询问“什么是人生意义”了。这种天真的质疑,是由于把人生看成借着积极创造某种有价值的东西而实现某个目标所致。我们早已彻悟,人生意义的涵盖面不止于此,它包括生存与死亡,临终与痛苦。

一旦看透了痛苦的奥秘,我们就不愿再以忽视、幻想或矫情的乐观态度来减轻或缓和集中营内种种折磨所带来的痛苦,反而把痛苦看作是值得承担的负荷。我们不再退缩,只因为我们已了解痛苦暗含成就的机运。正是这种机运,使德国诗人里尔克(Ralner Maria Rilke 1873-1926)写出:“有待了结的痛苦,何其多也!”(Wie vielist aufzuleiden!)所谓“有待了结”的痛苦,与一般常说的“有待完成的工作”用意相类。的确,有待我们了结、完成的痛苦,实在非常繁多。所以,我们有必要勇于面对所有的痛苦,并把软弱的时刻和暗弹的泪水减到最低量。然而,我们并不必以流泪为耻;毕竟眼泪证明了我们有承担痛苦的最大勇气。只可惜了解这个道理的人少之又少。有的人偶尔会赧颜表白自己哭过。我就曾问过一位难友,他的水肿是怎么治好的,他红着脸答道:“我用眼泪把它哭好的。”


\section{寻出生命的意义}
心理治疗或心理卫生法,在集中营内十分不易进行。一旦有此机会,则进行之初,不是采取个人方式就是采取集体方式。个人心理治疗通常是一种“救生步骤”,以防止自杀为主。营中若有人企图自杀,按营规是严禁施救的。比如有名俘虏企图上吊,任何人都不可割断绳索将他救下。因而,在自杀企图萌生之前即防患未然,最为重要。

我还记得两个极其类似的自杀未遂案例。两名当事人都曾吐露过自杀的意图,并且都运用了典型的论调:他们对生命再也没什么指望了。碰到这种情形,最重要的便是让当事人了解“生命对他仍有指望,未来仍有某件事等着他去完成”。事实上,我们发觉这所谓的“某件事”,对其中一位而言,是指他的爱子——后者正在外国等着他。对另一位,则是指一件事,而不是一个人。此人是个科学家,已经撰写了一系列尚待完竣的书籍。这件工作别人是无法代劳的;正如上述那位父亲在他爱子心目中的地位,任何人都无法取代一样。

这种独一无二的特性,使得每个人都与众不同,也使得每个人的存在有其意义。这种特质与创造性的工作和人类之爱息息相关。一个人一旦了解他的地位无可替代,自然容易尽最大心力为自己的存在负起最大责任。他只要知道自己有责任为某件尚待完成的工作或某个殷盼他早归的人而善自珍重,必定无法抛弃生命。他了解自己“为何”而活,因而承受得住“任何”煎熬。
\endnote{生命赋予每个人不同的使命,不同的挑战。我们每个人都需要去面对,哪怕只是苦中作乐,哪怕只是在几乎没有希望的情景之下挣扎,我们的态度,我们的所作所为也赋予那不同的意义。朋友们,去勇敢向前,大胆地追寻你的生命的意义吧。}


\section{集体精神治疗}
在集中营,集体精神治疗的机会十分有限。合适的榜样远比空泛的言辞还要有效。因此,如果有一名资深舍监不与当局同流合污,则他的正义并鼓舞人心的作为,将使他有千万次的机会对辖下诸俘虏发挥他惊人的影响力。行动的影响,向来比言辞还具有立竿见影之效。不过,如果内心的感受力为外在某个情境所增强,则口头劝勉仍然有相当的功效。记得有一次,某营舍发生事故,全舍俘虏皆深受震撼,内心的感受力因之大增。当时,舍监便安排了一场集体精神治疗。

那天真够糟糕。大伙儿在集台场上听训,当局宣称,今后许多行为,譬如从旧毯子割下一段一段的小布条(用来垫脚踝)或其他极其微不足道的“偷窃”等等,都将视同捣乱,因此当立刻以吊刑处决。几天前,一名饿得半死的俘虏闯入囤放马铃薯的储仓,偷了几磅的马铃薯;结果东窗事发,还被几个难友认了出来。营区当局获悉此事,便下令大伙儿把该俘虏交出来,否则全营要挨饿一天。不用说,全营二千五百名俘虏都宁愿绝食。

绝食当天的傍晚,我们躺在茅舍里,心情十分恶劣。每个人都闷不吭声,即使出声,也显得恶声恶气。更倒霉的是,后来连灯火都熄了。大伙儿的心情真是恶劣到极点。所幸,资深的舍监非常睿智。他针对大家当时的心境,临时来一段训勉。他提到近几天来因病或自杀而死的许多难友;并指出他们真正的死因是在于放弃了希望,然后,他要大家设法防止类似的惨例发生.并且指定我替大家“打打气”。

天晓得我当时有没有心情去说教打气。我又饿又冷又累,加上心情不佳,根本没兴致为难友提供任何精神治疗。然而,我又不能不把握这难得的机会。在当时,大伙儿最迫切需要的莫过于鼓励了。

因此,我先以最琐碎的舒适问题作为开场白。我说,即使在二次世界大战已届六年的欧洲本土,我们的处境仍然不算是想像中最悲惨的。我建议每个人问问自己:截至当时为止,有哪些损失是无可挽回的?据我推测,对大多数的俘虏而言,这种损失实际上几等于零。任何人只要活着,就有理由去怀抱希望。健康、家庭、幸福、专业技能、运气、社会地位等等,这一切都是可以重整旗鼓、东山再起的。毕竟,我们的一身硬骨,都还完好如初。过去不论经历了什么,都可以成为来日的一笔资产。说到这儿,我引用了尼采的一句话:“打不垮我的,将使我更形坚强。”(Was mich nicht umbringt , macht michstarker.)

随后,我谈到了未来。我说,平心而论,未来似乎是无何希望。每个人都可以料定自己的生还机会极其渺茫。营中虽尚未流行斑疹伤寒,我个人估量自己大约只有二十分之一的存活机遇。然而我又说,尽管如此,我仍不打算放弃,也不愿失去希望。毕竟,连下一个钟头会有什么变化都没有人知道,而况是未来? 我们虽不能预料这几天内能发生什么重大的军事变化,然而以我们在集中营的经验,谁又比我们更清楚大好的时机有时往往乍然降临——至少降临在某个人身上? 譬如,你我很可能意外地被分发到一个工作环境特佳的支队上,只因为集中营俘虏的“运气”,便是由这类事情凑合起来的。

我不只谈到未来及其阴影,更提到往昔和往昔的一切欢乐,也谈到过去的光辉如何照耀着此刻的昏暗。为了避免流于说教,我再度引用一位诗人的诗句:“尔之经历,无人能夺。”不只我们的经验,连我们做过的一切事、受过的一切痛苦,甚至脑海中有过的一切重大思考虽然已成过去,但全都未曾消失;只因为我们已把它孕育成形,使其存乎人间、曾出现过的也是一种存在,而且可能还是最明确的存在。

接着,我又谈到许多能使生命有其意义的机会。我告诉这些难友(他们全都静静地躺着,偶尔哀叹一、两声),人类的生命无论处在任何情况下,仍都有其意义。这种无限的人生意义,涵盖了痛苦和濒死、困顿和死亡。我请求这些在昏暗营舍中倾听着我的可怜人正视我们当前处境的严肃性,我要他们绝不能放弃希望,而该坚信目前的挣扎纵然徒劳,亦无损其意义与尊严,因而值得大家保住勇气、奋斗到底。我说,在艰难的时刻里,有人——一位朋友、妻子、一个存亡不知的亲人,或造物主——正俯视着我们每个人。他一定不愿意我们使他失望。他一定希望看到我们充满尊严——而非可怜兮兮地承受痛苦,并且懂得怎样面对死亡。

最后我谈到我们的牺牲,并说这牺牲无论如何都有其意义,在正常的环境或有所成就的情况里也许不然,但事实上的确有其价值;而这一点,有宗教信仰的人一定不难理解。我更举一个难友为例。此人在抵达集中营时,曾试着和上苍约定:他要以自己的痛苦和死亡,作为超渡他所深爱的人的代价。在他看来,死亡和痛苦乃深具意义;他的痛苦和死亡,是意味深长的牺牲。他不愿平白无故地死去,任何人都不愿这样子死去。

我这番话的用意,无非是想在那种黯淡无望的处境中,为我们当时当地的生命,寻出一个圆满的意义来。我看得出,我这番努力发挥了极大的效果。当电灯泡重又大放光明,我看到许多难友拖着憔悴的躯体蹒跚地走过来,噙着泪直向我道谢。然而此际,我却不能不承认我当时所拥有的内在力量实在太过薄弱;否则,我一定更能够和难友在患难中互相砥砺。而且我也相信,我一定错过了许多这样的机会。


\section{天使和恶魔}
现在,我们要谈的是俘虏心理反应的第三阶段:重获自由之后的心理。在此之前,我们且先考虑心理学家(尤其是曾在集中营里呆过的心理学家)经常被问到的一个问题:集中营警卫的心理结构究竟是怎么一回事? 同样是有血有肉的人,怎么可能像许多俘虏所说的那样,用尽各种残酷手段来凌虐别人?任何人一旦听到这样的指控,且一旦相信这种事的确发生,就不能不问,这一切由心理学的观点来看怎么可能发生的。如果要给这个问题一个简单明了的答复,就不能不先指出几个事实:

第一、警卫中有几名是虐待狂,而且是医学临床上纯粹的虐待狂。

第二、一旦有必要组成一支真正严格的警卫队,这些虐待狂一定入选。

在严霜刺骨的工地作了两个钟头的苦役后,我们如能获准在一个喂满树枝木屑的小炉前烤火暖身,那可真是天大的享受。然而,总有几个监工专以剥夺我们这份享受为乐。他们不只严禁我们靠近炉火,还把炉子踢翻,并把可爱的炭火塞到雪堆里弄熄,然后脸上便现出得意之色。挺进队只要看某个俘虏不顺眼,永远找得到一个以酷虐手段出名的虐待狂,来整这名倒霉的俘虏。

第三、大多数警卫由于多年来在集中营里目睹过太多太多的残酷手段,感觉早已经迟钝了。这些心肠硬化、道德感僵化的人虽然不愿主动采取残酷的手腕,却也不阻止别人施暴。

第四、众警卫当中也有几个人对我们甚为同情。这是必须声明的一点。即以我最后羁留的集中营为例,该营的司令官就是个好人。他为了替自己辖下的俘虏购置药品,常常自掏腰包,花了许多钱。而这件事,一直等到我们获释之后才传扬开来;在此之前,则只有营医(也是一名俘虏)知道。然而那个老资格的总舍监,虽然也是个俘虏,心肠却比任一名挺进队警卫还要狠。他老是为了细故,殴打别的俘虏;然而据我所知,这位司令官从没出手打过任何人。

因此,光知道一个人究竟是警卫或俘虏,显然并不能据以了解此人的性格。人类的同情心,在任何一群人当中皆可发现,即使是容易招致诟病的一伙亦然。群伙之间的界线原都是重叠的。我们不能以草率的二分法来断定谁是天使,谁是恶魔。当然,身为警卫或监工,却能善待俘虏,出污泥而不染,确是值得称道。反之,同为俘虏却虐待其他难友,则卑鄙可耻莫此为甚。众俘虏对于这种毫无品格的无赖,显然特别感到懊恼,但对于任何一位警卫所施予的小德小惠,则特别能刻骨铭心。我就记得有一天,一位监工偷偷塞给我一片面包,而我知道,那一定是他由自己的口粮中节省下来的。当时,那一小片面包所代表的心意,令我感动得流泪。那位监工所给予我的,除了一片面包之外,更有包含在他辞色之间的一股温情啊!

由此可知,世界上有——且只有——两种人:正人君子与卑鄙小人。两种人处处都有,散见于社会的各阶层。任一阶层任一团体的人,都不会是清一色君子或清一色小人。所以,即使是挺进队警卫,偶尔也会有一、两个正人君子。

集中营的生活揭露了人心深处的隐秘。如果我们从这些隐秘中再度窥知人性其实不过是善恶的混和会惊奇不置吗? 善恶的分界线,竟划过了天下众生,直抵人性的最深层;即连在集中营所揭露的深渊底层,亦如此清晰可辨,宁不令人慨叹吗?


\section{获释后的营俘心理}
现在,该正式来谈谈最后一个阶段的集中营心理——也就是俘虏获释后的心理了。要描述这种个人色彩必然特别浓厚的获释经验,我们自当循着故事部分的线索,回到营门挂起白旗的那天早上。

经过多日来高度的紧张,营区当局终于宣告投降;营中俘虏的心境随即由极度的紧张转为全然的松懈。不过,如果你以为我们一定都乐得发狂,你可就大错特错了。

我们拖着疲惫的身子,蹒跚走向营门,而后怯怯地四下张望,再狐疑地互相瞥视,最后则试探性地走出营门外几步。这回,没有人对我们吆喝,我们也不必再急忙闪避突如其来的一击或一踢。这回,警卫请我们抽烟了! 起先,我们几乎认不出他们来,因为他们早已经迅速换上便衣。大伙儿沿着营门口的大路慢慢走着,不久,每个人两腿都酸痛得像要断了般的。然而大伙儿还是一跛一跛地往前直走,想以初为自由人的眼光看一看集中营的周围环境。“自由”——我们脑海里反复思索这个字眼,然而却无法领略它的真义。过去这些年来,我们无时无刻不梦想着自由,梦到后来却忘了自由的含义。如今,我们已拥有自由,无奈这个事实,我们一时间还不能心领神会。真切的自由滋味尚未渗入我们的意识层中。

大伙儿走到缀满鲜花的草地,虽然看得到,且也知道草地就在眼前,无奈心里却空空茫茫,毫无感觉。后来,大伙儿看到一只尾部羽毛极其鲜艳的公鸡,第一丝喜悦的火花才绽现出来,然而绽现的时间十分短促。毕竟,我们还不属于这个自由而美丽的世界啊!

当晚,大伙儿重在营台里相聚,有人悄悄问另一个人:“告诉我,你今天快不快乐?”

此人似乎不知道大家的感觉全都跟他一样。他害臊地答道:“老实说,一点也不。”其实,我们早就丧失了感受快乐的能力,必须慢慢再学习才行\footnote{正如前谈及的感官迟钝是一种自我保护机制。}。


\section{人格解体}
获释俘虏的这一切反应,照心理学的说法,便是由于所谓的“人格解体”(depersonalization)使然。每样事物都显得不真实、不可能,恍如梦幻一般,令人不敢置信。过去这些年来,我们有多少次为梦境所骗啊!我们梦到获释的日子来到了,我们重获自由,并且重返故乡,会见朋友,拥抱妻子,还坐在餐桌旁边,畅谈营中的一切经历——还说我们常梦到获释的光景。然而不多时,耳边响起尖锐的哨音;起床的讯号,惊醒了我们的“自由大梦”。如今,梦境是实现了,然而我们真能相信吗?


\section{宣泄}
身体所受的压抑,远较心理的为少;因此,打从获释的第一刻开始,身体就懂得善加运用。大伙儿开始狼吞虎咽,一连吃喝了好几个小时和好几天甚至还吃到半夜。每个人的食量都大得惊人。有个俘虏就曾被附近的农家请去作客,当时,他吃了又吃,然后又喝了咖啡,打开了话匣子,从此便喋喋不休个没完。多年来积压在他心里的一切,终于得到了宣泄的良机。听他说话,他会有一种感觉:他“非讲不可”!他说话的欲望是无可抗拒的。大凡曾在重大压力之下度过一段时间——即使只是短短数天,譬如被盖世太保反复盘问——的人,都会有相同反应(这种人我就认识了几个)。一直要等到许多天以后,当事人不光是舌头喋够了,连心头的重荷也宣泄够了,才会突然意识到自己已冲破了重重羁绊,真正感到轻松起来。


\section{重获新生}
获释几天后的某一天,我穿过缀满鲜花的草地,在乡野间一连走了好几里路到营区附近的市镇去。我听得到云雀振翅高飞时欢欣的啁啾。几里方圆之内,一无人迹,只有广阔的大地、无垠的晴空,云雀的欢跃、以及自由的空间。我停下脚步,四下张望,再仰视穹苍,然后,我跪了下来。那一刻间,我对自己和世界,可以说一无所知;而我的心里,永远只回荡着同一句话:“我从窄小的牢狱里向天主呼号,而它在广袤的穹苍间答复了我。”

我究竟在那儿跪了多久,把那句话重复说了几遍,如今已不复记忆。然而我却知道,就在那一天的那一刻中,我的新生命展开了。我一步步地前进,直到我重又成为一个真正的人为止。


\section{精神失调的危机}
要由营中最后数日的剧烈精神紧张归于内心的平静,当然不是轻而易举。如果你以为获释的俘虏不再需要任何精神上的帮助,那可就大错特错了。一个人承受如此巨大的精神压力达如此长久的时间,获释之后(尤其释放得相当突然)自然会遭遇危险。这种危险,由心理卫生的观点来看,便是指心理平衡的问题。潜水员一旦突然离开海底,巨大的水压顿告消失,他的身体健康势必受到威胁。同样地,一个人一旦突然解除其精神压力,精神健康也一定会面临考验。

在这个心理阶段当中,禀性较粗朴原始的人,必然逃不过营中残酷暴行的影响。他们一旦获释,就自以为可以随便且毫不容情地使用自由。在他们眼里,改观了的只有一件事:他们已经摇身一变,成为压迫者\footnote{那些反抗压迫的革命者最后不是很多都成了另外一个残暴的压迫者了吗。},而不再是被压迫者。如今,他们是强权和不公的煽动者,而非饱受凌辱的落水狗。他们根据过去的恐怖经验,认定自己所行不偏;而这想法,常常在小事情上可以看出。譬如,我有次和一位友人要到集中营去,途中经过一片麦田。我很自然地绕道而行,但友人却抓住我的手,把我拖着走过麦田。我结结巴巴地说不该践踏农作物,不料他勃然大怒,瞪着我咆哮道:“不用说了,我们难道被剥夺得还不够多? 我的妻小统统死在煤气间里,而别的东西当然更不必提。现在,你居然还禁止我践踏几棵麦子!”

这种人唯有经过苦口婆心的劝导,才会慢慢领悟到一个极其平凡的事实:没有人有权做坏事,即便是受尽欺凌的人亦然。这种苦口婆心的劝导工作,必须有人肯予承担;否则后果势必远比损失几千棵小麦还严重得多。我仍然记得有个难友卷起袖子,指着我的鼻子大叫:“等我回家以后,这只手要是没沾满血腥,我一定把它剁掉!”在此我必须声明,说这话的人并不是个坏家伙。他在集中营里和出营以后,一直是我最好的伙伴。

精神压力骤然解除,固然易于导致道德的畸型,但另有两个基本经验,也很可能破坏获释俘虏的人品。那就是回复正常生活时很容易产生的愤恨和幻灭。

愤恨是故乡的一大堆令他不满的事情所引起的。譬如,当他返归故里,发现许多乡人一看到他,只是耸耸肩或打哈哈而已,不免兴起满腹的尖酸愤恨,因而不禁自问干嘛要承受过去那一切痛苦。当他到处都听到差不多雷同的反应:“我们并不知道啊!”“我们也吃了许多苦”,不禁要自问这些乡亲难道真的没别的话好对我说?

幻灭的经验就不同了。当其时,显得残酷的,不是乡人(他们的肤浅和淡漠,真叫人厌恶得恨不得找个洞钻进去,永远不再看到任何人);而是命运本身。一个多年来自以为已尝遍人世间最惨烈痛苦的人,却发觉痛苦永无止境;发觉自己可能还要再吃更多的苦,而且苦得更厉害。

我们曾谈到,要鼓舞营中俘虏活下去的勇气,就必须为他指出一个可堪期待的未来,必须提醒他人生仍大有可为、有待开创,并告诉他也许有个人等着和他团圆等等。然而获释以后呢?有的人发觉根本没有人等着他回去。他在集中营里为之梦魂萦牵,且为之振作精神的那一位早已芳踪杳然!他日夜思盼的这一天终于实现了,然而摆在眼前的这一切,却与他所渴望的大相径庭!这多么凄惨啊:他也许搭上电车,兴冲冲回到他多年来朝思暮想的家园。他按了门铃,一如他在千万次的梦里所渴望做的一样。结果却发现该来应门的那人已经不在,且永远不在了。

在集中营里,我们彼此间常说,人世间恐怕没有一种幸福足以弥补我们所受的一切痛苦。我们并不是希求幸福——使我们有勇气,使我们的痛苦、牺牲及死亡有其意义的并不是幸福本身。然而,我们也没有面对不幸的心理准备。正因为这样,为数甚众的俘虏在重返故里之时才受不了幻灭感的打击而消沉颓丧难再振作。精神医师也很难帮助他们克服这一层心理障碍,重新展望人生。尽管如此,身为精神医生的人仍不应就此束手,反而该把这个心理障碍看作是一项额外的刺激物。


\section{直如一场噩梦}
终有一天,每位获释的俘虏在回顾集中营的种种经历之时,将不复了解自己是如何熬过来的。当获释的日子终于来临,每样事物在他看来都像是一场美梦。同样地,终有一天,一切集中营的经验在他看来,也将不过如一场噩梦。

而重归故里的人最重要的一个体验,便是历尽沧桑之后所享有的一个美妙感觉:从今以后,除了上苍,什么都用不着畏惧了。

(第一部分 集中营历劫 到此结束)



\section{注释}
\showendnotes

\chapter{编者语}
我在想这样一个问题,中国人的国民性是如何形成的。下面我提出一个说法,中国的国民性在很大程度上是长期处于独裁专制的类监狱环境下自然而形成的性格或者文化,并没有其他高深不可测的神秘力量。下面我列出一个清单,清单中的各条都在本书各个地方列出和有所说明,大家看看是不是这样。
\begin{enumerate}
\item 适应忍耐能力很强 \pageref{适应}
\item 冷漠 \pageref{冷漠}
\item 爱好食物 \pageref{爱好食物}
\item 性压抑 \pageref{性压抑}
\item 甚少欢乐 \pageref{甚少快乐}
\item 避免与众不同 \pageref{不要与众不同}
\item 躁急易怒 \pageref{躁急易怒}
\item 自卑情结 \pageref{自卑情结}
\end{enumerate}

我列出这个清单的目的不是为了证明中国国民性多么的糟糕,更多的是试著列出来在监狱或者类监狱环境下将带来那些负面情绪。同时引起人们的反思,最好是引来有识之士的内心自我批判,原来我一直以来内心的那些性格上的倾向和喜好都不过是一种文化的熏陶,而这种文化说到底只是一种政治大环境下的产物。我无意要做出这样的结论,只是提出这种可能性,好让智者沉思。




\chapter{意义治疗法的基本概念}
阅读我的短篇自传式故事的读者们,时常请求我更充分及更直接地说明意义治疗的学说。因此我在本书第一版中,增加简短的篇幅来说明意义治疗学。但这不够.许多要求围困住我,希望能作更详尽的解释。因此在这一版中,我重新并较充分地说明。

但这件工作何其困难!要以十四大本德文书写的资料,用短短的篇幅通俗地介绍给读者,真是不太可能。我记得有一位美国医师曾到我在维也纳的诊所问我:“请问医师,您是心理分析学家吗?”我回答道:“不完全是心理分析学家,最好说是心理治疗家吧!”然而他继续问我:“那么您代表哪一种学派呢?”我答说:“是我自己创建的学说,称为'意义治疗学'(Logotherapy)”那么,您能用一句话告诉我什么是意义治疗学吗?“他又问”至少告诉我心理分析(Psychoanalysis)与意义治疗的不同点如何?“”好吧!“我说;”但是首先请你用一句话告诉我心理分析的本质是什么好吗?“下面就是他的答案:”在作心理分析时,病人要躺在睡椅上,向你诉说那些有时是非常令人讨厌去讲的事。“于是我立刻用下面即兴的话去反驳他说:”唉哟!作意义治疗时,病人可以笔直坐着,但他必须聆听有时是令人非常讨厌听的一些话。"

当然,上面这样说法有点滑稽,也不可作为意义治疗学的外在的说明。但是其中也有些道理。同心理分析比起来,意义治疗是较少回顾与较少内省的方法。意义治疗的焦点是放在将来,也就是说,焦点是放在病人将来要完成的工作与意义上。同时意义治疗尽量不强调所有“恶性循环的形成”(vicious circle formation)及“反馈机质”(feedback mechanisms)因为这两者恰恰足以助长“神经官能症”。这样一来,神经官能症患者典型的自我中心遂告瓦解,不再益形增强、恶化。

当然,以上的叙述是过于简略了,但在意义治疗中,病人终必遭遇到生命意义的问题而再次予以探索。上面我即兴而作的意义治疗法的定义事实上也包含一些真理。精神官能症病人企图逃避他的生命课题,不愿力求领会;若使他醒觉,清晰意识到自己的生命课题,则能激起他的潜力以克服神经官能症。

容我解释一下为何要用“意义治疗法”(Logotherapy)一词作为我的理论术语。“Logos”是希腊字,它表示“意义”(Meaning)。“意义治疗法”或如某些学者所称的“第三维也纳心理治疗学派”,其焦点放在“人存在的意义”以及“人对此存在意义的追寻”上。按意义治疗法的基础而言,这种追寻生命意义的企图是一个人最基本的动机。因此我所提出的“求意义的意志”(a will to meaning)与弗洛依德心理分析学派(Freudian Psychoanalysls)所强调的“快乐原则”(Pleasure principle),以及与阿德勒心理学派(Adlerian psychofogy)所强调的“求权力的意志”(the will to power)大不相同。


\section{求意义的意志}
人要寻求意义是其生命中原始的力量,而非因“本能驱策力”(instinctual drives)而造成的“续发性的合理化作用”。这个意义是唯一的、独特的,唯有人能够且必须予以实践;也唯有当它获致实践才能够满足人求意义的意志。有些学者认为所谓的意义与价值只是“心理自卫机制”(defense mechanisms),反向作用(reaction formatioa)以及“升华作用”(sublimatious)罢了!但为我而言,我不愿意只是因“心理自卫机制”而活,也不准备为了“反向作用”的缘故而死。但是,人,是能够为着他的理想与价值而生,也甚至能够为着他的理想与价值而死。

数年前法国曾作过一项民意测验,其结果显示89%的人承认需要为了某些因素而话。更甚者,61%的人承认他们肯为了生命中的某个人或事物去死。我在维也纳的诊所中用此问卷测验工作人员及病人,结果与法国的数千人样本测验雷同,只有2%差距而已。换言之,求意义的意志对大多数人是一“事实”,而非一“信条”。

当然,会有某些个别情况,其表示的价值观只是潜藏的内在冲突之伪装。如果如此,它们只代表法则的某些例外,而不能视之为法则本身。对于这些情况,精神动力学的解释可以揭发其潜意识底下的因素;而我们必须处理其虚伪的价值观(最好的例子是执迷不悟的顽固分子),这样便可揭开其面具,暴露其真相。不过,一旦面临到人真实的一面(也就是人渴望一生尽可能充满意义的事实),就该立刻停止揭穿面具的举动。如果不立刻停止,适足以显示揭发者有意贬低他人灵性上的渴望。


\section{存在的挫折}
一个人求意义的意志也能遭受挫折即意义治疗学所称谓的“存在的挫折”。“存在”(Existential)一词有三种用法:(1)表达“存有”本身,例如独特的人类存在型式。(2)表达存在的意义。(3)在个人的存在中努力去寻找具体的意义,这就是上面所说的“求意义的意志”。

存在的挫折也能导致神经官能症,但意义治疗学为这类型的神经官能症,刨造一新名词,称之为“心灵性神经官能症”,以区别于一般常用的“心因性神经官能症”(psychogenic neuroses)。“心灵性神经官能症”(Noogenic neufoses)非起源于心理因素,而是源自人类存在的心灵层次(希腊字“noos”意指心灵mind)。这新创的名词表示所有个人人格中有关灵性的部分。我们必须记住,在意义治疗学的架构中,“灵性”一词(spiritual)并非只是宗教上的含义,而是指人类生命中一特殊的层次。


\section{心灵性神经官能症}
心灵性精神官能症并非由于“驱策力”与“本能”之间的冲突所引起,而是由于不同的价值冲突所引起。换言之,是来自道德的冲突。或用更通俗一点的说法,是由于灵性的问题。在这样的问题中间,存在的挫折常扮演一个重要的角色。

因此,对于心灵性神经官能症患者,适当而正确的治疗,显然通常并不是心理治疗,而是意义治疗,此治疗法者胆敢进入人类生命中的灵性层次中去。事实上希腊字“Logos”不只表示“意义”也有“灵性”的意思。人类灵性的产物,诸如渴望存在的意义以及这种渴望的受挫,都必须用含有灵性意味的意义治疗法来施予治疗。治疗者热诚且认真地面对存在意义的问题,而不是去追溯潜意识的根源且处理本能的问题。

当一位医师无法区别到底是灵性的层次还是本能的层次时就会发生危险的混淆。容我引证下面的例子作为说明。

有一位高级美国外交官到我在维也纳的诊所,希望能继续接受精神分析治疗。他已经在纽约接受一位分析家治疗长达五年之久。一开头,我就问他为什么会认为自己需要分析?第一次分析是在何种情况下开始的?结果我弄清楚原来这位病人不满意他的事业,并且感觉要遵从美国的外交政策非常困难。然而过去那位精神分析家一再地告诉他应该跟父亲和解亲善,因为美国政府及其上司皆为他父亲的“心像”,因此他对工作的不满是由于潜意识里隐藏着对父亲的憎恨之故。分析持续了五年,病人愈来愈接受分析家的解释,直到最后他已经看不见现实的森林,而只见到幻影的树木。我与他会谈了几次之后,问题清晰呈现,他的职业使其求意义的意志受挫,他真正地渴望从事其它种类的工作。我认为根本没有理由不放弃这份事业,他如此做了,结果非常满意喜悦。最近他向我报告,虽然从事新工作已经过了五年多,他仍然深感满意。对于这个病例,我怀疑是否要像对待一位神经官能症患者那样去处理,我认为他不需要心理治疗,甚至也不需要意义治疗,理由很简单,因为事实上他根本不是病人!并非每个冲突都必然是病态的,有许多冲突可以是正常而且健康的。类似的概念,“痛苦”并非总是神经官能症者的病理症状,“痛苦”有时可以是人性的伟业,尤其是因存在的挫折所产生的痛苦。我要断然否认,一个人寻找他的存在意义,或怀疑其存在意义,皆是源自某种疾病。“存在的挫折”既非病理学亦非病源学的名词。一个人的忧虑或失望超过他生命的价值感时,只能说是一种“灵性的灾难”,而不能视之为一种心理疾病。如果一位医师将灵性的灾准视为心理疾病,就会将其病人的“存在性失望”用大量的安神剂埋葬掉了。医师的真正工作是引导病人通过存在的危机而获得成长与发展。

“意义治疗”的任务,在于协助病人找出他生命中的意义,亦即尽量使他随着分析的过程理会到存在中隐藏的意义。从这方面看来,意义治疗与精神分析有些相像。然而,意义治疗努力使人再意识到某些东西,因此它不光是注意人潜意识内的本能因素,还关心灵性的事实,诸如:人潜伏而尚待实现的存在意义,及其求意义的意志。但是,任何分析法,甚至是那些未涉及心灵或灵性层次的分析法,在其治疗过程当中,都会企图使病人理会到,他内心深处所渴望的到底是什么。意义治疗与精神分析最主要的差异是在于前者认为:作为一个人,最重要的关怀是实现意义与价值,而不仅仅为了满足驱策力及本能,或只是为平衡协调原我,自我、超我间的冲突;或只是为了去适应社会与环境而已。


\section{心灵动力学}
我应该肯定,人的寻求意义与价值可能会引起内在的紧张而非内在的平衡,然而这种紧张为心理健康是不可缺少的先决条件。我要大胆地说,这世界上并没有什么东西能帮助人在最坏的情况中还能活下去.除非人体认到他的生命有一意义。正如尼采充满智慧的名言:“参透'为何',才能迎接'任何'。”(He who has a “why” to live for can bear almost any “how”)我认为对任何心理治疗,这句话都可作为座右铭。在纳粹集中营内,我们可以亲眼看到,那些知道还有一件任务等待他去完成的人,最容易活下去。后来美国的精神医学家在日本与朝鲜的集中营内也证实了此点。

就拿我自己作例子,当我被抓进奥斯维辛集中营时,我正准备付梓的原稿被没收了。(这是我第一本书的笫一个版本,英译本在一九五五年由Alfred A.Knopf出版,书名为《医师与心灵:意义治疗法简介》——The Doctor and the Soul:An Introduction to Logotherapy)。当然啦,那时我最深的心愿是要再次写出这本书,而这竟帮助我活着度过集中营的严酷。例如当我患伤寒感觉很难受时,我用碎纸片记下许多摘要,以备重获自由时能重新著书。我确信这份动机协助我度过巴伐利亚集中营黑暗的牢房,而克服了令人崩溃的危机。

因此,我们可以看出心理健康是奠基于某种程度的紧张:人“已经达成”与“还应该完成”二者之间的紧张,或者是:人“是什么”与“应该成为什么”之间的紧张。这种紧张是人类生命中固有的属性,为心理健康是不可或缺的条件,因此,我们不必再迟疑去要求人实现自身潜在的意义了;也只有这样作,才能唤醒人潜伏状态中的求意义之意志。我认为我们如果以为人最主要的需求是“平衡”(生物学上称为Homeostasis),是没有紧张的状态,那将是心理卫生上的一种危险的错误观念。人真正最需要的并非不紧张,而是为了某一值得的目标而奋斗挣扎。他所需要的不是不惜任何代价地解除紧张,而是唤醒那等待他去实现的潜在意义。人所需要的不是生物学的平衡,而是我所称的“心灵动力学”——心灵动力在紧张的两极之中,一极代表需实现的“意义”,另一极代表必须实现此意义的“人”。我们不要以为这只有对正常情况中的人才是如此,其实对于精神官能症病人甚至更加确实。如果建筑师想要巩固一座老朽的拱门,便需增加覆盖在拱门上的负荷,如此拱门的各部分才能接合得更紧密坚固。所以,如果治疗者希望增长病人的心理健康,就不必害怕经由再次探索其生命的意义而增加他的负荷。

提出上述意义治疗的优点之后,继而我要说明它的缺点。今天有太多的病人抱怨,对自己的生命感到全部无意义,以及终极的无意义。他们无法体认到为了它就值得活下去的某种意义,他们被内在的空虚所萦扰纠缠,他们中有一种虚幻的寂寞。这种境遇我称它为“存在的空虚”。


\section{存在的空虚}
“存在的空虚”是二十世纪的一种普遍现象。这是可以理解的,因为人类要成为真正的“人”时,必须经历双重的失落,由此而产生存在的空虚。人类历史之初,“人”就丧失了一些基本的动物性本能,而这些本能却深深嵌入其他动物的行为中,而使它们的生命安全稳固。这种安全感就如同伊甸乐园一样,永远与人类绝缘,人必须自作抉择。除此之外,人类在新近的发展阶段中,又经历到另一种失落的痛苦,即一向作为他行为支柱的传统已迅速地削弱了。本能冲动不告诉他应该作什么;传统也不告诉他必须作什么,很快地他就不知道自己要作什么了,于是他愈来愈听从别人要他去作什么,于是他就愈来愈成为顺从主义的牺牲者了。

我在维也纳综合医院神经科的工作同仁对全院病人及护理人员作了一项全面的统计调查,结果显示55%的问卷呈现或多或少的“存在的空虚感”。换言之,一半以上的人感到过生命无意义。

“存在的空虚”所表现最主要的现象是无聊厌烦。现在我们可以领悟叔本华所说的,“人类注定永远在两极之间游移:不是灾难疾病,就是无聊厌烦。”事实上,现时代中所兴起的无聊厌烦感,比起灾难疾病要给精神科医师带来更多的问题。而且这类问题必定会日益增加,因为自动化机器不断进步,使一般人增多了闲暇的时间,但可怜的是其中有许多人根本不知道要用这些新获得的自由时间来作些什么?

举个例子,让我们仔细想想“星期天神经官能症”(Sunday Neurosis)这回事。当一周的匆忙结束,而内在的空虚浮现,使人理会到他对自己的生命不满意时,就会发生此类忧郁症了。很多自杀的案例都可以追溯到这种存在的空虚上面。现代如此广泛普遍的酒瘾(Alcoholism)及少年犯,除非我们能意识到问题底下的存在空虚,否则就无法理解为何有此种现象了。领养老金者及老年人的危机问题也是如此。

此外,还有许多种不同的面具及伪装隐藏着存在的空虚。有时求意义的意志受到挫折,于是用其他代替者作为朴偿,例如求权力的意志(包括最原始型态的权力意志)以及求金钱的意志。也有些时候.这种受挫的求意义意志被求享乐的意志所取代,因此成为性的代偿作用(compensation)。在这些案例身上我们可以看到,因为存在的空虚,性欲遂猖獗泛滥。

在神经官能症患者中发生一种类似的情况,就是某种型式的反馈机制与恶性循环作用。这我后面会再提到。我们可以再三再四地观察到,神经官能症之症状一旦侵入存在的空虚中,就会继续兴风作浪。但是如果在心理治疗中我们不增补意义治疗法,那就永远不能成功地使病人克服他的病况。要预防病人将来再复发,我们必须充填其存在的空虚。因此,意义治疗法不只适用于上面所述的心灵性病患,也适用于心因性病人,特别是对“假性神经官能症”病人更为适当。玛加达曾说过,“每一种治疗方法,无论它多么受到限制,在某方面都可以成为意义治疗法。”

现在让我们深思一下,如果有位病人问你:“到底我的生命的意义是什么呢?”此时我们要怎么办才好?


\section{生命的意义}
我怀疑一个医师是否能用概括性的措辞来回答这个问题。因为生命的意义因人而异,因日而异,甚至因时而异。因此,我们不是问生命的一般意义为何,而是问在一个人存在的某一时刻中的特殊的生命意义为何。用概括性的措辞来回答这问题,正如我们去问一位下棋圣手说:“大师,请告诉我在这世界上最好的一步棋如何下法?”根本没有所谓最好的一步棋,甚至也没有不错的一步棋,而要看弈局中某一特殊局势,以及对手的人格型态而定。人类的存在也是如此,一个人不能去寻找抽象的生命意义,每个人都有他自己的特殊天职或使命,而此使命是需要具体地去实现的。他的生命无法重复,也不可取代。所以每一个人都是独特的,也只有他具特殊的机遇去完成其独特的天赋使命。

生命中的每一种情境向人提出挑战,同时提出疑难要他去解决,因此生命意义的问题事实上应该颠倒过来。人不应该去问他的生命意义是什么。他必须要认清,“他”才是被询问的人。一言以蔽之,每一个人都被生命询问,而他只有用自己的生命才能回答此问题;只有以“负责”来答复生命。因此,意义治疗学认为“能够负责”(Responsibleness)是人类存在最重要的本质。


\section{存在的本质}
这种强调“能够负责”的特色,就显示在意义治疗法的一句金玉良言中:“假设你已经生活在第二度的生命中,并假设你第一次作错了,而现在还可能作错一样。”依我看来,再没有其它辞汇,可以比这句金玉良言更能激发起一个人的负责精神了。它叫人先假想现在已成过往,再假想过往可能无可改动、弥补。这种训示,使人意识到生命的有限,体悟到人由自身及生命中所创获的一切都具有决定的意义。

意义治疗法企图使病人深深体会到他自己的责任,因此必须让他自由抉择为了什么,对什么人或什么事负责。他了解是他自己要负责。这就是为什么意义治疗家在所有的心理治疗家中,是最少把价值判断塞给病人的,他决不允许病人把判断的责任交给医师。

因此,病人必须自行决定他究竟该对社会负责,抑或对良知负责。不过,也有不少人自认为该对上苍,对天主负责。他们不只以承担责任的角度,更以承行上天旨意的角度来诠释自己的生命。

意义治疗并非一种教训,也非传道。它不是逻辑的推理,亦不是伦理的劝诫。打个比喻来说。意义治疗家所扮演的角色与其说是一个画家,毋宁说是一名眼科医师。画家企图把他所看见的浮世图通传给我们,而眼科医师则是要我们自己去看见真实的世界。意义治疗者的角色在于放宽及开阔病人的视野,以使他能意识到整个的意义与价值体系。意义治疗不需要硬加给病人任何判断,因为事实上,真理会自行呈现,无需他人干涉或居间介入。

人是一种能够负责的物种,他必须实现他潜在的生命意义。我这样说,是希望强调:生命的真谛,必须在世界中找寻,而非在人身上或内在精神中找寻,因为它不是个封闭的体系。同样地,我们无法在所谓的“自我实现”上找到人类存在的真正目标;因为人类的存在,本质上是要“自我超越”(Self-transcendence)而非自我实现(Self-actualization)。事实上,自我实现也不可能作为存在的目标,理由很简单,因为一个人愈是拚命追求它,愈是得不到它。一个人为实践其生命意义而投注了多少心血,他就会有多少程度的自我实现。换言之,“自我实现”如果作为目的,是永不能获得的,它只是当“自我超越”之后的副产品而已。

人不能把世界看成光为了表现自己,也不该将它视为只是一种自我实现的工具或途径。这两种态度,都会使所谓的世界观(Weltanschauung)变成“世界评价”(Weltentwertung),因而瞧不起世界。

到此,我们已经指出了生命的意义是会改变的,但永远不失其为意义,按照意义治疗学,我们能以三种不同的途径去发现这意义:

(1)创造、工作。

(2)体认价值。

(3)受苦。

第一种,显然是功绩或成就之路。第二与第三种途径,则需要进一步的详细说明。

第二种途径是经由体验某种事物,如工作的本质或文化;或经由体验某个人,如爱情,来发现生命的意义。


\section{爱的意义}
爱是进入另一个人最深人格核心之内的唯一方法。没有一个人能完全了解另一个人的本质精髓,除非爱他。借着心灵的爱情,我们才能看到所爱者的真髓特性。更甚者,我们还能看出所爱者潜藏着什么,这些潜力是应该实现却还未实现的。而且由于爱情,还可以使所爱者真的去实现那些潜能。凭借使他理会到自己能够成为什么,应该成为什么,面使他原有的潜能发掘出来。

意义治疗学并没有将爱情解释作性冲动及性本能的升华的“次级现象”(由原始现象所产生之结果)。爱与性一样是属于原始现象。正常言之,性是表达爱的一种方式。性是无罪的,甚至是神圣的——当它作为传达爱的媒介时。如果只将爱情作为性的副作用,那么我们不会了解它便是两心永相契的体验,也是表达此体验的一种方式。

第三种发现生命意义的途径,是借助于受苦受难。


\section{苦难的意义}
当一个人遭遇到一种无可避免的、不能逃脱的情境,当他必须面对一个无法改变的命运——比如罹患了绝症或开刀也无效的癌症等等——他就等于得到一个最后机会,去实现最高的价值与最深的意义,即苦难的意义。这时,最重要的便是:他对苦难采取了什么态度?他用怎样的态度来承担他的痛苦?

我下面要引证一个清晰的例子:

有次一位年老的全科医师来看我,他患了严重的忧郁症。两年前,他最挚爱的妻子死了,此后,他就一直无法克服丧妻的沮丧,现在我能怎样帮助他呢?我又应该跟他说些什么呢?我避免直接告诉他任何话语,反而问他:“请问医师,如果您先离世,而尊夫人继续活着,那会是怎样的情境呢?”他说:“喔!对她来说这是可怕的!她会遭受多大的痛苦啊!”于是我回答他说:“您看,现在她免除了这痛苦,而那是因为您才使她免除的。现在您必须付代价,以继续活下去及哀悼来偿付您心爱的人免除痛苦的代价。”他不发一语但却紧紧握住我的手,然后平静地离开我的诊所。痛苦在发现意义的时候,就不成为痛苦了,例如具有意义的牺牲便是。

当然,认真说来,这根本就不算是一种治疗。因为第一,他的失望并非疾病;第二,我不能改变他的命运,我不能使他的妻子复活。但是在那瞬间,我成功地转变了他面对自己不可改变之命运的态度;或在那一刻,至少他了解了他的痛苦的意义。意义治疗学的基本信条之一即是:人主要的关心并不在于获得快乐或避免痛苦,而是要了解生命中的意义。这就是为什么人在某些情况下,宁愿受苦,只要他确定自己的苦难具有意义即可。

不用说,除非痛苦是绝对必须,否则它就没有意义。例如可用手术治疗的癌症,就不应该要病人像背十字架一样平白忍受。果真忍受下来了,那也只能算是一种“被虐狂”,不能算是英雄气概。不过,如果医师既不能治愈这种疾病,也无法减轻病人的痛苦,就应该激发他的潜能去实现痛苦的意义。传统的心理治疗目的在于恢复病人的能力,使他能工作及享受生命。意义治疗法也包括这些,但更进一步还要使病人再获得受苦的能力,因此需要去发掘痛苦中的意义。

关于这一点,美国普渡大学心理学教授Edith Weiss Kopf-Joelson,在其有关意义冶疗的文章上说:“心理卫生哲学日趋强调人应该快乐;而不快乐是适应困难的一种症状。这样的价值现应该对我们四周许多由不快乐所引起的不幸负责。”在她另一篇论文中,表达了希望意义治疗学可以“抵制当前美国文化不健康的趋势。当前的趋势是以痛苦为耻而非为荣,因此使得一个人不但不快乐,还要因不快乐而羞耻。”

在有些情况下人可能丧失工作的机会或生话的乐趣,但人永不能排除痛苦的不可避免性。如果勇敢地接受苦难的挑战,生命至最后一刻都仍具意义。换言之,生命的意义是绝对的,它甚至包括潜伏的痛苦的意义。

容我回忆一下在集中营里一次可能是最深刻的经验。众难友之中,历劫而犹能生还的人,不超过二十分之一,这是很容易以正确的统计来证实的。抵达奥斯维辛集中营之后,我暗藏在衣袋中的第一本书原稿,是不可能不被搜走的,因此我必须经历及克服丧失我灵魂之子的悲痛。当时,似乎已没有什么东西能使我继续活下去了,我既失去了身体的儿子,又失去了灵魂的产儿。我发现自己正面对着一个疑问:我是否在这样的情势之中,生命是终极的虚无而无任何意义?

但我发现我如此激动奋力要找寻的答案,已经贮藏在我内,并且不久后就显露了出来。事情是这样的,我必须交出自己的衣服,而继承另一位犯人的破衣服(那犯人在到达奥斯维辛火车站后就立刻被送进煤气间了)。我失去了我的著作原稿,却在别个犯人的破衣服口袋中发现犹太祈祷书中撕下来的一页纸,其中还包括犹太人最主要的祷文Shema Yisrael这样的巧合。我除了把它当成一种挑战,一种要我活出自己的思想,而不光是纸上谈兵的启示之外,又能作何解释呢?

后来几天,我记得自己感到快要死了。但在这种危机时候,我内心的问题与大多数难友不同。他们的疑问是:“我们能在集中营内活下去吗?如果不能,所有的痛苦便没有意义。”但困扰我的疑惑却是:“所有生命中的痛苦,我们四周的死亡,有意义吗?如果没有,那么人的生命终究毫无意义。如果生命的意义只依赖一些偶发事件——可以脱逃或不能脱逃的偶发事件——那么人生终究不值得一活。”


\section{形而上的临床问题}
(Meta-Clinical Problems)

一位医师愈来愈会面临这个问题:生命是什么?痛苦到底是什么?今天,精神医学家不断遭遇到的人类问题比精神官能症的症状要多得多。昔日人们去见神父,牧师或大师(佛教或犹太宗教中的大师),当时所问的问题,有些人现在转而去问精神科医师了。因此医师现今要面对哲学上的问题而非仅仅情绪的冲突罢了。

\section{演剧意义治疗}
我很喜欢举下面的例子:有个十一岁的小男孩死了,他的母亲因自杀未遂而住进我的诊所。同事柯葛里医师(Dr.Kocourek)遂请她加入一个治疗性团体。当柯医师正指导一出演剧心理治疗时,我恰好进入,听到她投诉她的故事。小儿子去世后,她就与患过小儿麻痹而需坐轮椅的大儿子住在一起。这位母亲要反抗她的命运.她想与大儿子一起自杀,却是这个可怜的有残疾儿子阻止了她。他要活下去,因为对他来说,生命仍具有意义。但为什么对她的母亲就不是如此呢?她的生命要如何才有意义?并且我们要怎样去帮助她体会到这个意义?

我即席加进了他们的讨论,我询问团体中另一位妇人芳龄多少,她答:“三十岁。”我说:“不!你不是三十岁而是八十岁,并且正躺在临终的床上。现在你回顾一下你的生活,生活中没有儿女却充满财富及社会名望。”然后我教她想像在这样的情境中她感觉到什么。“你想些什么呢?你跟自己怎么说?”下面就是引用她录音下来的一段实际谈话:“喔!我嫁了一位大富翁,我过着充满财富的舒适生活,我与男人调情并且戏弄他们。但是现在,我已经八十岁了,我没有自己的孩子,像个老女人那样回忆,我看不出一切到底是为了什么?事实上,我不得不说,我的生活是一场失败!”

然后我邀请那位残疾儿的母亲作同样的想像,去回忆她的生活。让我们听听她录音下来的谈话:“我希望有孩子,而这个希望实现了。一个儿子死了,还有一个残疾儿子可能要送进孤儿院——如果我自己不照顾的话。虽然他又有残疾又无助,毕竟是我的亲生儿子,因此我要尽量使他的生活更加圆满,我要使他成为一个好人。”刹时,她突然大哭,流着泪继续说:“至于我自己,我能够宁静平安地回顾我的生活,目为我可以说我的生命充满意义,我已经努力去实现它,我已经尽了最大的努力——为了我的儿子尽了最大的努力。我的生命没有失败!”她因为假想自己在临终前回顾一生,而突然能够看到其中的意义,这意义甚至包含了她所有的痛苦。用这样的方法,我们也可发现生命是何其短暂,像她那死去的儿子。但是那么短的生命却可以充满爱与幸福,其包含的意义或比八十岁还要多。

隔了一会儿我提出另一个问题,这次是向整个团体询问。问题是有一种猿类很容易发生脊髓灰白质炎,因此必须时常给它打针,那么它是否可以了解其所受痛苦的意义?全体一致回答它当然不能了解,因为它有限的智能是无法进入人类世界的,而人类是唯一能了解痛苦的意义。然后我再把问题往前推:“那么人类又怎样呢?你能确知人类世界就是宇宙演化的终点?是否超越人类世界之上还可能有另一层次?在那里人生痛苦的终极意义就可以找到答案?”


\section{超越的意义}
这个终极的意义,必超越并凌驾于人类有限的智能之上!在意义治疗学中,我们就称之为“超越的意义”。人所要求的并非如同某些存在主义哲学家所言,是去忍受生命的无意义;而是要忍受自身无能力以理性抓住生命的绝对意义。“意义”比“逻辑”更加幽深。

一个没有“超越的意义”概念的神经科医师,迟早会被他病人的问题所困扰,就如同我被我六岁小女儿的问题所困倒一样。她问我:“爸爸,为什么我们要称天主是好天主呢?”我说:“几星期以前你生麻疹,是好天主赏赐你完全康复了呀!”但是这小女孩却不满意而反驳道:“嘿!爸爸不要忘了,首先也是他让我出麻疹的。”

然而,一个病人如果具有坚定的宗教信仰,我们就没有理由不借重他的宗教信念及精神力源来发挥医疗上的效果。为了这样作,神经科医师可以站在病人的立场上为他设想。我个人就有过一次这样的经验:有位近东国家的犹太经师来看我,并告诉我他的故事。在奥斯维辛的集中营里,他的妻子与六个孩子都被煤气杀死了;而现在,他的继室竟然不孕。我认为生育并不是生命的唯一意义,果真如此,生命本身就变得毫无意义了。而本身没有意义的东西并不能因为其永久存在而变为有意义。这位经师的绝望是因为在他死后没有儿子可为他诵经祈祷,而这对正统犹太人是很重要的。

但我仍不放弃,便作最后的尝试来帮助他。我问他是否不希望在天堂上再见他的孩子?不料,我的问题引得他放声大哭,而他绝望的真正理由也跟着浮现出来。他解释说,因为他的孩子死得像殉道的圣人,因此在天堂上地位最高,但是像他这样的有罪老人,却不能期望在天堂上有相同地位。我一听,立刻接道:“这就是了,先生,这就是你比你孩子长寿的意义。这样,你才可以用这些年来的苦难来净化自己。直到最后,你即使不能像孩子们那般圣洁,也会有资格在天堂与他们在一起。圣咏上不是写说天主不忘你的眼泪吗?因此,你的一切痛苦都应该不是白费的。”多少年来,第一次.他终于能从痛苦中解脱出来,因为我给他开辟了另一条路,而让他从新的观点来看自已的生命。


\section{生命的短暂性}
使得人生看起来没有意义的事,除痛苦之外还有濒死;除疾病之外,还有死亡。但是我愿意强调,生命中唯一真正短暂无常的部分是它的潜在力,这些潜力一旦成为事实,立刻就变成过去。然而,凡存在过的,会永恒地存在,因此它们就从短暂性中被解救及被保存起来。

如此说来,我们存在的短暂性决不会是没有意义的,反而构成了我们的责任感。因为每样事物的关键点就在于我们知道它是短暂的,所以人必须不断地抉择,哪些要做,哪些不要做,何种抉择可成为一种不朽的生命痕迹?在任何时刻,人都要决定(不管是因而变得更好或是更坏)什么样的事物,将成为他存在的里程碑。

人通常只注意到“短暂性”所余下的残株败梗,却忽略了过往所带来的丰盈谷仓(于其间,他收藏了那曾属于他且永远属于他的言行、喜乐及痛苦)。事实上,没有一样东西可以被毁灭,也没有一样东西可以被废除。存在过了就是一种最确实的存在。

意义治疗学牢记着人类存在的短暂性本质,它不仅不悲观,反而非常积极。若用比喻来表明此一观点,我们可以说:悲观主义者就好像一个人,既悲且惧地发现他每天撕去一页的日历愈来愈薄。积极解决人生问题的人,则像似一个人撕下日历的一页后,在背面摘记一些日志,然后按序归档。他能够骄傲及喜悦地从那些摘记中回忆生活的丰盈,品尝所有他已经充分活过的岁月。即使他注意到自己正逐渐老迈,那又有什么要紧呢? 难道他要去嫉妒所看见的年轻人?或者伤感地怀念自己失去的青春?他何必要去羡慕年轻人?就因为年轻人有许多“可能性”,并且有“将来”吗?“不!”他会想:“为我来说,这已经不是'可能性',而是'事实'了。我不只拥有已经作过的工作,还有爱过的人,受过的痛苦。这些都使我引以为骄傲,因此我才不会去羡慕年轻人呢!”


\section{意义治疗是一种技术}
符合现实的畏惧(比如对死亡的畏惧),不能以精神动力学的解释予以免除;反过来说,神经官能性的害怕,如惧旷症,也不能用哲学的了解来治愈。但是意义治疗法却发展出一种特殊的技术来处理这些病例。为了解使用这种技术的出发点为何,我们选择了一种在神经官能症者身上常常会碰到的情况,所谓“预期的焦虑”。其特征为如果病人预期会害怕什么东西,到时就真的害怕了。例如一个人进入一间大厅面对许多人时,他害怕自己会脸红,结果真的脸红了。因此我们可把“愿望乃思想之父”(The wish is father to the thought)这句格言改成“害怕为事故之母”(The fear is mother of the event)。

但是,如果一个人强烈地意图什么东西,反而会使愿望落空。这种过分的意愿在性的神经官能症患者中最易看到。如果一个男人愈是想要表现其性能力,或一个女人要表现性感,则他们愈是不能成功。快乐是,而且必须是一种副作用或附带产品,如果将它视为目标则会消灭或破坏了快乐。

除了上述“过分意愿”之外,意义治疗学上还有一种“过分注意”或“过分反射”(Hyper-reflection)的情况,也会导致病症。下面的临床报告可以说明我的意思:有一位年轻的妇人来看我,主诉她性冷淡。其病历显出她小时候曾被父亲强暴,但是并非由于这创伤经验造成她性的神经官能症,这很容易证明。从很多流行的精神分析文献中可发现,这个病人所害怕是其创伤经验有一天会重演。这种预期性的焦虑造成“过分意愿”想确证自己的女性化,以及过分注意自己,因而忽略了她的伴侣。此二者足够使病人无法达到性高潮,性的快感成为意愿的目标以及注意的焦点,因而使她无法注意到伴侣并且无法反应了。在作了极短期的意义治疗之后,病人的过分注意自己及过分意图获得快感都减轻了,并进入另一个意义治疗的阶段。当她的注意焦点转向应当的目标——她的伴侣一时,性快感自然地就获得了。(附记,为了治疗性无能的病例,意义治疗发展出一种特殊的技术,是基于“过分意愿”与“过分反射”的理论。当然,我们在这简短的解释中是无法详细说明其处理原则的。)

基于上述两顶事实,即人所预期的害怕会变成真的,而人过分想要得到的却反而得不到,,意义治疗就发展出一种称作“矛盾取向法”的技术(Paradoxical intention)。此法是使畏惧症的病人故意去要他所害怕的东西,甚至只一刹那时间也好。

我想起一个病例,有一位年轻医师因畏惧出汗而来看我。无论什么时候,当他预料会突发性地出汗,这种预期的焦虑就果真使他大汗淋漓。为了阻断这种循环作用,我劝他当下次出汗时,他要下决心从容地给别人显示到底他能流多少汗。一星期后他回来报告结果,当他在别人面前发生预期的焦虑,他就对自己说:“我只冒了一夸脱的汗,但是现在我打算至少要冒十夸脱汗才行。”于是困扰他四年的畏惧症,只经过一次会谈,就在一周内永远地解除了。

读者自然已经注意到这种方法是使病人的态度颠倒,如果他害怕什幺,就以矛盾的希望来代替。凭借这种治疗,使风带走了焦虑之帆。

不过,这种方法必须利用人类所特有且附属于幽默感的自我超越能力(Self一detachment)。任何时候只要一使用意义治疗的“矛盾取向法”,这种能力便会展现出来。而同时,病人也能够使自己与他的神经官能症保持一段距离。关于这点,欧伯氏在他的著作《个人与其宗教》一书中说得很好(Gordon W.Allport's book “The Individual and His Religion” New York,The MacMillan Co.1956,P.92):“如果一个神经官能症患者已经能嘲笑他自己,那么他就已经在作自我治疗了,有时还可能痊愈。”欧伯氏在临床上使用矛盾取向法的经验证明相当有效。

再多几位案例报告或许可以进一步阐明这种方法。下述的病人是一位簿记员,他进过好几家诊所,接受过许多医师的治疗都无效。当他来看我时已是极端失望,很想自杀,好多年来他罹患手抽痉的毛病,而最近更显严重,还有失去工作的危险。因此只有立即的短期疗法才能缓和这种情势。在治疗之初,我的同事建议病人做与他平时所行相反的事,也就是说,他平常极力要写字写得既整洁又清晰,而现在,却极力地潦草。他劝他自己说:“现在我要给剐人看看我是多么乱写的潦草专家!”然而就在他从容地企图乱写之时,他竟无法这样作,第二天又来时他说:“我尝试潦草写字,可是不能够。”四十八小对之内,这个病人就解除了手抽痉的毛病,并且在尔后的一段观察时期仍然很好,现在他再度是个快乐的人,并且工作顺利。

另一个类似的病例是语言的而非书写的障碍。是维也纳综合医院咽喉科我的一位同事转诊给我的病人,这个病人是他行医多年来从未遇到过这么严重的口吃病人。就病人记忆所及,在他一生中从未免除口吃的困扰,只有一次例外。那是在他十二岁时有次偷坐电车被车掌抓到,他想唯一的逃脱方法是表现他的口吃,那么也许车掌会同情可怜这口吃小男孩而放了他。可是,谁知他企图口吃说话时,竟不能口吃了。因此彼时虽然不是为了治疗的目的,他正用了矛盾取向法。

然而上述的说明不该给读者一个印象,以为矛盾取向法只是对单一症状的病人才有效。我在维也纳综合医院的同事们,以这种意义冶疗法的技术,相当成功地缓解了最严重的强迫性神经官能症患者之症状。例如有一位六十五岁的妇人,六年之久遭受严重的强迫洗手症之苦,其严重度使我以为只有用“脑叶白质切除术”才能缓解症状。但我的同事却尝试使用矛盾取向法,两个月以后,这病人竟能过相当正常的生活了。在住院以前,她曾说:“生活对我而言如同地狱一样。”由于“惧细菌症”的强迫思想及强迫行为,使她只能整天躺在床上而不能做任何家务事。当然,如果要说她现在已经完全康复是不正确的,因为“强迫思想”仍紫绕她的心头,但是应用矛盾取向法之后,她已经可以嘲笑自己的强迫思想了。

矛盾取向法也可以应用在睡眠障碍的病例。害怕失眠造成一种“过分意愿”希望入睡,于是反而睡不着了。(附记。对大多数害怕失眠的病人来说,是因为他们对于生物体真正最低睡眠量的无知所致。)为了克服这种惧怕,我通常劝病人不要努力想入睡,反而努力保持清醒状态。换言之,想入睡的“过分意愿”引起了预期的焦虑,使他更无法入睡,而应用矛盾取向法反其道而行时,就很快可以入睡了。

上述说明了矛盾取向法对治疗强迫症、畏惧症是一种有用的工具,尤其是那些潜伏性预期的焦虑之病患。再者它又是一种短期治疗法。然而我们不要以为像这样的短期治疗只会保持暂时的效果。艾弥尔说:正统弗洛依德学派最常见的错觉之一即认为治疗期间的长短与其效果成正比。(Emil A.Gutheil,American Journal of Psychotherapy l0:134,1956)。在我的档案中有份病历报告,他二十多年前用过矛盾取向法,至今仍能证明其治疗效果。

矛盾取向法最值得注意的一点是不管病人的病因为何。Edith Weisskopf Joelson曾说:“虽然传统的心理治疗认为必须要找出病因,但是很可能童年期有一些引起神经官能症的原因,到成年期却另以纯然不同的因素使神经官能症获得缓解。”

心理情结、冲突及创伤,常被认为是神经官能的原因。其实,有时候更好说它们是神经官能症的症状而非病因。当潮汐低落,暗礁就显现了出来,但暗礁并非是低潮的原因。如果没有某种情绪的低潮,忧郁症又是什么呢!同样的,内因性忧郁症患者(endogenous depressions,千万不可与神经官能性忧郁症相混),很典型的症状之一是“罪恶感”。其罪恶感并非造成忧郁症的原因,而是因情绪低潮,使其浮现出意识的层面来了。

至于神经官能症的真正原因,姑先不论其为生理性或心理性的精抻官能症,可能像“预期的焦虑”这类“反馈机质”是重要的病理因素。畏惧症有其症状,此症状又加强病人的畏惧。类似的链状情况也可在强迫症患者身上看到。强迫症病人与萦扰他的思想奋战,然却反而加强其作祟的力量,因为压力引起反弹作用。但是如果病人停止与他的强迫思想作战,而用讽刺的方法嘲笑它们一番;于是就在应用这种矛盾取向法之时,其恶性循环作用被切断了。症状获得减轻最后终至萎缩。那些并非由于“存在的空虚”而引发症状的幸运病例,不只可成功地减轻他们病态的畏惧,甚至到最后可完全痊愈。

我再重复一次:预期的焦虑要以矛盾取向法来克服;过分的意愿与过度的反应要以“减反应作用”来克服。然而最后,这些方法都只是方法;除非我们引导病人再度发现其生命的意义及天赋使命,否则仍不能彻底协助他。

神经官能症患者的自我关心,无论是自怜或自我轻视,都将使情况更坏,而其治愈的生机却是“自我超越”(self-transcendence)。


\section{集体性神经官能症}
每一个时代都有其集体性的神经官能症,同时每一个时代都需要它自己的心理治疗法以应付之。现时代的集体神经官能症可以说是“存在的空虚”。“存在的空虚”是一种个人性的“虚无主义”(nihilism),而虚无主义可界说为“生命没有意义”。如果现时代的心理治疗学无法脱离虚无主义哲学的撞击与影响,那么充其量它只代表了集体性的症状,而没有治疗的效果。这种脱离不了虚无主义的心理治疗,通传给病人的并非真正“人”的图像,而只是一张漫画而已,虽然它不是故意这样做。

首先,心理治疗学有一个危险,将人看成“不过尔尔”(nothingbutness),人只是生物学的、心理学的及社会学的状态,或只是遗传与环境的产物。用这样的观点来看,人就不成其为人,而成为机器人了。并且因为否认了人的自由,反而更助长了神经官能性的宿命论主义。

当然,人是“有限”的,因此他的自由也受到限制。但是人并非具有脱离情境的自由,而是面对各种情境时,他有采取立场的自由。举个例子,我当然对自己的灰头发没有责任,但是我没有去理发店染发(如同许多女士所作的)却由我自己负责。因此每一个人都有大量的自由,即使是像选择发色这样的小事亦然。


\section{泛决定论的批判}
精神分析时常为人所诟病的即其所谓的“泛性主义”(pan-sexualism)。但我怀疑这样的谴责是否正当。因为我认为其更错误与更危险的是所谓的“泛决定论”,“泛决定论”忽略了人面对任何情境时有采取立场的能力。人并非完全被制约及被决定的,而是他自己要决定向情境屈服还是与之对抗?换言之,人最后是自我决定的。人不仅仅是活着而已,他总是要决定他的存在到底应成为什么?下一刻他到底要变成什么?

同样地,每一个人在任何时刻都有改变的自由,因此只有在有关整个群体的统计学研究之庞大架构中,我们才能预测一个人的将来,至于个体的人格(personality),仍然是不可预测的。任何预测的基础,皆是用生物学、心理学或社会学的条件来表示。然而人存在的主要特征之一,却在于他具有超越上述条件的能力。而且,人终究要以同样方式来超越他自己。人之所以为人是因为他可以自我超越。

让我引用齐博士的例子作为说明:我敢说齐博士是我一生中所遇见过唯一像魔鬼般的人物。那时人们称他为“斯坦霍夫的大屠杀者”,斯坦霍夫(Steinhof)是维也纳一座大型精神病院的名字。当纳粹开始其“安乐死计划”时,他手中握有所有的线索,并且对当局委派绐他的任务极为热衷,因此他尽量不让任何一个精神病患者逃过煤气间。战争结束后我返回维也纳,也就是说我逃脱了奥斯维辛集中营的煤气间之后,我问及齐博士的遭遇如何?他们告诉我:“他被俄国人囚禁在斯坦霍夫精神病院的一间密室里,但是第二天发现门被打开了,从此再也没见到齐博士。”以后我与其他人一样,相信他由同伴的帮助逃到南美洲去了。然而一直到最近,有个过去曾任奥地利外交官的人来我这里看病,他在铁幕中被囚禁了许多年——首先在西伯利亚,后来在著名的莫斯科卢布拉卡监狱。当我正给他作神经科的检查时,他突然问我是否认识齐博士?在我肯定回答后他继续说:“我在卢布拉卡认识齐博士,他死在那里,大约四十岁,是因为膀胱癌病死的。但是,你很难想像他死以前竟会是十足的一个好难友、好同伴。他去安慰每一个人,他活出最高的道德标准。他是我在监狱的漫长岁月中所能遇到最好的朋友了。”

这就是齐博士的故事:“斯坦霍夫的大屠杀者”。你怎么敢预测一个人的行为呢? 你可以预测一部机器的运转或预测机器人的行动!更甚者,你甚且也可以预测一个人精神的机制或动力(mechanisms or dynamisms of the human psyche),但是,人比精神(psyche)还要复杂多了。


\section{精神医学的信条}
我简直无法想像会有一个人是完全被制约住而没有丝毫的自由存留。因此无论是神经官能症甚或精神病患者,都会有残余的自由,不管所存留的自由有多少。事实上,即使是精神病,也不能触碰到一个人最深的人格核心。我记得有位大约六十岁的病人被带来看我,他罹患听幻觉已有数十年之久。 当时,我面对着一个衰败的人格(personality)。不出所料,他四周的人都认为他是白痴,但我却发现他身上辐射出一种多么奇异的魅力!他从小就想要作神父,然而他却只能享受到星期天早上在教堂唱诗班中唱圣歌的快乐。陪同他一起来看我的姊姊诉说,他有时会非常激动,但最后总是能恢复自我控制。我对这个病人非常的精神动力很感兴趣(underlying psychodynamics),我以为他对他姊姊的关系有很强的“固定现象”;因此我问他:“为什么你能够恢复自我控制?是为了谁的缘故?”停了几秒钟后,病人回答:“为了天主的缘故。”在这时刻,他最深处的人格显露了,而在这样的深底,不论其智能的秉赋是多么贫乏可怜,却显现了美丽的宗教生活。

一个无法治疗的精神病患者可能失去了他的“有用性”,但是仍然保有作为一个“人”的尊严。这就是我的“精神医学信条”。如果没有此信条,我就认为实在不值得去作一个精神科医师。为了谁的缘故?只是为了损坏而不能修理的“脑机器”吗?如果病人再也没有人性的尊严了,那么“安乐死”就可以使用了。


\section{再赋予人性的精神医学} 
良久以来(事实上已有半世纪之久了)。精神医学试图以机械作用的观点来解析人类心灵,因而对心理疾病的冶疗,也光是从技术上着眼。我相信这场梦已经过去了,现在地平线上隐约可见的已是“人性化的精神医学”,不再是“心理学化的医术”了。

一位医师如果仍然认为自己的角色主要是一个“技术员”(technician),那么他应该坦白承认他所看到的病人只不过是一部机器,而无法看到疾病后面的“人”。

一个“人”并非许多事物中的一件事物,“事物”是互相牵连决定的,而“人”最终是自我决定的。他要成为什么——在天赋资质与环境的限制之下——他就成为什么。举例来说,在集中营这个生活实验室与考验场中,我们发现并且见证有些难友的行径像个恶棍,有些却宛如圣人。人在他自身内有两种可能性,去实现哪一种是由他自己所抉择,而非视情境所定。

我们这一代处于现实主义的时代,因此我们要知道“人”真正是什么。毕竟,“人”是发明奥斯维辛煤气间的“存在”;但同时,“人”也是笔直走进煤气间,口中念着天主经或犹太祈祷文的“存在”!

【全书完】


\section{注释}
\showendnotes



\end{common-format}  
\end{document}



