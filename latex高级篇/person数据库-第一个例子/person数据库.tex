% !Mode:: "TeX:UTF-8"%確保文檔utf-8編碼
%新加入的命令如下:addchtoc addsectoc reduline showendnotes hlabel
%新加入的环境如下:common-format  fig scalefig xverbatim

\documentclass[11pt,oneside]{book}
\newlength{\textpt}
\setlength{\textpt}{11pt}
\newif\ifphone
\phonefalse


\usepackage{myconfig}
\usepackage{mytitle}
\usepackage{python}



\begin{document}
\frontmatter

\titlea{书籍}
\titleb{使用\LaTeX 排版}
\titlec{一种良好的风格}
\author{作者}
\authorinfo{作者:}
\editor{编者}
\email{a358003542@gmail.com}
\editorinfo{编者:}
\version{1.0}
\titleLC

\addchtoc{前言}
\chapter*{前言}
\begin{common-format}
这里说明你写这个python项目的基本思路和想法。


\begin{python}
#!/usr/bin/env python3
# -*- coding: utf-8 -*-

##################################
#序言部分
\end{python}

%这里空一行。

\end{common-format}


\addchtoc{目录}
\setcounter{tocdepth}{2}
\tableofcontents

\begin{common-format}
\mainmatter

\chapter{类}
\section{Person}
\begin{python}
class Person:
    def __init__(self, name, job=None, pay=0):
        self.name = name
        self.job = job
        self.pay = pay
\end{python}

类名一般都大写。

特殊的\verb+__init__+ 用于这个类具体创建instance实例的时候执行的动作。self表示创建的那个实例,self.name表示实例的名字,self.name =name表示接受的name将会传递值给self.name,同时创建的那个实例将会拥有一个自己的name属性。其他属性操作类似。

和一般函数的做法一样,job=None,表示job这个参数是一个可选参数,它有一个默认值None。



\chapter{输出结果}
%\printpython
\debugpython

%这里空一行

\end{common-format}
\end{document}



