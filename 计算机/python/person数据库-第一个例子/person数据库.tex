% !Mode:: "TeX:UTF-8"%確保文檔utf-8編碼
%新加入的命令如下:addchtoc addsectoc reduline showendnotes hlabel
%新加入的环境如下:common-format  fig scalefig xverbatim

\documentclass[11pt,oneside]{book}
\newlength{\textpt}
\setlength{\textpt}{11pt}
\newif\ifphone
\phonefalse


\usepackage{myconfig}
\usepackage{mytitle}
\usepackage{python}



\begin{document}
\frontmatter

\titlea{书籍}
\titleb{使用\LaTeX 排版}
\titlec{一种良好的风格}
\author{作者}
\authorinfo{作者:}
\editor{编者}
\email{a358003542@gmail.com}
\editorinfo{编者:}
\version{1.0}
\titleLC

\addchtoc{前言}
\chapter*{前言}
\begin{common-format}
这里说明你写这个python项目的基本思路和想法。




%这里空一行。

\end{common-format}


\addchtoc{目录}
\setcounter{tocdepth}{2}
\tableofcontents

\begin{common-format}
\mainmatter

\chapter{classtools.py}
\begin{python}
#!/usr/bin/env python3
# -*- coding: utf-8 -*-

##########################
\end{python}

\section{AttrDisplay}
\begin{python}
class AttrDisplay:
    def gatherattrs(self):
        attrs = []
        for key in sorted(self.__dict__):
            attrs.append('%s=%s' % (key,getattr(self,key)))
        return ', '.join(attrs)
             
    def __repr__(self):
        return '[%s: %s]' % (self.__class__.__name__,
        self.gatherattrs())
\end{python}


\section{测试代码}
\begin{python}
if __name__ == '__main__':
    class TopTest(AttrDisplay):
        count=0
        def __init__(self):
            self.attr1 = TopTest.count
            self.attr2 = TopTest.count+1
            TopTest.count +=2
            
    class SubTest(TopTest):
        pass
        
    X,Y=TopTest(),SubTest()
    print(X)
    print(Y)
\end{python}


\section{输出结果}
\dopython[0]{classtools}



\chapter{person.py}
\begin{python}
#!/usr/bin/env python3
# -*- coding: utf-8 -*-

##################################
#序言部分
\end{python}



\section{Person}
\begin{python}
from classtools import AttrDisplay
class Person(AttrDisplay):
    def __init__(self, name, job=None, pay=0):
        self.name = name
        self.job = job
        self.pay = pay
    def lastname(self):
        return self.name.split()[-1]
    def giveraise(self,percent):
        self.pay = int(self.pay * (1+percent))
#    def __repr__(self):
#        return '[Person: %s,%s]' % (self.name,self.pay)
\end{python}

类名一般都大写。

特殊的\verb+__init__+ 用于这个类具体创建instance实例的时候执行的动作。self表示创建的那个实例,self.name表示实例的名字,self.name =name表示接受的name将会传递值给self.name,同时创建的那个实例将会拥有一个自己的name属性。其他属性操作类似。\verb+__init__+ 这里实际上也是进行了函数重载。

和一般函数的做法一样,job=None,表示job这个参数是一个可选参数,它有一个默认值None。

这里新建了一个Person类,这个类有三个属性:name,姓名,job,工作,和pay,薪酬。

新建了lastname方法,将会该实例的名字的最后的姓氏。

新建了giveRaise方法,还需要一个参数percent,这样该实例的pay属性将会提高这么多百分比。这里的int函数是将数值转化为整数。

重新定义\verb+__repr__+,将会影响print函数的行为。


\subsection{Manager}
\begin{python}
class Manager(Person):
    def __init__(self,name,pay):
        Person.__init__(self,name,'mgr',pay)
    def giveraise(self,percent,bonus=0.10):
        Person.giveraise(self,percent+bonus)
\end{python}
这里定义了一个类Manager,它还接受一个参数Person,表示它是Person的子类,即一切Person类的属性它都将继承。

这里重新定义了giveraise方法,用一种巧妙的方式。直接借用Person类原有的giveraise方法,对参数输入稍作修正。

重载\verb+__init__+方法,提供更加灵活的本地方案。

在python中:\\
\verb+instance.method(args...)+\\
都会化成这样的形式:\\
\verb+class.method(instance, args...)+\\
这种转换只针对实例。

python中超类,子类,实例的重载是由一种搜索机制实现的:
\begin{fig}{类搜索结构图}
\label{fig:类搜索结构图}
\end{fig}
python首先搜索self有没有这个属性或者方法,如果没有,就向上搜索。比如说实例没有,就向上搜索那个子类,子类没有就向上搜索那个超类。



\section{测试代码}
\begin{python}
if __name__ == '__main__':
    bob = Person('Bob Smith')
    sue = Person('Sue Jones', job='dev', pay=100000)
    print(bob)
    print(sue)
    print(bob.lastname())
    sue.giveraise(0.10)
    print(sue)
    tom=Manager('Tom Jones',50000)
    tom.giveraise(0.10)
    print(tom.lastname())
    print(tom)
\end{python}
如果py文件是import的形式,那么这段代码将不会执行。只有以脚本python3 test.py之类的形式才执行。

可见继承过来的类将不会有自己默认定义的\verb+__init__+等方法。

\section{输出结果}
\dopython{person}








%这里空一行

\end{common-format}
\end{document}



