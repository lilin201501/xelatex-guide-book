% !Mode:: "TeX:UTF-8"%確保文檔utf-8編碼
%新加入的命令如下: reduline showendnotes 
%新加入的环境如下:solution solutionorbox solutionorlines solutionordottedlines

\documentclass[12pt]{exam}
\newlength{\textpt}
\setlength{\textpt}{12pt}

\usepackage{teachingplan}

%输出方案 
%学生版 学霸版 老师版

%写上答案或者不写上答案%1  
\printanswers  

%题目类型划分:A类题,基本知识题;B类题,中等难度;C类题, 难题;D类题,中考题。
\excludecomment{Aquestions}
%\includecomment{Aquestions}
\excludecomment{Bquestions}
%\includecomment{Bquestions}
\excludecomment{Cquestions}
%\includecomment{Cquestions}
\excludecomment{Dquestions}
%\includecomment{Dquestions}

\CenterWallPaper{1}{教案模板-2.pdf}


\newcommand{\keti}{光现象}
\newcommand{\zhongdian}{1.光的传播 2.}
\firstpageheader{}{}{\today}

\begin{document}
\ThisCenterWallPaper{1}{教案模板-1.pdf}
\vspace*{80pt}
\keti \par
\zhongdian \par
\section{光的传播}
我们放眼望去,能够看到很多东西,甚至有些东西还很亮,比如月亮,但只有那些能够发光的物体叫光源,如太阳、灯泡等。

光的传播的规律是:在同一\textbf{均匀}介质中是沿\textbf{直线}传播的。\raisebox{0.5em}{\tikz{\draw[ultra thick, color=red,-latex] (0,0) --(1,0);}}。通常我们用带箭头的直线表示光的轨迹和方向,这样的直线叫做\textbf{光线}。

这里均匀介质的意思是光一直在真空中或者空气中或者水中或者玻璃中行走,如果光从一种介质进入另一种介质,通常会发生\textbf{折射}现象。(这里通常的意思指一般两个介质\textit{折射率}是不同的。)

光在不同的介质中有不同的运动速度,在真空或空气中,光的速度是。




\subsection{激光应用}


\subsection{影子}



\subsection{小孔成像}


\subsection{日食和月食}




\section{光的反射}



\section{光的折射}
一般折射

生活中的折射  早上的太阳  海市蜃楼



4、应用及现象
  
  ①激光准直。
②影子的形成:光在传播过程中,遇到不透明的物体,在物体的后面形成黑色区域即影子。
③日食月食的形成:当地球 在中间时可形成月食。如图:在月球后1的位置可看到日全食,在2的位置看到日偏食,在3的位置看到日环食。
④小孔成像:小孔成像实验早在《墨经》中就有记载小孔成像成倒立的实像,其像的形状与孔的形状无关。

5、光速:
光在真空中速度C=3×108m/s=3×105km/s;光在空气中速度约为3×108m/s。光在水中速度为真空中光速的3/4,在玻璃中速度为真空中速度的2/3 。
二、光的反射
1、定义:光从一种介质射向另一种介质表面时,一部分光被反射回原来介质的现象叫光的反射。
2、反射定律:三线同面,法线居中,两角相等,光路可逆.即:反射光线与入射光线、法线在同一平面上,反射光线和入射光线分居于法线的两侧,反射角等于入射角。光的反射过程中光路是可逆的。不发光物体把照在它上面的光反射进入我们的眼睛

3、分类:

(1)镜面反射:
   定义:射到物面上的平行光反射后仍然平行
   条件:反射面 平滑。
   应用:迎着太阳看平静的水面,特别亮。黑板“反光”等,都是因为发生了镜面反射

(2)漫反射:
   定义:射到物面上的平行光反射后向着不同的方向 ,每条光线遵守光的反射定律。
   条件:反射面凹凸不平。
   应用:能从各个方向看到本身不发光的物体,是由于光射到物体上发生漫反射的缘故。
三、平面镜成像
1、平面镜:
成像特点:等大,等距,垂直,虚像
   ①像、物大小相等
   ②像、物到镜面的距离相等。
   ③像、物的连线与镜面垂直
   ④物体在平面镜里所成的像是像。
成像原理:光的反射定理;作用:成像、 改变光路。
实像和虚像:
   实像:实际光线会聚点所成的像
   虚像:反射光线反向延长线的会聚点所成的像
2、球面镜:
    定义:用球面的内表面作反射面。
凹面镜  性质:凹镜能把射向它的平行光线会聚在一点;从焦点射向凹镜的反射光是平行光
    应用:太阳灶、手电筒、汽车头灯。
    定义:用球面的外表面做反射面。
凸面镜  性质:凸镜对光线起发散作用。凸镜所成的象是缩小的虚像
    应用:汽车后视镜
四、光的折射
1、折射:光从一种介质斜射入另一种介质时,传播方向发生偏折,这种现象叫做光的折射。当发生折射现象时,一定也发生了反射现象。当光线垂直射向两种物质的界面时,传播方向不变。
2、光的折射规律:在折射现象中,折射光线、入射光线和法线都在同一个平面内;光从空气斜射入水中或其他介质中时,折射光线向法线方向偏折(折射角<入射角);光从水或其他介质中斜射入空气中时,折射光线向界面方向偏折(折射角>入射角)。在折射现象中,光路是可逆的。在光的折射现象中,入射角增大,折射角也随之增大。在光的折射现象中,介质的密度越小,光速越大,与法线形成的角越大。
3、折射的现象:①从岸上向水中看,水好像很浅,沿着看见鱼的方向叉,却叉不到;从水中看岸上的东西,好像变高了。②筷子在水中好像“折”了。③海市蜃楼。④彩虹。

从岸边看水中鱼N的光路图(图1): 图中的N点是鱼所在的真正位置,N'点是我们看到的鱼,从图中可以得知,我们看到的鱼比实际位置高。像点就是两条折射光线的反向延长线的交点。在完成折射的光路图时可画一条垂直于介质交界面的光线,便于绘制。
五、光的色散
1、光的色散:光的色散属于光的折射现象。1666年,英国物理学家牛顿用玻璃三棱镜使太阳光发生了色散(图2)。太阳光通过棱镜后,被分解成各种颜色的光,用一个白屏来承接,在白屏上就形成一条颜色依次是红、橙、黄、绿、蓝、靛、紫的彩带。牛顿的实验说明白光是由各种色光混合而成的。
2、色光的三原色:红、绿、蓝。红、绿、蓝三种色光,按不同比例混合,可以产生各种颜色的光。(图3)
光的色散色光的三原色颜料的三原色
图2                                 图3
3、物体的颜色:透明物体的颜色由通过它的色光来决定。如图4,如果在白屏前放置一块红色玻璃,则白屏上其他颜色的光消失,只留下红色。这表明,其他色光都被红色玻璃吸收了,只有红光能够透过。不透明物体的颜色是由它反射的色光决定的。如图4,如果把一张绿纸贴在白屏上,则在绿纸上看不到彩色光带,只有被绿光照射的地方是亮的(反射绿光),其他地方是暗的(不反射光)。如果一个物体能反射所有色光,则该物体呈现白色。如果一个物体能吸收所有色光,则该物体呈现黑色。如果一个物体能透过所有色光,则该物体是无色透明的。



平面镜的作用:改变光的传播方向,使得射向天花板的光能够在屏幕上成像。	
实像和虚像(见下图):照相机和投影仪所成的像,是光通过凸透镜射出后会聚在那里所成的,如果把感光胶片放在那里,真的能记录下所成的像。这种像叫做实像。物体和实像分别位于凸透镜的两侧。
凸透镜成实像情景:光屏能承接到所形成的像,物和实像在凸透镜两侧。

凸透镜成虚像情景:光屏不能承接所形成的像,物和虚像在凸透镜同侧。




2、光现象
光源:能够 发 光 的物体叫光源。
光的直线传播:光在 同一均匀介质 中是沿 直 线 传播。
光在真空中传播速度最大,是3×108米/秒,而在空气中传播速度也认为是3×108米/秒。
我们能看到不发光的物体是因为这些物体 反 射 的 光 射 入 了 我们的眼睛。
光的反射定律:反射光线与入射光线、法线在 同 一 平 面 上,反射光线与入射光线 分 居 法线 两 侧,反 射 角 等于 入 射 角。(注:光路是 可 逆 的)
        入射光线                    法线                反射光线

         
   镜面

漫反射和镜面反射一样遵循光的 反 射 定 律。
平面镜成像特点:(1)像与物体大小 相 同 (2)像到镜面的距离 等 于 物体到镜面的距离(3)像与物体的连线与镜面 垂 直 (4)平面镜成的是 虚 像 。
平面镜应用:(1)  成 像      (2)  改变   光  路 。
色散: 色光的三原色 红 、绿 、 蓝    
  透明物体的顏色由通过它的 色 光 决 定 的
   不透明物体的颜色是由它反射的 色 光 决 定 的
10、光的折射:光从一种介质 射 入 另一种介质时,传 播 方 向 一般发生变化的现象。
11、光的折射规律:光从空气 斜 射 入水或玻璃表面时,折射光线与入射光线、法线在 同一 平 面 上;折射光线和入射光线 分 居 法 线 两 侧,折射光线向法线 靠 拢 ,折射角 小 于 入射角;入射角增大时,折射角也随着 增 大 ;当光线垂直射向介质表面时,传播方向 不 改 变。



\begin{Aquestions}
\newpage
\section{练习题}


\end{Aquestions}


\begin{Bquestions}
\newpage
\section{一般题}
\end{Bquestions}




\begin{Cquestions}
\newpage
\section{难题}
\end{Cquestions}



\begin{Dquestions}
\newpage
\section{中考题}
\end{Dquestions}




%
\ThisCenterWallPaper{1}{教案模板-3.pdf}

\end{document}



