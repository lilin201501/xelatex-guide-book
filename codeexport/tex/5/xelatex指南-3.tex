
% 一种书籍的风格,对中文处理较好。
%
% by wanze
% LPPL LaTeX Public Project License
%
%

\NeedsTeXFormat{LaTeX2e}[1994/06/01]
\ProvidesPackage{myconfig}

%\def\hi{Hello, this is my own package}
%\let\myDate\date
%\newcommand\GoodBye[1][\bfseries]{{#1Good Bye}}

\RequirePackage{calc,float,multicol,moresize}
\RequirePackage[doublespacing]{setspace}
\newcommand{\addchtoc}[1]{
\cleardoublepage
\phantomsection
\addcontentsline{toc}{chapter}{#1}}

\newcommand{\addsectoc}[1]{
\phantomsection
\addcontentsline{toc}{section}{#1}}

\newenvironment{common-format}{ %
\setlength{\parskip}{1.6ex plus 0.2ex minus 0.2ex}   %段落間距
\setlength{\parindent}{\baselineskip * \real{0.06} + \textpt * \real{2.4}}}
    {}


\ifphone
%for phone
\RequirePackage[
  paperwidth=105mm, %除去旁註其他沒變,115,再稍微小點
  paperheight=190mm,%太長了縮短點,
  bindingoffset=0mm,%裝訂線
  top=15mm,  %上邊距 包括頁眉
  bottom=15mm,%下邊距 包括頁腳
  left=5mm,  %左邊距or inner
  right=5mm,  %右邊距or  outer
  headheight=10mm,%頁眉
  footskip=10mm,%頁腳
  includemp=true,% 旁註寬度計入width
  marginparsep=0mm, %沒有旁註
  marginparwidth=0mm,  %沒有旁註
  ]{geometry}
\else

\RequirePackage[a4paper, %a4paper size 297:210 mm
  bindingoffset=0mm,%裝訂線
  top=45mm,  %上邊距 包括頁眉
  bottom=40mm,%下邊距 包括頁腳
  left=35mm,  %左邊距or inner
  right=40mm,  %右邊距or  outer
  headheight=25mm,%頁眉
  footskip=25mm,%頁腳
 includemp=true,% 旁註寬度計入width
 marginparsep=0mm, %旁註與正文間距
marginparwidth=0mm,  %旁註寬度
  ]{geometry}
\fi

\RequirePackage[table]{xcolor}
\definecolor{bgcolor-0}{HTML}{CCFFCC}

\newsavebox{\tempbox}
\newenvironment{fancycolorbox}[1][bgcolor-0]
 {\noindent%%
  \renewcommand{\tempcolor}{#1}
  \begin{lrbox}{\tempbox}%
  \begin{minipage}{\linewidth-10pt}
   \setlength{\parskip}{1.6ex plus 0.2ex minus 0.2ex}   %段落間距
\setlength{\parindent}{\baselineskip * \real{0.06} + \textpt * \real{2.4}}}
 {\ignorespacesafterend%
  \end{minipage}%
  \end{lrbox}%
  \colorbox{\tempcolor}{\usebox{\tempbox}}}


%================字體================%
\RequirePackage{xltxtra,fontspec,xunicode} %必備三件套
\RequirePackage[CJKnumber=true]{xeCJK} %中文環境宏
\xeCJKsetup{PunctStyle=plain}
\defaultCJKfontfeatures{Scale=1.2}   %放大全局CJK字體。中文字應該稍微高於英文字
\setCJKmainfont[BoldFont=Adobe 黑体 Std,ItalicFont=Adobe 楷体 Std]
    {Adobe 宋体 Std}%影響rmfamily字體
\setCJKsansfont{Adobe 黑体 Std}%影響sffamily字體
\setCJKmonofont{Adobe 楷体 Std}%影響ttfamily字體
 %設置英文字體
\setmainfont[Mapping=tex-text]{DejaVu Serif}
\setsansfont[Mapping=tex-text]{DejaVu Sans}
\setmonofont[Mapping=tex-text]{DejaVu Sans Mono}

%=============新的字符===========%
\newfontfamily{\libertine}[Scale=1.5]{Linux Libertine O}
\newfontfamily{\ubuntu}[Scale=3]{Ubuntu}
\RequirePackage{newunicodechar}
\newunicodechar{Ⓐ}{{\libertine{Ⓐ}}}
\newunicodechar{Ⓑ}{{\libertine{Ⓑ}}}
\newunicodechar{Ⓒ}{{\libertine{Ⓒ}}}
\newunicodechar{Ⓓ}{{\libertine{Ⓓ}}}
\newunicodechar{①}{{\libertine{①}}}
\newunicodechar{②}{{\libertine{②}}}
\newunicodechar{③}{{\libertine{③}}}
\newunicodechar{④}{{\libertine{④}}}
\newunicodechar{⑤}{{\libertine{⑤}}}
\newunicodechar{}{{\ubuntu{}}}

%%===============中文化=========%
\renewcommand\contentsname{目~录}
\renewcommand\listfigurename{插图目录}
\renewcommand\listtablename{表格目录}
\renewcommand\bibname{参~考~文~献}
\renewcommand\indexname{索~引}
\renewcommand\figurename{图}
\renewcommand\tablename{表}
\renewcommand\partname{部分}
\renewcommand\appendixname{附录}
\renewcommand\today{\number\year年\number\month月\number\day日}

\RequirePackage{fancyhdr}   %頁眉頁腳
\pagestyle{fancy}
\fancypagestyle{plain}{
    \fancyhf{}
    \renewcommand{\headrulewidth}{0pt}
    \renewcommand{\footrulewidth}{0pt}
    \renewcommand{\chaptermark}[1]{\markboth{第\CJKnumber{\arabic{chapter}}章~~#1}{}}
     \renewcommand{\sectionmark}[1]{\markright{第\CJKnumber{\arabic{section}}節~~#1}{}}
%    \fancyhf[HL]{\ttfamily \footnotesize \leftmark }
    \fancyhf[HR]{\ttfamily \footnotesize \rightmark }
    \fancyhf[FR]{\thepage}
    \fancyhfoffset[R]{\marginparwidth+\marginparsep}
    }
\pagestyle{plain}

%=========章節標題設計=========%
\RequirePackage{titlesec}
%修改part
\titleformat{\part}{\huge\sffamily}{}{0em}{}
%修改chapter
\titleformat{\chapter}{\LARGE\sffamily}{}{0em}{}
%修改section
\titleformat{\section}{\Large\sffamily}{}{0em}{}
%修改subsection
\titleformat{\subsection}{\large\sffamily}{}{0em}{}
%修改subsubsection
\titleformat{\subsubsection}{\normalsize\sffamily}{}{0em}{}


%================目录===============%
\RequirePackage{titletoc}

%==============超鏈接===============%
\RequirePackage[colorlinks=true,linkcolor=blue,citecolor=blue]{hyperref} %設置書簽和目錄鏈接等

%=================文字強調=========%
\RequirePackage{ulem} %下劃線,加點
\normalem

\newcommand\reduline{\bgroup\markoverwith
{\textcolor{red}{\rule[-0.8ex]{1em}{0.4pt}}}\ULon}
\renewcommand\emshape{\color{red}}

%==================插入圖片=======%
\RequirePackage{graphicx}
\graphicspath{{figures/}}
\newenvironment{fig}[1]
{\begin{figure}[h]
\includegraphics[width=\linewidth ,totalheight=\textheight , keepaspectratio]{#1}
\caption{#1}
\end{figure}}
{}
\newenvironment{scalefig}[2][0.4]
{\begin{figure}[h]
\includegraphics[scale=#1]{#2}
\caption{#2}
\end{figure}}
{}

%==============插入表格========%
\RequirePackage{booktabs}
\renewcommand{\arraystretch}{1.3}

%============插入代码============%
\RequirePackage{fancyvrb}
\DefineVerbatimEnvironment%
{verbatim}{Verbatim}
{numbers=left,frame=lines,tabsize=4 ,baselinestretch=2,
xleftmargin=6pt, fontsize=\footnotesize , numbersep=2pt}

\newlength{\fancyvrbtopsep}
\newlength{\fancyvrbpartopsep}
\makeatletter
\FV@AddToHook{\FV@ListParameterHook}
{\topsep=\fancyvrbtopsep\partopsep=\fancyvrbpartopsep}
\makeatother
\setlength{\fancyvrbtopsep}{-15pt}   %代碼環境之上空白高度
\setlength{\fancyvrbpartopsep}{0pt}  %段落之上空白高度

%=============插入注释=========%
\RequirePackage{endnotes}
\RequirePackage{hyperendnotes}
\renewcommand\makeenmark{(\theenmark)}
\renewcommand\notesname {注释和参考资料}
\newcommand{\printendnotes}
    {\theendnotes}






\endinput


