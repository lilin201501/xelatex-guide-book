\part{后面的珍宝}
\begin{quotation}
"The opera ain't over until the fat lady sings."

好戏还在后头,等著瞧吧。
\end{quotation}

我在Ubuntu的模板文件夹里面新建了一个tex.tex文件。方便我用鼠标右键快速创建模板。

第一个模板我打算使用一个通用的配置,写一个myconfig宏包,用于加载这个通用配置。还有一个mytitle宏包用于加载封面。然后整个模板就专注于写作,很简洁的样子。

然后我在Ubuntu的home文件目录或者称之为主文件夹下面新建了一个文件夹texmf,然后里面新建了一个文件夹tex,然后里面新建了一个文件夹latex,然后在里面新建了一个文件夹myconfig。myconfig.sty文件就放在里面。然后在终端上运行命令:\verb+sudo~ texhash+即可。mytitle宏包的安装类似。

myconfig宏包的内容是我以前之前的XeLaTeX指南的精华部分,很多冗余删除了。关于myconfig和mytitle宏包的编写,我参考了\href{http://tex.stackexchange.com/questions/70166/create-a-function-that-generates-a-title-page}{这个网站}。

三个文件的代码在本文后面列出。现在就一些必要的信息说明如下:

首先sty文档里面@符号不需要用makeatletter这个命令了,@在里面已经默认就是一个符号。
其次文档的基本结构
\begin{verbatim}
\NeedsTeXFormat{LaTeX2e}
\ProvidesPackage{mytitle}
…
\endinput
\end{verbatim}
这个不用多说。

然后就是sty文档里面所有usepackage都换成RequirePackage。这样sty文档兼容性更好,前面没有documentclass也行。

myconfig宏包的详细讨论我在XeLaTeX指南一书中会详细讨论。
mytitle宏包前面那部分是定义新的输入命令,第二行是用于处理可能出现错误的反应。就照这个模式写。然后gdef命令后面就跟著新定义的命令的形式,gdef和def命令的不同是他定义的是global的,我估计之前我那种写法不行可能就是没有gdef。
后面就是基本的封面设计了。

特别提醒在Ubuntu下右键新建之后,是未命名 然后空格 然后tex,这种文件名XeTeX处理似乎不很好。建议改成没有空格的文件名。

\section{tex.tex代码}
\begin{verbatim}
% !Mode:: "TeX:UTF-8"%確保文檔utf-8編碼
%新加入的命令如下:addchtoc addsectoc reduline printendnotes 
%新加入的环境如下:common-format  fig scalefig 

\documentclass[11pt,oneside]{book}
\newlength{\textpt}
\setlength{\textpt}{11pt} 
\newif\ifphone
\phonefalse

\usepackage{myconfig}
\usepackage{mytitle}

\begin{document}
\frontmatter   

\titlea{书籍}
\titleb{使用\LaTeX排版}
\titlec{一种良好的风格}
\author{作者}
\authorinfo{作者:}
\editor{编者}
\email{a358003542@gmail.com}
\editorinfo{编者:}
\version{1.0}
\mytitle

\addchtoc{开头说的话}
\chapter*{开头说的话}
\begin{common-format}
开头说的话

%这里空一行。

\end{common-format}


\addchtoc{目錄}
\setcounter{tocdepth}{1}    
\tableofcontents

\begin{common-format}
\mainmatter 

\chapter{章节开始}

开始写作。。。




%这里空一行

\end{common-format}  
\end{document}




\end{verbatim}


\section{myconfig.sty}
\begin{verbatim}
% 一种书籍的风格,对中文处理较好。
% 
% by wanze
% LPPL LaTeX Public Project License
%  
%

\NeedsTeXFormat{LaTeX2e}[1994/06/01]
\ProvidesPackage{myconfig}

%\def\hi{Hello, this is my own package}
%\let\myDate\date
%\newcommand\GoodBye[1][\bfseries]{{#1Good Bye}}

\RequirePackage{calc,float,multicol,moresize} 
\RequirePackage[doublespacing]{setspace}
\newcommand{\addchtoc}[1]{ 
	\cleardoublepage   
	\phantomsection    
	\addcontentsline{toc}{chapter}{#1}}
	
\newcommand{\addsectoc}[1]{ 
	\phantomsection    
	\addcontentsline{toc}{section}{#1}}

\newenvironment{common-format}{ %
	\setlength{\parskip}{1.6ex plus 0.2ex minus 0.2ex}   %段落間距
	\setlength{\parindent}{\baselineskip * \real{0.06} + \textpt * \real{2.4}}}  
    {}


\ifphone
%for phone
\RequirePackage[
  paperwidth=105mm, %除去旁註其他沒變,115,再稍微小點
  paperheight=190mm,%太長了縮短點,
  bindingoffset=0mm,%裝訂線
  top=15mm,  %上邊距 包括頁眉
  bottom=15mm,%下邊距 包括頁腳
  left=5mm,  %左邊距or inner
  right=5mm,  %右邊距or  outer
  headheight=10mm,%頁眉
  footskip=10mm,%頁腳
  includemp=true,% 旁註寬度計入width
  marginparsep=0mm, %沒有旁註
  marginparwidth=0mm,  %沒有旁註
  ]{geometry}
\else

\RequirePackage[a4paper, %a4paper size 297:210 mm
  bindingoffset=0mm,%裝訂線
  top=45mm,  %上邊距 包括頁眉
  bottom=40mm,%下邊距 包括頁腳
  left=35mm,  %左邊距or inner
  right=40mm,  %右邊距or  outer
  headheight=25mm,%頁眉
  footskip=25mm,%頁腳
 includemp=true,% 旁註寬度計入width
 marginparsep=0mm, %旁註與正文間距
marginparwidth=0mm,  %旁註寬度
  ]{geometry}
\fi

\RequirePackage[table]{xcolor}  
\definecolor{bgcolor-0}{HTML}{CCFFCC} 

\newsavebox{\tempbox}
\newenvironment{fancycolorbox}[1][bgcolor-0]
 {\noindent%%
  \renewcommand{\tempcolor}{#1}
  \begin{lrbox}{\tempbox}%
  \begin{minipage}{\linewidth-10pt}
  	\setlength{\parskip}{1.6ex plus 0.2ex minus 0.2ex}   %段落間距
	\setlength{\parindent}{\baselineskip * \real{0.06} + \textpt * \real{2.4}}}  	
 {\ignorespacesafterend%
  \end{minipage}%
  \end{lrbox}%
  \colorbox{\tempcolor}{\usebox{\tempbox}}}  


%================字體================%
\RequirePackage{xltxtra,fontspec,xunicode} %必備三件套
\RequirePackage[CJKnumber=true]{xeCJK} %中文環境宏
\xeCJKsetup{PunctStyle=plain}
\defaultCJKfontfeatures{Scale=1.2}   %放大全局CJK字體。中文字應該稍微高於英文字
\setCJKmainfont[BoldFont=Adobe 黑体 Std,ItalicFont=Adobe 楷体 Std]
    {Adobe 宋体 Std}%影響rmfamily字體
\setCJKsansfont{Adobe 黑体 Std}%影響sffamily字體
\setCJKmonofont{Adobe 楷体 Std}%影響ttfamily字體
 %設置英文字體
\setmainfont[Mapping=tex-text]{DejaVu Serif} 
\setsansfont[Mapping=tex-text]{DejaVu Sans}
\setmonofont[Mapping=tex-text]{DejaVu Sans Mono}

%=============新的字符===========%
\newfontfamily{\libertine}[Scale=1.5]{Linux Libertine O}
\newfontfamily{\ubuntu}[Scale=3]{Ubuntu}
\RequirePackage{newunicodechar}
\newunicodechar{Ⓐ}{{\libertine{Ⓐ}}}
\newunicodechar{Ⓑ}{{\libertine{Ⓑ}}}
\newunicodechar{Ⓒ}{{\libertine{Ⓒ}}} 
\newunicodechar{Ⓓ}{{\libertine{Ⓓ}}}
\newunicodechar{①}{{\libertine{①}}}
\newunicodechar{②}{{\libertine{②}}}
\newunicodechar{③}{{\libertine{③}}}
\newunicodechar{④}{{\libertine{④}}}
\newunicodechar{⑤}{{\libertine{⑤}}}
\newunicodechar{}{{\ubuntu{}}}

%%===============中文化=========%
\renewcommand\contentsname{目~录}
\renewcommand\listfigurename{插图目录}
\renewcommand\listtablename{表格目录}
\renewcommand\bibname{参~考~文~献}
\renewcommand\indexname{索~引}
\renewcommand\figurename{图}
\renewcommand\tablename{表}
\renewcommand\partname{部分}
\renewcommand\appendixname{附录}
\renewcommand\today{\number\year年\number\month月\number\day日}

\RequirePackage{fancyhdr}   %頁眉頁腳
\pagestyle{fancy}
\fancypagestyle{plain}{
    \fancyhf{}
    \renewcommand{\headrulewidth}{0pt}
    \renewcommand{\footrulewidth}{0pt}
    \renewcommand{\chaptermark}[1]{\markboth{第\CJKnumber{\arabic{chapter}}章~~#1}{}} 
     \renewcommand{\sectionmark}[1]{\markright{第\CJKnumber{\arabic{section}}節~~#1}{}} 
%    \fancyhf[HL]{\ttfamily \footnotesize \leftmark }
    \fancyhf[HR]{\ttfamily \footnotesize \rightmark }
    \fancyhf[FR]{\thepage}
    \fancyhfoffset[R]{\marginparwidth+\marginparsep}
    }
\pagestyle{plain} 

%=========章節標題設計=========%
\RequirePackage{titlesec}
%修改part
\titleformat{\part}{\huge\sffamily}{}{0em}{} 
%修改chapter
\titleformat{\chapter}{\LARGE\sffamily}{}{0em}{} 
%修改section
\titleformat{\section}{\Large\sffamily}{}{0em}{}
%修改subsection
\titleformat{\subsection}{\large\sffamily}{}{0em}{}
%修改subsubsection
\titleformat{\subsubsection}{\normalsize\sffamily}{}{0em}{}


%================目录===============%
\RequirePackage{titletoc}

%==============超鏈接===============%
\RequirePackage[colorlinks=true,linkcolor=blue,citecolor=blue]{hyperref} %設置書簽和目錄鏈接等

%=================文字強調=========%
\RequirePackage{ulem} %下劃線,加點
\normalem

\newcommand\reduline{\bgroup\markoverwith
{\textcolor{red}{\rule[-0.8ex]{1em}{0.4pt}}}\ULon}
\renewcommand\emshape{\color{red}}

%==================插入圖片=======%
\RequirePackage{graphicx}
\graphicspath{{figures/}}
\newenvironment{fig}[1]
	{\begin{figure}[h]
	\includegraphics[width=\linewidth ,totalheight=\textheight , keepaspectratio]{#1}
	\caption{#1}
	\end{figure}}
	{}	
\newenvironment{scalefig}[2][0.4]
	{\begin{figure}[h]
	\includegraphics[scale=#1]{#2}
	\caption{#2}
	\end{figure}}
	{}	

%==============插入表格========%
\RequirePackage{booktabs}
\renewcommand{\arraystretch}{1.3}

%============插入代码============%
\RequirePackage{fancyvrb} 
\DefineVerbatimEnvironment%
	{verbatim}{Verbatim}
	{numbers=left,frame=lines,tabsize=4 ,baselinestretch=2,
	xleftmargin=6pt, fontsize=\footnotesize , numbersep=2pt}  	
	
\newlength{\fancyvrbtopsep}
\newlength{\fancyvrbpartopsep}
\makeatletter
	\FV@AddToHook{\FV@ListParameterHook}
	{\topsep=\fancyvrbtopsep\partopsep=\fancyvrbpartopsep}
\makeatother
\setlength{\fancyvrbtopsep}{-15pt}   %代碼環境之上空白高度
\setlength{\fancyvrbpartopsep}{0pt}  %段落之上空白高度

%=============插入注释=========%
\RequirePackage{endnotes}
\RequirePackage{hyperendnotes}
\renewcommand\makeenmark{(\theenmark)}
\renewcommand\notesname {注释和参考资料}
\newcommand{\printendnotes}
    {\theendnotes}






\endinput

\end{verbatim}


\section{mytitle.sty}
\begin{verbatim}
\NeedsTeXFormat{LaTeX2e}
\ProvidesPackage{mytitle}


\def\titlea#1{\gdef\@titlea{#1}}
\def\@titlea{\@latex@warning@no@line{No \noexpand\titlea given}}
\def\titleb#1{\gdef\@titleb{#1}}
\def\@titleb{\@latex@warning@no@line{No \noexpand\titleb given}}
\def\titlec#1{\gdef\@titlec{#1}}
\def\@titlec{\@latex@warning@no@line{No \noexpand\titlec given}}

\def\authorinfo#1{\gdef\@authorinfo{#1}}
\def\@authorinfo{\@latex@warning@no@line{No \noexpand\authorinfo given}}
\def\email#1{\gdef\@email{#1}}
\def\@email{\@latex@warning@no@line{No \noexpand\email given}}
\def\editorinfo#1{\gdef\@editorinfo{#1邮箱:\href{mailto: \@email}{\@email}。}}
\def\@editorinfo{\@latex@warning@no@line{No \noexpand\editorinfo given}}

\def\editor#1{\gdef\@editor{#1}}
\def\@editor{\@latex@warning@no@line{No \noexpand\editor given}}
\def\version#1{\gdef\@version{#1}}
\def\@version{\@latex@warning@no@line{No \noexpand\version given}}


%默认的title样式  继续添加的有mytitlea,mytitleb,mytitlec。。。
\newcommand{\mytitle}{
	\begin{titlepage}
	\begin{flushleft}	
	 \vspace*{\stretch{2}}
{\HUGE\bfseries \@titlea}\\[\stretch{1}]
{\Huge\bfseries \@titleb}\\[\stretch{1}]
{\LARGE\itshape \@titlec}\\[\stretch{2}]
{\Large \@author\footnote{\@authorinfo}}\quad\rule{0.8pt}{3ex}\quad
{\large \@editor\footnote{\@editorinfo}}\\[\stretch{2}]
\vfill
{\ttfamily 版本: \@version}
\end{flushleft}
\end{titlepage}
}


\endinput

\end{verbatim}

\chapter{源码和参考资料打包下载地址}